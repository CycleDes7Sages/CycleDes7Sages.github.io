\documentclass{article}
\usepackage[T1]{fontenc}
\usepackage{lmodern}
\usepackage{microtype}
\usepackage[pdfusetitle,hidelinks]{hyperref}
\usepackage[english]{babel}
\usepackage[series={},nocritical,noend,noeledsec,nofamiliar,noledgroup]{reledmac}
\usepackage{reledpar}


\begin{document}

\date{}
        \title{Roman de Pelyarmenus}
        \author{Camille Carnaille}
\maketitle

\begin{abstract}
Abstract to be added
\end{abstract}
\begin{pages}
\beginnumbering
        
   
  
      
         Trois lignes sont effacées et restent illisibles avant le début du texte donné dans V3. 
          Comment Helcanus issi de prison et ala en la queste
         de Nera, 
            s'amiesa s'amie.
            
               Enluminure sur 2 colonnes (a-b) et 16 UR. 
                  Helcanus sort de prison et 
                  part à la recherche de Nera son amie, 
                  en armes vers forêt hors d’un château grillagé avec deux hommes le saluant à sa sortie.
            
\pstart  Ci endroit dist li comptes que, quant 
   Helcanus ot entendu la verité de Nera 
   s’amie, que plus pres li point
   lese cerchier et le querre que ne feist a retourner en 
   Grece pour retourner a 
   son pere et por lui faire aide ne 
   aus autres barons. Adont se mist hors de Romme 
   et nene ne finefina
   de cerchier et de querre tout ainssi commeque il cuidoit que 
   la pucele fust venue a lui. Dont ne fina d’aler l’un jour aprés l’autre tant que il s’embati 
   par aventure en i trespas ou moult passoit de pluseurs manieres de
   gens, et tantcaut
   Explication paléographique à cette variante, probablement issue de la copie de l'exemplaire
   dont est issu V3, d'une mélecture du t en c et du n
   en u. que il fu hebergiez en i 
   chastel, aussi comme il faisoit souvent, et enquist de sa besoingne si comme il 
   li couvenoitcouvient 
   fairefaire ceulz qui ce vont chacant. Dont le regarda
   li hostes de l’ostel et li dist :
   AutreSire, autre 
      i fois 
      a
      esté demandeedemandee ceens 
      celle dont vous parlez. Mais or me dites s’ele vous est riens.
   Par foi, siresire, dist il, si est !
   Dont li dist que ele li estoit si pres que li uns ne devoit faillir a l’autre.Sire, dist il, je le vous di por une chose. Bien 
      aa ce quit x semaines passees 
      queque queque c
      vindrentjurent marcheant de Pise,
      et venoient d’Atheines. Si conterent ceens que ii 
         fames s’estoient mises avec eulz en mer et furent detenues pour leur mesfait par devers 
         les mariniersmaronniers.
      Et croiquit que eles ont puis eu a souffrir. Je ne sai mie se ce sont 
      celes que vous alez querant. \pend
\pstart Pas de nouveau § B Quant Helcanus oÿ 
   ce, si li dist li cuer que ce estoient
      elesceles que il aloient querant. Maintenant sailli em piez et 
   s’apresta a cele heure meismes et vint a son cheval et monta sus et se mist 
   cele part ouque li hostes 
   li avoit enseignié. Quantil se fu mis au chemin, il fist poi de sejour, si vint au port 
   tout droit dont cil estoient descendu qui la nouvele 
   en avoient contee. Or avint si conme Diex 
   L'intervention divine nous paraît souligner l'importance de la réunion de Helcanus et Nera
   tant pour la survie de Nera que pour l'actualisation finalement de leur relation, longtemps retardée.
   lelui voult la mener que la nouvele estoit espandue 
   partout de la pucele qui estoit encusee de 
   l’enfant et meismes des ii 
   mariniersmaronniers
   qui avoient esté trouvé mort en la nef avec celes. Et quant 
   HelcanusHelcanu 
   oÿ de ce parlerce, si fu a trop grant meschief et vint a 
   i preudonme et li dist :
   Biaus tres douz sires, dites moi se je porrai 
      trouver qui la men me 
      menast.Par foi, siresire, dist cil,
      si ferez, car ja doivent mouvoir li ami et li parent de ceulz qui
      en la nefensi furent trouvé. \pend
\pstart Quant Helcanus oÿ ce, si se pensa que
   il se metroitmetroit volentiers
      avec eulz et saroit se ce estoientseroient
      celes dont il estoit en la questeelles. 
   Et lors fist tant que il se mist avec eulz en mer, et orent i vent 
   que Diex leur avoit apresté, et vindrent aussi comme par miracle au 
   chastel, ainssi comme touz li paÿs 
   y estoit assemblez pour la pucele jugier.
   Et quant cil virent la nef, si sorent maintenant que ce estoit lacele
   que on disoit que murdriz estoient li marinier
      il, et dont firent i duel merveilleus et vindrent au 
   seigneur du chastel
   et firent leur plainte au plus grevablementgrevable que il porent.
   Et quant li sires entendi ce, si leur demanda 
    se il savoient qui celes estoient.  Il distrent que
      ii mauvaises fames estoient
   qui moult avoient fait de maus, dont fu 
   Helcanus presens, qui li dist :
   Sire, se je les veoie, bien vous en savroie a 
      diredire la 
      si com je croi, car tout aussi sui je en la queste d’une dame,
      dont je sui en moult grant doutance.
   Quant li sires ot oÿ Helcanus, 
   si l’esgarda et li dist :
      Sire bien soiez vous venus !, car il sot bien en son corage 
      que ilet vit bien a son samblant qu’ilet li dist bien en corage qu’il 
   estoit de moult grant affaire.
   Il est intéressant que le seigneur de Cartageus se fasse d'emblée cette idée d'Helcanus, toujours
   soucieux, tout au long de ses aventures, de passer incognito.
   Si li dist :
   Ja 
      assezassez i 
      tost les verrez au feu, car eles sont jugiees selonc la loy de ceste terre,
      se ensi n’ est que eles ne se puissent desfendre. \pend
\pstart Quant 
   ilHelcanus oÿ ce, si se souffri
   tant qu'il verroitsavroit que on de les feroit. 
   Dont furent li prince du paÿs mis ensemble. Et lorsleur fu conté 
   conment la pucele avoit 
      la petitepetite
         puceletedamoisele en garde et
      conmecomment ele avoit geté le cri par quoi tout estoient assemblé.
   Dont y ot i preudonme qui dist :
   SeigneursBiau seigneur,
      je de ma partie ne porroie mie croire queque que 
      cele que on ditdist qui cest murdre a fait 
      fust tele que ja eust crié quant ele cria se ele meismes l’eust fait.Par foy, dientdistrent
      li autre, tout aussiensi le nous semble il.
   Dont manderent a leur seigneur que il leur feist 
      venir cele qui estoit accusee, et il si fist. Et quant il la virent, 
   si l’esgarderent et en celui regart en orent 
   si grant pitié que chascuns encommença a souspirer,
   et leur vindrent communement les lermes aus iex.
   Dont la prist 
      li uns desunspreudonmes et la mist a raison et li dist :
   Amie, riens estn'est de vous, a ce que 
      j’entens. Dites veritév verité, 
      et je vous asseur que ja n’en arez pis.
   Quant la pucele l’entendi, si ne pot plorer, comme cele qui tant l’avoit fait que bien le pooit on 
   veoir, et dist :
   Sire, puisqu’ensi est que Diex m’a consenti ceste mort a avoir, dire veul qui je sui et conment Diex 
      se venge de moi au jour d’ui.
   LorsDont leur conta tout son errement et 
      qui ele estoit et leur dist tout du commencement jusques en la fin comment il li estoit avenu et comment 
      li seneschaus l’avoit requise et comment ele l’avoit refusé, 
      par quoi ele meismes savoit bien 
      que il en tele coupe l’avoit mise. \pend
\pstart  Pas de nouveau § BQuant cil oïrent ce, si esgarderent li uns
   l’autre et firent la pucele traire arriere. Et lors parla 
   li preudons et dist : 
   Avez vous oÿ merveilles ?se ce puet estrePar foi, distrent il, voirementvoirs 
      est ce grant merveille se ce puet estre voir !Voir ? dist li preudons. 
      Par foi, ce vous ferai je savoir, mais que vous m'en laissiez couvenir.
   Quant il oÿrent ce, si distrent :
   Par Dieu, sire, oïl, faites ent toute 
      vostre volenté, et nous du tout nous y acorderonsacordons. 
   Quant cilz oÿrent ce, si manderent leur seigneur que
   il venist a eulz, et il si fist. Dont 
   distdist dist cilz :
   Sire, il nous couvient savoir qui cele damoisele encoupe de ce 
      fait. 
   Et quant li sires oÿ ce, si respondi et dist que 
   son fait l’encoupoit.SireSi voire, dist 
      li preudons. Mais nous trouvons en nostre loy que nus ne puet 
      encouper autrui de mort d’ommemurdre se li fais n’est prouvez.Comment donc ! 
      distce distfait
      li seneschaus. N'est donc mie li fais prouvez quant li coutiaus fu trouvez 
      en son lit touz sanglens ?Sire, dist donc li preudons, 
      ce n’est mie souffisaument prouvé, quar la damoisele dit que 
      vous meismes li boutastes et feistes meismes le fait ou feistes faire et par quoi ele est en tel 
         peril. \pend
\pstart Quant li seneschaus oÿ ce, si passa avant et dist :
   Comment ! Dites vous cede ?
      Je sui cilz qui envers touz hommes ferai estable qu’ele meismes l’a murdri et ceulz de la nef
      aussi.ParPar i 
      Dieu, dist li preudons, se ele desfendre ne se puet,
      si est bien raison qu’ele y perde.
   Dont s’en vint li preudons
   a la pucele et li conseilla que elle deist hardiement que
      ce avoit fait li seneschaus et que ce li feroit
         ele recongnoistre avant que 
         li doi jour fussent passé.
   Ainssi comme il li dist le fist la pucele. Dont fu li jours mis a l’endemain, que, 
   c’ele ne se pooit descouper de ce dontque li 
   seneschaus l’encoupoit, ele estoit jugiee a estre 
   arsearse et brullee. \pend
\pstart Pas de nouveau § BCeste nouvele ala par 
   la ville, et sot Helcanus que cele n'avoit nul champion.
   Si dist en soi meismes que ja pour ce 
      celeicele ne 
      perdroit la vie, quar avant feroit il la bataille que li seneschaus
      n’en eust 
      sa deserteencontre, tout ne seust il qui elle estoit.
   Nouveau § BEt quant ce vint a l’endemain, si commanderent 
   li jugeour que i
      championchamp et i feu fust fais
      de dehors la ville.
   Et fust li seneschaus aussi aprestez 
   comme pour son corslui
   desfendre de si vilain criemme
      c criemme,
   se ainssi estoit que la pucele peust avoir champion. Ainssi comme il le commanderent le firent.
   Dont vint la meschine a la pucele au
   seigneur 
      du chastel et li dist :
   Sire, puisque ensi est que vous voulez ma damoisele metre a mort,
      je meismes ne veul pas vivre, car tout aussi sui je coupable de la mort comme elle est.Par mon chief, dist il, bien le croi !
   Dont furent les pucelesamenees toutes ii menees aus 
   champs et furent mises en haut, si que touz lesle porent bien veoir.
   Li seneschaus fu touz aprestez et dist aus 
   jugeours que il feissent ce pour quoi il estoient la venus.
   Lors fu crié, si que il fu bien oÿ, que, se il fust nulz qui la verité seust du murdre, que il le deist, et se il estoit seu par autre, 
   cil qui le savroitsavroient et ne le 
   diroitdeistdeissent tout 
   seroitseroient coupables 
   de celui fait et de 
   la mortmort a recevoir. 
   Dont n'en y ot nul qui mot deist adont 
   forsfors que li seneschaus, qui dist :
   Biaus seigneurs, je me presente devant vous que je ferai estable que la desloiaus a
      murdri la filletefille 
      de mon seigneur. Et se il est nul qui se pouroffre pour li,
      je sui cilz quiqui i 
      sousfisaument le ferai estable.
   Dont vindrent li jugeour a la pucele et 
   li distrent :
   Damoisele, que dites vous du seneschal,
      qui vous accuse de ceste chose ?Je di, dist la pucele, que il meismes l’a
      faitfuit et pourchacié, 
      de quoidont je prie a Dieu, qui set
      tout le fait, que il par sa pitié me sequeure aujourd’ui,
      comme cele qui est desconfortee. Et n’ai qui me sequeure 
      autrea autre que Dieu. \pend
\pstart A cest mot ot congneue Helcanus
   la pucele et ne failli gaires que il ne se pasma de la douleur
   que il ot quantque 
   il la vitvit qu'ele fu si tainte et descoloree. Il desrompi la presse, 
   comme cilz qui bien fu a cheval montez a son voloir, et vint devant les 
   jugeours, qui faisoient huchier que, s’il y avoit nul 
      qui la pucele vousist desfendreescondire, 
      venist avant et il seroit moult volentiers receuz.
   Et adonc parla Helcanus etqui dist :
   Biaus seigneurs, ne cuidiez mie que la pucele soit ci venue pour 
      tele chose. Et mauvaisement savez qui elle est. Et je sui ci pour li, que elle n’i a coupes, a ce que 
      li fel traitres 
      lique je ci voi li mis sus, ainz li ferai
      jehirgehui gehir si avant, comme vous le voudrez jugier, 
      que il meismes fist le murtre de sa main. Et se il est tel, que il le desdie : je le ferai estable
      et voir ! \pend
\pstart Maintenant que laque la-que la
   pucele ot veu Helcanus, si le congnut et ot tel joie que ele 
   tressailli voiant touz ceulz qui la furent. Et dont, quant 
   li seneschaus ot veu Helcanus, si fist semblant 
   queque il de nule 
   riens ne le doutast. Li i et li autre furent apresté pour la bataille faire. Dont s’esmerveilla 
   li sires et touz ceulz qui la furent qui cilz estoit qui la bataille vouloit faire 
   pour la pucele. Dont en vint 
   li sires 
      du chastel a lui 
   et li dist :
   Vassal, qui estes vous, qui ci vous pouroffrez de faire la bataille encontre 
      mon seneschal, qui ci est,
      pour une murtrieremurdrie qui 
      mon enfant a mauvaisement 
      murdri ?
   Et quant Helcanus oÿ cel'oÿ, 
   si ot tel duel, comme vous porrez oïr, et dist :
   Dant filz, a mauvais je vous prouveraiprometerai
      aprés cestui que vous y mentez et que la pucele ne fist onques murtre, si comme je vous ferai 
      congnoistre, se Diex plaist, en qui je croi, en ceste journee d’ui,
      se vous tiex estes que vous savoir le veulliez !
   Et lors quant li sires l’oÿ si
   hautementhardiement parler, si pensa que 
   il en lui moult se fioit et dist :
   Voirement le veul je savoir avant que vous m’eschapés ne ele aussi. \pend
\pstart Pas de nouveau § BDontMout 
   firent le champ widier et furent andui enz mis, et ot chascuns armeures teles comme a
   championchevalier
   afferoitappartenoit qui 
   enen la bataille devoient estre. Il furent mis a leur liex et orent leur 
   escuz et leur glaives tiex comme a eulz couvint. Li 
   champsc champs, 
   qui par raison avoit esté souffisaument jugiez, fu bien gardez. Et quant li dui chevalier s'entrechoisirent mal fu congneus li uns 
   de l’autre, car li seneschaus le cuida metre a mort en poi d’eure.
   Et dont mistrent leur escuz a leur pis et estrainstrent leur glaives. Si mistrent trop durement grant paine a ferir des esperons que 
   en poi d’eure se furent assemblé. Et feri li seneschaus si adroit que il mist son 
   glaive en pieces et en asteles. Helcanus,
   qui avoit tel duel que onques mais il n’avoit eu greigneur, ne volt mie faillir, car
   si durement et si aigrement feri 
   li seneschaus que il, 
   vousist ou non, abatique abati
   le cheval et lui 
   touta terre tout en i mont.
   Lors fu li seneschaus
   trop durement bleciez au cheoir.
   Et li bons Helcanus se mist jus du destrier
   moult vistement au plus tost que il le pot faire et s’en vint, l’espee toute nue 
   sachiee, vers lui. Mais onques si tost ne le pot faire que il fu em piez et ot s’espee 
   sachieesachiee hors. Et Helcanus s’en 
   vint vers lui de si grant ire plains et embrasez
   fu de si grant maltalent que il le feri i cop si tres dur que 
   il, vousist ou non, il le couvint metre a genoulz
   le mauvais seneschaus
   traitres, comme cilz qui tiex cops n’avoit pas aprisacoustumez
   a sentira sentir fu bien eslais.
   La description donnée ici d'Helcanus, qualifié de "bon" et animé de la colère typique de la
   narration épique, souligne la portée de l'épisode qui renoue et joue de la tradition en la matière pour louer la valeur 
   d'Helcanus et le sauvetage si opportun de Nera.
   Dont le hasta Helcanus et le feri
   trop durement du pié en la poitrine,
   si qu’ilet l’abati tout envers. 
   Si sailli maintenant Helcanus sus lui
   et li deslacha son hyaume, et adont li dist :
   Rent toi, mauvais traytres desloiaus prouvez, 
      et descoupe la pucele, qui coupes n’a en cestel 
      fait que tu li as mis sus mauvesement, car tu l’as toi meismes 
      fait et pourchacié. 
   Cilz.Cil vit
   ###Difficile de dire ce qu’il s’est passé : ça a l’air d’un problème commun à la tradition, 
      ergo problème ancien (même d’auteur ?). Mais il est aussi possible qu’il y ait une lacune, peut-être un saut : 
      Cil vit … Quant vit… ? Le cas me semble intéressant pour 
      montrer comment la leçon a varié, et G s'est amélioré sur la base de cette faute de V qui commence par garder 
      une version commune avec B et X2 ?
   Quant le traitre seneschal vit que il fu ainssi atrapé
      et mis au dessouz et que il ot le chief tout nu et que il li couperoit
   se il vouloit, si li a crié merci et dist :
   Ha ! vassal, dist li traitres,
      por Dieu souffrez vous de moi metre a la mort
      maintenant. Je vous en veul proier et requerre comme a 
         franc chevalier que vous estes. Et je vous promet queet je vous ferai riche homme. 
      Et bien sachiez que je n’ai 
      miei mie le murtre fait,
      ainz l’a cele fait pour qui vous me voulez occirre.
   Et quant Helcanus oÿ 
   le traitrecelui 
   ainssi parler, si le feri du pié
   derechief en la poitrine i si grant cop que
   il le fist pasmer tout estendu.
   Et adont l’eust Helcanus occis, mais il volt que
   le mauvais seneschalil
   recongneust sonle mesfait. Dont revint cilz a lui et li dist :
   Recongnois tost
      quiex tu eston fait, ou je maintenant te couperai la teste !
   Et le mauvaiscil ot paour 
   et cuida venir a merci par ses fausses paroles
   a son seigneur et dist :
   Je dirai volentiers verité,
      mais que tu ne me touchestoucches plus.Non ferai je, dist 
      Helcanusil.
   Dont hucha Helcanus ceulz qui le champ gardoient, et meismes le
   seigneur du chastel,
   qui la estoit present. Et quant il y furent touz assemblez, si recongnut
   li seneschaus comment il avoit esploitié 
      du murtre et pour quele raison 
      il l’avoit fait. \pend
\pstart Quant li jugeeur oÿrent ce, si distrent :
   Biaus sire, assez en avez fait !Par foi, dist il, voire de cestui. Mais encore en y a i autre qui il escouvient jehir 
      ou je meismesi meismes 
      l’apele de traÿson et de felonnie !
   Et quant li sires l’oÿ, si en ot tele paour que pour tot le monde ne se fust 
   aatis a lui de bataille. Et dont dist :
   Biau sire chevalier, bien croi que vous aiez
      la pucele noblement
      delivree par vostre proesce et par vostre bonne chevalerie, et bien est quite 
      partant. Et jeide (sic) ma 
      partiepartie lipartie le 
      vous pardoing la mort de 
      mon enfant comme a cele qui coupes n’i a, dont je sui 
      moult liez, ainz le comparra cil qui mes anemis est que je tant amer souloie.
   Dont n’i pot nul autre conseil mettre que, voiant touz 
   ceus qui la furent, que
   li senechaus ne feust pendus et trainez.
   Helcanus vint a la pucelle
   et la trouva moult mal atiree. Quant ele le vit, si 
   s’entrefirentse firent tel joie que bien
   s’ense pot on esmerveillier, meismement cil qui riens n’en sorent. 
   Dont vint li sires du chastel a 
   eulz et les conjura sus touz les sains de paradisdu monde que il li 
   deissent verité qui il estoient. Helcanus 
   ot paour, conme cil quil ne vouloit pas que on le seust ou paÿs 
   quequi il feussent devant ce que on seust la verité de
   Grescela guerre, si dist :
   Biau seingneur, se vous courtoisie nous fesiez,
      moult bien vous seroit merié. Et vous diroievouldroie
      volentiers
      quique vous sceusses qui nous sonmes, sauf ce 
      qu’il ne fust seu en 
                     nul lieu 
         oudont
                      que nul sceust que 
                  nous eussions a souffrir.
   Dont li jura li sires sus l’ordre de chevalerie que de 
   par lui ne 
               damesa dame 
               dont il sequi lequi li
               peustceustdeust garder il n’avroit garde se 
   bonne non. Adonc li dist il qui il estoit et dont il venoit et de 
      la pucele li dist il toute la verité. Quant il oï ce, si fu touz esbahis et dist :
   Ha ! sire, pourcoi ne me feistes vous 
      sagesage de ceste chose avant que
      encontre le traitre vous 
      deussiez combatre ne entrercombatissiez ne entrissiez en champ ?Pour ce, dist il, que je i sui a temps venus. \pend
\pstart Lors pristrent la pucelle et 
   l’en amenerent en la ville a moult grant joie. 
   Et commanda li sires que touz fussent aprestez de fere feste, 
   car moult avoient mespris vers eulz sanz raison. 
   Si ne fu onques mes tel joie menee a nule gent tant com il fu a la pucelle et 
   a Helcanus. 
   MeismementMeimes, li sires 
   fist venir sa fenme deschaucé 
   et en lange et eschevelee, 
   et la fist cheoir aus piez a 
   la pucele et 
   crierprier merci de ce que elle li ot fet et dit. 
   Dont la leva la pucele et 
   li dist :
   Dame, je le vous pardoing moult bonement, 
      car je sai bien que vous n’estiez pas sage de ce dont vous estesl'estes 
       maintenant. 
   Si en est ore seu le droit par quoi il n’i doit point avoir de hayne, 
   car tant mn’avez 
      fet en autre maniere que je le vous pardoing de bon cuer, et Dieu si face ! \pend
            \pstart Pas de nouveau § B
               Ainsi fu la pucele delivree 
               par grant miracle de Dieu, et li enquist Helcanus 
               de tout son estre puisque il ne l’ot veue, et ele li dist 
               tout 
                  ce que elle avoit eu a faire,
               et il meismes si li dist du sien affaire toute 
                  sa volentéla verité. 
               Et aprés furent tant avec le chevalier que 
               xiii jours furent passé. 
               Li prince dont j’ai devant parlé avoientorent
               tant alé 
               qu’il avoientorent 
               oïes nouvelles au port ou il avoient passé 
               que ii damoiselles estoient ainsi acusees a 
               Cartagieu de tel fait. 
               Dont firent nes aprester et ne finerenttargierent 
               de nagier, si furent 
               la arrivéa l'arivé. 
               Si cuidierent ou chastel estre honni 
               et deceu. 
               Il clostrent lesleur portes 
               et vindrent au seingneur et 
               distrent que grant gent estoient au port arrivé. 
               Qui sont il ? dist il.Par foy, sire,
                  nous ne savons, mesdistrent il 
                  il dient que il sont de Gresce 
               et quierent le chevalier qui a outré nostre 
                  seneschal.
               Quant il oï ce, si sot bien que il queroient Helcanus. 
               Li sires 
               si vint a lui et li dist :
               LainsSire, laiens 
                  sont arrivé grant gent qui vous demandent.
               Dont sailli Helcanus em piez et vindrent 
               avalaval entre lui et le 
               seingneur. 
               Adonc regarda Helcanus, si vit 
               Daphus, Josyas, Mirrus, 
               Japhus et le duc de Nise 
               et moult des autres qui moult 
               lor plutli durent plaireli dut plaire. 
               Il s’escria en hauthaut et li dist :
               Bien viengnent tuit mi bon ami, que je tant couvoite a 
                  avoirveoir ! 
               Et quant il orent aperceu que ce fu ille virent, 
               si ne fu onques tel joie menee commeque on pot la veoir. 
               Dont leur enquist la verité de Grece, 
               et il li distrent tost et a briez mos. 
               Dont se mistrent ens outout el 
               chastel et 
               furentfurent bien 
               viiv 
               censcens chevaliers, que povres que riches. \pend
            \pstart Quant li sires les vit touz assemblez, 
               si ot paour que Helcanus ne se plainsist de lui, 
               mes non fist, ainz s’en loa moult, et entendirent li baron a la pucele et a lui fere feste, 
               conme cele qui auques estoit revenue a point selonc ce que elle avoit esté menee. 
               Quant Josyas pot a li parler, si li dist :
               Suer, or avez fet vostre volenté. Frere, dist elle, vous dites verité. Je ne vousisse mie 
                  que je ne l’eusse fait. 
               Cette affirmation de Nera porte à réflexion : elle insiste ainsi sur sa volonté propre
               suite à cette longue série de mésaventures qu'elle a connues dans le but, affirmé à son départ, de retrouver Helcanus. 
               Quels que soient les malheurs qu'elle a endurés auprès des mariniers et du sénéchal, 
               elle a en effet atteint son objectif et permis, par là-même, que ces chevaliers retrouvent l'héritier du trône, mais aussi
               le mariage, qui va aussitôt se concrétiser, de Nera et Helcanus.
               Atant lece lessierent ester, 
               et Helcanus demanda nouvelles de son pere 
               et de Dorus, son frere, et des autres barons. 
               Il distrent qu’il estoient 
                  demourez en Grece et estoient a moult grant meschief de lui 
                  et que pour Dieu il reperassentrepairast tost 
                  arriere pour mettre 
                  em pesa point eulz 
                  et les siens. 
               Il dist que si feroit il. 
               Il pristrent congié au seingneur du 
                  chastel et apresterent leur 
               erreaffaire et 
               distrent que moult bien 
                  leurli seroit 
                  renduemerie la courtoisie qu’i leur avoient faite. 
               Li sires
                  Li sires leur dist que 
               il onques mes riens n’avoit fet si volentiers. \pend
            \pstart Pas de nouveau § B
               Atant ses'en departirent 
               de lui au matin et ne 
               finerentfinerent d'esrer par leur journees, 
               si vindrenttant que il furent 
               devant Constentinonble. Et quant il aprochierent 
               la cité, 
               si le firent asavoir a l’empereeur et a Dorus 
               et au conte de Flandres et 
               a Edypus, son oncle. Dont ne fu mes tel joie menee conme 
               on fist alorsil firent. 
               Li emperieres meismes conjoui la pucele 
               si souffisaument que moult li dut il souffire. 
               Et aprés tout ce, sanz nulle autre besongne faire, 
               volt Helcanus parler a l’empereeur a conseil 
               et demander a Josyas sa suer, 
               et il si fist. Et quant il ot fet sa requeste, 
               Josyas li otria moult 
               doucementhumelement, 
               et ainsi fu li mariages fais aus us et aus coustumes du paÿs. 
               Et quant ce fu fet, li empereres fist 
               sala feste fere si grande que au jour 
               d’adonc n’avoit eu greingneurgreigneur. 
                  Et quant li jours fu passez et ce vint au couchier, dont vint Helcanus a la pucele et orent moult de leur desiriers, 
                  ce poez vous bien savoir.. 
               Et quant se vint aprésaprés ce que la feste fu 
               fineefaite, 
               si parlerent li prince a l’empereeur de la pais de li et de ses enfanz. 
               Dont n’i ot nul qui soufisaument peust autre pes fere quequar 
               li emperieresli emperiereres 
                  (sic)
               demandoit celui qui les enfanz devoit avoir murdris, 
               meismement Cleodorus, que 
               li rois d’Arragon tenoit en sa prison. 
               Li rois s’escusoit de lui, mes il disoit que, 
               se il pooit venir a pes couvenable, que il li rendroit sain et hestiez. 
               Dont li dist li emperieres que de nule pais ne li parlast de si adonc 
               que il averoitraveroit 
               Cleodorus, l’onme du monde que il avoit trouvé 
               dede la plus grant foy. 
               Quant li rois oï ce, 
               si fist tant qu’il vint a luilui et sain et 
               haitieentier. 
               Et quant il fu venus, il meismes si fist tant envers l’empereeur 
               que il ot bonne pais et li jura foy et loiauté, et plus que a 
               touz les honmes du monde, 
               et dist que, se il de Rome ne faisoient du tot 
                  a sa volenté, que il n’avroient pieur anemi de lui. 
               Et quant Pelyarmenus sot que tout 
               estoitestoient venu a pais 
               couvenablecouvenablement fors lui, 
               si manda Helcanus que il pour Dieu venist a lui. \pend
            \pstart Quant Helcanus oï le mandement de 
               son frere, il i vint. Dont ne 
               fufu il nul , tant 
                  eusteust ilsi dur 
               cuercuer el monde que 
               cil eust veu Pelyarmenus, quequi 
               grant pitié n’en peust avoir, car il fondoit touz en lermes et en plors et disoit :
               Frere, se vous saviez la bonne repentance que j’ai de vous, 
               se ainsint estoit qu’en vous eust point de pitié, 
               je sai de voir que vous ne me tenriez ne jor ne heure. 
               Et bien vous fais asavoir que, se jeje ne puis avoir merci 
                  que vous estesn'estes mie mon frere 
                  sanz doute, 
                  quequar se en moi n’avoie pitié, si en 
                  doitdoit doit il avoir en vous selonc le bien que on 
                  en dist. 
               Tant li dist Pelyarmenus de 
               doucestoutez parolles 
               que Helcanus vint a touz les princes et 
               leur pria que il li aidassent a prier a l’empereeur 
                  de la paispais son frere. 
               Il furent moult joiant de ceste priere et vindrent a l’empereeur et 
               li chaïrent aus piez et li distrent que il eust 
                  pitiémerci de 
                  Pelyarmenus, son filz. Il dist a touz ensemble :
               Biaus seingneurs, alez, si en faites vostre volenté !
               Dont en jeterent le fes sus Helcanus. 
               Il vint a Dorus, son frere, et li dist que 
               il eust merci de Peliarmenus, son frere. 
               Il dist que il n’averoit pais convenable a lui pour honme qui 
                  en seust parlerl'en proiast 
                  se il ne li rendoit Dyalogus, 
                  qui lui et sa suer cuida avoir murdris. 
               PelyarmenusPelyarmenus li jura que 
               il li rendroit dedensdens 
                  brief temps a faire sa volenté. \pend
            \pstart Ainsi fu la pes ordenee, et se remist 
               li rois a aler vers Rome. 
               Li autre baron revindrent a l’empereeur et vodrent prendre congié. 
               Li empereres regarda chascun et dist que 
               nulz d’eulz ne se partiroient de lui
               si li avrezavroit 
                  rouvé un donjusquez qu'ilz seroient guerredonnés, 
               et celui leur donroit il 
                  quequoi 
                  qu’i li deust couster, mes qu’il en peust finer. 
               Et adonc se tut chascuns fors 
                  Dorus, qui dist :Dont dist Dorus qui dist a son 
                     pere : Peres
               Vez ci mon seingneur le duc, qui m'a
                  resuscité et nourri 
                  deet mis de mort a vie. 
                  Il doit premier faire sa demande.Filz, dist il, et je l’otroi.
               Quant li dux oï ce, si fu moult joians et dist :
               Sire, je ai une vostre fille, 
                  et cele vous demande je avec 
                  LeusBourleus, 
                  mon chier filz.
               Quant li empereres oï ce, si ne li volt mie escondire, 
               ainz li dist ceste chose :
               Sire dux, je ne vous vueil mie escondire, 
                  ainz en sui mout liez pour l’amour que je ai a vous et a lui. 
                  Je doing 
                     a Leus autre choseli doins encore autre chose, 
                  car je veil qu’il soit dux  d’Athainnes, 
                  quecar celui qui l’a esté ne 
                  revendraremaindra plus a ce qu’il le soit.
               Et quant Leus oï cest otroi, si l’en chaÿ aus piez et 
               l’en mercia si com il dut, et ainsi firent tuit li baron et distrent que 
               miex ne la pooit il emploier. 
               Nouveau § BDont aprés fu Mirrus 
               apelez, et li dist Helcanus :
               Amis, vous m’avez moult 
                  loiaumentsouffissaument servi. 
                  Je vueil que vous aiiez vostre deserte.Sire, dist il, moult sui bien paiez, 
                  la merci monseingneur vostre pere, 
                  qui m’a donnee la riens du monde que je plus amoie, 
                  par quoi je m’en doi bien tenir atant, car assez ai terre plus que a moi 
                  n’apartientaffiert.Par mon chief, dist 
                  li duxli empereres, 
                  ainz vueil que vous teniez de moi Laomedon.SireSire, sire,
                  dist il,il, grans mercis, si a gente promesse !
               Dont l’en cheï as piez, et li empereres si l’en leva et 
               li donna 
               la ducheela duchee de Laomedon
               qui moult iert souffisanz. 
               Elegius, son compaingnon, donna 
               il aussi grant terre, 
               et tout aussi fit il a maint autre. Mes de touz les autres dont j’ai fet mencion n’i 
               otot il nul qui plus vosist avoir 
               de l’empereeur ne de son filz, 
               car, tout estoient riche honme en leur païs, si ne vodrent demorer en la terre. 
               Meismes dist Daphus que 
               pour riens il ne demourroit que il n’alast arriere, 
               dont il s’iertestoit partis 
               ainsi conme devant ai dit. 
               Mes touz cistsez freres furent riche ou païs fors 
               Fremor, 
               qui dit que tout aussi se remetroit il en sa terre. \pend
            \pstart Quant il orentli empereres ot 
               ceste chose ordenee, si lor pria a touz ensemble que 
               il 
                  tant sejournassent o lui que il eust sa chiere fille 
                  veue, 
               car il la desirroit plus a veoir que riens du monde, et il meismes vouloit que touz feussent aus noces, 
               et il li otroierent. Dont se mist 
               DorusBourleus, o lui 
               BorleusDorus
               et moult d’autres barons qui au chemin 
               se mistrent, 
               si errerent tant que ilet vindrent a 
               Lembourc. 
               Et quant la duchesse les vit, si ot tel joie que
               nul ne doit demander la grant joie qu’ele ot
                  moult fait bien a croire. 
               Et aprés la joie qu’ele fist sot elle la besongne que il furent 
               venus querre, dont fu sa joie enforciee. Lors quant 
               Cassidore et le duc 
                  se furent conjoïil orent Cassidoire et la ducesse conjoï, si li distrent que 
               li empereres estoit en 
                  Gresce et avoient tout le paÿs aquité. 
               Dont ot moult grant joie et loa Dieu Nostre Seingneur de tout son cuer. 
               Et aprés ce si li dist Dorus :
               Ma suer, 
                  nous sommes ci venus a vous, car nostre pere vous a donnee a mon compaingnon 
                  LeusBourleus, 
               et sera duc d’Athainnes et vous duchesse. \pend
            \pstart Pas de nouveau § B
               Quant elle ot son frere oï, 
               si fu moult liee 
               et
               moult joiant de ce mariage, car 
               susdessus touz honmes amoit 
               elle le damoisel, si dist :
               Frere, je sui a la volenté de mon pere, 
                  si ferai ce qui li plera, car a nula‹insi›[nul] 
                  honme je 
                  ne vousisse estre donnee se a lui non.Par Dieu, dist il, aussi ne voudroie je.
               Dont apresterent leur affere et vint Cassidore au 
               duc et li dist :
               Sire, puisque il est ainsi que je aler m’en doi, sanz madamoiselle vostre fille ne m’en doi je mie aler, 
                  ainz vous pri que vous lale 
                  m'otroiezmander.Damoisele, 
                  ce dist Dorus, et je le vueil 
                     par mon chief,dist il, et je le vueil. Par mon chief, dist Dorus, 
                  puisqu’il vous plest. 
                  Et je sanz li me metroie moult a envis au chemin.Sire,Sire, ce 
                  dist li dux, 
                  et je l’otroi.Je voudroie, dist Dorus, que vous en fussiez si joians 
               conme je diroie par maniere que vous la me vousissiez donner, aussi com mon pere 
                  a fet ma suer a vostre filz.Par foy, sire, distce dist 
                  li dux, jaje 
                  mar le pensseroie fere, car li empereres donne grant chose a 
                  sa fille quant il li donne la duchee d’Athainnes,
                  et je de ma terre ne puis riens donner a ma fille se ce n’est par l’otroi de 
                  LeusBourleus, 
                  mon filz, et de ses freres. Se je cuidoie que vous la vousissiez avoir 
                  et li otrois de l’empereeur i fust, je feroie tant que 
                  je vous feroie avoirvous auriez la duchee 
                  de Lucembourc et la contee avant que li 
                  marchiezmariages 
                     n’enne fust fes.Sire, dist il, moult seroit grant chose, 
                  et je vous asseur que jaje autre n’avrai mes que je sanz terre 
                  lale deusse prendre. \pend
\pstart Quant se sot 
   la pucelela damoisele, 
   si ot tele joie que nule famefille 
   ne pot plus avoir,
   car ele l’amoit si en son cuer que moult en souffroit grant painne. Si avint que li dux ot 
   apresté son erre et ot dames et damoiselles a grant plenté. 
   Meismement la duchesce 
   volt il mener et iii autres damoisiaus a filz, que il avoit sanz Leus. 
   Dont se mistrent en leur chemin. Ci ne vueilore 
      fere mencion de leurs journees, 
   car si esploitierent qu’ilsi vindrent en 
   Gresce en mains d’un mois. 
   Dont firent savoir leur venue en Constentinonble. De toutes les festes 
   dont j’ai encore oï parler, 
   n’n'eni ot nule si grant conme ceste. 
   QuantQuar tout li prince et li duc qui demouré estoient 
   orent divers acesmemens et issirent hors de 
   la cité contr’eus, la ot mainte lance brisiee et maint biau cop donne. 
   Li empereres conjoï sa fille, 
   meismes la duchesce et les autres dames dont il i ot moult grant plenté. 
   Autresi firent li baron et entrerent en l'enfermeté de la ville a grant joie. \pend
            \pstart Pas de nouveau § BQuant il furent tuit descendu et monté amont ou palés, si vint Dorus 
   a son pere et li dist :
   Sire, se vous voulez loer une chose qui moult me plest, 
      vous feriez 
      adoncmoult 
      ma volenté, et vous en savroie moult 
         bon‹…›[b]on gré.Quele, filz ? 
      distce dist 
      le pere.Par foi, dist il, je lela vous dirai. 
      Vez ci le duc, a qui filz 
   vous avez donnee ma suer. 
      Il a une pucelle a fille, qui moult est gente 
      et plainne de grant biauté, 
      si avons esté nourris ensemble. Et tant esta la chose alee que ele a 
      mon cuer et elle m’a s’amour otroiee, et li dus si me veult donner avec li grant terre, 
      mes que ce soit vostre volentérequeste et la moie.
   Quant li empereres oï ce, si conmença a sourire et dist :
   Biau filz, 
      etja ne la voulez vous avoir ?Pere, dist il, oïl, se il vous plest.Par mon chief, dist ill'empereur, 
      voirement me plest il moult, car je ainme moult le duc, et vous aussi devez fere. \pend
\pstart Pas de nouveau § B
   Quant Dorus ot l’otroi de 
   l’empereeur, son pere, 
   si fu moult joians et moult liez et vint a 
   son frere et li dist en tele maniere 
   ce qu’il avoit 
      oï de son pere, 
   et son frereil li otria aussi. 
   Dont vindrent au duc ambedui 
   et li demanderent la pucele. Quant li dux oï ce, 
   si fu moult joiant et dist que tele requeste ne 
      doitdevoit il pas refuser et 
      leur en fist l’otroi moult amiablement. Dont vint a son filz et li dist :
   MoultFilz, moult il vous est 
      hui creüe grant honeur quant vous 
      avezavez l'otroi de
       la fille a l’empereeur 
      et la duchee d’Athainnes et encore plus ! 
      QuantQuar Dorus veult 
      avoir vostre sereur a fenme, mes il couvient que nous li dongnons terre.
   Quant Leus entendi ce, si ot tel joie qu’il ne sot que 
   respondredire fors que il dist :
   Pere, je lorli 
      vueil donner toute la terre qui me 
      pourroitdoit escheoir de vous, 
      car je en ai assez en Gresce, 
      se jeje adroit vueil 
      l’empereeur servir.Filz, dist il, moult estes sages, car je l’otroi, et maintenant li donrai la conté 
   de Lucembourc en sa main, et aprés moi tendra la duchee et sera sire de tout le paÿs.Certes, dist Leus, et je l’otroi.
   Dont vindrent a Dorus et li distrent ceste chose, 
   et il mie ne la volt refuser, ainçois en fist la pucelle arrester par la requeste de 
   ses freres et de son souverain. \pend       
\pstart Quant ces choses furent devisees, si espouserent 
   la damoiselleli damoisel 
      les damoiselesla damoisele et les damoiseles. 
   Ne demoura mie que la joie et la feste ne fust tele que toutes les autres n’avoient esté riens envers ceste, 
   car onques mes chevalier ne se penerent tant com firentil firent 
   ceus qui la furent. 
   Et distrent touz ensemble que les noces Dorus 
      etet les Leus furent 
   estraites de Nera, la fame Helcanus. 
   Et dont i ot iiidames qui moult se porent amer, et sanz 
   faille bien le moustrerent jusqu’en la fin, si conme l’estoire le raconte. 
   Quant se vint que‹ci›[que] la feste dut 
   departir si conme toutes les autres font, il pristrent tout congié a l’empereeur. 
   Et quant il vit que chascuns s’en vouloit aler, si dist :
   Biau tres dous seingneurs, a Dieu soiez vous conmandez, conme ceus 
      quequi jamés ne cuit veoir de ci au jour ou nous sommes touz semons, 
      ce est au jour espoentable ou cilz tenra sa cortsecourt, 
      qui est vrais deux et puissanz sus touz. \pend
\pstart Quant il ot ce dist, si li vindrent les lermes aus yeux, 
   et aussi firent ileus a touz les autres. 
   Dont le baisierentlaissierent tout li i aprés l’autre 
   et sanz plus dire s’en partirent, conme cil qui orent grant pitié de lui, 
   et orent li plusseur leur oirre aprestee. Japhus s’en ala en Frise, 
   li quens de Flandres et 
   JosyasJosias d'Espaigne en leur paÿs, 
   Edypus ende 
   Galille, li autre prince chascun selonc ce qu’il estoient en leur contree, mes li aucun demourerent 
   pour l’empereeur et ses enfanz faire compaingnie, 
   meismement Daphus, 
   Mirrus et li autre donttout 
   la compaingnie si estoit couvenablecouvenable et bele, 
   tant que a l’empereeur vint une volenté moult diverse, car ne demoura que, 
   ii mois aprés ceste departie, qu’il apela Daphus et li dist :
   Dous amis, mout aiai eu en vous grant fiance. 
   Et sachiez que moult ai eu a faire en divers lieus que onques 
   je ne me vols tant fïer en chevalier conme je ai fetsui en vous. 
   Pour ce vous vueil je dire une chose que je ai enpensséen propos a faire, 
   mes que elle soit celee de vous.Sire, dist il, moult grans mercis. Autre fois vous m’avez fet 
      honneur,honneur, et dit 
      si conmandez vostre volentévolenté sus moi, 
      et je sui appareilliez du 
      faire tout a vostre conmandement si conme drois est 
      et verité l’aporte.
   Amis, dist li empereres, moult grans mercis. 
      Et je vous dirai ma volenté quele ele est, la mercimere 
      de celui Dieu qui noustout
      a fet et crié de noient, 
      qui tant m’a donnees de graces en ce siecle desqueles je ne l’aila 
      mie a sa volenté servi entre ses choses et moult d’autres. \pend
\pstart Pas de nouveau § B
   Veritez est que les choses sont venues a bien 
   en pluseurs lieus et
   en pluseurs manieres endroit que de mon empire, qui ci a esté empeeschiez. 
   Et je meismes d’autre part que vous bien savezsavez ai este empegiez 
      et bien ai mespris envers mon Souverain, 
   par coi je sai vraiement que l’ame de moi en sera dampnee 
      se je n’enne fas l’amende a mon vivant.Sire, cedont dist 
      Daphus, quele chose est cece que vous dites ? 
      Ja n’avez vous fet envers Pelyarmenus ne envers les autres que vous ne doiez.Certes, amis, dist li empereres, je non, 
   mes vous savez bien que par ma folie je vous menai au Chastel Mignoit, 
   ou je ai moult mespris et fait de ma volenté et choses qui sont contre Dieu et le salut de maint.Ha ! sire, dist il, sachiez que je n’estoie pas la et croi bien que vous 
      aiezaiez bien bone entencion, et je m’i acort. 
   Dont prist li emperieres 
   Daphus en giemissant 
   par la main, en lermes et en soupirs, et dist :
   Dous amis, ce n'est pas seingneurie d’estre 
      en ce siecle ne d’avoir toutes ses volentez, 
      ainz est grant sens d’aquerre tresor en ceste mortel vie dont on puet avoir vie 
      pardurable sanz fin.
   Quant il ot ce dist, Daphus ne tint pas la parolle a truphle, ainz fu a ce menez qu’il dist :
   Sire, veritez est. Et quant jele 
      tout avroieavroie a-vroie
      avisé, seingneuries, richesces, chevaleries, riens ne valent se pour Dieu servir non.Par Dieu, frere, dist li empereres, verité avez dist. 
   Or regardez donc a moi meismes, se je doi pensser conment j’ai vescu, 
   quant Dieu m’a donné tant de belles graces 
      quecomme je puis et doie connoistre ! 
   Et je sai de voir qu’en plusseurs manieres je l’ai courroucié, par coi ma conscience me reprent, 
   et sai de voir que l’amende m’en couvient fere a mon vivant, 
   ou autrement je ne seroie mie mis avec ceus qui seront en gloire sanz fin. \pend
\pstart Pas de nouveau § BQuant Daphus l’ot entendu, 
   si fu mout bonnement convertis et penssa en lui meismes que 
   aussi avoit il vescu legierement, si dist :
   Sire, sachiez que tot aussi puis je dire et fere de moi, si regardez et me dites en quele maniere 
      vous en voulez ouvrer, et je aussiau-aussi 
      en voudrai faire.Amis, distce distfait 
      li emperieres, je ne le di pas pour vous, 
      car je croi que vous soiezestes preudoms 
      etet cuitet croi 
      que levostre cuer vous tent 
      a raler apar deversvers 
      vostre fame, et jeje je 
      le vueil
      , quecar il est drois. 
      Et je demourrai en cest païs ou je ferai ma penitance selonc le conseil que Dieu me donra.Ha ! sire, cuidiez vous que sans vostre compaingnie me doie mettre au retour ?Amis, dist il, oil, 
      quecar sachiez de voir 
      car, se Dieu plaist,que jamais ne me metrai 
   en l’ordure ou je ai tant demouré. Mes vous, qui par loiauté de mariage le poez faire 
   et devez estre avec vostre fenme, 
      a qui vous devez foy et loiauté a 
         touzjourstous les jours
         mes du monde, 
      vous vous metrezmetez au retour et 
      maintenezmaintendrez la terre et la bonne dame, 
      car aussi le vous conmant je et vueil que vous le faciésainsi le vueil je.
   Tant li dist li empereres et d’un et 
   d’eltel que il li ot en couvent 
   qu’ilil s'en iroit 
   arieresarrieres en son pays a sa femme, 
   mes avant voloit savoir a quoi il beoit. Dont li dist li emperieres que 
   par le conseil d’undu 
      saint hermite vouloit
      il ouvrer de sa penitance faire, 
      mes ilil ne vouloit que nul 
      nele seust 
      ceste chose fors lui 
      et le saint hermite 
      qui demoroit en une forest moult estrange et loing de 
      Constentinonble.
   Et quant Daphus l’ot entendu, si fu si 
   durementmoult temptez de fere aussi, 
   mes onques ne li empereres ne li volt souffrir, 
   ainz fist a Daphus aprester sa voie et prist congié a 
   l’empereeurlui et a ses enfans 
   en lermes et em pleurs. Et quant ce sot le duc de Lembourc, si ne volt plus ou païs demorer, 
   conme cil qui en sa contree n’avoit lonctemps 
      estéesté avoit, 
   meismementet Dorus 
   dist que il vouloit mettre sa terre a point et fere 
      soilui connoistre a ses honmes. 
   Si prist congié a son pere et a son frere, 
   et tout aussi fist li dus. Adonc se mistrent au chemin et 
   firent tant parde 
      leur journees conme 
      bonbel leur fu tant que chascuns 
   vintvindrent la ou il 
   devoitdevoient alervenir. 
   Si me vueil ore taire 
      deatant de eulztous, 
      fors deet remendrai a 
   l’empereeur qui sa voie emprise avoit pour aler au 
   saint hermite en unela 
   forest 
      moult estrange et loing de 
         Constentinonblesi comme dessus est dit. \pend
         
         
            Ci deviseEnsi 
               conmentcomme 
               l’emperere 
               trouva le lyon quant il
               alaaloit 
               parler a l’ermite.
            
               Enluminure sur 1 colonne et 12 UR. 
                  Cassidorus, en armes à ronds rouges sur 
                  fond jaune dans la forêt, rencontre le lion 
                  en allant parler à l’ermite
               
            
\pstart Or dist li contes que veritez 
   estVoirs fu, 
   si com l’estoire le raconte et tesmoingne,
   La formule introductive témoigne d'emblée de l'importance de l'épisode dans le cheminement de Cassidorus,
   et de sa dimension merveilleuse annoncée et assumée ainsi.
   que quant li emperieres ot la terre aquitiee, 
   si conme nous avonsest devant dit, 
   une nuit en son dormant li vint une avision et une vois, qui li dist :
   Lieve 
      toitoi et t’en va, 
      et si n’atens mie de la nuit jusqu’a l’endemain 
      tant qu’que tu aies riens a faire
          !
   Ceste vois entendi li empereres et sailli sus tous esbahis et se vesti au plus tost que il pot. 
   Lors fist tant que sanz le seu de nullui issi hors de sala 
   chambre etet puis du palais et 
   tout ainsi se mist hors de 
   la villecité tout a pié, 
   apuiéapuiant d’un bourdon en guise de nonpuissant et de si grant chose 
   conme estoit adonc de l’empereeur de Constentinonble. 
   Dont ne volt arrester en une ville que une nuit, 
   tant que il vint en la forest 
   ou li sains hermites 
   demouroitestoit, 
   dont il avoit oï parler. Quant il fu la venus, bien cuida estre assenez assener
   udu lieu ou il 
   conversoitconversoit conversoit, 
   mes onques tant ne sot aleraler ne venir 
   ne sus ne jus que il onques le peust trouver, 
   conme cil qui en la voie n’estoit pas. Quant ce vint a la nuit dont il ot alé le jour entier, 
   il ne pot trover ne borde ne meson ou il se peust herbergier, conme cilz qui moult estoit traveilliez. 
   Dont s'asists’aisist desous 
   i grant arbre et conmença a pensser et lui a regarder. Si se vit moult soilliez 
   et plain de grant lasté, si li vindrent les lermes aus yex, et li cuers li conmença a atendrir, 
   et la endroit fist i duel et une lamentacion moult piteuse, 
   et tant se demena que il s’endormi jusqu’dessia 
   l’endemain au matin que il s’esveilla 
   et vit que li solleus jetoitjeta 
   laja ses rais parmi le bois, qui tant 
   estoitestoient grans et haus. 
   Dont se dreça en son estant, et li semblafu 
   a son avis que 
   onques mesmes jor 
      enen toute sa vie n’avoit dormi si 
      doucementbonnement ne a si grant repos, 
   mes il se senti trop vuis et eust volentiers mengié se il eust eu quoi. Mes cil qui nule riens ne volt prendre du sien 
   ne de l’autrui dont il peust ne vousist 
   jourjour ne nuit vivre si s’estoit mis 
   en la forest sus 
   l’aventure de Dieu que il par sa pitié le pourveist de la poture de l’ame 
   et aprés de cele du cors. En ceste esperance se mist le bon preudoms a aler parmi 
   le bois, 
   si ne savoit ou, mes sa priere fist a Dieu que il l’asenast en aucun lieu 
   ou il peust prendre peutureaucune porcionpe...ture 
   dont illi cors 
   peust lors estreestre soustenus. 
   Moult longuementlonguement ala 
   par la forest que 
   onquesonques il n’i pot 
   trouvertrouver voie ne 
      sentiersentier trouver 
   par quoiquoi il peust savoir nul 
   absensa...sens de ce qu’il queroit. 
   Et aprésaprés la fain qu’il avoit 
   l'aloitlalo...
   siil si destraingnant 
   que a poup...u qu’il n’issoit du sens. \pend
\pstart En ce qu’il estoit 
  en... tel point 
   il oï 
   uneune vois de loing qui 
   crioit aidecrioi...i de :
               Diex, qui tout a a 
                  sauversauver 
                  ne sueffre que si anemisanemis 
                  me destruisse. 
   Et quantquant il ot ce entendu, si ne 
   voltvo... lessier que il cele part 
   ne sen...e meist 
   ou il avoit la voisvo... oie. 
   Dont conmença a aler...ler moult longuement 
   quequ... il la vois 
   ne voltpot plus entendre ne oïr.
   Et quant ili... vit ce, 
   si fu moult dolens et se conmença a lui meismes a courroucier et dist :
   Je cuit que folie me mainne.
   Quant il ot ce dist, si regarda devant soi, siqu'il vit i 
   chevalier armé de toutes armes et portoit une 
   trop gente dame sus l’arçon de sa selle.
   Lors li escria 
   l’empereresl'empereres et li dist :
   Vassal, atens moi et sueffre que je aie a toi parlé !
   Et cil s’avança tant qu’il vint a lui, si li dist :
   Amis, dites moi quel partie pourrai trouver le
      saint hermite 
      qui en ceste forest demeure.
   Quant il ot ce dist, la damoiselle li dist :
   Ha ! Deux aide, gentilz homs ! Je te prie pour cele pitié que li filz de Dieu ot
      de l’umaing lignage que tu de moi l’aies par quoi 
      cilz 
         folzfolz cilz desloiax
      ne me toille ma virginité que je ai de si a hore gardee. 
   Aprés dist li chevaliers :
   Amis, onbien 
      vous metramettrai a la voie que vous 
      me demandez, 
      mes que vous aleraler y vuieilliez.
   Quant li empereres oï ce, 
   sifu esbahis et penssa que
   ce estoit une estrange aventure, 
   sisi li dist :
   Biau tres dous sire, avant vous voudroie je prier de 
      cele damoisele
      que vous la delivrez de ce que j’ai entendu de lui que vous li volez faire,
      et je vous en savré bon gré.Frere, dist li chevaliers, 
      se vostre premiere requeste fust tele,
      moult eusse fet volentiers vostre priere. Mes sachiez que je orendroit riens n’en ferai.
   Dont s’en tourna cist sanz plus dire, et cele conmença derechief a crier :
   Ha ! homs de tres grant foy et de tres 
      grant chevalerie plains,
      faites moi secours, ou autrement se desloial 
         faus fera de moi sa volenté
      et je a tousjours serai honnie !
   Quant li empereres oï ce, si ot moult grant talent qu’il peust a cele faire aide. 
   Dont dist au chevalier :
   Par Dieu, moult mauvaisement emportez la pucele ! 
      Et sachiezsachiez que, se je eusse mes armes 
      en autel point 
      conmeque 
      jeje les ai aucune fois eues, 
      vous ne 
      l’emportissiezl'emportissiez hui sanz bataille ! 
   Quant cil oï l’empereeur, 
   si ne volt lessierlessier si ne volt lessier 
   qu’il ne retournast, si li dist :
   Conment ! Es tu donques tieux que tu la me voudroies contredire se tu 
      avoieseusses 
      tes armes ?Par Dieu, ce respondidistfait 
      li emperieresil, voirement 
      feroieouil. 
      Et se vous estes tel que vous vers moi la veilliez desresnier a gieu parti, 
      moult me metroie volentiers en aventure de lui tenser 
      et garentir selonc ce que elle me fet entendant.Par mon chief, dist li chevaliers, 
      tuvoirement tu en venras a gieu parti, mes que 
      tutu me die qui tu es.
   La condition de la bonne question à poser peut faire allusion à une tradition importante dans ce genre de
   scène de révélation, notamment dans les scènes du Graal dans la littérature arthurienne. Quant à la question de l'identité, elle 
   permet de renouer, une fois encore, avec le motif de l'incognito, cher à Cassidorus, avec les problèmes que cela a déjà pu lui 
   poser dans le précédent Roman d'Helcanus, mais aussi au moment de ses retrouvailles avec ses armées dans ce roman. La répétition
   du motif et les conséquences qui sont associées viennent y donner une coloration particulière.
 \pend
\pstart Quant li empereres l’entendi, 
   si ne sene fust fet connoistre a lui 
   pourde 
   nulle riens, ainz li dist :
   De moi connoistre 
      autrementautrement que vous 
         neore faites 
      n’avez vous orendroitore mestier, 
      quecar auques 
      poezpoez ore savoir que 
   je ne sui mie de grant affaire, car je sui i povre chevalier 
   et voudroie avoir parlé au saint hermite pour 
      avoir aucun conseil dont je ai 
      grant mestier. 
      Si faites tant que ma voie ne soitsoit mie empeeschiee, 
      si vousvous en tendré a courtois.Amis, distce dist cil, ta voie ne 
      te vueil je mie empeeschier, mes, 
      se tu meismes le fais, si te loeroie que tu alasses ton chemin et me lesses 
      couvenir de 
      ceste fame, qui est 
         fausse chose, car bien sachiez que tu tarttard en 
      venroies au repentir.
   Dont se refraint par semblant li empereres, et cele conmença moult fort a gemir et dist :
   ChevalierChevalier de Jhesucrist, 
      ne me faitesfaites mie
      faillance, car bien poez savoir qui vous ça amene pour fere ceste besonne.
   Dont saisi li empereres le 
   chevalchevalier par le frain 
   et dist :
   Par Dieu, sire chevalier, il vous couvient 
      ceste puceledamoisele 
      desresnier a jeu parti se vous vilainement nen'en voulez esploitier.
   Dont dist le chevalier a l’empereeur 
   qu’il ostast sa main de son frain et, s’il vouloit 
      la damoiselle 
      fere secors, si alast aprés luilui si alast aprés lui, 
      etet que 
      il 
      n’iroitiroient pas loing 
      qu’ilsi trouveroient leur recet et 
      lala li 
      feroientferoit
      iluecil 
      leurle droit du gieu. 
   Dont s’en retourna cist parmi une viez voiesente, 
   et li empereres 
   fu touz esbahis, si ne volt lessier que tout autel point conme il estoit ne 
   se meist aprés le chevalier 
   a aler, 
   si le suivisuy tant qu’il vindrent en une grant valee. 
   Ens ouEl fons de ce val si couroit une moult grant riviere 
   qui descendoit d’une montaingne qui n’estoit mie mains roide pour ce ne fors, 
   ainz l'estoit telesi que bien 
   sembloitsembla a l’empereeur 
   queque il meismesque il en toute sa vie n’eust tele veue. 
   Lors vit que li chevaliers 
   aloit costoiantcostoia la riviere et 
   l’empereeuril tousjours aprés, 
   tant que il vint aussi conme a i pont qui d’autre part la riviere estoit levez. 
   Dont s’arresta li chevaliers et 
   pristdist i cor qui 
   a son coste pendoit et le sonna iii foismos, et ne demoura guieres que 
   i honme de grant aage vint et li avalaaval 
   le pontune grant planche, 
   et li chevaliers monta sus et puis passa outre. 
   Mes avant que li emperieres fust la venus resacha cilz 
   le pontla planche amont, 
   et li empereres si demoura de l’autre part. Dont hucha 
   l’empereeurcelui 
   etet li dist :
   Amis, avalez le pont et me lessiez aler aprés 
      le chevalier qui la pucele
      enporte outre son grésans sa volonté !
   Onques li empereres si tost n’ot ce dit qu’il ne vit ne honme ne pont 
   ne rienschose nule que l’yaue et la valee, qui moult estoit hideuse, 
   et la forest grant, quique trop estoit laide chose a veoir. \pend
\pstart De ceste chose fu li emperieres moult esbahis et penssa que 
   ce povoit estre, si s’asist, conme cil qui estoit plain de grans 
   travaustravail. 
   Dont il dist a lui meismes :
   DousBiaus dous Deux, 
      que m’est il avenu et queles decevances me sont avenues au devant par quoi je 
      ne puis venir a ce que je vouloie faire ?
   Lors li vint une aventure en 
   avisonavison devant, qui li dist que 
   tout ce qu’il avoit veu devant 
      estoit temptacion d’anemi pour lui decevoir et mettre enen aucune desperacion. 
   Lors se conforta et fist une priere a Nostre Seingneur, moult piteuse, que 
   par sa grace le feist alerl'envoiast en 
      tel lieu par coi le cors ne perisist a ce qu’il 
      eust peture a faire la penitance dont il l’avoit courroucié. \pend
\pstart Pas de nouveau § BLettrine oubliée dans G
   Quant 
   li empereresil ot ce dit, 
   si li vint une famine si grant que a pou qu’il ne crioit aussi com s’il fust hors de sondu sens. 
   Et lors se leva sus sesdreça ens 
   piezson estant et se penssa que, 
   de quele part que ce fust, Deux li envoieroit chose par quoi il ne periroit mie. 
   Il se prist a aler et regardaregarda, 
   si vit en la costierecostiere
   de la montaingne i merveilliexmerveilliex 
   desruban de rochesroches 
   et de pierres mossuesmossues. 
   Il montapuia 
   tanttant et erra 
   qu’ilqu’ilqu'il vint la et dont 
   trouva i recet ausi 
   commeco...me se gent i eussent 
   converséconversé aucunes fois. 
   Et dontdont ala avant et trova i 
   lyonlyon qui 
   menoitmengoit sa proie, 
   mesmesmes il onques si tost ne 
   l'aperçutl'aper...t quant 
   il sailli vers 
   l'empereeurl’emperereeur 
   et li fist une joie et une... une feste si 
   tres grant 
   que...ue nule 
   b‹d›[b]este 
   mue ne fit 
   onques...nquesmaisonques mais si grant a honme 
   nulnul. 
   Et quant il vit ceste aventureaventure, 
   si ot si grant joie queque 
   onques en 
   jour de sa vie il n’avoit euot 
   greingnorsi grant. 
   Lors l’aplania li empereres moult 
   doucementdevotement et 
   sotsot de voir que ce estoit miracle de 
   Nostre Seingneur. Il pristprist de la beste qu’il trouva 
   menantmenjant au lyon, 
   si espraint le sanc de la char hors au miex que il pot et enconmença a mengier par moult grant talent, 
   conme cil quil n’avoit mengiéqui fait ne l'avoit 
   dede pres de ii jours. Quant il ot ce fet, 
   si rendi graces a Nostre Seingneur et fist sa priere que 
   il li otriast que il li consentist 
      le lyon a compaignoncompagnie 
      et d’iluec ne se partiroit de si a tant que autre conseil avroit dont il 
      pouroitpouroit bien miex 
      DieuNostre Seigneur servir.
   ###Qqch sur le lion compagnon, dans la lignée de Yvain notamment \pend
\pstart Lors conmença li emperieres
   le lyonlieu 
   a cerchier et trouva le recet 
   assez couvenable a son oeus, se li sembla, 
   et vit que li lyons se tenoit touzjors avec lui. 
   Si fu moult joians et ne li souvint de chose nule qu’il eust lessiee arriere, ainz mist son cuer a Nostre Seingneur 
   si forment comque l’estoire raconte 
   que touz ses travaus estoit a faire priere, 
   et ne menjoit que une fois le jour, et ce que il usoit li aportoit le lyon, 
   tout ausi apensseement conme ilDiex le vouloit, 
   carqui bien estoit puissant 
   dude faire et du consentir, 
   car il savoit bien sa conscience, qui bonne estoit. \pend
\pstart Pas de nouveau § B
   Ainsi demoura li bons 
   emperierespreudoms 
   grant piece qu’il ne fu sceu de nuluinului en la cité de Constentinoble. 
   Quant sa gentil virent que il ainsi s’en 
   fuestoient partis sanz le seu de nullui, 
   Helcanus, qui bien penssoit le fait de lui, ne volt lessier que il em plusseurs lieus 
   ne le feist querre et cerchier. Et quant il sot que novelles on n’en 
   potpovoit oïr, si fu moult dolens. 
   Il ne volt lessier l’empire, ainz lale tint a droit 
   et se fist craindrecremir et amer 
   et fist touz ceus seingneurs qui valu li avoient encontre ses anemis. Le parallélisme qui présente
   le souci d'Helcanus de retrouver son père et de tenir au mieux l'empire à sa place dénote bien sûr le lien de cause à effet, 
   mais souligne aussi le fait que Cassidorus ne tienne pas lui-même son empire. C'est la première occurrence de ce schéma qu'on retrouvera
   aussi au prochain, et dernier, départ de Cassidorus, et qui empreint pareillement Helcanus du souhait de tenir son trône, 
   comme au contraire de son père, qui lui a certes cette fois bien passé le pouvoir. La nuance entre ces deux cas de figure qui se 
   suivent dans la trame du texte connote elle aussi le départ de Cassidorus.
   Si me vueilvueil maintenantvueil ore 
      de toutes ces choses taire fors que de Dorus, 
      qui partis s'estoit 
      de Gresce, si conme 
      nous avonsa estéest 
      devant dit el conte.
         \pend
         
            Ci conmeComment 
               Dorus prist 
               Pelyarmenus en la forest en chaçant 
                  le porc 
               pource que il ne li avoit envoié 
               DyalogusDyalogus le bastart.
            
            
               Enluminure sur 2 colonnes (b-c) et 13 UR
                  Pelyarmenus en armes noir et orange à cheval, 
                  chassant dans la forêt le porc à l’aide de deux chiens, 
                  est capturé par Dorus à pieds sans armes avec une corde.
               
            
\pstart Aprés ce dist li contes que, 
   puisque la pais fu ordenee des enfanz 
   l’empereeur de Constentinoble, 
   Pelyarmenus, qui la descorde y avoita 
   du tout mise, si conme devant a esté dit, n’envoia ne ne revint aprés la covenance qu’il avoit fete, 
   neent plus conme cil ou il n’avoita 
   foyne foy ne loyauté. 
   Dorus, qui n’atendoit el fors que on li envoiast
   Dyalogus, si vitsi-vit si vit 
   que li ans estoit passez que il devoit estre envoiez.
   Si en fu moult iriez et se penssa que,
   se il ne l’avoit, que moult demourroit la chose descouvenable
      et grant reprouche seroit a li.
   Il se parti dude son chastel et 
   alavint a Lembourc, ou
   li dus estoit, et lors li moustra ceste chose,
   et li duxli dux li respondi :
   Biau filz, je vous doi moult amer, 
      et je si fas. Sachiez de voir, conment
      que les choses soient avenuesalées, 
      vous avez ma fille, que je 
      ai moult ameemoult aime.
      Aprés mon filz si a vostre sereur,
      et cil ne sont pas en cest paÿs. Il n’afiert pas a moi que je vous mette a descorde entre 
      vosvous freres.
      Et d’autre part, li emperieres n’est pas en lieu par coi 
      on puist ovrerfaire par son conseil. 
      Si covient que vostre frere Helcanus soit principal
      du conseil et couvient que vous envoiez a lui par quoi vous sachiez sa volenté.Sire, dist il, et je m’i acort.
   Dont firent ceste chose escrire et fu pris i mesage et envoiez en Gresce.
   Quant cil fu venus la ou Helcanus estoit, si le salua de par son frere 
   Dorus et li bailladonna 
   la lettre que il li envoioit. \pend
\pstart Quant
   Helcanusil ot le mandement oï et seu, 
   si fu tost conseilliez et fist escrire sa volenté. Lors se remist cil en son chemin et ne fina par ses journees tant 
   que il revint arrieres et trouva Dorus a Lembourc 
   et li mist la lettre en sala 
   main, que son frere li envoioit. 
   Dont trouva escript que il meist ses choses en respit et il envoieroit a Rome 
   et savroit que Peliarmenus voudroit fere de ceste chose, et il si fist. 
   Helcanus ne se mist pas en oubli, ainz envoia a Rome a 
   ses freres et manda a Peliarmenus que il ne s’estoit mie aquitez 
      de son serement qu’il otfist 
      en tele maniere‹en›[m]aniere 
      enque il l'ot en couvent 
      par coi il peussent demourer em pes et bon ami. \pend
\pstart Pas de nouveau § BQuant Pelyarmenus oï
   , si fetce mandement, si dist que 
   ja ne seroit en leu ou Dyalogus receust de ceste 
      chose pis que il meismes feroit. Dont fu leur response tele que, 
   sese bien 
   ses'en 
   tenissent
      tenissent Helcanus et Dorus, 
   atant quecar il autre choses n’en emporteroient de eulz et parmi 
   tantce 
   seroientseroient il ami et non autrement. 
   Quant Helcanus oï ce, si fu moult iriez et ne sot sus ce que 
   il peust mander 
   a son frere Dorus, 
   car ilil a envis 
   esmouvoitesmouveroit guerre contre eulz. 
   Si se souffri de ceste chose 
   jusqu’dessia i jour que 
   Dorus envoia a lui, et il ne pot mie ne ne volt celer le mandement que 
   son frere et li autre li orent mandé et que pour 
   Dieu il souffrist encore de ci adonc qu’il pourroient oïr nouvelles de 
   leur perel'emperor. 
   Quant Dorus oï ce, si fu 
   plustanttant plus iriez 
   qu’il n’avoit onques esté devantque nul plus de lui. 
   Il vit bien que, s’il ne se metoit en aventure, il ne porroit estre vengiez a son talent 
   qu’il ne s’en couvenist meller 
   plusgrant plenté de gent. 
   Dont penssa a une chose moult merveilleuse, car il prist i 
   escuiersien escuier, qui moult estoit preudom et plains de tres grant proesce et 
   ot non Mardocheus, si li dist :
   Amis, moult me vueil fïer en vous quant mon cors et ma vie et mon honneur vueil du tout mettre en vous.
   Cil respondi :
   Sire, pourquoi pour mort a 
      recevoir ne vous faudroiefaudroie je mie ? 
      Dites moi vostre volenté, et je sui appareilliez du fere.
   Et lors li dist il que il beoit a faire, et 
   cilcil li respondi que 
   moult volentiers le feroit. Dont s’atournerent il a la guise des ii marcheans 
   et si pristrentpri‹e›[s]trent leurs
   armes et grant avoir que il firent trousser et se mistrent ouel chemin 
   devers Rome 
   et ne finerent par leurs journees tant que ilsi vindrent en la cité de 
   Rome. Dont se traitrent en la ville, 
      conme cil qui bien furent apenssé de ce que il orent em proposet se hostellerent a leur devis. \pend
\pstart Dorus, qui jadis avoit esté a Rome, 
   si conme devant est dit, si penssa qu’il couvenoit savoir conment 
      il estoit au roy pescheeur 
      La leçon de B et X2 semble plus correcte, le pêcheur en question n'étant jamais associée à une 
         figure royale quand il en est question au début du roman (voir §1-15 en particulier). La leçon de V3 et G dénote sûrement 
         d'un écho au roi Pêcheur du "Conte du Graal" de Chrétien de Troyes, dans une association symbolique forte évidemment.
      et a sa fame, quiqui jadis 
      l’avoient norri ainsi conme devant est dit ou conte. 
   Il se mist entre lui et MardocheusMar‹ch›[do]cheus cele part ou il cuidoient que il demourassent. 
   Moult alerent cerchant par la ville ainz qu’il peussent assener au lieu 
   ou il si petis avoit esté. Tant cerchierent que par aventure trouverent le preudonme 
   qui ce seoit en sa meson assez povrement, et tout aussi sa fame, et 
   estoientestoient tout viel 
   devenu. 
   Lors mist Dorus le pescheeur a reson et le tret d’une part, 
   si li demanda ou ci enfant estoient. Cil 
   leur respondidist 
   qu’il estoient en lor gaaingnage, et pourcoi le 
      demandoient ildemandez vous.
   Biau sire, 
      distce dist Dorus, 
      ce vous dirai je bien, mes que il soit celé.Sire, distce dist cilz, 
      se s’est chose dont jeje y soie tenus 
      a vous dire, 
      cele vous 
      diraiferai 
      je volentiers.Par Dieu, 
      dist Dorusil, oïl.
   Lors li dist :
   Il fu i jour que vous feustes d’un enfant sesi,
      que vous secoureustes en une yaue deen 
      ceste cité, 
      si vous privoudroie prier que vous 
      nousme deissiez qu’il devint. \pend
\pstart Pas de nouveau § BQuant li peschierres l’ot entendu, 
   si ne pot dire motmot, ainz commenca a sospirer. Lors se traist moult pres de lui 
   etet li dist : 
      Ha : sire, por Dieu merci, dites moi qui vous estes.
      "Amis, dist il, sachiez que je sui moult amis a celui se il est encore en vie." 
      "Par Dieu, sire, dist cil, je quit que vous me volez decevoir aussi comme aucun out fait maintes fois." 
      Quant Dorus oy ce, si ot merveille pour quoi il disoit ce, si li dist que il le feist sage de la decevance. 
      "Sire, dist cil, volentiers. Veritez est que aucun de ceste vile ont eu moult a souffrir de celui enfant que 
      vous demandez et vous dirai comment. Li enfant a l'empereris de Romme avoient amene en ceste vile .ii. enfans, liquel durent estre
      peri et ne sot on comment fors tant que li uns fu jetez en une yaue et fu secourus d'un pescheur de ceste vile. Or est puis avenu que 
      il a esté demandé a pluseurs et maint en ont eu a souffrir et je meismes en ai jeu .ii. ans en prison pour ce que j'ai esté retez que 
      je l'avoie trouvé si ne l'avoie delivré si comme je devoie, si vous pri pour Dieu que vous ne m'en faites pis que j'en ai eu." 
      Nouveau § BQuant Dorus l'entendi, si ot moult grant pitié de li et dist :
   Conment, biau tres dous sire ! Ce que vous me demandez, je ne sai ou il est, 
      mes bien est la verité que je oi jadis i enfant en garde pour cui je ai esté moult longuement em prison 
      et tout pour celui que vous me demandez et ai
      sire ! Dites vous que vous en avez jeu em prison pour celui dont je vous 
         ci 
         demant?" "Certes, biaus tres douz sires, voirement y 
         ai gege geu et 
      perdu tout quanque 
      j’avoieen avoie vaillant.
   La leçon de B témoigne de ce goût, marqué tout au long du roman, de revenir sur les évènements passés.
   Par mon chief, distce dist 
      Dorusil, bien vous sera rendue vostre perte, 
      se jeje le puis esploitier, car je vous fas bien asavoir que 
      je sui celui meismes pour qui vous avez eu tant de donmages.
   Quant li preudoms oï ce, si li dist :
   Pour Dieu, siredous sire, 
      dites vous verité ?Amis, dist il, se savrez vous par temps.
   Lors mena li preudoms une joie si grant que sa fame 
   entendi tout l’afaire et li demanda que ce estoit et qu’il avoit, 
      et lors li conta que ce estoit. Quant ele oï ce, 
      si en orentfurent tel mené que moult en peust on grant 
   pitié et furent moult liez de ce qu’il connurentpitié avoir comment comment 
      li uns et li autres connoissoitconnoissoient 
   Dorus. \pend
\pstart Pas de nouveau § B En ce qu’il estoient en tel point, 
   atant ez vous lesvindrent 
   iii filzvalles 
   auque li 
   preudonme preudoms avoit a filz, 
   quicilz avoient esté nourris 
   petit enfançonnetensemblez 
   aveca l'eure que 
   Dorus quant ilainsi dut estre 
   perisperis et avoient esté nourri ensamble tant 
      qu'il fucomme il avoit esté norrisavec eulz. 
   Quant il furent venus, si orent moult grant joie et 
   furent mout esbahismoult grant merveille qui il estoient quant 
   leur pere leur dist :
   Mi bel enfant, recouvré avons nostre 
      perte !perte, car 
      Vez ci celui par qui nous avons eu tant a souffrir.
   Quant cil entendirent ceste chose, si furent tout esbahi et li 
   firentfirent moult grant feste 
   de ce qu’il porent. 
   Dont, quant il se furent grant piece conjoui, si dist Mardocheus :
   Sire, il nous couvient sagement esploitier.
   Lors dist Dorus que tout ce estoit verité, 
   et dont pristrent il 
   congié au preudonme de ci a leur revenue et menerent les iii freres avec eulz a leur hostel. 
   Dont leur fist Dorus donner c mars et leur conmanda que 
   il s’aprestassent en tele maniere que li uns d’eulz menast 
      son pere et sa mere en son paÿs, 
   et li autre dui demourassent en leur compaingnie, et il les feroient riches a touz les jours de lor vies. 
   Cil furent lié et distrent que sa volenté feroient il volentiers. \pend
\pstart Lors s’en vindrent arriere et aporterent l’avoir et distrent a 
   leur pere et a lor mere que 
   li sires qui la avoit esté lor avoit donné cent mars 
      et vouloit qu’il s’en alassent en son païs, 
   vousil et 
      masa mere 
   etet li un 
   de leurdes freres, et li dui 
   demourroientdemourraissent avec eulz. 
   Quant li preudoms oï ce, 
   si fu moultsi liez 
   plus que onques mes n’avoit esté, 
   et encore le fu plus sa fame. 
   Que vous iroiediroie je 
      lonc conte faisant ?
   IlSi apresterent lor afaire 
   tant que ilet 
   se mistrent au chemin par devers Alemaingne, et demourerent li dui ainsné des 
   freresfreres est 
   avecpar devers Dorus pour 
   eulzlui fere secors et aide. 
   Mardocheus, qui moult estoit sages et plains de grant enging, mist 
   Dorus a raison et lilor dist que 
   painne leur convenroitconvenoit mettre 
      a ce que il peussent besongnier de ce qu’il avoient a faire.
   Par Dieu, distce dist 
      Dorus, nous ne poons neentriens 
      faire de nostre besongne, se ainsi n’est que Fortune et Aventure nous vueille aidier.Tout ce est voir, dist Mardocheus, mes nous ne connoissons pas 
   bien ceus que nous volons seurprendre.
   Lors apelerentparlerent a 
   l’ainsné des filz au 
      pescheeur et li distrentdist :
   Amis, gardez bien que ce soit celé ce que 
      nous vous dironsvous saurez de nous.Biau seingneurs, dist il, n’en doutez ja, 
      car avant me leroieleroie je destruire que je ne feisse vostre volenté !Dont volons nous, distrent il, 
      que vous nous faciez connoistre Dyalogus, 
   qui si est privez de l’empereris et de ses enfans.Sire, distce distfait 
      le varletcil, 
      sice vous ferai je moult volentiers. \pend
\pstart Lors s’atourna Dorus en tel maniere que nul ne le peust connoistre.
   Ce redoublement de l'incognito de Dorus dans cette même aventure, 
      sous ses habits de marchand puis sous cette guise anonyme ici, joue du motif important, tout au long du roman, 
      de l'incognito et du brouillage d'identité qu'il peut occasionner. 
      La visée de Dorus est cependant, par contraste notamment avec celles de ses père et frère, bien plus pragmatique
      et bien plus directement associée aux enjeux chevaleresques de l'honneur.
   Il s’en vindrentvint a la court ou Fastidorus 
   etet ses freres Pelyarmenus 
   estoient. Moult avoit grant gent avec eulz. Et regarda, si vitvit le vallet 
   Dyalogus, 
   qui moult avoientavoie‹n›[t] grant suite aprés lui. 
   Dont lile moustra 
   le traïteur a 
   Mardocheus et dist :
   Vez vousle la 
      celuile qui tant nous a donné a souffrir et mis a povreté !
   Lors l'esgarderentesgarderent andui et mistrent retenance 
   en leur cuer, 
   si que il n’eussent jamés failli a lui reconnoistre. 
   Lors s’ense tournerent a leur hostel et distrent 
   qu’il couvenoit qu’il fust espié quant il alast en bois ou en riviere, 
   par quoi il le peussent sorprendre et fere leur volenté. 
   Dont dit le varlet que, 
   s’il le vouloient sorprendre, que il le 
      prendroientporroient faire bien prochainnement 
      et qu’ouil ne seroit pas plus de lui tiers.
   Par mon chief, dist 
      li duxDorus, bien 
      let'en croi !
   Dont li dist Mardocheus 
   qu’il leleur couvenoit 
      pourveoir de ce faire.
   Par Dieu, distce dist 
      Dorus, verité dites.
   Lors firent fere ii paniers grans et merveillieus et 
   troussertrousser seur sus 
   leur.i. cheval, 
   et eulz meismes estoient touz
      toustous les jours 
   garnis de leurs armes et maintes fois en furent en aguait, 
   mes leur chose ne venoit mie si a point que demi an ne sejournassent a Rome 
   ainz que il peussentpe-il peussent riens esploitier de leur besongne. 
   i jour avint, si conme faire dut, 
   que nouvellesnouvelles leur vindrent que 
   Pelyarmenus devoitdevoit aler chacier 
   i porc en une forest a iiii liues de 
   Rome. Quant Dorus
   le sot, si fu moult 
   liezjoians, 
   car bien vousist qu’i le peust deschevauchierentrechevauchier,
   si li eust cousté toute sa terre. 
   Lors atournerent leur affaire, et Pelyarmenus fist ses chevaucheeurs venir 
   devant lui, si leur dist qu’il s’aprestassent a l’endemain, car il avoit oï nouvelles 
      d’un porc qui estoit en la Forest du Martrai. 
   Cil distrent que tot estoient prest 
      et appareilliez du mouvoir quant il voudroit conmander.
Cet épisode de chasse dédoublée, du porc et de Pelyarmenus, fait ainsi à la fois chasseur et proie, 
joue de la tradition de l'aventure à caractère merveilleux, du moins symbolique dans le cheminement du personnage, de la chasse au porc
dans une forêt dont la narration souligne le caractère merveilleux. L'éloignement de Pelyarmenus et Dyalogus à sa poursuite est donc à 
la fois attendu dans ce cadre, mais bien entendu aussi pour permettre leur capture par Dorus et Mardocheus. \pend
\pstart Pas de nouveau § BLors s’apresta Pelyarmenus a l’endemain, 
   et meismement Dyalogus 
   et autres chevaliers de cijusques a v, et pendirent leur 
   corsescus a leur 
   colsco‹r›[l]s et pristrent 
   les fors espiez en leurs 
   poinzmains et tout ainsi se mistrent hors de 
   Rome a grant bruit et ne finerent, si vindrent en la 
   Forest du Martrai. 
   Et le venierres cercha tant qu’il trouva le porc. Lors descouplerent leurs chiens, et fu la noise a merveilles grans, 
   dont le porcle porc fist si grant abail et deffense que moult 
   leur en gita mort des plus soufissans, 
   mes tant en i ot d’uns et d’autres qu’il ne vot leur noise endurer, 
   ainz se dreça en son estant et fu veus d’aucuns qui disoientdistrent 
   que bien avoit xi piez entre la keue et la teste. 
   Nouveau § BLors en fu la nouvelle dite a Peliarmenus,
   qui moult en fu joianss'en esjoi. 
   Il s’en esbaudistestendi et semont les veneurs de bien fere la besongne, 
   et il si firent, par quoi li pors, qui telz estoit conme dist est, se mist a la voie, et n’i ot nul qui pooir 
   eust de lui suivre tant estoit la forest orrible ou il s’embati. 
   Pelyarmenus, qui le cuer ot fier et orgueilliex, se mist aprés en 
   la forest et le suivi tant que nul de sa mesniee 
   ne li potn'ot pooir de lui tenir compaingnie. 
   Dorus et Mardocheus, qui leur afere avoient pourveu, 
   les avoient tant suivis qu’i s’estoient mis en la chace. Et tant esploita Dorus que par aventure 
   il s’estoit mis parmi les esclos aprés Pelyarmenus. 
   Dont alerent tant et li uns et li autres que li pors ne pot eschaper, ainz rendi estal a ii chiens 
   qui moult l’avoient tenu le jour court. Quant Pelyarmenus vit 
   cele, si fu molt joians. 
   Lors descendi du cheval a terre et prist l’espie, si vint au porc 
   mout esforcieement, et ne demora guieres que ilet le mist a mort. 
   Dorus, qui si pres le suivoit 
   conmeque vous porrez oïr, 
   mist tele painne a lui consuivir que avant que il peust cornersonner la prise 
   li vint devant et mist pié a terre du cheval et l’escria et dist :
   Par Dieu, Peliarmenus, trop avez atendu de ma couvenance ! 
      Si est hui le jour que chier lele vous covient comparer. \pend
\pstart Pas de nouveau § B
   Quant Pelyarmenus 
   le vitl'entent, 
   si l’a prisié moult pou, conme cilz qui de riens ne le connut. Lors saisi l’espie et dist :
   Qui es tu, qui si tost m’as conquis ?Assez atant le savras, dist Dorus. 
   Lors vindrent ensemble conme li plus courageus du monde, mes tost fust la bataille 
   desd'eus ii outree, 
   ne fust Dorus, qui vif et sanz bleciersain 
   le voltvoloit prendre, mes il 
   n’n'enot le pooir, 
   quecar tele desfensse 
   metoitmist Peliarmenus 
   en soilui que il eust ocis Dorus 
   se il ne fustne fust ce que il estoit armez. Mes cil, 
   qui avoit son temps passé de touz les chevaliers du monde de proesce, si le mist a merci, vousist ou non. \pend
\pstart Quant il ainsi du tout fu au desous mis, 
   Mardocheus vint par la forest chevauchant, qui avoit oï la noise des chiens, 
   si trova Dorus quiou il 
   tenoit son frere si court que pou s’en failloit qu’il ne li metoit jus la teste 
   par desus lesdes 
   espaulesespau-les espaules. 
   Il mist pié a terre et vint vers eulz et mist li un et l’autre a reson et dist :
   Biaus seingneurs, ceste escremie couvient demorer. L’un couvient obeïr a l’autre. 
      Et qui tort a, bien en doitcovient fere l’amende
      Légère variante du proverbe "Qui a tort si l'ament" (Morawski, 1822)..
   Lors esgarda Pelyarmenus Mardocheus, 
   si ne le reconnut de riens, et non fist il Dorus, son frere. 
   La narration insiste donc sur le fait que Pelyarmenus n'est pas capable de reconnaître son frère, qu'il a 
   pourtant déjà vu, dans un nouveau jeu sur la méconnaissance entre frères qui vient comme sous-tendre leurs désaccords.
   Dont respondi, moult iriez :
   Biau seingneurs, quiqui estes vous qui 
      sisi hardis estes, 
      conmequant 
      a tel honme com je sui osastes fere tele envaïe ?
   Dorus respondili respondi :
   Tu n’es mie homs a qui on doie dire veritévoir, 
      car il n’a en toi foi ne loiauté, ainz es parjures faus et desloiaus, et pour tel te rendrai 
      je 
      ae‹n›[a] 
      l’emperere de Coustentinoble, envers 
      qui tu es 
      faus parjures de ce que tu avoies en couventtelz comme 
         je devant ai dit. \pend
\pstart Quant Pelyarmenus entendi Dorus, 
   si le reconnut et se penssa que de li se departiroit par son enging. 
   Lors se mist a jenous devant lui et li dist :
   Ha ! chevaliers de tres grant proesce 
      qui as non Dorus, 
      voirement vous doit onon aorer et prisier sus touz les filz de mere. 
      Et sachiez que je du tout me metraimes en vostre merci, 
      et le maufetteurtraiteur 
      metrai en vostre main voiant touz les princes du mondeRome, 
      et en sera faite justise a vostre volenté, si que touz diront que onques 
      mes en nulnule 
      jourjour du monde ne fu veue si grant.Par Dieu, dist Dorus, autre seurté 
      m’enne seraseroit 
      faite avant que vous m’eschapez !
   Lors l’ont amdui saisi. Porque chose que il seust dire ne li valut riens 
   qu’il ne li liassent lesles les piez et les mains, 
   et l’ont en tele manieremaniere si fort 
   atornéchevillié qu’il 
   n'otn’ont pooir de nul mot 
   dire. Lors l’ont mis en un lieu ou il jamés ne feust trouvez qu’il ne le seust, puis orent conseil que il son cheval destourneroient 
   et puis corneroient la prise pource que Dyalogus vouloient avoir. \pend
\pstart Pas de nouveau § B
   Ainsi com ilil le deviserent, 
   ainsi le firent.
   Lors ne demoura guieres que Dyalogus entendi le son du 
   corcors et se trait cele 
   part au plus tost que il pot. Mardocheus, qui bien le connut, 
   si volt de lui l’afaire esploitier. Dont li vint a son encontre et le prist en tel maniere qu’il n’ot pooir de lui desfendre, 
   et le lierent aussi conme le chevrel que on mainne au marchié. Lors fu chevillié tout entour, aussi 
   commeque il avoient fet 
   Pelyarmenus, et furent mis li uns delez l’autre, si que bien se porent connoistre. \pend
\pstart Quant il orent ainsi esploitié, Mardocheus dist a 
   Dorus que il demourast avec eulz 
      etet le 
      lessast convenirconvenist
      du seurplus. DontDont il vint au porc ou il gisoit, 
   si grant que bien s’en pot on esmerveillier, mes ne demoura mie longuement que li veneeur ne soient venu, 
   qui moult estoient traveillié et lassé. Lors esgarderent Mardocheus, si ne le connurent mie, ainz li 
   demanderentdemanderent a 
   qui il estoit.
   Je sui, dist il, le forestier de ce bois et 
      lale garde.
   Dont li enquistrent qui le porc avoit ocis. Il leur dist que 
   Pelyarmenus l’avoit mort mes il estoit i pou bleciez, 
      par coi il s’estoit acheminezmis vers 
      Rome, 
   et le royleur 
   La leçon de B et X2 paraît plus vraisemblable, Pelyarmenus n'occupant, à son grand dam, aucun trône.
   mandoit par lui qu’il en feissent l’office, car il se doutoit de sa plaie. 
   Adonc furent dolent et esbahiesmari 
   et i en ot aucuns qui se mistrent 
   aprés luiaprés, et meismement 
   lisi chevalier. 
   Mardocheus s’en vint a sa mesniee, qu’il avoit lessiee en 
   la forest. 
   Dont les hasta que il se meissent aprés lui, et eulzil si firent 
   et ne finerentfinererent, 
   si vindrent a Dorus ou ilqui les atendoit. \pend
\pstart Pas de nouveau § B
   Quant il furent la venus, il orent leur chose si aprestee qu’il mistrent 
   Pelyarmenus en i panier et Dyalogus en 
   i autre et puis les mistrent sus un fort cheval. Et n’i otne fust 
   nul quiqui jamais 
   s’ense peust apercevoir que se peust estre 
   tel chose. 
   Et lors se mistrent a chevauchier par 
   la forest, qui moult estoit grant et hydeuse. 
   Moult chevauchierent longuementpar la forest 
   celetoute la nuit, mes onques tant ne 
   sorent esploitierle porent faire que il peussent venir a voie 
   que il vousissent. Et nonpourquant, siil les menerent 
   li dui varlet pescheeur sanz qui il ne peussent fere ce qu’il avoient empris. 
   Quant il orent tant alé que bien cuidierent eslongnieravoir alongié lor voie, 
   si s’arresterent, conme cil qui jusques a heure de vespres avoient alé de ci pres du jour. Lors descendirent et se vodrent aaisier et 
   reposer, conme cil qui moult estoientestoient laz et traveilliez. 
   Il avoient viandes assez et ce que mestier leur estoit. Dorus vint a
   Pelyarmenus et li dist :
   Se tu veuls mengier, maintenant le pués fere. 
      Et se tu ne lele pues faire et tu ne le fais fais, 
      tu ne le feras devant demain a ceste heure. \pend
\pstart Pas de nouveau § B
   Quant Pelyarmenus oï ce, si li fist signe 
   par quoiquepour quoi 
   onil peust parler a lui, et il si fist. Dont li demanda et dist :
   Chier frere, conment estes vous telz que vous ainsi me menez ?Pource, dist Dorus, que je vous vueil rendre a mon frere Helcanus, 
   et pource que je vous vueil mener plus seurement le fas je ainsi, ne autrement ne vous cuit je mener de cest païs, 
   si avraiavra plus grant force.Frere, dist il, se ne ferez vous mie, ainz vous asseur sus Dieu et sus 
      moim'ame que je ne voudroie estre quite 
      de ceste amende, puisqu’il est ainsi que vous en tele maniere m’avez pris et 
      conmeconmme vous maintenant avez pooir de moi tolir la vie, 
   que jeje avec vous partout m’en irai ausi debonnerement conme 
      feroit la plus simple chose qui soit par tout le monde, 
   car, puisque je sai que vous me deliverrez a mon chier frere Helcanus, 
      je ne feraiferai ferai chose qui soit contre sa volenté.Pelyarmenus, dist Dorus, 
      ne vous vaut chose que vous puissiez faire ne dire, car foy queque je doi 
   l’ame mon pere ne Helcanus, mon frere et le vostre,
      vous ne m’eschaperez autrement de ci adonc que 
      vous soiezje vendraisoie 
      en lieu ou je vous puisse mener a ma volenté. Mes faites ce que je vous di 
      debonnairement, ou par la foy que je doi a 
      mon peretous ceulz que vos m'avez ci fait nommer, 
      je vous metrai ml’espee parmi le cors aussi com se vous 
      eussiezm'eussiez mort la riens ou monde que je plus amai !
   Nouveau § BLors quant il entendi ce, si ot tele paour que il onques 
   mes jour de sa vie tele n’ot. 
   Dont li dist adonc Pelyarmenus :
   Frere, aiezaurez merci de moi 
      et je ferai vostre volenté. Et sachiez, se vous saviez la moie volenté, vous feriez ma requeste.Je n’en ferai autre chose, dist Dorus. 
   Dont mengierentmengierent tout conmunement et burent. 
   Et lorsqu’Lors quantil orent 
      ce fait,Et puis si se mistrent a la voie. 
   Mes il n’orent alé guieres loing quequant le jor 
   aparutappercut, 
   et alerent tant que il virent i chastel devant eulz 
   qui seoit en la costiere d’unede la montaingne. 
   Dorus demanda a ceusvalles qui les menoient 
   quel chastel s’estoitestoit la. 
   Dont furent cil esbahiesbahi et (sic), 
   car il autre fois i avoient esté, 
      et sorent que s’estoit i 
   lieuchastel ou 
      Fastidorus demoroit souvent. Si orent moult grant doute que 
   il n’i fustfussent, 
   par quoi ilil ne fussent conneu. 
   Lors furent sage et apenssé et ne vodrent mie lor seingneur esmaier. Et nonpourquant 
   avoient ilavoient il a-voient il  
   aprochiéaprochié de 
   Rome de la voie de leur chemin, si distrent :
   Sire, nous sonmes bienvenus ! 
   Lors distdist li uns a Mardocheus 
   en l’oreille :
   Sire, cist chastiaus est li retés de cest paÿs 
      ou Fastidorus repairedemeure et repaire 
      plusle plus sovent.
   Quant Mardocheus oï ce, si dist :
   Nous couvient il par la passer ?Certes, sire, distrent il, oïl, 
      se nous voulons aler le droit chemin.Alons, dist il, 
      hardiement.hardiement ou 
         autrement 
      Nous n’avons paorpooir de faire chose qui bonne 
      ne soit. \pend
\pstart Pas de nouveau § B
   Dont s’adrecierent cele part et ne finerent, si furent la venus. Lors arrouterent leurs sonmiers, 
   conme cil qui estoient en guise de marcheans. Dorus aloit ou chief 
   devantpremier et 
   Mardocheus aprés. Ainsi entrerent ou chastel. Et 
   lorsleur fu enquis 
   quel gent ilce 
      estoient. 
   Il distrent qu’il estoienterent 
      marcheans d’Alemaingne. Lors leur couvint paier le travers selonc ce 
   que il distrent que li avoirs estoit. \pend
\pstart Pas de nouveau § B
   Quant Dyalogus entendi qu’il estoient en 
   une ville, 
   si conmença tot maintenant a fere noise 
   la plus grant 
   quede ce que il 
   onquesonques le 
   pot fere, mes onques sitost 
   conme cil qui les menoientcil ne 
   les entendirent, sique il conmencierent a chanter 
   a hautes voissi haut que noise que 
   Dyalogus feistil feissent 
   ne pot estre entendue. 
   Et nonpourquant, si estoit lors Fastidorus el chastel et 
   avoit grant plenté de barons avec lui. Mais cil, qui sage furent de leur fait, passerent outre 
   la ville sanz nul arrest 
   et sanz nul empeeschement de nullui et se mistrent en leur chemin 
   et chevaucherent tant letoutetout le jour sanz encontrer chose qui 
   grever les peust. MesCar il estoient si apenssé et si sage 
   que enen nul lieu ou il 
   eust gent autre que les leur il ne s’arrestoient 
   point, 
   ainz buvoient et mengoient et dormoientreposoient touzjours aus champs 
   et estoient sus leur garde adés, conme cil qui bien en avoient mestier et besoing. 
   Mais i pou me vueil taire d’eulz et 
      me couvient venirretournerai a ceus qui 
   avecques Peliarmenus avoient le porc chacié le jour que il fu pris.
 \pend
         
            Ci aprés devise 
               conment li chevalierla gent Peliarmenus 
               quistrentse mistrent en la queste de 
               Peliarmenusleur seigneur 
               et Dyalogus, que Dorus 
                  avoit pris en la forest
                     la forest en chacant le port
                  et comment il trouverent son cheval en une forest 
               etet les 
               amenezamena o 
                  eulzluy.
            
               Enluminure sur 1 colonne et 12 UR
                  Les chevaliers en armure en forêt se mettent à la recherche de 
                  Pelyarmenus et 
                  Dialogus pris par Dorus 
                  sans qu’ils ne le sachent.
               
            
            \pstart Or dist li contes que li 
   chaceeurchevalier furent a 
   Rome reperié, et bien cuidierent 
               trouveravoir trouvé 
   leur seingneur. 
   IlIl le 
   demanderent‹l›[d]emanderent
   partout, 
   mes il ne fu nul qui leur en 
   seustpeust 
   riensnouvelles dire. 
   Moult fu tost la cité esmeue,
   et fu quis suschiez tous 
   lesceus qui mires 
   qui adonc estoient adesont a 
   Rome, mes moult s’en esbahissoient, conme ceus qui riens n’en savoient. 
   Dont cuidierent qu’il fust demorez en la forest pour sa plaie. 
   Lors s’en vindrent plus de cent et le cerchierent em pluseurs lieus. Tant alerent li i et li autre que par aventure 
   trouverent le cheval Pelyarmenus ou cil l’avoient lessié pource qu’il ne vouloient 
   mie estre conneu. 
   Nouveau § X2Quant il ne virent leur seingneur, sachiez que 
   ilil enil dont furent a moult grant meschief, 
   si le quistrent tant et 
   le huchierent partout, et 
   meismement iluec pres trouverent le 
   cheval Dyalogus. Dont furent moult esbahis et ne sorent que pensser et 
   se mistrent au chemin par devers Renniers – 
   Renniersce 
   estoit li chastiaus ou Fastidorus estoit 
   quantque Dorus et 
   ses gens i passerent. Dont s’en vindrent a lui et ne li vodrent celer ceste chose. 
   Quant Fastidorus l’entendi, si fu esbahis 
   moult durementsi c’onques mes si ne fu. 
   Dont conmença a pensser en plusseurs manieres moult durement. 
   IlMais onques siMais il
   nene le 
   sotpot 
   tant pensserfaire 
               queen nule maniere que
   ilde riens ses'en donnast garde 
   de ce que avenu lor estoit, ainz fist la forest cerchier en tel maniere que, 
   dedensde ii mois, 
   il ne fist autre chose, il et plus de 
   viivc, 
   que uns que autres, d’enquerre et d’encerchier, mes n’en porent oïr nouvelles. 
   Et dedenz cel termine qu’il le queroient porent 
   bienbien cil 
   estreavoir 
      eslongniécherchier et avoir vuidié 
   dule paÿs 
   Dorus et li sienses gens, 
   car ilqui n’arresterent 
   onquesnul jour 
   neet nule nuit 
   fors tantse tant non que
   il menjoient et buvoient et 
   dormoientdormoient au mains qu'ils pooient. 
   Si errerent tant qu'il
   vindrent a Lucenbourc le chastel dedenz i mois. \pend
\pstart Quant Dorus fu ainsi saisi de ceus 
   dont vous avez oï parler, 
   si fu moult joians et fist son frere deslier et Dyalogus 
   aussi. Si les fist venir par devant lui et dist :
   Que vous semble ? Avez vous esploitié envers mon frere et envers moi si conme vous devez ?Ha ! frere, dist Pelyarmenus, por Dieu merci ! 
      Je sui tel atornezmenez 
      que jamés, si com je 
      croicuit, n’avrai pooir de moi aidier.Ha ! dist Dorus, conme vous faites 
      ore le mauvés ! 
      Et vous, maistre Dyalogus, conment vous 
      semblesemble ore de la besongne ?SireSire, dist il, 
      il me semble que chose que je pourroie dire ne faire ne m’ime pourroit aidier.Par Dieu, dist il, vous dites verité.
   Lors conmanda Dorus qu’il fust mis en une fort prison et descouvenable, 
   et il si fu. Aprés siil conmanda que 
   Pelyarmenus fust mis en une tour et vi
   serjans fors et seürs por lui garder sus leur vies, et il si fu. Dont prist Dorus 
   i mesagier et l’envoia au 
   duc, son seingneur, et li fist asavoir ceste chose. \pend
\pstart Pas de nouveau § BQuant 
   li duxil 
      entendil'entendi 
   ceste chosece, si ne fu 
   onques mesmes homs si esbahis, 
   et si ne pot croire que ce fust verité. Dont s’en vint a Luxemborc et trouva 
   Dorus qui aisier se faisoit, conme cil qui avoit 
   sousfert moult de durs lis. 
   Quant il vit le duc, si se dreça encontre lui, et s’entrefirent joie de ce qu’il porent. 
   Lors s’asistrent l’un delez l’autre et li requist li dux qu’
   il li deist verité pourquoi il l’avoit mandé. 
   Dorus li conta 
   tout ainsi conme il i estoit avenu, car il de riens ne li 
   en volt mentir. 
   Quant li dux l’ot entendu, si dist queque il 
   onques mes en toutejour de sa vie n’avoit 
      oï parler de greingneur proesce, car il dist qu’il avoit fet greingneur proesce que 
         c mille honmes n’eussent fait. 
   Lors dist li dux que 
   il vouloit parler a Peliarmenus.
   Alez, dist Dorus, et sachiez que il voudra 
      direfaire ne dire, 
      car il convient mon frere 
      savoir ceste chose et nous avoir conseil que nous en 
      pourronspourrions fere.Par Dieu, dist il, c’est toute verité.
   Dont s’en vint li dux 
   a Pelyarmenus, 
   quiou il estoit en une tour amont et la avoit il joué aus eschés 
   a i serjant que il tenoit moult court. 
   Quant le duc vint amont, si dist on ali fu 
   
      PelyarmenusPelyarmemenus :
   Sire, vez ci le duc de Lembourc
      , qui vous vient veoir.
   Dont sailli sus Pelyarmenus et vint encontre lui et le salua 
   tant conme il pot. 
   Li dux, de l’autre part, le conjouy conme cil qui bien le sot faire. 
   Lors le trait a une part et li dist :
   Par foy, sirefire, 
      onques mes jour de ma vie n’oÿ si 
      grant merveille conme de ce que Dorus vous a amené en ce païs en tel maniere.Sire, dist il, je ne cuit mie que tele merveille si avenist mes ne ne doie fere jamés. 
   Et sachiez que Dorus est li plus souverains de proesce 
      et de sens quequi onques jor du monde nasquist de mere, 
      ne je ne puis savoir conment nuls homs peust penser ne osast ce faire que il a fait. Et sanz faille, je connois bien, 
   et chascun le doit savoir, que de son droit il a fait ceste chose et je de mon tort si le doi comparer.Certes, sire, dist 
   li dux, vous dites ce que vous devezverité. 
      Et soiez certainsachiez que, se Dieu plest, 
      les choses vendront a bien de vostre partie. 
      MaisMais cilz Dyalogus 
      n’en puet partir que il ne le couviengne comparer.Sire, dist‹...›[c]e dist
      Pelyarmenus, pour Dieu merci ! 
   Et sachiez que il n’i a courpes sanz moi et nous sonmes freres, si pourrons encore estre bon ami ensemble.  
      S’il estoit aucuns preudoms qui painne vousist mettre en mala pais, 
      et je le vous voudroie prier que vous painne
   i meissiez, conme cilz qui moult y a de pooir. Et sachiez que vous i pourrezporriez
      moult faire de vostre honneur et de vostre proufistpreu.Sire, dist li dux, sachiez que j’en sui moult priez 
      et voudroie que la pais que j’en diroiediroie en 
      vousistvousissiez tenir l’une partie 
      et l’autre par tel maniere que ele seroit plus couvenable a vous 
      quequ'ele ne seroit a eulz.Sachiez, sire, dist Pelyarmenus, 
      que je entendrai du haut et du bas tout ce que vous en vodrez faire et dire 
      et en vueilcuide 
      donner bonne seurté, quoi que je doie perdre ne gaaingnier.Sire, distce dist 
      li duxil, 
      ceste chose ne puet pas estre faite si tost, ainz le couvient fere par le 
      conseilconseil Dorus de 
      vosvous amis et des nostres.Certes, sire, dist il, vous dites verité. 
   Dont parlerent de plussors choses. 
   EtEt quant il orent parlé assez ensemble, dont se departi 
   li dus de Peliarmenus moult debonnairement et 
   l’aseura de ce qu’il pot 
   que il toute la pais que il pourroit y metroitde la paix faire. \pend
\pstart Pas de nouveau § BAtant prist congié li i a l’autre, 
   et s’en revint li dus
   a Dorus et li dist la verité de tout ce qu’il avoit trouvé en 
      son frere. Lors firent faire unes lettres, si envoierent i 
   serjant en 
   Gresce, qui se parti de court ainçois que li dux ne fist. 
   Cilz ne fina par ses journees, l’un jour plus, l’autre mains,
   et errad'aler 
   tant qu’il vint en la cité de 
   Constentinonble et trouva Helcanus et o lui 
   le duc Leus d’Athainnes et Mirrus 
   et autres, dont la compaingnie estoit moult belle et moult gente. 
   Nouveau § BQuant le mesage fu descendus de son cheval, 
   si monta amont ou palais et salua Helcanus de par 
   son frere et li mist la lettre en la main. Dont rendi Helcanus 
   a celui son salu et froissa la cire et desplia la lettre et conmença a lire 
   et trouva ens escript la prise de 
   Pelyarmenus et de Dyalogus, son frere. 
   Quant il l’otot ce leue, si fu si 
   durement esbahis que il 
   nen'en sot 
   que dire ne que fere, 
   ne il ne savoit ce s’estoit verité ou nonse il avant le deist pour verité. 
   Lors trait le mesage d’une part et li 
   demandadist :
   Amis, conment puet ce estre voirs de ce 
      que mon frere me mande ?SireSire, dist cil, 
      je ne cuit mie que vostre frere vous certefiast chose par son seel qui ne feust 
      veritable.veritable.
      "Or me di, est Peliarmenus et Dyalogus en sa prison ?" 
      "Sire, dist cil, voirement y sont il. Estoit ce que vous ne poiez croire ?"
      "Par foi, dist il, voirement le puis je trop a envis croire."
      Et lors sacha cil une autre lettre de par le duc et li dist : 
      "Sire, veez ent ci encore unes autres que li dus de Luxembourc vous envoie et bien croi que il le vous tesmoigneront."
      Dont brisa la cire et trouva cilz moult d'amours que il li mandoit et li tesmoingnoit ce que il avoit trouvé de Peliarmenus et
      que il en feist savoir sa volenté. Dont fu si joians Helcanus que moult le pot on bien savoir, car il apela le duc d'Atheines et leur moustra 
      et a moult d'autres les lettres qui li estoient envoiées de son frere. Quant il oyrent ce, si ont li uns l'autre esgardé et n'i ot nul 
      qui ne quidast que ce fust truffe. Et distrent tout a un concille : "Comment puet c'estre voirs que ces lettres dient 
      que Peliarmenus et Dyalogus soient en la prison Dorus ?"Par foy, dist 
      HelcanusHelca-canus, 
      de ce sui je touz esbahis ! Et nonpourquant, ilcil ne le m’eust 
      ja mandé se ce ne feust voirs.Certes, dient cil baronil, 
      se devons nous bien croire, car bien savons que, 
      ce nul chevalier peust faire ceste chose, que Dorus l’a faite, 
         car ilende ce faire est bien puissans de sens et de chevalerie.Certes, dist chascuns, veritez dites. \pend
\pstart Dont assembla
   mist ensamble 
   Helcanus son conseil et leur 
   demandademanda conseil 
   quele response il pourroit arriere mander de cele chose. Lors esgarderent tout que bonne chose seroit que 
   li prison fussent amené en Constentinonble porce que le principal du 
   fetmeffait 
   avoit esté meffetfait en la terre et ou paÿs, et il meismes amenast 
   le duc, son seingneur. 
   Adonc avroient conseil que on en pourroit faire. Ainsi conme ilil le 
   deviserent fu escript, et se remist 
   le mesage a voieen son chemin 
   et ne fina, si vint a LucembourcLembourc 
   le chastel. Dorus n’estoit mie en la ville, 
   mes il ne demoura que ii jours que il reperarevint et 
   sotso‹i›[t] i 
   le mandement de Helcanus, son frere. \pend
\pstart Pas de nouveau § BQuant il oï 
   ce 
   que la les couvintcouvenoit envoier, si le fist savoir 
   le duc, sanz qui conseil il ne voltvoloit 
   rienspetit fere. Li dus li manda 
   que tout ce estoit resons 
      et que il seroit prest de ce qu’il vodroit faire de quele heure qu’il 
      voudroit‹li›[v]oudroit mouvoir. 
   Si ne demoura guieres qu’il orent leur affaire apresté, et furent c armeures de fer 
   quiet ne finerent, i jour plus, l’autre mains, 
   sanz nul empeeschement, tant qu’il vindrent en la cité de Costentinoble. 
   Et trouverent Helcanus et moult d’autres, dont li contes ne fet nule mencion. \pend 
\pstart Quant Helcanus vit le duc et 
   son frere, si leur fist si grant joie que grans anuis seroit du 
   recorderracompter. 
   Et sanz faille aussi firent tuit li autre baron. Mirrus, qui touzjours estoit a cort, 
   et Clyodorus ne s’en porentpooient 
   rasazier. 
   Quant il se furent ainsi conjoui, 
   Helcanus volt que on li amenast Peliarmenus, et on si fist. 
   Et quant il vint devant lui, si se lessa cheoir a ses piez devant 
   touztouz ceus qui i estoient et dist :
   Frere, moult m’avez donné par maintes fois a souffrir, et je a vous aussi. 
   Or est du tout ainsi que vous estes venus au desus de moi. Si vous requier merci, conme frere courpable envers autre.
   Lors dit Helcanus :
   Lieve toi, ou autrement je ne parlerai de riens a toi.
   Le refus d'Helcanus d'échanger avec Pelyarmenus ainsi à ses genoux témoigne de la dimension 
   théâtrale, et certainement surjouée, de l'intervention de Pelyarmenus, et de sa fausseté ainsi aussi.
   Dont le firent drecier li baronbaron qui la estoient. 
   HelcanusEt quant il ce fu dreciez, Helcanus 
   li dist :
   Par Dieu, Pelyarmenus, voirement dites vous verité que moult avons donné a sousfrir 
   li un a l’autre. Mes je ne sai ne ne voi que toute la soffrance vous n’aiez pourchaciee. 
   Et qui moult vous en eust lessié couvenir, vous en eussiez ouvré ordement 
      et feussent les choses alees autrement qu’i ne sont.Frere, dist Pelyarmenus, de riens que vous vueilliez dire sachiez
      que jamés ne vous en vueil desdire. Et sachiez que vous du tottot en 
      dites verité, car je du tout y ai mise la descorde. 
      Et encoreoncore (sic), 
      quise il 
      le me lessastloisist 
      le feisse je ne ne feusse encore venus au repentir, mes que vous me creez. 
   Je veil que vous sachiez que je du tout en sui vrai repentant et appareilliez de fere l’amende tele conme vous la voudrez prendre. 
   Et se je ne la puis fere, si querez qui la face, car mon pooir ne s’estent plus avant. \pend
\pstart Quant il ot ce dit, si orent tuit pitié de li fors 
   que Dorus, qui dist :
   Par Dieu, Pelyarmenus, qui vos parolles voudroit metre a oeuvre 
      et vos fais, moult avroit 
      griefgrant 
      choseoeuvre 
      a fere, 
      car vos parolleselles sont contraires les unes et les autres !Conment ! distce dist Helcanus. 
      Faites moi entendre ce que vous dites.Par foy, dist il, volentiers. Je di que en ses fais puet on trouver petit de foy et de loiauté. 
   Et en ses parolles a a la fois du sens et du refui, par coi je di que en li a moult grant estrif souvent.
   La lucidité de Dorus, qui semble aller de pair avec sa grande combativité et peut se comprendre 
   en raison du caractère très personnel des méfaits de Pelyarmenus à son égard, met en lumière la propre naïeveté d'Helcanus face 
   aux ruses de leur vil demi-frère.
   Dont y ot pou de ceus qui moult volentiers n’eussent sourris 
   s'il osassents'il l'osassent. 
   Mes le duc de Lembourc se feri es parolles 
   Dorus, et fu Peliarmenus menez en la prison, 
   qui moult fu couvenable pour lui. Si parlerentparloient ensemble li baron 
   conmentpar quoi il pourroient trouver une voie 
      parpour 
      quoi la venjance fust prisefaite couvenable et 
   que li frere peussent demourer bon ami ensemble. Dont n’i ot nul d’eulz qui la voie peust trouver, 
   car il n’i avoit cil quinulz ne pleiast
   Pelyarmenus. 
   Si l’avoient trouvé faus et traïteur et plain de mal enging, car il ne tenoit chose qu’il eust en couvent, 
   caret bien savoient que ainsi 
   le feroit il 
   qu’i les ense on le 
   lesseroitlaissoit aler. Si distrent tout de conmun que 
   bon seroit que il fust forment 
      tenus de ci adonc que Fastidorus et autres s’en mellassent, par quoi on fust seürs de lui 
      et des autres que, lequel qui jamés forferoitforfeist 
      l’un vers l’autre, que il perdist 
      la chose par quoi il fust desheritez a touz les jours de sa vie. \pend
\pstart Biaus seingneurs, dist Dorus, 
   je di de ma partie de tant conme a moi en apentapartient 
   quecar, par cele foy que je doi 
   a touz bons chevaliers et 
   au cors et a l’ame mon pere et que je doi a Helcanus, 
   mon frere qui ci estest et foi que je doi a tous chevaliers, 
   Pelyarmenus ne Dyalogus 
   n’istront de prison tant comque je 
   lesles y pourrai tenir 
   devantdevant ce que cil i 
   serontseront, se il sont en vie, qui a l’autre fois i furent. 
   Quant il de la pais et de l’amende distrent leur volenté, et par ceus derechief l’amende en iert jugiee 
   et faitefaire si couvenablement que 
   je m’en tenrai a paiez. \pend
\pstart Pas de nouveau § BQuant 
   li baron qui la furentil oïrent ce, 
   si furent tout lié de celui serement et distrent :
   Sire, sachiez que de ceste chose ne faites vous pas a blasmer, et nous tous 
      l’nous yacordons.
   Et adonc furent li prisonnier conmandé fort a tenir, et 
   meismement 
   Dyalogus fu a tel meschief que grant pitié fust de lui se il feust honme 
   de quidontde quoi on 
   lela deust avoir. 
   Quant Pelyarmenus sot ce, si pria a Helcanus que 
   il feist savoir a Rome que il feust secourus, 
      car il savoit bien que on ne savoit nule nouvelle de lui. 
   Dont li dist Helcanus que ce feroit il volentiers et en feust a pes, 
      quecar il n’avroit pis 
      que le cors de lui, mes qu’il vousist fere droiture. 
   Il dist que de ce seroit il prest en toutes les manieres que 
      onil le vourroit 
      mener. \pend
\pstart Ainsi demoura ceste chose. Dorus et 
   li duxdux de Lembourc s’en vodrent partir 
   de ci adonc qu’il avroient oï nouvelles de Fastidorus. Il apresterent leur voie pour aler veoir 
   le duc et la duchesse. 
   Helcanus ne volt mie lessier que il ne feist escrire sa volenté, 
   si prist i mesage, si l’envoia a Rome. 
   Cil ne fu mie lens, ainzcar il ne fina 
      de ci ase au mains non que il potfist 
   tant que il vint en la cité dea Rome. 
   Lors enquist cist se 
      Fastidorus estoit en la ville, et on li dist que 
   voirement i estoit il, mes moult estoitert 
      courrouciez et plains de grant maladie. 
   Cil demanda de quoi il estoit courouciez.
   
      Amis,Amis, dist il,Et on li dist que de son frere, 
      qui estestoit mis a mort, si conme il cuide. \pend
\pstart Pas de nouveau § BLors ne volt plus 
   le mesage enquerre de ceste chose, 
   ainz en vint au Palais Majour, ou il avoit grant baronnie. 
   Cil fist tant qu’i parla a un chevalier qui Gasus avoit non. 
   Et cil Gasus estoit si bien ami au 
   damoiseldamoisiaus de 
   Costentinoble. Et quant Gasus vit 
   le mesage, si li fist moult grant 
   joiefeste, et cil li 
   bailladonna unes lettres que Helcanus 
   envoioitli envoioit par lui. 
   Il froissa la cire et trouva ens escript : \pend
\pstart Pas de nouveau § B
   Helcanus, 
   ainzné filz a noble honme et couvenable 
   Cassidorus de Costentinoble, 
   a son tres chier et feable ami Gasus, 
   seingneur de Durmont et baillif de Rome, 
   salut et appareillié de servise. Conme il soit ainsi que discorde ait eu entre nous 
   et la devant dite cité et les damoisiaus de 
   Rome et que les choses venissent a ce que pais et concorde 
   en fust faite par 
   le dit de vostre pere et de maint vaillant chevalier, 
   endroit de ce que Peliarmenus 
   devant nonmé oteust eu couvent par son serement
   que il nous rendroit Dyalogus, 
   qui avoit esté trouvé en son tort, si conme ilque bien est seu 
   pour voir, il n’est venu ne n’a envoié ne couvenances 
   tenues, pour ce fas je savoir a vostre 
   vueillancevaillance que, par l’aide de Nostre Seingneur et de vous, 
   que Pelyarmenus et Dyalogus 
   nous tenonsest en nostre prison par devers nous, 
   en telle maniere que nous ne poons mie veoir que legierement il se puissent partir de nostre prison. 
   Si en avrons eu nostrevostre volenté et nostre droit, et Dieu vous tiengne en 
   voie de verité, si conme je croi que vous touzjours avezayez vescu 
   et ferez tant comque vous vivrez. \pend
\pstart Quant Gasus ot ce entendu, si fu si joians que moult le pot 
   on bien veoir, car il couvoitoit moult que les damoisiaus de Costentinoble eussent 
   leur raison de ceus de Rome. MeismementMeismemen‹...›[t]
   quant il sot que Peliarmenus estoit en vie, si li plot moult, 
   car il en estoit moult durement courrouciez. Lors conmanda que 
   le mesage 
      feustfust fust bien aaisiez et li enquist 
   se il avoit nulles lettres qui alassent a Fastidorus ne ailleurs.
   Il dist que non. Dont s’en vint es chambres ou 
   Fastidorus estoit a moult grant anui. Lors li dist :
   Sire, faites bonne chiere, car bonnes nouvelles ai de monseingneur vostre frere. \pend
\pstart Pas de nouveau § BQuant Fastidorus oï ce, 
   si se dreça en son estant et dist :
   Chiers amis, dites moi queles elles sont.
   Lors li lut Gasus la lettre qu’il ot eue de Helcanus. 
   Quant il ot ce entendu, si fu si esbahis qu’il ne sot que respondre autrement que il dist :
   Qui sont cil qui tel chose pueent avoir fait en ma terre ?Sire, dist Gasus, se n’ai je mie seu ne autre chose n’en sai 
      que ce que ceste lettre m’en chante.
   Dont distfist Fastidorus que 
   on li feist venir 
      le mesage par devant lui 
      qui avoit les lettres aportees, et on si fist. 
   Quant il le vit, si le salua et dist :
   Amis, dites moi se vous savez quantcomment 
      mon frere fu pris et menez en Gresce.
   Cil dist que mauvessement le savoit, car 
      il estoit (sic) 
      n’avoit il pas esté au prendre, mes ce qu’il en avoit entendu li diroit il volentiers. 
   Lors li conta auques pres l’afaire, ainsi conme il avoit alé, 
   mes il onques ne le pot croire, si dist :
   Il ne puet estre ainsi avenu ! 
      Bien sai de voir que mi honme meismes l’ont trahy, et ne puet estre qu’il ne soit seu en aucune maniere.Sire, dist Gasus, souffrez vous de ce dire, 
      car je ne croirécroi 
      jamie qu’il ait honme en ce païs qui de nule chose s’en fust mellez. 
      Et bien sachiez que, se vous mem'en voulez croire, 
      ja tel chose n’en direz devant que vous miex en savrezsachiez la verité.
   Adonc mescrut Fastidorus Gasus de maintenant, 
   car il savoit bien que il amoit l’autre partie, mes il onques n’en fist semblant de riens, car il 
   sele doutoit pour le sens 
   et pour le bien que il avoit maintes fois veutrouvé en li. 
   Le jeu de camps qui se dessine entre Rome et Constantinople est explicite ici avec la position de Gasus, 
   bailli de Rome, mais allié dans son coeur des héritiers de Constantinople. La lucidité de Fastidorus sur la question est révélatrice
   de l'importance de ces lignes de tension, mais aussi de la forme d'impartialité de Fastidorus, qui peut valoriser la qualité de Gasus
   même s'il ne soutient pas sa "partie".
   Dont li dist :
   Chiers amis, moult vous ai trouvé feable par plusseurs fois. 
      IlIl vous couvient que vous me conseilliez, 
      carcar bien sai que vous estes l’onme el monde de vostre pooir 
      qui miex i puet besongnier. \pend
\pstart Sire, dist Gasus, 
   toute la bonne painne que je i pourré mettre, je l'ii metré, et en sui tout 
   prestpriez. 
   Mes aiez vostre conseil pres de vous et sachiez qu’il vous en diront.
   Dont fist savoir a l’empereris, sa mere, ceste nouvelle, 
   qui moult estoit a grant meschiefau dessous de corrous et de tourment. 
   Quant ele le sot, si fu i petit plus aise a sa volenté, 
   car on li fist savoir queque il de leur volenté estoient alé en 
   Gresce pource qu’il l’orent en couvent. 
   Lors furent tuit li prince a Rome mis ensemble et orent conseil qu’il envoieroient en 
   Costentinoble Gasus pour savoir et 
   pour aprendre conment on pourroit ceste chose metremener miex a point. 
   Dont s’apresta Gazus, 
   et meismes Ysidore, la fame Pelyarmenus, 
   qui ne volt lessier pour nullui que elle n’alastne vousist 
   veoir son seingneur. 
   Lors ne se pot tenir sa suer Affode qu’ele ne venist a lui et li dist en l’oreille :
   Par Dieu, vous i alez plus pour veoir 
      Helcanus que vous ne faites pour vostre mari.SuerSuer, dist ele, 
      vous ne faites pas bien ne 
      ne dites voir, ainz estes pou sage, 
      quique vous 
      tel parollece dites !
   L'affection que portent l'ensemble des quatre filles du roi d'Espagne au preux Helcanus, et la jalousie
   que cela suscite entre elles, est au coeur des épisodes relatés dans le "Roman d'Helcanus", §169 et suivants.
   Lors s’en partirent atant l’une de l’autre, et fu leur affaire prest, si se mistrent 
   auen leur chemin. 
   Si ne vueil fere mencion de leurs journees, 
      car moult i avroitot a dire 
      ainz que je eusse tout conté, 
      que il errerent 
   tanttant firent par leur jornees 
   qu’il vindrentvenissent 
   en Costentinoble. \pend
\pstart Pas de nouveau § BQuant 
   Helcanus sot la venue de la dame, 
   si pensapenna de li fere moult grant honneur. 
   MeismementMeimes de 
   Gasus fu il moult joians. 
   Si conmanda que li bourgois et les dames qui auques valoient ississent hors de la cité 
   contre eus. Mesmement Helcanus 
   et mout d’autres chevaliers issirent hors, et i ot une joie mout merveilleuse, 
   carquant 
   HelcanusHelcanus meismes et 
   Nera, 
   sa fenme, vinrent a l’encontre de la fenme Pelyarmenus. 
   Si la prist entre ses bras et la conjoui et la baisa devantvoiant 
   sa fenme par moult grant signe d’amour. 
   LorsLors aprés s’entracollerent les ii dames, 
   que mie ne se porent taire qu’eles ne deissent :
   Ne soions mie courrouciee l’unel’u‹il›[n]e 
      vers l’autre pourpou‹...›[r] chose que 
   Helcanus ait fait. \pend
\pstart Aprés fu Gasus conjoïs de 
   Helcanus et des autres chevaliers qui la estoient. 
   Ainsi entrerent en la cité, 
   moulta moult grant joie 
   faisant. 
   Quant il furent descendu, si monterent amont ou palais. Et Ysodore s’apresta et 
   volt aler veoira son seingneur. 
   Dont la prist Gasus et 
   Mirrus et Nera et vindrent en la tour ou 
   Pelyarmenus avoit entendu les nouvellesestoit 
   que sa fenme et Gasus 
      estoient venus. Quant ilQuant 
   vit sa fenme, si fu moult joiant, si qu’il ne vousist 
   pourpar nul tresor que elle ne fust venue. 
   Lors se trestrent a une part et parlerent de plusseurs choses, et sot Pelyarmenus 
   la volenté de Fastidorus, son frere, et du conseil de Rome. 
   Et dont li enquist Gasus conment il avoit esté pris. 
   Dont ne li volt riens celer, ainz li dit 
   tout en tele maniere conme il li estoit avenu. 
   Et quant Gasusil oï ce, si dist :
   Sire, je loeroie que vous feissiez pespes qui 
      couvenablecouvenable fust et feussiez bon ami 
      ensembleensemble ensemble a touzjours 
      mes. 
      MaisCar 
      sachiezsoiez sachiez (sic)
      jamésque l’une partie ne l’autre n’avra 
      jaméspais ne repos se vous painne n’i metez.Certes, Gasus, verité dites. Et bien sai que a la concorde avez bien pooir, 
      et je du tot vous enchargeen gete la besongne sus vous, 
      si en fetes tant que je vous en sache gré,gré." 
         "Sire, faitdist il, j'en sui tous priez 
      mais que vous sachieznous sachions que il voudront dire ne faire. \pend
\pstart Dont se departirentdepartirent tout 
   de lui, fors Ysodore, 
   qui demoura avec son seingneur,sa femme 
   et li autre s’en revindrent ou palais, 
   et fu li mengiers appareilliez. Puis ont lavé, si se sont assis 
   etau mengier ou il 
   furent moult bien servi. Et quant se vint aprés mengier, 
   si mist Gasus Helcanus a reson et li dist :
   Sire, moult me puis esmerveillier conment vous estes en tele maniere saisi de 
      mon seingneur et de Dyalogus.Amis, dist Helcanus, 
   touzjors li biens vient a son droit et li maus a son tortCette expression
   sentencieuse n'apparaît pas chez Morawski ni Le Roux de Lincy. PD, stp, veux-tu bien vérifier, je suis sûre que cela doit y être..Et pour Dieu, aiez enen vous 
      bon conseil conment on puist ces choses mettre a point.Gasus, dist il, la chose ne gist pas du tout en moi, ainz gist a 
   Dorus mon frere.
   Adonc fu li jours et li termes mis que il devoient faire l’acordance, en tele maniere que dedenz xl jours i seroient 
   tout li prince qui avoient esté a l’acordance faire. Lors firent fere chartres et briez, 
   et fu chascuns mandez moult couvenablement selonc ce qu’il estoient. 
   Peliarmenus volt que Gasus demorast o lui
   et renvoia arriere a son frere, que il fust a sa journee qui mise i estoit. 
   Si me vueil de ceste chose taire et vueil 
      repairier a l’empereeur, dont je me sui une piece teus.
 \pend
         
            Ci revient li contes a l’empereeur 
               Cassidorus, 
               qui est en l’ermitage avec 
                  le lyon
                  comment le lyon le mena a l'ermite.
            
               Enluminure sur 2 colonnes (e-f) et 13 UR. Cassidorus 
                  en armes rouges à ronds blancs dans la forêt devant 
                  l’ermitage où l’attendent 
                  l’ermite lisant et le lion assis à ses pieds.
               
            
\pstart Or dist li contes que,
   quant li emperieres Cassidorus ce fu mis en la roche,
   si conme devant est dit, il fu moult obedient envers Nostre Seigneur,
   et sans faille il souffroit moult de 
   griezgrans penitances, 
   conme cilz quiqui riens n’avoit de coi vivre,
   fors de ce que li lyons 
   li aportoit chascun jour sa porcion de quoidont il vivoit. 
   Et si li avenoit que, quant il vouloit boivre, il aloit a la riviere qui couroit le fons du val. 
   Cele vie mena bien i an que onquesonques de la ne s’en parti, 
   et tant que i jour avint que li lyons demoura outre heure que il souloit venir. 
               Dont s’esmerveilla moult li empereres durement que il demoroit tant. 
   Lors le regarda et le vit venir ausi conme i grant quartier de 
   chevalchevrel en sala 
   bouche. Et quant il vint aussi conme a l’entree de la roche, si esgarda l’emperere aussi 
   com s’i li deistdi me tu : 
   Venez avant, se vous voulez avoir chose que j’aie !
               Li empereres s’en donna bien garde et vint a lui. 
               Et quant li lyons le vit venir vers li, si se trait ensus. 
   Li empereres 
   vit qu’il se traioit ensus, si regarda quele part il aloit. 
      Et lors le suivi 
   li empereres, 
   et tant le suivi qu’il vint devant la meson a l’ermite 
   qu’il avoit tant quis quant il vint premiers en la forest. 
   Dont regarda et vit que li lyons s’arresta a l’uis 
   dedu 
   l’ermitepreudomme. 
               Quant il l’apercut, il fu si joians qu’il loa Nostre Seingneur de tout son cuer. Lors hucha 
               li empereres que on le lessast entrer ens. 
               Li preudoms estoit en oroisons et n’estoit pas encore couchiez, 
               si ot paor, car il n’atendoit mie tel oste. Si dist :
   Qui estes vous, qui a teleceste heure 
      me huchiez ?Amis, dist il, je sui uns homs qui moult couvoite 
      que je puissea parler a vous, et ne vous vueil se touz bien non. \pend
\pstart Lors a 
   l’ermitecil ouvert son huis, 
   et li empereres et li lyons entrerent ens. 
   Et quant li hermites vit le lyon, 
   si oten ot grant paour et dist :
   Aide Deux ! Soiez moi anuithui 
      aidantgarant que ceste beste ne me face mal !
   Dont li dist li emperieres :
   Amis, vous n’avrezavez garde se bonne non, 
      car sachiez de verité que je croiquit 
      que ce soit miracle de Nostre Seingneur quantque il ceens 
      m’amenam'a amené.Ha ! sire, dist ilil, pour Dieu, 
      dites moi qui vous estes, car je ai grant volenté que je le sache.Amis, dist il, tout ausint sui je desirrant que vous le sachiez.
   Lors s’asistrent li uns delez l’autre et li lyons entre 
   sesleurs piez, sa proie entre 
   sesles siens patespieds. 
   Adonc conmença li empereres a dire 
   son errement de chief en chief
      chief comment il avoit esploitié 
   puisque li empereris avoit esté morte, et tot ainsi conme il avoit vuidié son empire 
   et conment il avoit ouvré au Chastel Mignot ; 
   de la pucele qu'il ainsi avoit menee par sa volenté ; 
   aprés conment il s’en parti et conment il avoit trouvé sa terre empeeschiee, 
   si conme devant est dit ; aprés conment ele avoit esté aquitee par la volenté de Dieu 
   et de ses bons amis, qui aidié li avoient ;
   aprés li dist conment la volenté li vint en son cuer de faire sa penitence, 
      meismement de 
      l’avisionla vois qui li vint 
      enen son dormant, 
      pourquoipar quoi il avoit tout lessié et en estoit 
      venuspartis pour venir a lui, 
   conme cist qui bien cuidoit trouver son recet ; 
      conmequant il trouva 
      le chevalier qui ainsi le mena, conme devant est dit ; 
      meismement de la valee de l’yaue, du pont et de 
      la roche du lyon : 
   il nene li volt de riens 
   mentir, car tout ainsi conme avenu li 
   estoitestoit dessi a laavoit li 
   a ditdist toute la pure verité. 
   Une fois encore, le récit témoigne de ce goût de revenir sur les derniers évènements, nombreux et 
   complexes, qui ont mené à la présente situation. La dépendance de Cassidorus au lion, pour manger comme pour trouver cet 
   ermite dont il souhaitait tant la compagnie, joue de la dimension miraculeuse de tout cet épisode.
   Nouveau § BQuant li hermites oï ce, 
   si loa moult Nostre Seingneur de tout son coragecuer et dist :
   SirePar foi, sire, 
      moult puet on bien savoir que Nostre Sires vous ainme, 
      quecar voirs est que maint biau miracle a Dieu fet pour vous 
      puisque vous venistestenistes a terre 
      tenir, 
      et il est bien resons et droisdrois et raisons est 
      que vous faciez sa volenté a ce que li cors soit traveilliez 
      par coia ce que l’ame 
      ne soit empeeschieeempeeschiee quant ce vendra au grant jour du 
      jugementsalu.Sire, dist l’emperiere, ainsi 
      l’enli pri je qu’i le me consente jusqu’a la mort.
   Dont fist chascun s’oroison a Nostre Seingneur moult devotement. Et quant il furent 
   au chief de piece redrecié, 
   si li demanda li hermites 
   se il huiméshuimés il voloit mengier. 
   Li emperieres ditli dit que 
   il n’avoitne l'avoit 
      mengiéfait de tout le jour. 
   Lors li donna de telz biens conme il avoit, mes 
   li empereresil 
   respondidist que il ne mengeroit fors 
      tele viande conmece que son pourveeur l’avoit pourveu, 
      c’est a entendre ce queque li 
         lion li avoit aporté. \pend
\pstart Maintenant prist sa proie entre 
   lesses piez du lyon 
   et puis la torditorst aussi conme 
   ferefere le souloit et en 
   menga par moult bonne volenté. Quant il ot ce fait, 
   si fu alé moult grant piece de la nuit. Dont s’alerent reposer de ci a l’endemain que il se leverent. 
   Et quant li hermites ot son huis ouvert, 
   li lyons si sailli hors tout aussi com s’il eust esté em prison. 
   Et quant li empereres vitoy ce, si 
   fu esmerveilliezot merveilles porquoi il 
   otavoit ce fait. 
   Lors le hucha, mesmes il onques 
   nene se retorna, ainz se feri en 
   la forest aussi conme s’il eust paour de lui. 
   Dont demoura li empereres avec l’ermite, 
   qui moult estoit joiant de sa compaingnie. Quant se vint a l’eure de nonne, que li lyons 
   souloit reperier en la roche ou li empereres 
   souloit estre, si cuida que il deust venir a lui tot aussi pourveus conme il souloit, 
   mesmes mais pour noient le fist, car il ne vint ne ala, 
   ainz passa l’eure qu’il devoitsoloit venir. 
   Lors fu li emperieres moult dolens, car il cuida bien que Nostre Sires 
   se fust courrouciez a lui de ce qu’il avoit 
   vuidiélaissié son recet de la roche. 
   Dont apela l’ermite, si li dist :
   Sire, moult sui dolens de mon lyon, qui ne repaire 
      mesplus.Sire, dist il, or ne vous anuit, car je cuit que sanz aucune bonne reson n’est ce mie. 
      Si vous pri et requier que vous veingnez prendre avec moi telz biens conme Dieu m’a pourveu et 
      consenticonsentir a avoir, 
   car pource que je vueil bien croire 
      queque pource que vous n’avez de quoi vivre, 
      sique ce
      a estéfu sa volenté
      tele que il ça vous 
      amena et envoia a moi et 
      que vous preissiez ce que il vous voudroit prester. \pend
\pstart Pas de nouveau § BQuant li empereres 
   l’entendi, si penssa que veritéverité se disoit et dist :
   Sire, je ferai vostre requeste.
   Maintenant ot li hermites pain d’orge apresté et clere yaue de fontainne, 
   si mengierent par bonne saveur. 
   On peut peut-être voir le départ du lion et le changement de régime qu'il occasionne chez Cassidorus 
   comme une évolution dans sa pénitence, de la nourriture toujours carnée, chevaleresque, à celle plus propre à la vie érémitique,
   du pain et de l'eau.
   Ainsi demoura li emperieres avec
   le preudonme 
      hermite, 
   qui mout de biens li disoit, et metoit avant par plusseurs fois 
   moult de choses qui bien fesoient a metre en 
   remembranceretenance 
   et qui orent puis mestier a lui et aus autresaus plusieurs gens et d'autres 
   de la contree. 
   Si est drois que je d’eulz me taise de si adonc que 
      temps et leupoins 
         et heure sera
      en, 
      si revendrai aus barons qui mandez estoient par divers paÿs pouret 
      la concordeconcorde faire des damoisiaus de 
      Costentinoble et 
      dede ceulz de Rome, 
      dont je ai parlé devantdevant et fait mention ariere 
      en monou conte.
 \pend
         
            Ci devise conment 
               li prince et li baron sont venus 
               a la journee que Pelyarmenus et 
               Dyalogus avoient encontre 
               Dorus et Helcanus, et i vint 
                  Fastidorus
                     Fastidorus, le frere germain Peliarmenus, 
                  et toute sa baronnie o luilui. Si comme s'en suit aprés
                  s'assemblerent pour faire la pais entre Peliarmenus et ses freres
                  vindrent au jour qui leur fu mis de la pais.
            
               Enluminure sur 2 colonnes (e-f) et 12 UR. Rencontre des princes et barons pour 
                  le jugement de Pelyarmenus et 
                  Dialogus par 
                  Helcanus et Dorus. 
                  Un groupe d’hommes ni armés ni couronnés en accueille un autre devant la ville. 
               
            
\pstart En ceste partie dist li contes que 
   veritez est, si conme j’aiest dit devant, que, 
   quant li baron sorent que la requeste dudes 
   filz l’empereeur 
   estoit tele, 
   il n’i ot nul qui endroit soi vosistne voudrent lessier pour nule chose 
   que il eussent a ffaire qu’il ne venissent au jour que mis i estoit. 
   Sachiez que dont i ot maint bon chevalier quiqui joie et 
   feste s’entrefirent. MeismementMeimes Helcanus 
   et Dorus connoissoient leurs bons amis, qui aide et confort leur 
   avoientavoient jadis fait. 
               De l’autre partie se tenoit l’autre partie ensemble, et conmencierent a traitier de la pais 
   si conmeselonc ce que devant 
      esta esté dit. 
   Mes il nene le sorent onques tant fere 
   quique il onques la peussent trouver, car 
   Dorus vouloit une merveilleuse 
   amende avoir, 
   mes Pelyarmenus ne s’i voloit 
      acorderou nuls ne pooit Peliarmenus amener, 
   car je truis en escript que 
   il vouloit que Pelyarmenus 
               crevast les yeux a Dyalogus et li coupast les piez et les mains et encore plus : 
   que il li sachast le cuer du 
   ventrecors qui acordez s’estoit a la traïson qu’il avoit faite 
   et pourchaciee.Le sort que souhaite réserver Dorus à Dyalogus et Pelyarmenus paraît signaler sa vigueur, 
   mais aussi dans ses excès, du jeune et vaillant Dorus, mais il permet aussi de mettre en lumière la cause sous-jacente à la colère
   de Dorus, liée également à l'absence de son père pour faire respecter l'accord qui avait été pris contre Dyalogus et Pelyarmenus. \pend
\pstart Quant Fastidorus sot ce, 
   si fu moult courrouciez et esmeus en ire, et lors dist conme homs sanz joie que 
   avant que son frere feist tel chose 
      en morroient il plus que fet n’eust. 
   Dorus sot ceste parolle, si dist que son serement tenroit il 
      conment que la pes en deust avenir. 
   Lors s’en vint Daphus a lui et li dist :
   Sire, parpour Dieu, quel fu vostre serement ? 
      Et ne l’avezavez vous gité par vostre volenté sus ceus 
   qui autrefois furent a la concorde par coi nous feusmes departis et ala chascun en sa contree ?Par Dieu, dist il, voirement fis.Et conment, dist il, ne voulez vous mie que nous vous apaisons ?Sire, dist ilDorus, 
      vous n’en avez pooir selonc mon serement.Pourcoi ? dist Daphus.Pour ce, dist il, que cil n’i est pas sanz qui je ne le puis faire, se du tout n’est a mon chois. \pend
\pstart Pas de nouveau § BQuant Daphus oï ce, si penssa bien qu’il s’escusoit par toute raison,
   si dist :
   Sire, voirs est que monseingneur vostre pere n’i est mie presens, 
      par coi vous derriezpoez detrier la pais et la concorde.Par Dieu, dist Dorus, 
      voir ditesverité est, se il meismes 
      enn'en fet sa volenté, que je la moie en avrai avant que 
      Pelyarmenus ne li autre issent de la prison.
   Quant il ot ce dist, si fu Daphus moult esperdus, 
   car il sotsavoitsoit bien que 
   li emperieres estoit en aucun lieu ou il fesoit sa penitance, 
   conme cil qui giehi li avoit aussi conme paren confession. 
   Si ne sot que faire, ainz vint au 
   conte de Flandres et au 
   duc de Lembourc et a moult d’autres barons et leur dist que 
   jamais il n’i pooit 
      avoirvenir 
      acort se li cors de l’emperere 
      ne lela faisoit, qui mie 
      n’n'iestoit presens. \pend
\pstart Pas de nouveau § BQuant cil l’ont entendu, 
   si ont l’un l’autre regardé et furent tout mu. 
   Dont parla li quens de Flandres et dist :
   Conment ! Ne poons 
      nousnou‹...›[s] mettre nul jour 
      par quoi nous ne puissons tout aussi bienbien ore fere concorde et pais 
      conmeque a i autre jour?Sire, distce dist 
      Daphus, sauve soit vostre grace, si ferons, 
   car li emperieres n’est ore pas ci endroit, 
      par coi Dorusdont on
      ne veult mie que la pais fustsoit faite ne confermee 
      sanz lui, car autrement il ne tendroit mie son serement.
   Dont rougist li quens 
      de 
      Flandres de corrous et d’ire 
   et ne se volt taire qu’il ne deist :
   Dont nous devons nous tenir pour fox quant ci sonmes venus et on ne nous velt croire ne que nous 
      soions souffisans d’eus apaisier, selonc ce que li uns ne doit a l’autre porter hayne ne courrous ! \pend
\pstart Quant le duc de Lemborc ot 
   le conte entendu, si vit qu’il fu courrouciez, si vint a lui et li dist :
   Ha ! gentilz sire, ne soiez de riensce esmeus, 
      car je vous di bienvous ai en couvent que, 
   avant que de ci vous departiez, pes et concorde i sera, se jamés i doit estre !
   EndementresEn ce qu’il estoient ainsi en tel point entra en la sale 
   le lyon a l'empereeur. Quant li baron le virent, 
   si n’i ot nul si hardi qui joievoie ne li feist, 
   et il tout maintenant ala chascun cerchant tout aussi conme cil qui queroit Daphus aussi 
   conme par miracle de Nostre Seingneur. Lors ala tant de l’un a l’autre que il le trouva delez 
   le conte de Flandres et 
   le duc de Nise. Quant il l’ot trouvé, si li moustra 
   i signe d’amour tel que tout li baron s’en porent esmerveillier, 
   car tout aussi com s’il me deistdi me tu :
   Je sui a toi venus de loing pour grant besoing, si vueil que vous o moi veingniez, 
   si fist signe la beste mue. Lors n’i ot il nul qui le veist quiqui a 
   ceste chose ne prisastpensast. 
   Dont li dist li dux de Nise : 
   Daphus, Daphus, 
      ci a moult grant aventure et belle ! Sachiez que moult ameroie qu’ele feust moie aussi conme ele est vostre.
   La réflexion de Karus, duc de Nise, se veut révélatrice du topos du départ pour une aventure merveilleuse
      qui se joue ici avec l'appel du lion et les préparatifs de Daphus pour y répondre.
   Dont mist Daphus sa main sus le chief au lyon 
   et le conjoui de ce que il pot. Maintenant se parti li lyons de lui et 
   se mist touzjours vers l’uis de la salle en regardant 
   Daphus. Et quant 
   lele bon chevalier vit ce, 
   il demandaregarda ses armes que on 
      les il'i aportast, 
   car il vouloit aler avecaprez 
   le lyonlui en 
   quel lieu que ce fust. Maintenant saillirent plus de x serjans, et furent ses armes 
   aporteesaprestees moult tost, 
   et l’ont adoubé li meilleur chevalier du monde qui 
      a ce tempsalors estoient. \pend
\pstart Quant Daphus fu de ses armes aprestez 
   conme chevalier nele pot miex estre, 
   si li demanderent li plusseurs 
   se il vouloit avoir compaingnie. Il dit que non. Li chevaus estoit enmi la court, 
   quiqui li estoit aprestez. Lors vint 
   au chevala lui et 
   montase mist en la selle, si pendi i escu fort et pesant a son col 
   et prist ii i fort glaive en sa main a fer agu 
   et trenchant. Dont se mist a la voie et prist congié aus barons et leur dist que 
   pour Dieu il priassent pour lui. Il distrent que 
   Deux le menast en tel lieu que il peust faire le preu de l’ame et l’onneur du cors. 
   Quant il orent ce dist, si leur pria a touz ensemble i don que il ne savoit
   mie se il jamés plusplus nul 
   enleur 
   reverroitrequerroit. 
   Dont n’i ot nul quil ne fust souples, et distrent que moult volentiers li otrieroient 
       tout ce que il voudroit demander de raison, sauve lor honneur. 
   Sa demanderequeste si fu tele que il demourassent tous en 
   la cité jusques
   a tant qu’il avroient certainnes nouvelles de sa revenue ou de sa demouree. Ceste chose ne li vodrent refuser, ainz li otroierent, 
   et il atant s’en 
   partirentparti, et si s’en ala 
   grant aleure le lyon devant et 
   Daphusil aprés, moult grant joie faisant. 
   De ses journees ne vous 
      vueilvueil je fere 
      lonc contemencion, car 
   il errerentalerent 
   tant que il s’embatirent en la Forest de Espiere – ce 
   estoitert 
   la forestcelela ou
   ou li emperieres 
   estoitestoit demourez 
   avec l’ermite. \pend
\pstart Quant il furent la venus, li lyons, 
   qui le chevalier menoit aussi conme par miracle, se mist en une viez sente et 
   conmençacommencierent a aler une grant piece du jor, et ne demoura 
   guieresfors tantque tant que 
   Daphus entendi la vois meismes que li emperere avoit 
   devant oïe, si conme je ai devant 
      ditfait mention sanz l’aventure mettre a fin. 
   Dont s’arresta Daphus, conme cilz qui savoir volt que c’estoit, 
   si se mist cele part qu’il otavoit la vois oïe et la plainte que 
   la damoisele faisoit.
   Quant li lyons vit cele, 
   si li anuia moult et donna i saut a Daphus et li fist aussi 
   conmeque i signe 
   qu’il n’i alast mie, mes cilz en qui chevalerie terrienne estoit plantee ne le volt lessier pour lui.
   L'ennui du lion pourrait presque prêter à sourire, il semble souligner la répétition de ces aventures que 
   propose le récit, entre celles de Cassidorus à l'entrée de cette forêt et celles qu'il relate maintenant pour Daphus.
   La caractérisation de la chevalerie terrienne semble pour sa part dénoter d'une orientation des préoccupations de Daphus,
   trop éloigné des enjeux spirituels qui se nouent dans cette forêt.
   Lors feri son chevaldestrier des esperons et vint au lieu ou ele crioit, 
   si vit i chevalier armé de toutes armes et 
   son cheval delez lui, et tenoit une 
   moult belle puceledamoisele 
   par semblant et en vouloit fere sa volenté. Dont s’aprocha moult tost Daphus 
   de luieulz 
   etet en conmença moult 
   durement a blasmer 
   le chevaliercelui de ce 
   qu’il vouloit cele a force violer. Quant cil entendi Daphus, si lessa 
   la damoiselle et vint a son cheval et monta en la selle 
   si isnellement que moult s’en esmerveilla. Daphus si s’apresta de ce que il pot, 
   mes onques si tost il ne pot faire que cilz li donna i cop si grant 
   queque il onques mes n’avoit receu si grant, ce li sembla. \pend
\pstart Quant Daphus ot 
   cele cop receu, si ne fu pas lens, ainz ot l’espee sachiee et mist l’escu 
   avantamont et vint a celui et li donna i cop, 
   etou il mist moult de sa force. Tout aussi conme se il eust feru sus une 
   enclume couvint il 
   l’espeele cop resortir et traire ensus. 
   Dont le hasta cistcist aprés et 
   lile 
   donnarendi i greingneur cop que 
   cilz 
   devant n’avoit esté. Mes cil, qui au jour de adonc n’avoit honme douté, 
   li courut sus si esforciement que bien leles cuida mettre au desous par force de 
   chevalerie. Lors conmença escremie et bataille si grant que riens n’estoitne fust 
   au regart de toutes celes dontque j’ai encore parlé en mon conte,
      a ceste dont je me sui entremis de dire. Et pource que li aucuns tenroient a trufle ce que li contes en tesmoingne, si venrai a la 
      finfin d'eulz conment ne liquelz 
      en fu outrez. \pend
\pstart Pas de nouveau § BVoirs fu que tant dura li estris de eulz ii 
   que par semblant pou pooit conquerre l’un sus l’autre, et leur cop estoient moult afebloié. Dont esgarda 
   Daphus celui a qui il avoit la bataille et vit bien
   que point n’isoit de sanc de lui, conme celui meismes qui armeure n’avoit empiriee. 
   DontTout aussi vit il de lui, mes si se sentoit 
   froissiezfroissiez et blechiez que a bien pou qu’il ne chieoit jus du cheval a 
   terre, mes il nule volenté n’avoit de li requerre. En ce qu’il estoit en tel point, 
   si esgarda vers le ciel et dist :
   Biau 
      dousdous sire Dex, qui tout feistes de noient, 
      secourez moi si vraiement que pour nule mauvaise volenté que je vueille ne fais a mon espoir ceste 
      batailledeffense !
   Lors vit le ciel partir et fuaussi conme en crois, 
   et il maintenant se seingna conme ende la crois
       et fist lecelui signe sus lui 
   de la crois, mes onques si tost ne l’ot faite quant un grant escrois et 
   i grant cri si grant et hydeus oï, que il vousist ou non l’estuet 
   cheoir pasmé a terreterre jus de son cheval. \pend
\pstart Quant il fu revenusrepairiez de paumoisons, 
   si regarda entour lui et ne vit que son cheval et le lyon, qui l’atendoit en 
   grant signe d’amour. Dont s’apercut li bons chevaliers et vit bien que par son 
   orgueil estoitil avoit esté deceus. 
   Lors se trouva si bleciez et si doulereus que por tout l’avoir du monde il ne fust montez sus son cheval, et 
   leil li couvintestuet 
   si tres grant piece reposer. Et estoit ausi conme soleil couchiez, car tant avoit 
   duré laduré bataille, 
   si conme il a de nonne jusqu’dessia celle heure que 
   je devant ai dit. 
   Nouveau § BDaphus, qui en tel point se vit que il en son cheval ne 
   se pot mettremonter, 
   et d’autre part il estoit nuit, si ne sot que il pot faire. Lors esgarda, 
   si vit que li lyons conmença a gronnoier et a fere semblant que il se dreçast por aler, 
   et il si fist. Dont cuida monter, mes pour nulle choseriens ne l’eust 
   fait, car il estoit tel atourné que il ne se pooit aidier. 
   Si prist son glaive et son escu et le pendi a l’arçon de sa selle 
   et abati la resne du cheval et se 
   mistmist a aler tout a pié aprés 
   le lyon, qui le mena grant piece de la nuit, tant qu’il vindrent a 
   la meson de l’ermite. 
   Lors conmença le lyon a grater a l’uis 
   de sondes pié moult forment. 
   Quant li hermites l’oï, si ot grant paour et vint a 
   l’empereeur et li dist :
   Sire, je ne sai quele beste grate a nostre 
      huishuis moult fort, et bien sai que ele veult ceens entrer, 
   mes je ne sai se c’est pour bien ou pour mal. \pend
\pstart Dont sailli em piez 
   li empereres et sotson
   maintenant que ce estoit son lyon 
   et vint a l’uis et li ouvri et trouva lui et Daphus, si conme j’ai devant conté.
   Lors prist li emperieres le lyon
   par la teste et le conmença a aplanier et distdist a-aplanier et dist
   a Daphus :
   Biau sire,sire, ce 
      dist li emperieres
      aua 
      chevalierDaphus, qui estes 
      vousvous, dites le moi, qui avec 
   mon lyon estesestes ci venus ?
   Lors l’entendi Daphus, si le connut a la parolle et li dist :
   Sire, je sui Daphus, vostre ami.
   Quant li emperieres l’entendi, 
   sifi (sic) 
   fu moult joians qu'il ne pot mot dire, ainz 
   vintvint a lui en lermes et en pleurs, si l’embraça si fort que il le fist fremir,
   et ne failli guieres qu’il ne gita i cri de fine doleur de ses bleceures. Si dist :
   Ha ! sire, pour Dieu merci ! Souffrez que je puisse a vous parler tant que je vous aie conté m’aventure, 
               car de la vostre cuide je estre sages.
   Dont sailli li hermites au cheval et s’entremist de lui aaisier.
   Li emperieres et Daphus entrerent en 
   la seule et s’asistrent li uns delez l’autre, car plus ne pot Daphus demourer em piez de fine 
   lasseté.lasseté. Si dist li empereres a Daphus :
   Amis, ce 
      dist li empereres, tart m’est que je sache dont vous venez. 
      Et par quele aventure mon 
         compaingnon le lyon s’acompaingna a vous ?
   Lors li conta Daphus toute la verité 
      pourquoipar quoi il estoit la venus et conment 
      Dorus avoit esploitié de Pelyarmenus 
      et de Dyalogus, que il avoit mis en sa merci du tout, et conment li baron 
      estoientestoi‹t›[e]nt mandé 
      pour la pais asouvir, et Dorus n’en 
      voloitvoudroit riens fere pour eulz autrement que devant a esté dit, 
      et voloit que il meismes i fust ou a touzjours il
      leslel'en tenroit em prison. 
   Et quant il li ot tout ce conté, si li dist conment 
      li contes de Flandres en avoit parlé. 
      Et en tieux parolles qu’il parloient ensemble, 
      lison 
         lyons entraestoit entrez en la salle 
      et estoit venus a lui faisant tel signe que il o lui s’estoit mis a aler partout ou il 
      aloitiroit, et li avoient en couvent li baron qui la 
      estoientestoient que il ne se departiroientde ci a tant que il savroient nouvelles de moilui, 
      fust de reperier ou de demourersa demouree.
      Et quant je oi pris a eulz congié, je me mis aprés 
      le lyon, qui m’amena en 
      ceste 
         rochecontree et en ceste forestroche, se mist devant et 
            moi aprésforest. Mes quant nous eusmes alé de ci a nonne, 
      si entendi i cri moult piteus, aussi conme d’une pucelle.
   Lors li conta Daphus tout ainsi conme il li estoit avenu 
      de cele aventure la ou il estoitseoit delez lui.
L'orientation, d'emblée, du récit de Daphus sur les enjeux politiques de la situation qu'a interrompue le lion
à son arrivée à la cour souligne la direction que va prendre le récit, en-dehors des considérations spirituelles potentielles 
du miracle de l'appel du lion ou du repentir de Cassidorus auprès de l'ermite. La narration marque ainsi son retour aux questions
politiques qui doivent continuer à animer Cassidorus. \pend
\pstart Quant li empereres oï ce, si ot moult grant merveille de ce qu’il ot
   oï, et ne l’eust mie creu s’il ne le sentistl'eust seule seut 
   a si loial chevalier.
   Dont distDont
      il
      amisa mis moult a apensser a ce que vous m’avez 
      ditconté. Il vous couvient oster vos armes 
      et vous aaisier de ce que nous pourrons avoir.Sire, dist il, a vostre volenté.
   Lors le desgarnil'a desgarni li empereres 
   de ses armes et puis le vesti de telles robes conme il avoient, et fu tant avec eulz que il fu garis de ses plaies. Et quant il 
   fufurent 
   garisgaris de ses plaiessanez, 
   si se mistrentmist li emperieres 
   et Daphus a la voie ensemble a aler vers 
   Costentinoble et pristrent congié a l’ermite. 
   Se mistrent hors de la forest, et touzjours 
   le lyon avec eulz, et ne finerent d’aler, l’un jor plus, l’autre mains, 
   tant que il vindrent a une liueliue prez de 
   Costentinoble. Adonc firent savoir leur venue en 
   la cité, qui donc veist mouvoir chevaliers pour aler a l’encontre d’euls, et 
   adonc vindrent en Costentinoble a grant joie. 
   Quant se vint a l’endemain, si parlerent de la pes et de la concorde de 
   Pelyarmenus et de Dyalogus, le bastart. \pend
\pstart Dont se mistmistrent chascune partie ensemble 
   et troverent en leur conseil qu’il metroient tout l’afaire sus 
   le conte de Flandres 
   et sus le duc de Nise, 
   etet le duc Josyas 
   d’Espaingne et Japhus le 
   Frison et 
   le duc de Lucembourc, cil v 
   pour l’une partie ; et pour l’autre pristrent la seurté 
   de l’une partie et de l’autredes deux parties de tenir tout ce que 
   il en diroient puis, si orent conseil deque la pais et 
   de la concordeconcorde seroit 
   en tele maniere que Dyalogus seroit banys hors de 
   l’empire de Rome 
   etet de tout l'empire et de l'empire 
   de Costentinoble, de ci au rapel 
   l’empereeur et 
      Dorus, et ainsi que 
   Fastidorusses freres, Fastidorus 
   et Pelyarmenus 
   obligeroients'obligeroient cors et avoir a fere le conmandement 
   l’empereeur et son conseil, se il jamés a nul jour du monde i eust descorde ne male pais, 
   ne de par eulz ne de par leur conseil. 
    Ainsi fu acordéfu fait l'acort 
      de l’une partie et de l’autre par acort. \pend
\pstart Atant demourerent les parolles, et fu la feste et 
   la joiejoie el palais si grant que devant n’i ot greingneur eue. 
   Et fu Dyalogus mis hors de prison, qui bien cuidoit chascun jour estre mis a mort deshonneste.
   Lors s’en vint au conte de Flandres et 
   se lessa cheoir a ses piez, si ne fu onques honme qui tant se peust humelier conme il fist, et si li dist 
   telz parolles :
   Ha ! sire, gentilz homs, jeté m’avez de la tres 
      vilainnelaide et vilainne et cruele mort que je atendoie. Si poez 
      de moi fere conme de vostre serf.
   Adonc si li dist li quens que, pour l’amour de 
   Pelyarmenus, il vouloit qu’il alast avec li en Flandres, 
   et il si fist, dont il en ouvra puis mauvessement et fausement envers le conte, 
   dontde quoi vous 
      en orrez bien parler avant que li livres faille. 
   Il ne sera pas question des méfaits de Dyalogus auprès du comte de Flandres, mais cette précision
   vient encore signaler toute la vilennie de ce personnage.
   Et ainsi demorerent puis li prince viii jours touz entiersplains 
   en cele joie et en si grant feste que jamés ne s’en queissent partir, ne fu ce que chascuns avoit a faire en son pais. \pend
\pstart Quant ce 
   fuvint que 
   ilil s'en durent partir, 
   si firentfist li empereres 
   et li autre Gasus 
   venirvenir par devant lui, 
   et il i vint moult appareillieement. Quant il vint par devant lui, si le trait a une part et li dist :
   Amis chiers, mout vous ai trouvétrouvé plain 
      de grant foy envers moi et envers ma dame l’empereris. 
      Si vueil ouvrer par vostre conseil, car j’aiai ma volenté 
      dede li rapeler avec moi, si conme il me semble que raison l’aporte. 
   Si m’aprenez conment je le puisse faire plus couvenablement, par quoi je aie le gré et l’amour de ceus a qui la greingneur 
      partie de l’afaire monteen tient.
   Quant Gasus ot ce entendu, si ot si grant joie que il se lessa cheoir 
   ausa ses piez de 
      l’empereeur et li baisa le soler. Et quant il 
   lese leva, si n’ot pooir de respondre fors en larmes et en pleurs, et dist :
   Ha ! gentilz sirehoms, 
      conme gente requeste a ci ! OrCar sai je bien que Deux a oï 
      priere que je ai tant desirree et covoitiee. Si vous en dirai mon avis selonc ce que tout le monde en iert joians 
      et vous en prisera, si que bien le verrez.
   Lors li dist :
   Sire, et puisqu’il est ainsi que Nostre Sire Deux vous a donné tele volenté, je irai a mes damoisiaux de 
      Rome, en tele maniereaussi 
      que je ne sache mie ceste chose, si leur prierai que il facent prier a leur freres que il viengnent avec eulz a vous, 
   si vous charront aus piez et vous prieront pour Dieu que vous rapelez leur mere, 
      par quoi il soient plus couvenable, et que il liy puist avoir reson 
      et greingnor affinité. Et il vous prendra pitié de eulz, si serez si apensez que bien leur savrez respondre a leur requeste. \pend
\pstart Pas de nouveau § BQuant li empereres 
   ot Gasusl'oy 
   ainsi parlerentendu, si 
   distdist a 
   Gasusamis : 
   Bien avez dit ! Et je vousvou‹...›[s] 
      em 
      pri que vous en faciez ce queque vous quidiez que bon soit.
   Maintenant s’est Gasus de lui partis et vint a ses damoisiaus, 
   si les mist d’une part et leur dist :
   Biaus seingneurs, je me sui d’une chose apenssez qui moult seroit couvenable a vous, 
      se vous lela poiez fere. Et je croi que vous le ferez, se vous jamés la devez 
      faire, si en est poins.Quel chose est ce ? 
      distsefait 
      Fastidorus.Sire, dist il, la merci de Nostre Seingnor, nous avons bien besoingnié selonc ce que les choses 
      sontont alé. Mes encore 
      le pourrez, se vous voulez, miex faire une chose 
      que je vous dirai. Vous ferez a chascun des barons qui ci sont une requeste et leur direz qu’il vous 
         aidentaidassent a prier vos freres touz premiers et aprés 
         vostre pere que il vousist rapeler et venir avec vostre mere
         par coi il fussent ensemble a touzjours, si en seriez plus prisiez.
   Adonc se sont avisié queq‹...›[u]e il iront premierement a 
   leur oncleoncle, Dyomarques, 
      le roy d’Arragon. \pend
\pstart Lors vindrent a lui et li distrent ceste chose, dont il fu moult joians. 
   Aprés, il vindrent au 
   duc de Lucembourc, 
   si li firent ceste requesterequeste requeste ; aprés au 
   conte de Flandres 
   et a Japhus le Frison, et aussi a touz les autres, 
   qui furent moult lié de ceste requeste ; et adonc s’en vindrent a Helcanus et 
   a Dorus. Lors parla Daphus mout couvenablement pour 
   les damoisiausdamoisiau‹...›[s] de Rome. 
   Maintenant vindrent tout ensemble a l’emperere et se lessierent tuit iiii cheoir 
   de freres a ses piez. 
   DontDont il leur dist que 
   il se dreçassent amont et 
      levassent sus et 
      que il deissent lor volenté. 
   Dont dist Helcanus qu’il 
      ne len'en feroient mie se il ne leur otroioit leur requeste.
      Par mon chief, dist il, je savrai ainçois quele ele doit estre.
   Dont li dist Helcanus :
   Je et mon frere Dorus vous prions, 
      et ileulz du tout avec nous, que vous vueilliez rapeler 
      l’empereris, nostre mere, 
      avec vous et que vous 
      soiez li uns avec l’autre, si conme vous 
      devezsi comme vous devez, 
      et nous touz vousa vous en requerons que 
      vous le vueilliezil vous plaise a fere.
   B et X2 semblent avoir raison de se passer la précision qui fait de Fastige leur mère, puisque c'est 
   celle de Pelyarmenus et Fastidorus. On pourrait cependant voir cette dénomination comme une trace de la volonté des 4 frères d'agir
   de concert à la réunification familiale qui se joue ici.
   Quant Helcanus ot ceste raison dite, li empereres 
   se tut i pou, que il pou ne grant.I. mot 
      ne respondi. Adonc se mistrent tout li baron a jenous et distrent :
   SireHa ! sire, la requeste de vos enfanz 
      ne devez vous pas escondire se vous nous voulez servir a gré.
   Lors parla li empereres en haut et dist :
   Biaus seigneurs, 
      fetefetei est vostre 
      requeste. Or levez sus apertement.
   Adonc se drecierentdrecierent tout et i ot une 
   joiefeste moult merveilleuse 
   et grant. \pend
\pstart Pas de nouveau § BMaintenant fu demandé a 
   l’empereeur conment il vouloit que ceste assemblee fust 
      faite. Il respondi qu’il vouloit aler a Rome, 
   car grant piece avoit qu’il n’i avoit esté. Nouveau § B
   Adonc s’apareilla li empereres, o lui ses iiii filz et 
   tout lili autre baron qui la estoient, et se mistrent au chemin dedenz 
   viii jours. Dont ce fu Gasus 
   devant aprestez pour la nouvelle porter a l’empereris de Rome 
   et a ceus de la cité. Il ne fina par ses journees tant que il vint 
   a Romeen la cité et 
   descendientra ou Palés Majour ou 
   l’empereris estoit, o lui grant plenté de dames et de damoiselles. \pend
\pstart Pas de nouveau § BQuant il vint devant lui, si la salua de par 
   l’empereeur Cassidorus de Costentinoble. 
   Quant la dameelle 
   entendi Gasusl'entendi, 
   si ot moult grant joie de cestson 
   salutparole et salut et dist : 
   Dites moi :Amis Gasus, quelles nouvelles 
      me direzdites vous de 
      l’empereeur, mon seigneurlui ?Dame, dist il, bonnes nouvelles, se Dieu plest.
   Lors li mist la lettre que li emperieres 
   li envoioit en la main. 
   Et elle froissa ens la cire et trouva 
   escriptens escript la pes de ses enfans 
   etet aprés comment il li mandoit conment il venoit a elle 
   apour fereli fere 
   toute sasa plainetoute sa plaine volenté. 
   Quant ele ot ce leu, si ot moult grant joie a son cuer, qu’ele tressailli 
   toute de grant leesce, et puis loa Nostre Seingneur de tot son cuer moult devotement. 
   Lors se dreça la dame et dist oiant touz et toutes 
   le mandement que 
      son seingnor l’empereeur de Coustentinoble 
      li mandoit. \pend
\pstart Pas de nouveau § BQuant il 
   orent cel'orent entendu, si 
   alaala maintenant la novelle parmi Rome, 
   et i ot une si tres perfaite joie que anui seroit du raconter. 
   PourEn cele joie fist l’empereris 
   apareillier la cité pour recevoir 
   son seingneur, si le fist ferefu fait 
   si noblement comque on le pooit fere. 
   Quant se vint que li emperieres et li baron 
   s’aprochierent 
   de Romela cité, 
   si fu l’empereris aprestee, o lui grant plenté de dames et de damoiselles. 
   Lors fu mise sus i tel palefroi comque il 
   apartenoitcovenoit a 
   tel dameli et fu adestree de maint vaillant prince, 
   et tout ainsi se partirent tout conmunement hors de la ville. Si encontrerent 
   l’empereeur, o lui grant plenté de gent et 
   de barons. Nouveau § BAdonc s’asemblerent 
   l’empereeur et l’empereris et conjouirent li uns l’autre 
   en tele maniere que tout conmunement en orent grant joie, et mervelieusement en 
   fufurent loez li uns et li autres de leur noble contenance. 
   Mes il ne me plest ore mie que j’en face autre devise, fors tant que il vindrent a la grant eglise de 
   Rome, et furent li empereres et 
   li empereris reconsilié du pape et des cardonnaus en tele maniere 
   quecomme il cuidierent que Reson l’aportast. Aprés s’en reperierent ou 
   Palais Majour, qui touz fu pourtendus de draps d’or et de moult d’autres aournemens plus chiers, 
   et tout ainsi le firentfirent il, par coi on puet savoir que le prince devant 
   nonmé furent pourveupourveu de l'empereris chascun de divers 
   ascesmemensaournemens 
      etde leur armes et por eulz et pour leurs chevaliers, 
   si que ce fu la chose de quoi li emperieres fu plus joians et en 
   sot gregneur gré a l’empereris. \pend
\pstart Pas de nouveau § BLi baron, d’autre part, furent prié des damoisiaus de 
   Rome queprincipaument que il ne lessassent 
   pour nul coustement a faire une feste qui a honneur leurde chevalerie apartenist, 
   car bien seust chascuns se pour avoir ou pour tresor peust on avoir mis guerre a fin dont i eussent la leur mise et assouvie, 
   mes il virent bien aus amis quequi leur contrepartie avoit que 
   il n’i eust mestier. Et pour ce fu bien dit cist proverbes que 
   mieux vaut amis que autre tresorLégère variante du proverbe
   "Muez vaut amis an place que argent an borse." (Morawski, 1240).. 
   Dont Pelyarmenus dist a ceste fois 
   au noble conte de Flandres :
   Sire, de cest tresor que 
      nostrevostre chiere mere a assemblé 
      est en vostre conmandementcommandement pour fere une noble 
      feste parpour coi nul de nous ne 
      puistpeust entrer en nule male couvoitise aprés ce que vous serez partis 
      de nous. \pend
\pstart Quant li quens ot entendu 
   Pelyarmenus, si le prisa moult en son cuer et li dist : 
   Pelyarmenus, 
      orencore voi je bien que la feste 
      sera dene puet demourer sans grant coust.
   A cest mot fetfufist l’yaue 
   cornercornee, et 
   alors donnerent l’yauelaverent. 
   Et quant il orent lavé, si 
   se vont asseoirs'assistrent au mengier, 
   et furentfu chascun servis a son devis si noblement que 
      grans anuis seroit du raconterraconter touz les mes qu’il orent. 
   Que vous feroiediroie je 
      lonc sermon ? Li jours passa, et vint au soir que chascun 
   se trait a son repos. 
   Li empereres et l’empereris 
   s’entracointierentse furent entracointié, 
   conme cil qui autres fois l’avoient fait, si furent toute la nuitiee en soulas et en joie, conme ceus qui pou dormirent cele nuit. \pend
\pstart Pas de nouveau § X2Quant ce vint au matin, que li empereres fu levez, 
   si issi hors desde sa chambres et trouva que iii chevaliers estoient 
   en la sale venu etet cil estoient armez de toutes armes fors de hyaumes et 
   d’espees. Si tenoient par la main i enfant de x ans, et 
   cil estoit garnis de toutes biautez.
   Et sitost comque il virent l’empereeur, 
   il le connurent. Lors vindrent par devant 
   luilui et li presenterent l'enfant en tele maniere, si 
   li distrent :
   Sire, a vous nous envoie la damoisele du 
      Chastel Mignot et vousa vous 
      fet par nous asavoir son derrain salut, et si vous envoie autant de joie conme de 
      vous li estoit remez. Or est ainsi que Deux a fait de lui sa volenté. 
      Si vous mande par nous que vous priez a Dieu pour 
         l’ame‹d›[l]’ame de li, 
      car elle est de ce siecle departie pour l’amour de vous. 
      Si vous prie que vous penssez de l’enfant et que vous ne souffrez 
         que la terre soit mise a gast, car ele en meta geté sus vous 
      toute la painne et le travail.Il s'agit donc du fils bâtard, fruit de l'union de Cassidorus 
         avec Celidoine au château Mignot, auquel il était resté sous les charmes de Celidoine avant d'en être libéré par Clyodorus. 
         Voir §295-316. \pend
\pstart Quant ce vit li emperieres 
   et oï, 
   touz li cuers li atendriaatendria
   et prist l’enfant entre ses bras. Lors 
   ne se pot tenir que il ne le baisast en lermes et em pleurs. 
   Dont apela Daphus et li dist : 
   Amis, faites ses chevaliers 
      aaisieraaisier et estre a 
         salor volenté, et puis si repairiez a moi, 
      car j’ai de vous a faire.
   Lors se leva li emperieres 
   a toutentre ses bras l’enfant 
   et s’en entra es chambres et vint la ou l’empereris
   estoit et li dist :
   Dame, encoreoncore (sic) est 
      cist enfés miens,
      dont je vous prie que vous le faciez garder conme le vostre.
   Quant la dame oï ce, si sailli sus et prist l’enfant entre ses bras 
   et le baisa moult amiablement et puis dit :
   Sire, je vueil bien croire que cist enfés soit vostres, car nul n’en 
      avez qui miex vous resemble. Et je vous pri mes qu’il ne vous anuit que vous me dites dont il vient, car moult le couvoite a savoir.
   Adonc li distdist a briez paroles 
   li empereres, conme cilz a qui ilqui 
   enl'en avoit autre fois parlé.
   Et quant ele sot ce, si fu moult joians pour ii choses : la premiere, pource qu’ele tenoit 
   l’enfant ; la seconde, pource que la damoiselle 
   estoit trespassee de ce siecle, par quoi ele doutast qu’ele perdist l’empereeur et 
   que il ralast a lui. \pend
\pstart Ainsi fu l’empereris joiant de ceste chose et 
   tint l’enfant en moult grant chierté. Daphus revint a 
   l’emperere et li dist :
   Par foy, sire, voirs est 
      ce dist 
      CelidoineCassidore
      ### Confusion de prénom assez difficile à comprendre, puisque c'est bien
         aux adieux de Celidoine, et non de Cassidore, à Cassidorus que le texte renvoie (voir §321). 
         Preuve de proximité de X2 avec V3 ?, 
   quant elle dist que jamés ne vous verroit ne vous li.Daphus, dist il, voirement le dist ele. Mes 
      orce me dites : conment dient li chevalier qu’ele morut ?Sire, dist il, n’estil n'est 
      pas grant besoing que vous le sachiez selonc ce que j’ai entendu.Par mon chief, dist li empereres, savoir le vueil, 
      ou je ne serai jamés a aise dusqu’a tant que je le sache ! Car se vous 
      m’amastesamez 
      onques de rienstant, 
      quesi vous me 
      ditesdiez conment Dyane le fet, 
   car bien sai que vous en avez oïes nouvelles !Sire, dist il, Dyane ne 
      le fet se bien non, conme cele qui vous mande moult d’amistiez et 
      de salus et moult volentiers vous verroit se estre pooit.Par Dieufoi, 
      distce respontfait 
      li emperieres, se feroie je li, mes 
      je vous pri tant come je puis, se vous m’amastes onques de riens, ne par la foy que vous m’i devez, 
         que vous mem'y 
      contezdites de 
      Celidoyne, car j’en ai le cuer molt dolent.Sire, distce dist 
      Daphus, verité fu que, ainsi com vous eustes otroié la requeste de vos enfanz, 
      et meismentmeimes de l’empereris, 
      que vous eustes fet l’acordance, car 
      Celydoineelle 
      le sot par art, 
      conme cele qui plus en savoit que touz les astronomienstoutes celes 
      qui a son temps estoientdu monde.
      La magie de Celidoine, qui lui avait également permis
         de garder auprès d'elle Cassidorus, recherché dans tout l'empire avec les tracas politiques que cela a causés, 
         est ainsi assumée ici au moment d'expliquer son trépas.
      Et tantost conme elleelle le sot, elle acoucha malade 
      de duel et ne vesqui que iii jours, 
      et conmanda et encharga a sa gent que il vous deissentsi a fait le mandement 
      que elle mandoit, ainsi com vous 
         le poez entendre 
      se vous voulez. 
      Et moi me 
      couvientcouvient il mettre au retour, 
      car Dyane 
      me mande que, tout aussi tost conmeque j’avrai 
         oïoï les nouvelles de lui, se je ne 
      me mes au retour que, 
      elle ne 
      me 
      cuidecuide pas que ne me doive jamés veoir. Si vueil a vous prendre congié, 
      conme cil dontde qui vous n’avez pas moult a faire, 
         pour coi vous ne vueilliez que 
         jeje ne trespasse ce conmandement et ceste requeste.Par Dieu, dist li emperieres, 
      
         vous 
         avez ditdites verité, 
      car ma volenté est que vous ne 
         partiezpartez mie de moi. Mais puisque ainssi est que vostre volenté si est d’aler, 
         je vous pri que vous demouriez jusqu’a tantsi vueil que vous repairiez quant il vous plaira, mais 
      que la feste soit passee.
   Sire, se respontdist 
      Daphus, je ferai vostre volenté.
   Ainsi demora Daphus par le conmandement de l’empereeur 
      jusqu’a tant que la feste fu passee et, aprés ce que ele fu passee, il vint a 
         l’empereeur et 
      lia l'empereeur 
      demandademanda Daphus congié, 
      et l’emperiere li donna mout bonnement. 
      Mes ainçoisainsi li pria et requist que 
   il se preistsoiez garde de la terre de 
    Celydoineet de mon filz,
       car je le vueil que vous en soiez garde conme de la 
          vostre, et je vous en 
       donrai mes chartres et mes seausmon seel.Sire, dist Daphus, je ferai du tout vostre volenté.
   Atant fu heure d’oïr messe, si que, aprés 
   ce que la messe 
      fu chantee, il i en ot 
   aucunsli pluseur qui se vodrent entremettre de chevalerie. 
   Et avoient fet fichier grans estaches aussi conme terraches pour eulz esbanier, 
   et li autre avoient fet drecier quintainnes pour jouster, 
   et li auquant avoient entrepris joustes a 
   chevalchevaus perdre ou 
   a chevalchevaus gaaingnier. \pend
         
         
            Ci devise conment 
               les barons de Rome 
               et cil de Coustentinoble firent 
               feste de l’empereeur pour l’amour de 
               l’empereris, qu’il avoit reprise, 
                  si qu’il en firent telle feste que elle dura xv 
                  jorsjors sans faillir
                  quant il reprist sa femme.
            
               Enluminure sur 1 colonne (c) et 12 UR + un personnage dans la marge hors cadre. 
                  Fête des barons de Rome et 
                  Constantinople 
                  pour le re-mariage de Cassidorus 
                  et Fastige. 
                  Ils se tiennent la main avec 
                  deux autres personnages féminins derrière et 
                  un musicien de l’autre côté joue un instrument à cordes avec archet hors cadre.
               
            
\pstart Ci endroitOr dit li contes que 
   DorusPorus, qui moult avoit le cuer a ce que on parlast de lui en aucune chevalerie, 
   atissoit l’afaire d’issir aus chans, si que tuit cil qui aucune chose voudront faire d'armes pristrent une souppe en vin. 
   Et aprésaprés ce se sont 
   fet‹p›[f]et 
   tosttout
   apresterapresté pour issir aus chans. 
   Les dames de Rome, ou il en avoit a celui temps plus de belles 
   qu’il n’avoit en tout leou remenant 
   de Rome ne de l’empire, 
   furent requises que elles ississent hors avec 
   l’empereeur et l’empereris aus chans en leurs chars, 
   dont i ot tel bruit et tel frainte qu’il ne fust nul qui a merveilles ne 
   leles tenist les
   grans honneurs des nobles chars de Rome. Dont il avint que 
   l’emperereempereris avoit 
   fet faire v chevalierschars touz garnis d’orfaverie, 
   et furent atournez de pierres precieuses et couvers de fins dras 
   d’oror finement doublez dedens et dehors, 
   de quoi il avint que, a cele journee, que 
   l’empereris de Rome si ne volt issir aus 
   chans aussi conme por parolles. Mes Nera, qui fesoit quanque elle cuidoit que bon fust, 
   la fist entrer, o lui en son char, aussi com se nul ne 
   le seust qui ele fust. 
   Et ses autres ii sereurs se mistrent en leurs chars et les autres ii demourerent pource 
   que la royneroyne cele 
      d’Arragon n’i fu pas, 
   pource que on cuidoit que elle i deust estre. Et l’autre demoura pource que il n’aferoit pas que 
   l’empereris i fust a cele journee. \pend
\pstart Lors fu l’eure que toute 
   Rome aussi conmunement issirent cil dedens la cité aus champs. Qui dont 
   veist conment les princes desus dis s’estoient mis chascuns d’une part por 
   soisoi miex moustrer, dire peust que nule autre chose n’estoit si belle a veoir. 
   Pelyarmenus, qui pas ne se volt mettre arriere, se mist premierement hors de 
   la cité, o lui maint noble chevalier. Aprés se mist Dorus, son frere, 
   qui n’avoit mie de science mains, ainzainsi 
   s’en issirent du tout aus champs, et aprés les dames, 
               qui fesoient moult grant bruit. Pelyarmenus et Dorus 
   avoient plusseurs rens aprestez, car li uns se devoient essaieressaier a ferir 
   en la quintainne, li autre a ferir de dars en l’uneestache et li autre a jouster 
   de droit d’armes pour cheval perdre et pour cheval gaaingnierle cheval. 
   Mes sus touz sesles esbatemens, 
   Pelyarmenus et Dorus
   s’acorderent a fere a l’endemain que li iiii frere par acort tournoieroient contre l’autre partie.
   Ceste nouvelle vint a l’emperere, qui moult s’en esjoï, si dist entre ses dens :
   Je aie mal dehait se je ne le vodroie volentiers ! Mes que je ne me doutasse 
      d’autruiaucun meschief qui avenir i pourroit.
   Ainsi conme il le penssa, ausi le dist 
   le duc de Borleus.
   Sire, que dites vous de ses damoissiaus qui a ce se sont acordé ?Sire dux, dist il, je me douteroie qu’il n’en avenist el que bien, car je sai si l’une partie et 
      l’autre plainne de chevalerie que moult y avroit fet d’armes avant que 
      l’une partie et l’autre i preist se pou non.Vous avez dit 
      voirverité, sire, 
      distfait li dus. \pend
\pstart Que vous iroie je plus disant de ceste chose ? 
   IlQuar il n’i avoit nul 
   quiqui ne 
   s’s'yacordast 
   autrement
   qu’il ne couvenist que leauque le tournoiement 
   feristne ferist,
   mes il n’estoit pas encore ordené. Et de l’autre part si veoient devant eulz le grant esbatement dont j’ai 
      devantdessus parlé. 
   Et de l’autre part les dames s’estoient mises es rens poura veoir ceus qui 
   miex le feroient, dont il y avoit ja maint cop feru et d’uns et d’autres, dont li contes ne fet pas mencion. 
   Pelyarmenus, a ce qu’ilqui avoit mise 
   s’entente et qu'ila ce qu'il avoit 
   et temps etveu leus
   de faire, s’est si habandonnez au ferir en la quintainne, 
   mes tout aussi conme aucun i avoit failli li couvint faire faute, si en fu trop courrociez, car il en fu escharni tout aussi 
   conme avoient esté li autre. \pend
\pstart Pas de nouveau § BAprés Pelyarmenus 
   i mist Japhus s’entente, mes se fu pour noient.
   Et tout aussi fist Josyas d’Espaingne
   pour porter compaingnie a Pelyarmenus et pour donner a 
   Leus l’onneur de ceste emprise. Dont il en avint que
   Borleus, son pere, si li vint au devant et li dist :
   Biau filz, vous n’avez rienz fet se vous encore n’i ferez, car chascun dist 
      quequ‹i›[e] se fu aussi conme par mescheance 
         de la quintainne que vous l’assenastes. 
      Mes faites que vous i puissiez ausi droit ferirreferir 
      encore com vous avez fait. Adonc dirons nous que vous en avez l’onneur aportee.
   Et que fist Leus ? Il 
   demandadenmanda le plus fort 
   espie que on pot trouver, et li dux, son pere, vint a lui, si li dist :
   Biaus amisfix, ne metez ore si grant 
      painne en la quintainne mettre a terre que vous i failliez, je vous em pri.Biau pere, dist il, ne vous en doutez ja. \pend
\pstart Pas de nouveau § BAtant 
   fuli fu 
   a Leus une lance 
   aporteebaillee qui par semblant li deust estre trop fort, mes il iert entalentez 
   de fereparfaire ce que il 
   otavoit em propos.
   Si prist la lance voiant maint baron, qui tout mistrent leur entente a veoir conment il se maintendroit. 
   Aprés s’est aprestez et se molla en ses armes etet monta 
   elou cheval au plus affaitieement qu’il 
   onques pot, conme cil qui bien savoit que touz et toutes le regardoient, 
   et il se maintint si noblement qu’il ot la grace de touz. Lors se mist au ferir des esperons petit et petit, 
   et le cheval l’emportoit sus frain de merveilleuse maniere, que tuit cil qui l’esgardoient en orent grant joie. 
   Il feri en la quintainne de si grant aïr qu’il abatiil l'abati 
   la quintainne tout en i mont. Et au passer outrebrisa sa lance, 
   si que li tronçontrouçon (sic) en volerent en haut. 
   Mes puis li li avintli avint li-avint
   il grant meschief, car nous trouvons ou conte que 
   il couvint Leus morir de la bleceure que il reçut au cop ferir, car avant qu’il peust estre ramenez en 
   la cité cheÿ il mort de son cheval. Dont il avint que, quant 
   li dux, ses peres, vit ce, sique il dist a
   l’empereeur :
   Ha ! sire, ci n’a nul recouvrier. Et qui d’un donmage feroit ii, 
      tant vaudroit pis !Cette expression sentencieuse n'est pas répertoriée par Morawski ni par 
         Le Roux de Lincy. PD stp vérifie :)Sire dux, dist li empereres, moult parlez sagement.
      Or gardez que vostre parolle ait fruit par quoi je vous tiengne a sage.Sire, ce dist 
      li dux, poi voit on d’onmes qui doivent estre ausi courrouciez conme je sui, qui 
      puissepuissent dire sens ne fere. 
      Mes tout si courrouciez conme je sui, 
      je loeroie quepour le miex 
      fust que nul autre que vous ne seust huimés ce meschief ne demain, car de ce 
      donmage pourrons nous greingneur avoir qu’il ne le voudroit fairefaire savoir 
      em point et en leu, car je sui touz certainscertains que, 
      se madame vostre fille le savoit, 
      qu’ele en avroit plus grant duel que je n’aie encore euque je de mon filz 
         n'ai eu que je n'atendisse de lui.Par mon chief, dist li empereres, je 
      vousvous en tieng a sage.
   ###La dissimulation de la mort de Leus, avant tout à l'intention de Cassidore, son épouse,
   tend probablement à dénoter la qualité de Bourleus, perspicace - puisque personne d'autre ne semble s'apercevoir de cette blessure
   fatale, mais aussi très altruiste envers sa belle-fille qu'il avait secourue enfant. \pend
\pstart Ainsi avint de ceste chose que on desfendi sus cors 
   et sus avoir et sus quanque on pot mesfaire a ceus qui ceste chose sorent, 
   que il ne fust nesun si hardi qui descouvrist 
   cestce cest afaire en lieu 
   ou plus de gent le seussent. Lors fu moult soutivement apareilliez et ensevelis en i hostel dedenz 
   Rome pour miex celer l’afaire, por laquele reson il me covient venir a ce 
   que, quant LeusLeus qui 
   ainsi conme j’ai devantdessus dit ot la
   quintainnequintainne mise a terre, 
   si conme vous avez oï,
   il n’i ot nule des dames dont je aiai devant fet mencion
   quil neque fussent moult esbahies de ceste aventure. 
   Nera, qui sus toutes avoit le parler, embraça Cassydore, 
   qui fenme estoit Leus, et dist :
   Conment, suer ! Moult avez hui faite mate chiere ! Or la faites bonne et
      si aiez joie en vous quequar de plus noble vassal 
      ne puet estrene sai jen'ai je je dame honnoree 
      quecomme vous estes aujourd’ui.Ha ! tres chiere dame, dist elle, merci. Or a primes me semble que je aie la pointe 
      d’un coustel qui au cuer me doie ferir. Cette forme de clairvoyance de Cassidore 
      paraît valoriser l'amour qu'elle porte à feu son époux, Leus, qu'elle continuera en effet à chérir, s'opposant de toutes ses
      forces au mariage qui lui sera finalement imposé dans la suite de ses aventures. Sa fidélité à la mémoire de Leus sera cependant
      assurée par la conversion de son nouvel époux, Raimfort, qui accède à sa demande de garder leur union platonique.
      Voir le Roman de Kanor, chapitre 24.
   Lors n’i ot nule d’elles quil ne le tenist a grant folie, et dist chascune :
   Je ne sai si grant courrouz en moi que, s’il estoit avenu a mon seingneur tel honneur 
      quilcomme il est faite 
      au vostrea vous et a vostre seigneur, que je ne feisse 
      toute autre chiere que nous ne veons que vous faitesfaciez.
   Lors se reprist a soi meismes et sot bien que ele mesprenoit selonc l’avis de 
   chascun et de chascune. Mes li cuers, qui par nature ne se pooit esjoïr, atendoit ce que Nostre Sire li avoit pourveu.
   Et lors se cuida esjoïrefforcier de joie fere et dist :
   Et ou est ore celui qui ceste proesce a faite ? Ne vendra il pas avant ?
   Nouveau § BPar foy, dist chascune, de ce avons nous grant merveille, 
      car il n’est pas usage que, 
      quant on fet tel chose que on tient a bien faire et a proesce, 
      que cil ne se doiventdoive moustrer devant les dames.
   A ce mot est venus le roy d’Arragon 
   et autres barons qui furent avisiez, et distrent :
   Ceste quintainne est mise a mort. Or avant, dames, a une autre !
   Lors furentfurent d'iluec menees a i 
   autreautre la 
   ou il i avoitavoit eu maint biau cop feru,
   mes si fort estoit que nul ne la pooit mettre a terre. Quant Pelyarmenus s’aati de lui verser 
   que qu’il en deust avenir, a ce cop fu li emperieres reperiez et vit que
   Pelyarmenus s’aprestoit de ferir en la quintainne. Si saisi l’espie la ou il le tenoit et dist :
   Biau filz, n’est mie grant mestier d’entreprendre chose dont 
      onvous et autres 
      viengneveingniez tart au repentirCette 
      expression sentencieuse n'est pas répertoriée chez Morawski ni chez Le Roux de Lincy. PD stp :) ! Et si n’i ait huimés si hardi qui se
      mette en painne de mettre 
      la quintainne a terre, car je ne vueil pas qu’il i ait honme quil soit pareil a Leus,
      mon chier filz, qui le los et le pris doit avoir des quintainnes aujourd’ui. \pend
\pstart Quant Pelyarmenus oï ce, si fu touz courrouciez, 
   car bien cuidoit acomplir ce qu’il cuidoitvoloit entreprendre, mes il n’en osa 
   aler plus avant. Adonc vint li empereres devant les dames et les 
   damoisellespucelles, qui toutes 
   demandentdemanderent a i ton :
   Sire, que avez vous fet de nostre meilleur chevalier qui a ce fait que nul autre ne pooit fere ?Par la foy que je doi a vousvous toutes, 
      je l’ai fet mettre en tel lieu ou nule de vous ne le verra huimés ne demain, car mal 
      am'a mise mala 
      quintainne a terre sanz mon gré.
   Quant ce ont entendu, les dames si en ont eu grant joie, car elles cuidierent que ce feust ou pour joie ou pour feste, 
   si conme il leur donna a entendre. Et ce fu une tres bele voie pour couvrir ce qu’il avoient 
   entreprisdit. \pend
\pstart Pas de nouveau § BAinsi conmença la feste et l’esbatement a demorer en sa vertu, et les dames furent conmandees a mener en 
   i renc qui fu atirez pour jouster. Si avint que 
   DaphusDaphus le grisois 
   et JosyasJosyas d'Espaigne 
   devoient par acort jouster li uns a l’autre. Et lors vint li emperieres et leur dist :
   Biau seingneur, n’i ait nul de vous, se il ne veult perdre l’amour de moi,
      qui s’abandongne a nule vilainne empriseentreprise ! 
      Mes chevauchiez sagement, car on doit trop douter les perilz.
   Lors n’i ot nul quil ne s’apareillast a fere sa volenté. Et qu’en avint il ? 
   Il brisierent si bien et si bel 
   leur lances que ce fu moult grant joie du veoir et
   greingneurgrant deduit du raconter qui lieu et temps en avroit. 
   Mes tout ausi com chascun n’est pas bien disposez a oïr volentiers ce que on doit et puet briefment conter, 
   m’estuet il venir a ce qu’il m’est avis que on n’en puet pas si corrompre la matire,
   que cil qui volentiers loentl'oient ne m’en puissent blasmer, 
   pour laquele raison je vueil a ce venir que 
   Dorus s’en vint a l’empereeur, son pere, et li dist :
   Sire, il me semble que vous avez aussi conme desffendu que nul ne jouste a moi.Biau filz, sauve soit vostre grace, ains vous 
      sentset chascun si plains  de vostre volenté que il vous redoutent 
      en toutes manieres.Et conment ? dist il. Ne demeure il pour autre chose ?Ainsi le cuidoiecuide je, dist 
      li empereres.Par mon chief, dist il, je n’i ferrai mes hui cop d’espie ne de lance !Vien avant, folsot ! 
      dit li empereres. Veuls tu fere chose dont on parolle em bien et dont tu 
      seroiesseras plus honnorez que se tu abatoies toutes les quintainnes de ceste 
      place ?Sire, dist il, or me lessiez dont oïr que c’est.Volentiers, dist 
      li empereresil.
      Leus, qui sereur vous avez, si est i poi bleciez, 
      par quoi il ne puet pas a la journee de demain porter armes. Si vous loeroie que vous feussiez armez de ses armes et la montrissiez 
      conbien vostre valeur se puet estendre.Sire, dist il, de ce que Leus, mon frere, est bleciez me 
      doitdoit il moult anuier !
      Mais de ce que je serésoie 
      demain atournezau tournoi pour lui me ferai je a merveille joiant !
   Et lors vindrentvint Helcanus et
   li contes de Flandres a 
   l’emperere et li distrent :
   Sire, que avez vous fet de Leus ?Biau seigneurs, dist il, 
      je en ai fet ce que vous en poez oïr.Sire, distdist dont 
      Helcanus, nous nous doutons qu’il ne soit bleciez.Biau filz, dist li empereres,
      or ne vous en doutez ja, car il 
      estest encore tout autrement que vous ne cuidiez. 
      Et si ne m’en enquerez huimés autrementautrement de lui que j’ai dit a 
      Dorus, vostre frere.
   Adonc s’est traitse trait 
   li quens de Flandres 
   vers l’empereeur et li dist :
   SireSire, dist il, de ceste journee 
      ne puis je huimés veoir qu’il ne soit 
      bien heure de reperier en la cité, selonc ce que je entens, a demain le tournoy. \pend
\pstart Sire quens, dist li empereres, 
   or avez voir dist. Adonc fist li empereres sonner la retraite en
   la cité, et il tot aussi ordeneement 
   conme il en issirent 
   rentrerentsont rentré il 
   en la villey. 
   Il fu temps et heure que vespres durent sonner, et on si fist, et puis les sont alees oïr 
   li emperere et li baron qui aprés l’esbatement vindrent tout a court. Et 
   aprésaprés ce si fu heure de souper, et on si fist plus tost que on ne souloit 
   pource que li plusseur avoient pou mengié tout le jour. \pend
\pstart Pas de nouveau § BQui dont veist conment li keu s’estoient pené 
   et traveillié du souper noblement 
   appareillieraprester, dire peust que aussi bien se penassent d’eulz faire loer
   conme li chevalier se penassent de chevalerie fere. Ainsi avint que il firent 
   l’yaue corner. Lors laverentlava 
   li empereres et l’empereris
   et puis tuit li baron qui aprés l’empereeur sont assis tout a leur droit. 
   Cil qui de servir se durent entremettre le firent si ordeneement que moult fu 
   belbele chose a veoir de ceus qui de tel mestier se connoissoient. 
   Et sanz faille, qui que le tiengne a truffle, bele chose est en hostel de grant seingneur de servir ordeneement selonc ce que chascun vaut. \pend
\pstart A cele nuit ne fu nuls botez hors puisqu’il fust 
   assisentrez dedens. La ne fist on nului lever du mengier puis que il fu assis. 
   La peust on veoirsavoir toute honneur et toute courtoisie d’ostel a maintenir. 
   Cilz soupers dura moult longuement, et fu parlé du tournoiement sus table a l’endemain. 
   Mes je me vueil atant partir de cest souper, si vueil parler des dames,
   qui se sont mises aus karolles aprés ce que les tables furent ostees. Nera et 
   Cassydore en sont venues 
   devanta 
   l’empereeur et le mistrent a reson que il avoit fet de Leus.
   Li empereres se 
   pourpenssapourpenssa a ce et dist :
   Mes belles filles, et li une et li autre, 
      il est usage en hostel de grant seingnor, si conme il doit estre en hostel 
      d’empereeur de Rome, seque quant 
      il y a i chevalier quel que il soit et il fet ce que Leus a fait,
      que il doit estre mis d’une part tant que reson en soit faite aussi conme je 
      vous avoie hui dit.Sire, distrent elles, et nous ne cuidons pas que Leus
      ait fait chose dont il doie blasme recevoir.Par mon chief, dist li emperieres, je ne cuit mie,
      se il avoitil l'avoit fet 
      chose
      qui bonne ne fust, 
      que je n’en feusse aussi courrouciez conme nule de vous seroit. Mes vous savez bien qu’il couvient tenir usage de pais.
   Atant se dreça li empereres pource que il ne volt mie
   que elles le tenissent plus a plait de ceste chose.
   SiSi en prist l’une a une main 
   et l’autre a l’autre et s’en vint aus querolles et mist painne a ce que il 
   et ses ii filles deissent icest rondel :
   Moult vaut miex amener joie que estre trop 
   sus pensersouplet. \pend
\pstart Quant li emperieres et ses ii filles 
   orent leur chançon finee, si i en ot moult qui puis se penerent de chançons dire et de joie fere,
   et meismement l’empereris fu moult courte tenue de chanter,
   ainz ne se volt fere tenir ne a folle ne a nice, car ele priapria a ses 
   ii autres filles, qui furent fames aus 
   ii damoisiaus de Rome 
   dont ele estoit mere, que elles deissent ceste chançon :
   Ne se doit nuls desconforter qui a vraie amour s’atent. 
      Rondel, je te vueil noter, d’autre chançon que raconter,
      conment amours veulent montermoter en haut touz ceus 
      qui sont de leur couvent.
      Ne se doit nuls desconforterdesconter 
      qui a vraie amour s’atentetcetera.
   ### Y a-t-il un message à tirer de ce rondeau chanté, à la demande de Fastige, par ses deux belles-filles,
      les épouses des héritiers du trône de Rome, à la fête de ces noces renouvelées avec Cassidorus ? \pend
\pstart Quant li empereres et ses ii filles
   orent alé lor tourtour, si fu heure de reposer selonc ce qu’il orent a faire
   a l’endemain. Atant se sont li baron trait a l’ostelas hosteus 
   et pristrent repos, li uns plus, les autres mains. Dont il avint que Dorus ot
   moult a pensser que il peust a l’endemain fere chose dont li emperieres
   se peust tenir a paié. Il avoit fet les chevaliers Leus venir devant lui 
   etet il leur dist 
   qu’commentil vouloit armes porter 
      en la semblance de Leus, leurson seingneur,
      et qu’il ne s’esbahisissent mie, car il ne descroistroit pas son pris,
      mes qu’il i deust metre la vie se il i vouloientvouloi‹t›[ent] aidier.
   Il respondirent que de ce ne s’esmaiast pas, car il li seroient 
      en aide et menu et pres, 
      mes bien gardast que il fust bien montez a sa guise.
   Il leur respondi que de ce ne se doutoit il mie, car il avoit tel cheval qui cuidoit 
      bienbien qu'en tout le mont n'eust son pareil et 
      quequ'en tout le jour ne li 
      fausistfaudroit il mie se il ne li failloit.
   ### Réflexion sur cette forme d'incognito de Dorus, sous les traits d'un chevalier chéri décédé ? \pend
\pstart Pas de nouveau § BCeste parolle oï i 
   Henuier,
   qui moult prisa le mot que Dorus dist, et dist :
   Sire, se je estoie aussi riche honme de valeur conme vous estes,
      moult ameroie i tel cheval.
   Lors esgardaesgarda celui Dorus, 
   si le vit aournez de toutes biautez et de joenne aage, si li dist :
   Et je vueill que vous l’aiez, 
      et puis sime direz qui vous estes.
   Adonc ne fu pas celui esbahis, ainz se trait vers lui et li dist :
   Ha ! sire, la vostre franchise ne se puet couvrir,
      mes ma folle couvoitise m’a descouvert a ce que vous avez oï.BienDieu oï, 
      dist Dorus, 
      vostrea vostre raison que vous n’estes pas du tout a aprendre. 
      Mes vostre non vueilvueil je savoir et en quel lieu vous fustes nés.Sire, dist il, je fui nés en la marche de Henaut, filz a i 
      vavassour de povre renon et d’assez petit afaire. Et je meismes sui apelez Henaut.Henaut, dist Dorus,
      je vous pri que vous me soiez demain pres au conmencier du tournoy.
   Lors lel'en volt il aler au pié 
   baisier quant Dorus l’embraça et li dist :
   Alez huimés a l’ostel, car aussi est il temps.
   Adont prist chascun congié et sont atant departi. \pend
\pstart LendemainA lendemain ot grant bruit aval 
   Rome, car chascun s’aprestoit de ce qu’il pooit selonc ce que il estoit.
   Mes je ne puis ore pas de chascun parler ne raconter son affaire, 
      car ce n’est pas droit, ne reson ne l’enseingne mie. 
      Si vueil venir au duc Borleus et a l’empereeur, 
   quil ne s’appareillierent pas de porter armes a cele jornee. Lors firent venir devant eulz touz les grans barons
   qui armes durent porter a celui jour. Si en y ot encore assez de ceus 
   dont je n’ai faite nule mencion,
   car il vint nouvelles que i damoisel de Puille
   qui avoit esté filz au noble Synador, qui jadis fu compains d’armes au
   bon seneschal de Rome qui Marques fu apelez
   – cilz damoisiaus fu apelez Galyens –, et i autre, qui estoit
   filz au prince de Calabre, qui avoit a non Percemont,
   cil dui avoient esté nourri ensemble, car il estoient cousin et s’entramoient moult, conme cil qui n’estoient pas encore chevalier. 
   Mes il avoient en propos et en covent li uns a l’autre
   que il ne seroient ja chevalier se ledu meilleur des meilleurs 
   ne les fesoit,
   et il avoient tant enquis et demandé que on leur avoit fet asavoir que
   li emperieres de Constentinonble
   avoit le non desus dit. Dont il avint que li tournoiemens fu respitiez de ci a l’endemain,
   que cilz i porroient estre plus couvenablement. \pend
\pstart Li empereres et li dux
   Borleus, quil ne porent pas leur anui couvrir si bien conme il vousissent,
   orent conseil que il metroient Leus en terre au plus priveement que il pourroient.
   Dont il avint parpour 
   la fille a l’empereeur,
      qui sa fenme estoit, que on feist entendant que ce estoit i povre chevalier qui estoit
   au conte de Flandres.
   Et pour lui honorer, on en faisoit la greingneur sollempnité que on pooit,
   si que il fu mis devant l’autel saint Michiel 
   en la grant eglisse de Rome,
   et la le pueent encore veoir tout cil qui garde s’en veulent donner. \pend
\pstart Pas de nouveau § BAinsi avint que ceste chose fu celee,
   pource qu’il couvient a la fois couvrir son anui au plus bel que on puet
      quant on ne puet avoir autre recouvrier que il ot a cestuiCette expression sentencieuse n'est 
      pas répertoriée comme un proverbe par Morawski ni par Le Roux de Lincy, mais il s'agit d'une formule révélatrice de l'enjeu de
      la dissimulation, du contrôle émotionnel..
   Et la plus grant reson de cestui si fu que, quant on miremue une tres grant 
   joie en i parfet duel,
   que trop de perilz i pueent avenir, et especiaumentmeesment de ame et de cors.
   Si ne vueil pas ore ci toutes dire les resons que on i pourroit dire. Si vous sousfise atant fors que
   tant qu’il i ot encore une autre raison, qui pas ne fu petite. Car se tous seussent le meschief,
   riens il n’eust valu, ne la joie ne la feste, que on avoit 
   entreprisempris a fere 
   en l’onneur de l’emperere et de l’empereris,
   car joie faite en tristesce n’est autre chose que blanches chauces muees en noires.
   Cette expression sentencieuse n'est pas répertoriée chez Morawski ni chez Le Roux de Lincy. à vérifier PD stp.
   La formule, à portée proverbiale claire par son caractère très imagée, se veut très explicite et quelque peu familière, 
   avec un écho intéressant à la mutabilité d'apparence émotionnelle que vient notamment incarner la figure de Faux Semblant, 
   blanc au-dehors, noir au-dedans. L'orientation de la morale émotionnelle est marquée ce faisant, la simulation de joie pour 
   dissimuler la tristesse étant ainsi assimilée à des chauces blanches muées en noires.
   Et pour ceste raison et pour autres fist l’empereeur et
   le duc Borleus cest afairechose 
   ainsi couvrir conme vous poez oïr.### Réflexion sur ce raisonnement \pend
\pstart Li damoisel de Puille
   et cil de Calabre
   chevauchierent tant et a si noble compaingnie que il estoit avis a touz ceus qui les veoient que
   touz li mondes deust estre enclins a euls, car
   je truis escript que il ot en leur compaingnie xxx chevaliers, dont les
   x portoient banieres si que entr’eus menerent tel harnois que a l’entree de Rome
   leur volt on veer l’entree. QuantCar la novelle en vint a 
   Gasus, le seneschal, qui en fu trop courrouciez qu’i 
   conmanda et dist que le chief de Rome n’estoit pas esbahis por 
      la venue de tel gent. Lors entrerent ensen Romme a grant harnois, 
   conme cil qui aprés eulz faisoient traire et mener le plus 
   belgrant charroy du monde, 
   si qu’il avint que Ronmains, qui moult sont covoiteus d’avoir, distrent,
   quant il virent telete richesce : 
   De ces choses 
      nous couvient avoir ! \pend
\pstart Pas de nouveau § BLi damoisel, qui a grant bruit venoient derriere,
   ne finerent de chevauchier tant que il encontrerent les iiii filz
   l’empereeur et maint autre baron en lor compaingnie,
   qui leur firent moult grant joie et merveilleuse honneur. Il entrerent en Rome
   et furent bien regardez de touz, conme cil qui bien et bel se maintindrent, et ne descendirent de ci 
   auau perron du Palais Majour.
   Puis sont monté amont tout vi, main a main, et tout ainsi vindrent devant l’empereeur, 
   qu’il ne fu pas lens de bel 
   apelerparler 
   les damoisiaus quant il l’orent salué. \pend
\pstart Aprés ce, que vous 
   feroiediroie je lonc conte 
   de chose que li emperere leur demandast ne qu’il 
   en respondissent 
   a l’encontre de la demande ? 
   Assez le vous ai dit entendanment par devant pourquoi 
   il estoient venuvenoient 
   a la court l’empereeur. 
   Si ne furent pas aprentisa aprendre de moustrer leur besongne a 
      l’empereeur, ne li empereres du respondre. Lors 
   fufu il temps et heure de l’yaue corner, 
   si i coururentl'ont fait 
      ceus qui s’en durent entremettreet ils si firent. 
   Li empereres lava et 
   l’empereris et 
   tout li autre aprés moult ordeneement, et puis s’asistrent aus tables. 
   Si ne truis pas ou conte que, se on 
   avoitavoit par devant nul jouravoit par devant servi noblement, 
   que encor le feist on plus 
   seingneuriementde grant seignouriesoigneusement. 
   Cil qui vint a moutamont ot grant merveille aus 
   bonsmieus entendans conment on puet si grant pueple servir si 
   paisiblementpaisib-blement et a si grant aise, 
   si que cil qui bien connurent l'estre de ceus qui ce avoient faitordené 
   respondirentrespondoient :
      Biau seingneurs, par i honme est i hostel maintenu 
         en honneur et par i autre est il 
         mis a destructiondestruitCette 
            expression sentencieuse n'est pas répertoriée chez Morawski ni chez Le Roux de Lincy. à vérifier PD stp. Elle dénote
         cependant d'un jugement moral de nature proverbiale sur les mauvais effets et le mauvais gouvernement que vient incarner
         Pelyarmenus. \pend
\pstart Pas de nouveau § BVez ci Gasus, le senechal de 
   Rome, qui ne chace mie a mettre avant ses parens 
   ne ceus qui acquierent par dons les servises des grans seigneurs qui ne sont digne de servir ne eulz ne autrui. 
   Mes touz les meilleurs et les miex esprouvez que il puet avoir ne tenir, 
   ceus acointeceus-acoince ceus acointe
   il et a acointié touzjours el servise de 
   son seingneur. 
   Et par ce poez vous veoir la noblesce de la cort, ainsi com vous la demandez. 
   En ceste chose se devisoient li auquant, non mie sanz plus em boivre ne en mengier. 
   Et parpour ce est il a savoir aujourd’ui au plus gentilz honmes que il sachent 
   que uns homs destruit i hostel et i autre le maintientCette 
      expression sentencieuse, variante de la précédente, n'est pas répertoriée chez Morawski ni chez Le Roux de Lincy. à vérifier PD stp. 
      Plus encore dans ce jeu de reprise dans lequel elle s'inscrit ainsi, elle dénote bien d'une portée proverbiale.. 
   Tout ainsi fu de ceste court desus nonmee servie et honnoree de touz. 
   Si avint que, quant on otot ensi servi et les tables furent 
   ostees,ostees que
   dui menestrel furent aprestés'apresterent de vieler souz prouvinciaus 
   et en la fin distrent une chançon de court en chantant en leurs vielles de si noble maniere et 
   de si merveilleusmerveilleus bons estrument que tout cil qui les entendirent 
   n’avoient onques mes si bien oï chanter. \pend
\pstart Pas de nouveau § BAinsi 
   comque il orentavoient
   leur chançon finee et il disoient leur issue, estes vous que uns lyons 
   s’estoitestoit d'iluec desliez,
   qui avoit esté mis en i travail pour ce que il ne feist a nullui mal.
   Et lors se sont tot arriere trait pour la paour de lui.
   Li empereres, qui amoit 
   le lyon et qui l’avoit 
   fait amener de Constentinonble a Rome, 
   ainsi conme dist est en l’ystoire devant, sailli em piez et vint a la 
   court acourt aval ou il avoit i chevalier qui ne savoit pas que
   li empereres l’amast tant conme il fesoit,
   qui avoit une espee traite et maintenant gita son mantel entour son bras et
   voloitdevoit aler au lyon et
   li lyons a lui quant
   li empereres s’liescria et dit :
   Compains, compains, mal vous seroit avenuavendroit
      se vous occiez mon lyon
          !
   Lors se regardamist le lyon 
   et entendi l’empereeur et 
   vint a luil'empereour a merci grant aleure 
   et se mist a ses piez si humblement conme il pot 
   miex.miex. Nouveau § BCeste chose esgarderent
   li baron trop volentiers et li empereres s'abaissa et le prist par le chief, puis li frota les orelles et li dist aussi comme 
   il eust fait a un serjant le roy : "Je vous commant que vous a nului ne faiciez vilonnie dont vous me puissiez courecier." 
   Lors se dreça li lions aussi comme s'il eust entendu et vousist faire sa volenté et li empereres fist crier par toute la court 
   que nuls ne fust si hardis qui au lyon feist vilenie se il ne voloit perdre s'amour. De ceste chose s'esmerveillierent li plusor qui du tout
   ne savoient pas ceste aventure dont l'estore devant fait mention. Si avint que li lyons sivoit l'empereour aussi comme 
   un levrier sieut son souvrain seigneur. \pend
            \pstart AdoncCi vous vueil ore atant laissier a parler 
de cest lyon et venir a ce qu'il couvient ordener comment cis tornoiemens peust venir par acort et que l'une partie n'en eust 
si le meilleur que l'autre s'ele se vousist deffendre que il ne peussent estre foule par nule male covoitise. Adonc 
   se sont mis ensemble ceus a qui il en apartenoit. 
   Si trouvons que li iiii frere furent mis 
   d’une part ; li roys d’Arragon,
   Japhus de Frise,
   Josyas d’Espaingne et
   li quens de Flandres de l’autre. Et chascune partie si ot 
   chevaliers de grant renon qui ci endroit ne sontseront mie nonmé.
   Mes quant se vendra aus batailles du tournoy, si pourrez adonc oïr
   de ceus qui miex le firentfirent si comme ce sera raisons et drois 
   pour coi nous voulons certefier le nombre de la chevalerie
   qui a ce tornoi serasavnasara.
   Si trouvons qu’il en y ot, que riches que povres,
   xviiic et xlvi dont il en y ot plus d’une part que d’autre jusqu’a
   xxv, si que cil qui plus en orent, se fu
   le roy d’Arragon et cil de sa partie.
   Cest acortoctroi et ceste partie fu faite et confermee entr’eulz. 
   Lors approcha la nuit que
   li joenne damoisel qui chevalier devoient estre se moustramoustrent, 
   lui et son compaingnon qui devantavec 
      luilui le fu,
   par devant l’empereeur. \pend
\pstart Gasus, 
   dont nous avons devant parlé en assez bonne maniere,
   avoit i filz, qu’il amoit a merveilles pource qu’il li estoit avis qu’il 
   avoiteust biau conmencement,
   et sanz faille si avoit il, car il iert biaus et bien tailliez de touz membres et 
   joennes 
   de l’aage de xxv ans et sages pour i 
   royaumeempire maintenir, et 
   estoit moult gracieus sus touz autres.
   Icestui amena Gazus par devant l’empereeur et li dist :
   Sire, vez ci i mien filz
      que je couvoite moult que il vous puisse faire servise, et autresi a vos enfans qu’il vous puisse plaire.
      Et pour cece le vous requier je que vous le faciez chevalier a 
      la loengel'onneur de Dieu et de Sainte Eglise. \pend
\pstart Pas de nouveau § BLi empereres regarda l’enfant, 
   si vit qu’il estoit plain deque n'avoit il pas failli a 
   grant biauté et qu’il n’avoit pas failli a ses membres. 
   Et lors li demanda li empereres son non, 
   et ilcil liil lor dist 
   qu’il avoitot a non 
      Ganor.
   Ganor, dist li empereres,
      moult sui liez de la bonne requeste que 
      vostrevostre bon pere m’a faite.
   Lors dist li damoisiausGanor :
   Ja a Dieu ne 
      plaiseplace, sire, 
      quese je
      nene puisse faire chose
      ainz que je muire qui vous puisse encore aidier et valoir, 
      si vraiement que je a touzjours mes m’en voudrai mettre en painne.
   Et lors li volt aler au pié quant li empereres 
   l’enle leva et 
   distli dist :
   Amis, or soiez prest demain au matin avec les autres.
   Aprés cestuicestui en revindrent maint autre, 
   dont li contes ne fet pas orendroit mencion. Adonc fu heure de vespres, 
   que li empereres et li autre baron 
   les ont oïes apréspuis alerent souper, 
   si conme il leur couvintet furent servi si qu'il couvint. Et aprés souper,. 
   Li damoisel de Puille
   et cil de Calabre et plusseurs autres alerent 
   veillierveillier si comme il durent avec eulz 
   et maint filz de prince a l’eglise
   de Rome, qui a l’endemain 
   furent apresté.
   Quant li emperieres ot oï messe en la chapelle 
   saint Michiel 
   et les joennes damoisiaus orent veillié, 
   si leur conmanda l’empereeur 
      que il venissent par‹d›[p]ar devant lui, 
   si leura chascun 
   donroit l’ordre de chevalerie. \pend
\pstart Et quant il leur ot donné, adoncce 
   avint que li damoisel de Puille 
   et de Calabre et de mainte autre region furent venus devant 
   l’empereeur, si lor donna a touz 
   garnemensgarne‹ns›[m]ens selonc ce que chascun estoit. 
   Ici ot noble compaingnie, quantcar il furent 
   iiiixxx chevaliers nouviaus entre l’aage de 
   xxxxx ans et de trente, 
   dont le maint puissant si porta le jour baniere. Ceste belle compaingnie retint toute
   li damoisiaus de Puille et
   cil de Calabre.
   Qui dont veist conment chascun se trait a sa partie, dire peust cest proverbe qui est escript en 
   latinlatin qui dist en tele maniere :
   Similis similem sibi querit, 
   c'estc'est adonc a dire que chascun franc honme se mist a son franc 
   labour et li autre, 
   qui de ce n’avoient cure, se mistrent de l’autre part. De leur boivre ne de leur mengier ne 
      mem'en couvient fere mencion, 
      car je cuit que il furent assez largementlegierement servi, 
   car chascun ne penssoit pas a emplir son ventre
   mes a faire chose dont on peustil porroient savoir qu’il seroient serjant de Sainte Eglisse entre ceus qui adonc le vodrent fouler, 
   et tel estoit leur entendement au temps d’adonc ; je ne sai quel il est ore aujord’ui. \pend
\pstart Pour ceste chose faire asavoir, Helcanus, 
   Dorus, Pelyarmenus se firent
   moult richementrichement et moult seurement aprester 
   et leurs batailles d’autre part, ou Daphus li Gris estoit, 
   a ii banieresbanieres de ses armes 
   et x chevaliers de ses armesarmes et y estoit Mirus le Fier. 
   GazusGazus de Rome et 
   ClyodorusClyodorus de Grece 
   firent en tele maniere. 
   Aussi avoit fet chascun des iiii freres une bataille 
   avec la seueseue qui li venit au secours se mestiers fust. 
   Et pour ce ne demourroit mie qu'il n'eussent autres chevaliers sans escuiers sans lesquelz il ne peussent longues durer, et tout
   ensi ravoient cil a qui il avoient a faire.. 
   Nouveau § B
   DorusDorus Dorus avoit fet adouber 
   Karus de 
      NiseNise, le bon chevalier de ses armes, 
   et il meismes s’estoit mis es armes 
   LeusLeus, 
      son frere et avoit une bataille de .x. chevaliers esleus.
   Dont li vaillans chevaliers dist une moult fiere parolle, car il dist :
   Biau seingneurs, je vous ai esleus en l’onneur d’un mort chevalier sauver son
      non. Et vous savez que quant li homs muert, ses nons ne puet ne ne doit mourir. 
      Et pour ce vous pri je a touz ensemble que vous m’aidiezaidiez hui 
      sonle sien non a 
      retenirmetre en retenanceen retenance, 
      touttout pour ce pourrisse le 
      sien cors en terre, 
      dont Envie est joieuse de ce qu’il est trespasseztrespassez et mors.
      Qu'ele nous pardonra tous mautalens ja soit ce chose que nous vueillons le sien non essaucier. Q.SireSire, sire, dist 
      cil HanuiersHenaut
      dont j’avoie devant parlé, ne nous preeschiez mie, car nous le sonmes 
      touttout et bien devons chascun de nous savoir que nous devons faire 
      pour vostre honneur sauver et la vostre." Quant Dorus ot entendu Henaut qui ainsi avoit respondu, si li dist : .Henaut, dist 
      Dorusil,
      je nel di pour autre chosece fors que vous 
      savezsachiez que 
      Leus doit avoir aujourd’ui chevaliers 
      esleusesleus." Quant tout li autre l'orent entendu, si dist chascuns :.Sire, dist chascuns, 
      nous entendonsavons oÿ 
      moult bien a ce que vous nous avez dit.
      SiSi loons et vous requerons chascuns de nous que nous 
      soions li premier au conmencier !
   Or avint que il fu heure que on issi aus chans, par quoi il furent li premier issu. \pend
\pstart Aprés ne demoura pas que 
   Helcanus et cil de sa partie issirent moult 
   ordeneement. Li roys d’Arragon et li sien de l’autre part si 
   ne furent pas a aprendre. Si ne vous puis ore fere du tout entendre liquel furent d’une part ne liquel d’autre 
   autrement queque se
      que je desus vous ai touchié, et a vous, 
      qui savez conment tel affaire doit estre racontez, si vous en souffise ce que j’en 
      savraisavrai dire et conter. \pend
         
         
            Ci devise i tournoiement moult grant, 
               dont les uns fierent encontre une quintainne et li autre tournoient ensemble 
               a cheval perdre et a cheval gaainginergaaignier. 
                  Comme s'ensuit
               Ci parole li contes des princes qui furent au tournoiement.
               Cette rubrique, tout comme l'enluminure qui suit, est donnée deux fois dans X2
                  au f.105rb et au f.106ra, respectivement présentés sous X2A et X2B.
               Ci vient li contes a Dorus qui fu au tournoi es armes Leus et des princes et des chevaliers qui y furent.
               Ci vient li contes a Dorus qui fu au tournoy es armes de Leus avecques les princes et les chevaliers qui
               furent au tournoy. Et fu Dorus un des miex faisans..
            
            
               Enluminure sur 2 colonnes (a-b) et 13 UR.
                  Tournoi de chevaliers s’affrontant en armures et à cheval dans une grande mêlée. 
               
            
\pstart Ce paragraphe est donné deux fois dans X2, au f.105rb, de manière incomplète 
   (le texte s'interrompant à la moitié de la colonne f.105rc, sans que suite ne soit donnée sur ce folio ou son verso) 
   et au f.106ra, respectivement présentés sous X2A et X2B.
   Ci endroit dist li contes que,
   quant li prince et li baron se furent mis tout hors de 
   Rome, 
   ainsi conme il estj'ai devant dit, 
   li empereres et 
   li dux de Lambourc alerent de
   l’une partie a l’autre pour savoir l’acordement d’euls. Si ne troverent chose qui les peust empeeschier
   fors tant que moult leur demouroit ce que il n’i erent assemblé piece avoit,
   car il en y ot de telz qui jaja assez a temps 
   n’ine lene 
   cuidoientquidierent 
   estrefaire a qui que il sambla puis que la journee leur fu trop longue
      faire.
   Dont ne demoura guieres que leur batailles ne fussent ordenees. Et a qui chascun devoit assembler, 
   si vueil dire lesquelz furent li premier. Dorus, qui avoit le cuer de touz, 
   aussi conme cil quil ne se vouloit partir de nul lieunului 
   se le meillieur n’en feust sien, fu contre la partie ses freres, si fist ses banieres conduire ou premier chief de leur gent.
   Et cilz qui de l’autre part vint ou premier front si fu
   li damoisiaus de Puille,
   qui ja assez a temps ne cuidoitcuida venir aus cops donner. 
   Cilz si avoit avec lui des meilleurs chevaliers du monde.
   Lors ont cil dui prince avisié l’un l’autre. 
   Adont ferirent chevaus des esperons,
   car je truis en escript que si ataingnaument feri li uns l’autre que
   moult s’en pooit on esmerveillier. Et puis aprés s’asemblerent les autres parties ensemble.
   Qui dont veist lesmaint grans cox 
   donnerferir les uns sus les autres 
   et chevaliers abatre de chevaus et puis remonter, dont peust dire que onques mes jour de sa vie 
   n’ot veune vit si bele 
      chosechose qui fust bonne a esgarderpareille. \pend
\pstart Moult dura longuement cilz tournoiement et d’une part et d’autre,
   si que li soleus avoit ja tant alé queque il fu venus d'Orient 
   jusques en Occident et avoient tant feru les uns et les autres que mal fust de celui qui volentiers 
   nene vousist que il feust reperiez.
   Li emperieres et li dux 
   fistfirent sonner laune 
   retraiteretraite, si se commencierent a retraire petit et petit. 
   EtMais quant ce
   vitsot Dorus, 
   quiqui adonc se fu raliez a sa bataille et montez 
   estoitfu sus un fort cheval, 
   lors se feri es plus forsgrans batailles que il veoit, 
   si qu’il li avint a cele fois que si chevalier le tindrent plus pres que il n’avoient fet en tout le jour, 
   si que en la fin n’ot a qui avoir 
   l’estrif fors que au 
   conte de Flandres,
   qui estoit de sa partiepartie. A celui se prist il, mais en poi d'eure eust cousté
   l'afaire quant li empereres et li dus les departirent. Lors s'en repairierent en la cité a merveilles tart.. 
   Nouveau § BQue vous iroiediroie je 
      desoremés plus 
      alongnantautre chose disantdisant 
   que tout conmunement vindrent a la court souper ? 
   Quant il se furent retrait qui faire le volt, 
   la peust on veoir nés et sorcilz rompus et plusseurs autres bleceures qui puis furent 
   garies. Lorsgaries. Et d'autre part, moult y ot de ceulz qui les couvint demourer as hostelz
   et les couvint faire garder que il ne cheissent en pieur point. 
   Mais de ceulz ne me couvient ore pas faire lonc conte fors de venir a ce que 
   quant il orent soupé, si se mistrenttraistrent li grant seingneur d’une part 
   et enquistrent les fais de chascun, 
               si trouverent tout de conmun absens que Leus avoit le pris du miex faisant des 
               quintainnes et du tournoiement. Ceste chose fu partout seue, et distrent entr’eus que c’estoit reson 
   quecar li pluseurs en estoient certains et sages. 
   Aprés il avint de ce que il couvint a l’empereeur querre maniere de 
   ferefere entendre et savoir a touz ensemble 
   l’aventure du bon chevalier Leus. Lors n’en 
   y ot nul d’euls qui n’en eust grant pitié et quil nen'en feist 
   pour lui mate chiere. 
   Aprés ce vintvint que li emperieres 
   a sa fille, 
      qui fame avoit esté au bon chevalier Leus, et li dist tant d’un et d’el que il li fist savoir 
   au plus amiablement que ilil onques pot.
   Ha ! biau pere, dist ele, conme 
      cesteceste fin et ceste mort me doit estre mise en retenance 
      quantquant je 
      onquesonques en jour de ma vie n’oi i seul jour de joie !Fille,Fille, ce 
      dist li emperieres, ainssi 
      couvientcouvient il vivre en cest siecle.
   Mes atant me couvientcouvient me-couvient 
      mettremettre ore fin en ceste chose et venir
   a ce que il covient mener 
      et muer ce duel en 
   joie de quanques on puet. Mais en la fin couvient a autre chose entendre. \pend
\pstart Pas de nouveau § B
   QuantCar li prince et li baron qui furent aussi conme 
   anuié de festoierfestoier et de faire joie 
   vindrent a l’empereeur, 
   et li distrentaussi comme tout ensamble :
   Ha ! sire emperieres, conme Nostre Sire Deux 
      vous‹n›[v]ous a fet grant honneur
      en cest siecle quant il vous a mis en la haute roe de Fortune ! Car vous avez seurmonté touz ceus que on puet savoir 
      qui contre vous ont esté, et encore plus, si conme on puet veoir. Vous avez fait de vos anemis vos meilleurs amis. 
      Et vous a mis Deux en ce siecle au derrain a parfaite honneur. \pend
\pstart Quant li baron orent ce dit, si parla li empereres 
   moult humblement a eulz et dist :
   Ha ! biau seingneurs, de ce vous doi je moult hautement mercier,
      si conme cil qui ne sui que un seul honme d’assez povre sens et non mie de trop grant valeur,
      car, ce n’eust esté la vostre grant proesce qui s’est estendue a parfaite chevalerie,
      ja n’en feusse venus a ce que vous em poez veoir.Sire, distrent il, autant y a il de l’un conme de l’autre,
      car, se vostre chevalerie etet la vostre proesce n’eust esté, 
      ja ne nous feussions mis en painne de ce que nous sonmes fait, pour laquel chose on ne voit mie sovent avenir aujourd’ui que povre 
      chief viengne a bonnebon‹...›[n]e besongne fere.Ha ! Deux, distce distfait 
      li empereres,
      conme ceste parolle doit estre couvenable a ceus qui les honneurs ont a maintenir !
   Ce passage, tant dans l'échange de Cassidorus et sa fille que dans celui avec les barons, dénote la 
   dimension de miroir aux princes dont peut être investi le roman. \pend
\pstart Ainsi ont moult longuement parlé de ceste chose ou
   je ne me vueil ore plus arrester fors que de ce que tout li prince
   dont j’ai devant 
      fait mencionparlé 
   s’apresterent de congié prendre et de raler chascun enen son pays et en sa terre.
   Si le pristrent moult couvenablement. Et avint que li emperieres dist :
   Biaus seingneurs, je me part moult a envis de vous. 
      S’il peust estre que nous feussionspeussions estre touzjours ensemble, 
      mes il ne puet avenir. Et pour ce pri je a celui de qui nous devons 
      estre serf que je li puisse fere servise qui
      li plaisepuisse plaire et 
      ouque vousvous y 
      puisiez touz partir en maniere qu’ilque je vous sache gré du servise 
      que vous m’avez fait.
   LorsIl respondirent tout que il leur estoit bien meri, quant la chose estoit venue 
      a ce que on pooit veoir. Dont il avint que apres cest congié,  fist li empereres 
   donner a chascunleur de moult riches dons selonc ce que chascuns estoit, 
   dont li plusseur s’en esmerveillierent moult dont si riche jouel estoient venu, qui porent souffire 
   a si riche baronnie et a si grant com il 
   avoitestoient la assemblé, pour coi aucun distrent que 
   nule chose ne pooit tant proufiter quecomme 
      sagement garder ce qui en temps et en lieu pooit avoir mestier. 
   Et sans faille, ceste chose avoit l’empereris 
   fet, qui avoit en lieu et en temps ceste chosece mise ensemble 
   qui puis li tourna a joie et a loenges.
   Le récit présente ici une certaine insistance sur les qualités de Fastige, et surtout sur sa largesse, 
   déjà lors de l'arrivée de Cassidorus, du début du tournoi et ici encore, comme pour valoriser celle qui est ainsi redevenue 
   l'épouse de Cassidorus. Si avint ainsi qu’il se mistrent en lor chemin, et 
   s’en ala chascun vers son 
   paÿspaÿs, et ne finerent l'un jour plus l'autre moins tant que, 
   et ily 
   vindrentvindrent en leur contrees 
   sainzsainz et saus et liez et joians de ce que avenu leur estoit.
   Si me vueil tere atant de eulz et venir a l’empereeur et
   a l’empereris et dire conment il se maintindrent aprés ce qu’il furent ainsi assemblé. \pend
         
         
            
               Ci devise conment 
               li empereresempereres Cassidorus 
               et l’empereris jouentjouent ensemble 
               as eschéseschés comme vous orres.
               Comment li baron se partirent de l'empereeur de Romme et de l'emperris.
            
               Enluminure sur 1 colonnes (e) et 13 UR.
                  Cassidorus 
                  et Fastige 
                  couronnées jouent aux échecs avec des barons derrière eux 
               
            
\pstart OrOr nous 
   distdist ci endroit li contes que, 
   aprés ce que li baron et li prince de 
   Romemainte region se furent partis 
   dede Rome de 
   l’empereeur et de l’empereris, 
   il demourerent ensemble en bonne pes et a grant joie a Rome
      Rome si comme cil qui avoient grans prosperitez et moult de leur desiriers acomplis. 
   Et fuestoit ainsi qu’il alerent par plusseurs fois de l’un empire a l’autre 
   et menerent bien ceste vie x ans. 
   Aprés ce avint i jour que li empereres et 
   l’empereris se jouoient aus eschés, et avint ainsi com par aventure 
   queque li gieus tourna a ce que 
   li empereres en ot le pieur, pourpar 
   coi l’empereris li dist en grant joie et en amour :
   Sire, or sachiez que vous ne chevaucherez pas touzjours a 
   lorain qu’il ne vous coviengne perdre en aucune maniere.
   Li empereres entendi l’empereris 
   et nota en son cuer ce que ele 
   distot dit en autre maniere que elle n’avoit fet. Lors si li dist 
   en sourriant :
   Dame, dame, il n’est pas merveille 
      se‹l›[s]e il est ainsi conme vous avez dit, 
      car ainsi estest il de ce siecle dont il avient que, 
   quant on si asseure plus, que on y est plus tost tourne ce desus desous
      Cette 
         expression sentencieuse n'est pas répertoriée chez Morawski ni chez Le Roux de Lincy. à vérifier PD stp j'avoue ne pas bien la 
      comprendre d'ailleurs, donc ne me sens pas de la commenter. \pend
\pstart Quant l’empereris ot oï
   l’empereeur ainsi parler, si mua couleur a grant merveilles et dist :
   Ha ! sire, ore amasse je miex que je me fusse teue !Dame, dist il, ja, se je puis, ne vous en repentirez 
      quecar touz les temps 
      et les heures si ont leur saison. Et je ne croiquit pas que ce que vous 
      avez dit doie porter fruitfruit fructefiant en leu et en temps,
      dont il nous sera de mieux quant nous serons trespassez.En non Dieu, siresire, dist elle, 
      dont vueil je que vous me dites ce dont je ne sui pas 
      sagesage comment ne a quoi vous avez notee la parolle que j’ai dite, 
      car sachiez que je ne l’ai ditela dis 
      fors 
      que par 
      par amoursen amours et en feste. 
      SiEt me vola ensi de la bouche, pour quoi je vous pri en amour et en guerredon que, 
      se je ai dit chose qui vous anuieait anuié, 
      que vous lela me pardongniez, et, se elle vous a esté belle, 
               que vous m’en faciez sage pour coi vous l’avez ainsi reprise.En non Dieu, 
      chiere amiema chiere dame et ma bonne amie, 
      or m’avez vous demandé une chose 
               dont je vueil que vous sachiez la verité, et je la vous dirai, 
      mesmes que ce ne sera pas ore, ainz sera une autre fois 
      quant point et lieu en sera. 
      EtEt ce sera quand verrai et je savrai en vous autre volenté 
      qu’ele n’i estque je n'i voie ore.De parPour Dieu, 
      siresire, merci, dist la dame, 
      or soit quant il vous plera, quecar je ne cuit pas que vous ne le me doiez bien dire
         avant que il m'en soit gaires de mestier ne ja ci endroit n’avrai la maniere d’aucunes dames qui ja a temps ne 
      cuidentcuideront 
      savoir ce qui lor puet tournertourner aussi tost a anui 
      oucomme a joie.
   Dame, dist il, sauve lasoit 
      vostre reverence, je ne cuit pas que ceste chose vous doie tourner a anui se a grant proufist non.
      Je vous dirai une partie de ce que vostre parolle m’a donné a entendre. Il est voirs que on dit que, tant que li giex est biaus, 
      lessier le doit on, aussi conme il avient que cil qui est en vertu d’abstinence, quant il mengue volentiers d’une viande, 
      il n’en doit mie tant prendre que par abhominacion li couviengne lessier, ainz se doit refraindre et mettre le arriere pource que 
      elle li face meilleur digestion d’aquerre 
         meilleurnaturel apetit.
      Tout autresi, ma chiere damedame et ma bonne amie, m’avez vous mis en voie 
      de ce qu’il me puet faire melieur digestion. En lieu et en temps 
      vousde vous dirai pourcoi 
      et conment. \pend
\pstart Conme il soit ainsi que Nostre Sires 
   m’aait presté tant de 
   gracehonneur qu’il m’a mis ou plus haut estage de la roe de Fortune, 
   si conme il fu dist i jour qui passez est, il ne puet avenir 
   quepar droit ne par nature ne je 
   puissepuis touzjours vivre en l’onneur de l’empire de 
   Rome et de Coustentinoble, car il sont né qui a ceste honneur 
   s’atendent. Si me sembleroit reson et droituredroit que, tant 
   conmeque li gieus 
   est miens et que je l’ai d’apetit, que je le lessasse, car je 
   ne puisje sai de voir que je ne porrai pas touzjours vivre 
   en daintiezne chevauchier a lorains, c'est a dire en ceste seinourie. 
   Et pour ce vous dis je, tres douce dame, que se je ne laisse ceste deintie en quoi je me sui pluseurs fois delitez et orgueillis,
   et je puis moult priveement faillir a greignour dentie que cist ne soit. C'est a plus haute honnour que ceste n'est ou je ne puis 
   pas tousjours demourer.. 
   Et pource que je n’i puis pas touzjours demourer ne chevauchier a lorain, si me couvient aquerre perfaite honneur.
   Quant l’empereris ot ce entendu, si dist :
   Ha ! sire, quecomme je dis a bonne hore 
      la parolle que vous avez mise si bien a moralité ! Se il avient que Nostre Sire vous mete en voie de perseverer, 
      si conme vous le m’avez donné a entendre, parpour quoi j’en puisse 
      estre perçonniere aussi, com j’en aroie la volenté du deservir !Dame, distce dist 
      li empereres, la deserte si n’enne 
      tenrra se a vous non.Sire, dist elle, moult grans mercis.
   La réflexion de Cassidorus est bien détaillée, sur la base des remarques présentées ainsi de manière
   consécutive de ses barons qui vantent sa position dans la roue de Fortune et de son épouse qui valorise sa bonne fortune. 
   Elle semble une fois encore tenir de la tendance du récit à se rapprocher du miroir aux princes et vient valoriser ainsi une forme
   de lucidité, de juste milieu et d'humilité dans l'exercice du pouvoir. \pend
\pstart Ceste chose ne mistmistrent pas ne li uns 
   ne li autres en oubli, car il avint en assez aprés petit de 
   tempsterme 
   qu’il s’apresterent d’une chose dont il furent puis merveilleusement prisié, 
   La valorisation semble autant porter sur le mouvement de pénitence que vont aller prendre Cassidorus et 
   Fastige que, surtout peut-être en regard des épisodes précédents, sur la remise officielle du pouvoir de Cassidorus qui ne laisse
   ici plus son trône sans défense organisée de ses terres. car li empereres vint 
   en l’empire de Costentinoble, et mena a ce les barons de l’empire qu’il 
   assemblerentasseurerent
   Helcanus, son filz, et le reçurent a seingneur,
   et puis donna a Dorus le royaume de Grece,
   si que il du tout se demist de l’empire et prist congié a eulz et vint arriere a Rome
   a l’empereris, qui tout en autel 
   manieremaniere comme il s'estoit demis de l'empire de Constantinople 
   se demistrent ildemist elle de l’empire de 
   RomeRome et de toute la terre que onques ne li uns
   ne li autres n'en retindrent denree, et puis atornerent si soutivement leur afaire que, sanz le seu de nulli, 
   il se mistrent hors de Rome, sisi desconneument
   que nul ne puet connoistre leur affaire. Mes je ne sai 
      paspas ne ne truis en l'istore s’il 
   enmenerentenmenerent ilec avec eulz nul 
   ame en qui il se 
   fiassentfiassent tant comme a ceste afaire appartenoit.
   Mes d’une chosechose merveilleuse fet li contes mencion, 
      car aussi conme j’avoie touchié en mon conte du lyon qui avoit porté 
   a l'empereour compagniea mon compaingnon,
   si com devant a esté dit, il gisoit en i lieu disposé du palais a 
   l’empereeur, si que, quant ce venoit au matin acoustumeement, il s’en venoit en la sale 
   sanz fere mal a nullui et pour faire entendre a l’empereeur aussi conme dis me tu :
   Je doi mon seingneur garder et lui porter compaingnie.
   Il ne s’oublia pas a celui jour dont l’empereeur et l’empereris 
   s’estoient parti la nuit devant, car il vint en la sale, aussi conme il souloit, dont il avint, quant les chambellens se furent aperceus 
   de ceste chose, que il furent a merveilles esbahi et ne sorent que fere ne que dire. La nouvelle en ala par tout l’ostel, 
   conme cele qui ne pouoit estre celee legierementlegierement et sans faille la court avoit esté
   vuidié a cele heure et a cele jornee de grans seigneurs plus qu'ele n'avoit esté pieca et a privee maisnie,
   et avint que li mestre de l’ostel se mistrent a conseil que il pourroient fere de ceste chose, tant qu’il i ot i 
   chevalier, qui moult iert sages, 
   qui li dist :
   Je ne puis veoir que une seule chose qui tant 
      facefait a douter conme de 
      nostre nouvel empereeur que il ne nous vueille demander 
      son pere et sa mere que il nous lessa en garde.
      Car, sanz faille, de si grant chose conmeconme est de l’empereeur de 
      Rome avenist il bien que il feust si sagement gardez que il ne peust issir des chambres sanz le seu 
      d’aucuns, dont il me semble, se il n’i a aucuns de nous qui parler en sache, que je me doute que nous n’en aions a souffrir.
   Mes il n’i ot nul qui bien ne deistdeist vraiement que 
   il n’en avoient riens seu. \pend
\pstart En ce que cil estoient a ce conseil, li lyons, 
   qui pas n’avoit apris que li empereres demourast tant que il n’issist des chambres,
   conmençaconmença forment a grondir et a rungier, si qu’il n’i ot nul si hardi en la salle 
   qui y osast demourer, ainz s’enfouirent touz qui miex miexpot. 
   Et lors vint li lyons a l’uis de la chambre ou 
   li empereres avoit geu, si lale trouva 
   fermee et conmença fort a grater. \pend
\pstart Pas de nouveau § BCil 
   dontque j’ai devant parlé, 
   qui tenoient leur conseil de l’empereeur, ont oÿ 
   le lyon qui merveilleusement metoit grant painne a ce qu’il peust en la chambre 
   entrerentrer et il ne trouvast l'empereur que il ne leur donnast a souffrir.
   Dont n’i oteust nul qui seust que fere fors que 
   le chevalier qui devant avoit
      parléfait mention de 
   l’empereeur, qui vint a l’uys de la chambre et li ouvri. Et quant il fu dedens, il cercha amont 
   et aval, mes il ne trouva pas ce que il queroit. Et que fist la mue beste ? 
   Tout aussien autele maniere conme li bons 
   levriersaumiers 
   qui suit le senglier et le cerf quant il a trouvé leurs voies, tout en autele maniere 
   poursuivoit il l’empereeur, et trouva qu’il estoit issus par une fenestre 
   qui estoit en une garde robe, et de la estoit issus par i faus huys et passé a une 
   large nacele parmi i 
   grant fossé qui bien avoit iic piez de lé, 
   et de la se mistrent en la cité de Rome et puis 
   issirentse mistrent hors par la porte 
   orientalpar devers Orient que on apele la porte oriental. 
   Et tout aussi conme je dis se mists’est mis 
   li lyons aprés eulz que il n’espargna onques riens qui tenir le peust. \pend
\pstart Quant ce orentont 
   veuveu et seu la mesniee, si orent moult grant paour du
   nouvel empereeur que il ne les vousist achoisonner de
   son pere et de sa mere. 
   Dont il avint que il y ot aucun quil vindrentvoudrent aler 
   avant pour eulz metre a voie 
   avecaprez 
   le lyon, mes il leur fu desfendu des plus souverains, 
   etqui distrent 
   quequi aprez eulz se metroit a la voie que ja gré n'en recevroit de l'un ne de 
      l'autre quar ilchascun 
         pourroientpooit bien savoir que, pource qu’il ne vouloient pas que on seust
         quel part il tourneroient, s’estoient il ainsi partis de 
         RomeRome et de l'onneur.
   Lors n’i ot nul qui i osast vertir, ainz ont fait asavoir a touz leurs enfans en quel lieu que il fussent, ou loing ou pres, 
   conment ilil lor estoit avenu de 
   l’empereeur et de l’empereris. \pend
\pstart Pas de nouveau § BQuant il oïrent la novelle, si furent touz esbahis et 
   n’en firentsorent autre chose fors que il sorent aussi conme par avis 
   qu’il estoient alez en essil aussi conme l’empereeur avoit autrefois fait. 
   Li autre dientdist :
   Or couvient que nous soionsaions esté 
      filz d’un hermite qui ne set vivrevivre ne mourir se en hermitage non.
   Li autre disoient qu’il ne cuidoient pas qu’il fust en son bon sens quant ainsi s’estoit partis.
   Nouveau § BAinsi ont tourné a l’empereour 
   et a l’empereris a mal ce qu’il 
   avoientavoit fet pour bien. Dont il avint que uns preudoms qui 
   entendia‹t›[e]ndi a 
   ce que onil parloit ainsi sus la partie de 
   l’empereeur, il ne se 
   potpooit tenir qu’il ne deist oiant touz :
   Vrais Dieux, dist il, conme est diverse chose 
      d’estrede vivre en cest siecle a la loenge de chascun !
      Car je voi que nuls si n’ine puet vivre ne faire chose por si 
      grantgr‹...›[a]nt bien que li envieus n’i puissent mal noter.
   Quant il ot ce dit, si n’i ot nul qui osast autre chose dire fors que moult avoit esté sages
      li empereres quant il au derrain s’estoit mis ou servise 
      DieuNostre Seigneur.
   Aprés avint que i damoisel dont j’ai devant fet mencion, 
   qui avoitot non 
   Celidus – icelui avoit engendré li empereres en 
               la damoisele du Chastel 
                  MignotMignot dont j'ai devant parle ; 
   cilz estoit tant biaus etet tant gent et tant 
   sagesavoit sens de son aage que on ne savoit son pareil en 
   nulson paÿs –, quant il sot que 
   li emperieres s’estoit ainsi partis de Rome,
   sique il ne fina 
   ettant que il vint la ou li joennes 
   Fastidorus
      empereres qui Fastidorus estoit apelez estoit.
   Quant il vint devantdeva‹...›[n]t lui, si ne se pot tenir qu’il ne plorast moult 
   tendrement. Lors dist en soupirant :
   Sire, il me couvient a vous prendre congié en maniere que aventure sera se je jamés vous voi. \pend
\pstart Quant Fastidorus entendi 
   le damoisel, si vit que il lermoioit des yeux, 
   si en ot pitié et li distDieu :
   BiauComment biau chiers dous frere, 
      ja Dieu ne placeplaise que 
      je puisse i seul 
      jourjour em pes vous vivre em pes 
      quant par ma volenté vous partirez de moi, si vous avrai plus de bienbien fait 
      que je n’ai encore.
   Li damoisiaus respondili respondi:
   Sire, la vostre merci. Mes une chose me grieve de ce que j’ai 
      mon seingneur demon pere perdu, 
      de qui j’atendoie a recevoir l’ordre de chevalerie.  Mes il m’est avis que l’atente m’est trop longue avant que ce soit mes fet.Biau frere, dist Fastidorus, ne cuidiez pas que vous 
      a ce ne puisiezpuisiez ne puisiez encore bien venir, car 
      nous n’avonsvous n'avez 
      encore pas tant atendu que nous n’oïonsoïons procainement aucunes nouvelles 
      par quoi nous ferons vostre volenté.Sire, ce dist 
      Celydus, jasauve soit vostre grace, ja 
      cece je n’n'i
      atendrai, car je sai de voir que trop longue seroit ceste atente. Si pri a Dieu que je ne puisse jamés 
      armesarmes de chevalier porter jusques a tant que je de lui les recevrai.
      Et pour ce ne vueil je ore ci 
      fere plus longue demorance, ainz 
      vueil prendre 
      congiécongié a vous,
      conme cil qui ne vous cuide jamés reveoir de ci adonc que aventure le face. \pend
\pstart Lors ne valut riens chose que 
   lili jones 
      empereresFastidorus 
   peustdeust dire 
   – c’est a dire Fastidorus –
   La précision de V3 paraît un peu superflue, mais dénote le souci de rendre plus lisible le passage 
   de pouvoir qui a été effectué ici, ainsi que l'identité de ce nouvel empereur qui n'aura qu'un règne très éphémère.
   que Celydus ne se partist de luilui, vaust 
   ou non. Et se mist en son chemin vers son paÿs, entre lui et son mestre et sa mesniee. 
   Si se test ore li contes a parler de lui et retourne a l’empereeur et 
   a l’empereris, qui se furent partis de Rome 
   pour aler ou servise de Nostre Seingneur. 
      Et si parolle ausi du 
      lyon, qui aprés 
      les suivi,
   ainsi conme devant est ditavez oÿ 
      ou conte. \pend
         
         
            Ci devise conment l’emperiere 
               Cassydorus
               et l’empereris s’en issirent
                  se partirent de Romme. Et comment il encontrent le lyon.
                  se mistrent en l'ermytage la ou li lyons les mena. 
               hors en tapinage et alerent en une 
               forest a un hermite, 
               et aprés conment li empereres devint ouvrier gaaingne pain, 
               et l’empereris demoroit en la ceule avec 
                  le lyon.
            
            
               Enluminure sur 2 colonnes (b-c) et 12 UR.
                  Cassidorus en armes bleues à ronds rouges rejoint, 
                  dans la forêt, Fastige couronnée devant leur cellule avec 
                  le lion à ses pieds.
               
            
\pstart Ci endroitOr ci 
   endroitOr 
   distnous dist li contes que, 
   quant li emperieres et l’empereris 
   se furentfu partis de Rome, 
   il n’orent pas alé une journee assez petite queque ce vint au matin que il 
   avoientorent jeu en une petite vilette ou il avoit pou de gent. 
   Si trouverent a l’issir de l’ostel le lyon, 
   dont nous avonsj'avoie devant parlé. 
   Quant li emperieresli uns vit 
   le lyonl'autre 
   et le lyon le vit, 
   onques honme ne fist greingneurtele joie de beste 
   ne beste de honme. 
   Ha ! sire, dist l’empereris, que avez vous empenssé de 
      ceste beste, qui ainsint vous a suivi et veult encore 
      fairefaire feste ?Dame, dist il, moult ai noble compaingnie en lui ! Je ne sai chevalier 
      en Gresce 
      pour cui je le chanjasse pas, car je sui touz certains que Dieu le 
      m’a envoié pour nousmoi 
      feretenir compaingnie. Et puisque il plaist a Nostre Seigneur, 
      il le me couvient souffrir et a moivous 
         d’autre part.Sire, dist ele, je me douteroie de lui que il ne nous courrouçast en aucune maniere ou feist destourbier.Dame, dist il, ne vous en doutez de nule rienja. \pend
\pstart Ainsi avint que il se mistrent au chemin, 
   et le lyon avec eulza tout, 
   qui aussi paisiblement les suivoit conme i petit chiennet, et ne feist jamés mal a nullui qui mal ne li vousist faire. 
   Moult alerent longuement et moult trouverent en aucunes villes qui moult a envis les 
   herberjoienthebergoit pour 
   le lyon dont li plusseur avoient paour. Dont il leur avint que, 
   quant il orent ainsi alé tant que il s’embatirent en la terre de Patras, 
   ou il avoit une moult merveilleuse forest, et en cele forest avoit 
   i honme ancien et de sainte vie, 
   dont alerent tant et quistrent que il le trouverent et s’acointierent de lui. 
   Et li emperieres li 
   distrentdist 
   une partie de son afaire. 
   Et quant li preudoms l’ot entendu, si en ot moult grant merveille et dist :
   Sire, j’avroie assez greingneur mestier de vostre conseil que vous n’avez du mien. 
      Si vous requier que vous priez DieuNostre Seigneur Dieu pour moi.
   Adonc dist li emperieres a l’ermite que 
   por Dieu il li enseingnast aucun lieu ou il pourroit vivre de son labour en aucune maniere,
   ausi conme li plusseur qui ne sont neent plus ne de fer ne d’acier que je sui.Ha ! sire, dist li hermites, 
   conment pourriez vous endurer le labour de travail, qui onques riens n’en feistes ?Amis, dist li emperieres, 
      qui de bon cuer fait la chose, il ne li grieve auques neentCette 
         expression sentencieuse n'est pas répertoriée chez Morawski ni chez Le Roux de Lincy. à vérifier PD stp. Sa portée 
      proverbiale semble cependant assurée par sa dimension sapientiale qui vient valoriser surtout la vertu de Cassidorus.. 
      Et de l’autre part, j’ai entendu que Nostre Sire Deux si distdist que : 
      Qui ne labourera ja, es cieus n’enterraCette 
         expression sentencieuse n'est pas répertoriée chez Morawski ni chez Le Roux de Lincy ("Qui ayme labeur parvient a honneur"). 
         à vérifier PD stp. Sa construction
         syntaxique comme sa portée morale dénotent cependant une formule proverbiale.. 
      Et pour ceste reson, je n’en vueil pas estre hors mis, car je sui assez grans 
      et fors pour porter i grant fais ou pour autre labour fere.Sire, dist li preudoms, bon cuer et vrai avez. 
   Or vous doint Deux venir a si tres vrai labour qui vous maint ou regne de paradis !
   Cassidorus prend ici une décision importante, renforce explicitement la portée spirituelle et symbolique 
      de leur pénitence, par sa volonté de dénuement et de travail manuel qu'il souhaite ainsi valoriser, selon une tendance 
   philosophique et dévotionnelle bien établie bien sûr. \pend
\pstart Or me dites, dist li empereres, en 
   quel lieu je pourroie miex demourer pour tele vie mener et pour mon pain gaaingnier.Sire, dist il, il a ii liues de ci a 
      i chastel ou 
      il a une bonne ville et sai de voirse de voir-et sai de voir que on 
      i fait une nouvelle eglisse de saint Nicholas 
      ou vous avrez a fere se il iert 
      ainsi que il vous venistfust a plessir que vous i vousissiez traveillier 
      a aidier a aporteraporter les 
      pierrespierres blanche et bise et mortier 
      et autres choses de quoidont maint i sont soustenu.En non Dieu, distce dist 
      li emperieres, bien ira la besongne, mes que il plaise a 
      DieuNostre Seigneur.
   Lors s’apenssa li preudoms d’un lieu qui est pres de yaue, 
   mes que moult i avoit lieu estrange. Et ce demandoit li empereres pource qu’il ne avoit pas moult 
   a faire que nuls s’i embatist qui le peust grever ne nuire. \pend
\pstart Li empereres et l’empereris, 
   ainsi com j’avoieje vous avoie devant dit que
   li hermites si s’estoit apenssez d’un lieu couvenable 
   qui est a demieune liue de Gomor, 
   i chastel de merveillieus pooir – c’estoit cil dont li preudoms avoit fet mencion,
   et cilz chastiaus si estoit assez pres de la voie qui 
   venoitvient de l’ermitage au
   leu ou li empereres devoit avoir son herberjage, 
   dont nous trouvons en l’ystoire que xxii liues avoit de Patras 
   jusques au chastel de Gomor, 
   et est a savoir que aucuns veulent dire que li benois cors de 
     monseingnor saint Nicholas le confés avoit une fois 
      atouchié a une dame qui avoit une maladie moult crueuse dont nulz homs ne cuidast pas que jamés 
      en deustpeust 
      garirmgarir, 
   et pour ce glorieus miracle fist on fere 
   celeune eglise 
   en l’onneur de monseingneur saint Nicholas, qui mout cousta d’avoir.
   Cilz chastiaus si estoit aussi conme sus la marine. La endroit, 
   ainsi conme je vous ai dit, fist li empereres fere une
   ceule pour lui et pour l’empereris demourer. 
   Dont il avint que, quant il ot sa besongnechosete atiriee ainsi 
   conme il li plot miex, et tout ensement conme uns homs diroit a i autre, l’emperere
   conmanda au lyon a garder l’empereris
   et qu’il ne se meust de devant lui tant que il fust 
      revenusrepairiez.
   Et ainsi le fist le lyon. 
   ToutSans faille, tout aussi sagement s’ala jesir a l’uys de 
   la ceule conme se ce fust i honme qui eust dist : 
   Je ferai tout vostre 
      conmandementcommandement." Li empereres et l'empereris et li hermites 
         qui la erent s'esmerveillierent moult de ce que 
      li lyons l'ot si bien enten et que si bien fist sa volente, et s'en tint li empereres a bien payez. \pend
\pstart Li 
   empereresLors li empereres et li hermites 
   qui la erent 
   se sont parti de la ceule et tournerent leur voie 
   au Chastel de Gomor. 
   Et Gomor avoit non li sires a qui 
   li chastiaus estoit, 
   qui assez nouvellement l’avoit fet fermer, non pas pource que li chastiaus 
   ne fust de grant antiquité, et i avoit marchié une fois la semainne. 
   Li hermites desus dit estoit ouvrier de fere 
   coffins et estuis a hapnas. Et quant il les avoit fes, 
   il les portoit au chastel pour vendre, 
   si que de ce vivoit li preudoms de jour en jour et moult saintement, et du seurplus 
   que il pooit espargnier 
   il le faisoit donner ou  
   il le donnoit a edefier 
   l’eglise du chastel, qui ert de 
      saint Nicholas. 
   Li emperieres, 
   queque li hermites que 
   li hermites avoit amené au chastel, 
   ainsi conme je vous ai ditvous avez oy, 
   fist tant qu’i l’acointa a ceus qui avoient l’ueuvre entre mains  et leur dist que, 
   pour l’amour au glorieus confessor, il le meissent en l’ueuvre, et il se traveilleroit tant que 
      il ses'en loeroient 
      de lui et de sa besongne. \pend
\pstart Quant li mestre oïrent l’ermite 
   qui demandoit de l’ueuvre pour l’empereeur, 
   si li demanderent de quele 
      ouvragechose il 
      savoitvoudroitvouloit ouvrer. 
   Il distleur a respondu qu'il n'estoit pas ouvriers de maçonnerie, mais 
   qu’il savroitsavoit 
      bien porter chaus et mortier et sablon. 
   Lors esgarderent l’empereeur et virent a lui parfaite biauté et distrent :
   Sire, ilil nous semble que se soit une 
      moqueriemoquerie de ce que vous dites, car nous ne veons pas en vous que vous soiez 
      pourtel pour faire tel afere conme vous avez dit, 
      car vous estes miex festailliez a tenir i empire ou 
      i royaune dont li pluseur sont aujourdui moult deceu, 
      car par la defaute d’eulz font leurs menistres moult de
      torfaistorçonneries et de meffais.
   Quant li empereres ot celice entendu, 
   si mua coleurune merveilleuse coleur et dist :
   Ha ! biau seigneur, il ne pueent pas estre tous roy n’empereeur, 
      et si ne doit on pas avoir en despit les laboreurs de 
      pluseurspslors (sic) mestiers, car miex aime a mes membres 
      gaaingnier mon vivre a painne et a travail et a sauver l’moname 
      de moien paradis que avoir m livrees de 
      terreterre et de rente a perdicion.
   Lors quant li empereres ot ce dist, 
   si sorent bien li mestres que il estoit preudoms. Si li ont dit :
   Biau sire, or ne vous anuit pas, car nous avons pou veu de telz ouvriers conme vous estes. 
      Si vous prions que vous veingniez quant il vous plaira et si 
      faites avous entremetez de faire vostre volenté, 
      et nous ne detenrons pas vostre labour.Biaus seingneurs, dist illi empereres, 
      grans mercis. Les doutes que laissent entendre les maîtres ouvriers, en écho à ceux que 
      formulait également l'ermite face aux capacités de Cassidorus à endurer ce travail manuel, dénotent surtout le contraste 
      entre ses précédentes fonctions, dont il ne peut ainsi véritablement se cacher, celles-ci semblant, selon la théorie 
      de concordance entre intérieur et extérieur très prégnante alors, inscrites directement sur son visage. L'impossibilité
      pour Cassidorus de dissimuler son statut se trouvera au coeur de ses dernières aventures, mais l'affirmation de son 
      souhait de mener à bien son nouveau travail symbolise bien sûr avant tout sa bonne volonté pour faire sa pénitence, affirmée 
      nettement dans sa préférence pour le bien de son âme que pour les biens temporels dont il était jusqu'alors investi. \pend
\pstart Lors avint que li empereres 
   se mist a fere ce que vous poez ouyravez oy a quoi il beoit, 
   par quoi il se mist a lui traveillier eten servir maçons de porter pierres 
   a la fois en une civiere et a la fois en une hotte, si que il n’estoit pas oiseus que il ne feist ce que on li conmandoit,
   aussi debonnerement conme li aingniaus sueffre que on le porte vendre. Et d’autre part, il n’i ot mie moult esté que il sot fere 
   quenque ilil y apartenoit. \pend
\pstart Pas de nouveau § BAinsi demoura 
   li empereres en cest labour une grant piece 
   et repairoit en sa ceulle toutes les nuis avec l’empereris, 
   a qui il anuioit moult de si grant travail conmeque il avoit empris, 
   car touz les jours il se levoit au point du jour pour aler labourer,
   ainsi conme il a esté dit, et au vespre reperoit, et lassez et debrisiez et povrement peüs de boivre et de mengier, 
   pour quoi il avint que en cest afaire plot a Nostre Seingneur que 
   li lyons, enqu'en 
   autel maniere conme ils'il 
   oteust sens et entendement d’onme, aloit 
   ilil la nuit em proie et puis 
   sis'en reperoit,
   ainsi conme li empereres se devoit lever pour aler en son labour. \pend
\pstart Ceste vie maintint li emperieres moult longuement.
   Et toutes ses choses ainsint maintenues saintement, il avint 
   si conmeque il plot a Nostre Seingneur que
   l’empereris se senti grosse d’enfant, si ne le volt pas celer a 
   son seingneur pourpar moult de raisons. 
   Quant li emperieres 
   sot ce, si en fu moult joians et en loa Nostre Seingneur de tout son cuer. Si ne demoura guieres que 
   li emperieres vint a l’ermite 
   desus dit et li a confessee sa volenté tout ainsint conme il cuida que bon fust.
   Cilz hermites s’apenssa d’une 
   pucele qui vivoit em povreté pour l’amour de Nostre Seingneur. 
   Cele pucele estoit de sainte vie et de bon sens.
   L'emphase mise sur la sainteté de chacun des protagonistes est révélatrice de l'importance symbolique 
   dont seront revêtus les évènements à venir, la fin de l'empereur comme la destinée de cette progéniture et de la jeune fille qui en 
   aura la charge. Si fist li hermites 
   tanttant en poi de temps 
   que elle vint demourer avec l’empereeur et avec l’empereris 
   pourpour lui tenir 
      a l’empereris 
      compaingnie. 
   Et quant l’empereris la connut, 
   si li plot moult sa maniere, et ot la pucelle a non Nichole. 
   Si distvint 
   ala pucelle a 
   l’empererisa l'empereris et li dist :
   Dame, je ne savroie estre oiseuse que je ne feisse aucune chose, 
      pour quoi je vous pri que nous nous entremetons de
      quelle choseaucune chose de 
      quequoi que ce soit.Amie, dist l’empereris, moult me pleroit, mes que 
      mon seingneur le vosist souffrir. Si nous en
      couvendra a lui prendre congié. Or me dites, 
      pucelle, ce distfait 
         l’empereris,
      de quoi vous savriez ouvrer.Dame, dist elle, je sai i pou ouvrer d’or et de soie, et en 
      gaaingnoiegaigneroie mon vivre 
      avant que je venisse a vous. \pend
\pstart Pas de nouveau § BQuant l’empereris 
   entendi la pucelece, si dist :
   Tout ausi ai je veu tele heure que j’en savoie 
      aucune chose faire, si nous i couvient entendre.
   Lors quant li emperieres revint de son labour, si le mistrent a reson 
   l’empereris 
      et la pucelle 
   et li 
   distrentrequistrent que 
   il leur donnast congié delaissast fere leur 
      volenté, et il respondi que il le voloit 
         en bonne heure. Si firent tant que elles orent les instrumens telz 
   conme a leur oeuvre couvenoit. 
   Lors conmença l’empereris a ouvrer et a fere joiaus d’or ou de soie. 
   Si avenoitil avint que, quant elle les avoit fes, 
   la pucele les portoit vendre au Chastel de Gomor. 
   Et fu li ouvrages si bons et si biaus que on ne s’enle pooit 
   saouler dutrop regarder. 
   Nouveau § BAdonc le convint savoir a 
   la dame du chastel,
   qui vit i parement d’une aube ou il avoit une ymage d’un myracle de 
   saint Nicholas. Lors demanda la dame du chastel
   qui cecele pooit estre qui ce avoit fet. 
   Dit li fu conment une fame aportoit 
      ces chosestele chose 
      a vendre chascun jeudi ou marchié.
   Par Sainte Crois, dist elle, je veil que on lale 
      me face venir. Et on si fist. Quant celela pucelece 
   vintfu venue devant la chastelainne, 
   sila dame li 
   demandademanda et enquist 
   qui ceteles choses faisoit, 
   et cele ne li volt dire, quequar 
      deffendu li estoitcomme cele a qui il avoit este deffendu
   que ele n'en feist nului sage por ce que nule parole n'en fust.
   Et quant la chastelainne oï que cele celoit sa dame ainsi, 
   adoncadonc en fufu ele 
   plus engrant de savoir le, mes onques on ne li sot tant dire ne d’un ne d’el 
   que elle reconnoistre le vousist.
   En non Dieu, dist la chastelainne,
      je sé de voir que cilz ouvrages ne vous vient pas de bon lieu, car je sé bien que vous ne l’avez pas 
      faitfait." Quant la pucele ot entendu la dame qui ainsi parloit, 
         si li dist :.DameDame, pour Dieu merci, 
      dist la pucelle, 
      je ne di pas que je l’aie fait, mes il m’est deffendu que je nene le die 
      pasa nullui 
      que je l’aiequi l'a fet.DontPar ma foy, dont 
      vous demourrez vous !, dist la chastelainne. \pend
\pstart Pas de nouveau § BAinsi demoura 
   cele damoiselle 
   jusqueset fu detenue jusques a tant que
   li chastelains fu revenus, qui n’estoit pas en la ville.
   Li empereres, qui estoit a l’ueuvre et avoit veu la pucele 
   au marchié a toute son ouvragesa besoigne, fu 
   touztousdis esbahis quant il vit 
   que elle ne revenoitrepairoit 
   aussia lui aussi conme elle souloit, 
   si ne sot qu’i li fu avenu. Adonc vints'en vint arriere 
   il a l’empereris et li dist que
   lasa pucele 
      ne savoit il pas qu’ele iert devenue. 
   Lors fu l’empereris esbahie moult durement 
   et ne sot que pensser, conme celle qui moult l’amoit, si dist :
   
      Ha !etHa ! 
      sire, si ne serai je aise,
      si en avrai eusavrai bonnes nouvelles. \pend
\pstart Celle nuit passa, et vint a l’endemain, que li emperieres 
    vint a l’ueuvre. Si coururent parolles d’une fenme qui estoit arrestee chiez 
   le chastelainla chastelaine
   G semble avoir raison de préciser qu'il s'agit de la châtelaine, clairement distincte dans l'emsemble 
   de ses mauvaises actions, du châtelain qui cherche au contraire à s'élever au contact de l'empereur en particulier et non à lui
   nuire à lui et à ses proches comme le cherche dès lors constamment son épouse. Cet épisode se veut ainsi indiciel de la 
   vilennie et de la curiosité dévorante de cette châtelaine., 
   qui avoit aportee joiaus en la ville 
   apor vendre et on li metoit sus que elle les avoit emblez. 
   Quant li emperieres oï ce, si fu moult esbahie et ne sot 
   quequ'il peust faire ne que 
   diredire de tel afaire. Dont vint au mestre des ouvriers
   et prist congiécongié a lui d’aler en i pou de lieu, 
   et adonc vint chiez l’ermite et li conta l’afaire de 
   la pucelela pucele.
      Quant li hermites l'ot entendu, si li dist : "Sire, nous irons au chastel et si savrons que l'en li demande."
   "Par foi, dist li empereres, ensi le vueil je et je vous en pri.". 
   LorsLors se mistrent a la voie et s’en vindrent 
   amdui 
   versa 
   le chastelchastel de Gomor.
   Et quant il vindrent a l’entree de la ville, si encontrerent 
   la pucelle, que l’en avoit delivree, et l’enmenerent a 
   l’empererisl'empereris et distrent : 
   "Dame, vesci vostre pucele que nous vous amenons. Et quant l’empereris la vit, 
   si en fu moult joiansjoians et moult lie. Puis li demanda et enquist la cause pour quoi
      ele avoit tant demouré, et la pucele li dist tout en la maniere que devant est ditliee. \pend
\pstart Pas de nouveau § BUn jour avint que 
   li chastelains oï 
   parlerconter d’un tel honme qui tant estoit 
      biaus et fors que chascun s’en esmerveilloit conment il pooit tant de labour fere com il faisoit. 
   À la beauté déjà louée par les maîtres ouvriers vient se superposer la force depuis démontrée par Cassidorus
   à son travail, dans une nouvelle valorisation de ses qualités et de celle de sa pénitence. Lors conmanda 
   li chastelains que on le feist venir par devers lui, et on si 
   fist. Et quant li emperieres fu venusvint 
   devers le chastelain, si le salua moult 
   humblementhautement et moult humblement et li demanda 
   qu’il li plaisoit. 
   AdoncQuant li chastelains le vit, si le regarda moult 
   li chastelains 
   et li sembla moult tres bien que autre fois l’avoit veu, et si ne 
   savoitse pooit pas bien aviser en quel lieu. 
   Adonc li distdist il :
   Dous amis, de Nostre Seingneur je vous pri et conjur 
      a jointes mains que vous ne me celez pas ce que je vous demanderai, car je vueil que vous me dites en quel lieu vous feustes nés,
      et lors me souffira ce que je vueil de vous savoir. Car je sui touz certains que je vous ai autre fois veu, mes je ne sé 
      paspas ou ne en quel lieu 
      ce fu.
   Quant li empereres vit que il estoit 
   si engrant de savoir ou il avoit esté nés, si dist :
   Certes, biau sire, je ne vueil pas mon nonlieu 
      celer puisqu’il est ainsi que savoir le voulez. Je fui filz d’un povre laboureur et fui 
      nésnés droit en la cité de 
      Costentinoble. \pend
\pstart Pas de nouveau § BQuant li chastelains ot 
   l’empereeur entendu, si se lessa cheoir a ses piez et dist :
   Ha ! emperere de Rome et de 
      Costentinoble ! Je savoie bien que je vous avoie 
      veuveu autre fois ailleurs que ci !
   Et adonc li baisa le chastelain 
   sonle soler, vousist ou non, 
   dont li emperieres fu touz courrouciez et dist :
   Je ne voi pas en vous tant de sens com je cuidoie qu’il i eust. Et ne vous anuit ore pas de ce que 
      je vous ai dit, et vous dirai pourquoi. Ce n’est ore pas grant honneur a vous ne a moi quant si hastivement voulez dire que 
   je sui empereeur de Rome et de 
      ConstentinonbleConstentinonble."
      Quant li chastelains l'ot entendu, si li demanda :.Et pourquoipourquoi dites vous ce, 
      dist li chastelains, 
      quant je vous en dirai bones enseingnes ?Ore donc ! dist li empereres.SireSire, dist cil, 
      je le diré moult volentiers. \pend
\pstart Lors li contacommença 
   le chastelains et dist :
   Sire, il fu jadis i jour 
      avant que vous fussiez mariez que vous venistes en Galille en une cité 
      que on apelequi est apelee Bethsaïda, 
      de laquele Edypus estoit sires a celui temps. 
      Et la demourastes vous avec lui une piece. En cele 
      acointanceacointance il avint que 
      Helcana, sa fille, vous ama sanz ce 
      queque james a lui n’aviez penssee ne riens n’en saviez. 
      Aprés il avint que Edypus avoit eu guerre au prince de 
      Tyberiade, qui avoit a non 
      LapsusJaphus. 
   Edypus, qui estoit chevalier de grant renonmee, vint a vous, 
   qui avez a non Cassydorus, et vous pria en amours que 
      vous li vousissiez aidier eta mettre celui 
         LapsusJaphus en sa merci, 
      qui son honme devoit estre. Et vous li aidastes en tel maniere que bien i parut, car vous li donnastes conseil qu’il feist 
      sa gent ensemble mettre, et il si fist, et ala Edypus sus 
      Lapsus, qui auques sot sa venue. Et sé de voir que vous et vostre bataille feistes 
      tant d’armes es premieres venues que cil dedenz qui issirentissirent fors 
      furent mis en la cité arrieres par force, et preistes terre a vostre volenté et vostre gent aussi. 
      Et vous sé bien a dire que sus touz 
      chevalierschevaliers qui onques i furent a cel tamps 
      vous i feustes li plusplus renommez et li plus doutez, 
      et vous en dirai bonnes enseingnes, car il avint a l’assaut que vous i feustes navrez et emportez, et cuidierent cil dedenz 
      que vous fussiez mors. Et fu une biere levee par ainsint que cil dedenz issirent hors conme cil qui moult furent joians de vostre mort. 
      Mes avant que la jornee passast tourna lor joiejoie un poi en duel, 
      carquant il furent tel mené que, 
      quant il entrerentcuidierent entrer en la cité, 
      vous leur venistes au devant et feistes tant que a painnes en y ot nul quil ne 
      fustfust malement navrez. 
   Lors en y ot tant des mors que li fossé en furent tuit empli.Sire chastelains, dist li emperieres, 
      vous n’estes pas li premiers de ceus qui m’ont encercié pour celui que vous 
      m’avez ci contéavez dit.Non, sire ? distce dist 
      li 
         chastelainsli chastelains." "Non, par Dieu, dist li empereres.".
      Et conment ! dist li chastelains. 
      N’estes vous pas li empereres Cassydorus ?Sire, dist il, voirement ne le sui je pas.Non, sire ? dist il. Encore en ai je bonnes enseingnes.Mais or me dites avant ce qu’il vous plaira.
   Sire, dist cilz, volentiers. \pend
\pstart Quant cil de Tyberiade s’en furent reperié en la cité, 
   la concordance fu faite en tele maniere qu’il s’ense metroient sus 
   ii chevaliers, dont cil qui devoit faire la bataille pour Lapsus 
   fu appareilliezestoit li plus doutez de toute sa chevalerie 
      et avoit a non Ascanus. Et il 
   n’i ot chevalier de la partie 
   Edypus qui osast porter armes contre lui fors que vous, 
   biau sire Cassydorus, que je voi ci en ma presence. 
   Vous feistes la bataille et meistes Ascanus au desous, et par vous fu la pes faite entre 
   Edypus et Lapsus. \pend
\pstart Pas de nouveau § BQuant li empereres 
   ot entendu le chastelain, sisi li dist :
   JeCertes, sire, que je 
      croi bien que vous feustes a la bataille si coma ce que 
      je l’aile vous ai 
      oïoï dire et retraire.Ce savez vous bienbien, sire, 
      distce dist li chastelains.Sire, dist li emperieres, or soit ainsi 
      comque il vous plera ! 
      Mes or dites se vous voulez autre chose dire.En non Dieu, sire, or a primes vous vueil je dire plusseurs choses qui 
      prouffistconquest a l’ame vous pourront porter et tout aussi a moi.Vraiement ? dist li empereres. Se vueil je bien oïr.Sire, dist li chastelains, 
      dont me dites conment je serai si privez de vous que je i 
      puissepuisse aucune fois parler a lesir.Sire chastelains, dist il, quant je savrai que poins en sera, je revendrai a vous. 
      Si vous pri que vous m’enme 
      lessiezlessiez de vous partir atant.Sire, dist il, ce ne vous puis je escondire, 
   mes que vous m’en couvenenciez que vous revendrez a moi au plus tost que vous porrez en bonne maniere.Ainsint le vous ai je en couvent, dist li emperieres. 
   Lors s’en parti atant et vint arrieres a son ouvrageoeuvre, 
   et li chastelains demourademoura espris et 
   esbahi si durement qu’il ne sot se il fu ou en ciel ou 
   enen la terre. La résistance de Cassidorus 
   à être reconnu est tout à fait vaine et ne semble qu'alimenter le récit des précédents exploits du alors tout jeune Cassidorus.
   Les aventures relatées ici datent en effet d'avant son mariage avec Helcana, et font l'objet du Roman de Cassidorus, §10-72. \pend
\pstart Pas de nouveau § BLa chastelainne, 
   qui iert en aguet que li chastelains, son seingneur, ne venoit a lui pour lui dire nouvelles 
   de celui a qui il avoit tant parlé, ele 
   atendiatendi a lui i pou, mes il ne vint pas, 
   car il conmença moult fort a pensser a 
   l’empereeur, qui si avoit tout deguerpi pour l’amour de 
   Nostre SeigneurDieu et por son cors metre a destruction comme j'ai traitié dessus. 
   Si dist li chastelains a soi meismes :
   N’as tu aussi une ame a sauvergarder 
      conmeaussi conme il a ? Par Dieu, oïl, 
      et quit que elle soitest plus empeeschiee 
      que la ton seingneur ne soit, et si est elle sanz faille, et bien m’en reprent ma conscience. \pend
\pstart En ce que li chastelains estoit en tel point, 
   estes vous venir la chastelainne, 
   quiqui a cel cop s’embati sus lui et le trouva moult fort soupirant. 
   Or escoutezescoutez une male aventure et de 
   la male aventureuse fenme qu’ele fistqui son seigneur trouva en tel point comme vous avez oy,
   et li enquist et dist :.
   Conment, sire ! Il me semble que ce paÿsant ne vous a pas 
      lessié en bon point, 
      car il ne vous a pas dit chose qui vous doiepuist releescier.DameDame, dame, dist 
      li chastelains, conment parlez vous ! Li paÿsant ne m’a dit 
      guiere chose qui apartiengnie aa vous n'a vostre personne, 
   car je croi et sai que c’est i saint honme et de bonne vie.Conment ! dist elle. En ferez vous dont i saint hermite ? 
   Dites moi qui il est, que je le vueil savoir.
   Sont ici posés tous les éléments essentiels à la fin du récit : les mauvaises intentions de la châtelaine,
   reboublées de sa curiosité maladive, selon un retour notable aux stéréotypes misogynes qui ont longtemps instillé le cycle; 
   en contraste avec les propres états d'âme du châtelain de Gomor, qui se montre, dès sa rencontre avec Cassidorus, 
   soucieux de suivre sa voie vertueuse. Notons d'ailleurs le contraste très révélateur posé entre ces deux lectures du personnage de Cassidorus
   que le châtelain fait saint ermite là où son épouse ne voyait en lui qu'un paysan. Il s'agit donc de la seule personne que Cassidorus
   parvient à berner, cette interprétation se voyant malheureusement directement corrélée à sa perte.
   Dame, dist il, or ne m’en demandez plus, car vous n’en savrez orendroit plus 
      parde 
      moimoi qui il est.Conment, sire ! dist elle. Avez vous donc couvenances a lui 
      quequi vous ne voulez pas que je 
      sachesache qui il est ?Dame, dist il, voirement les y ai je.Voire, dist elle. Par monseingneur saint Nicholas, mar 
      lesle celerez pour moi ! 
      Et sese il ainsi est que je n’en sache la verité, mal ira li asfaire !DameDame, dame, sousfrez vous de ce dire, 
      que par la foy que doidoi a l’ame mon pere, 
      se je savoie que vous neen 
      feissiezfeissiez ne desissiez chose qui a courrous me tournast, 
      je vous courrouceroie du cors ! Conment doncdeable ! 
      Se, dist il, couvient il queque i 
      preudom ou une preudefame ne 
      pourrapuist pas 
      estrevivre en ceste ville 
      que vous ne vueilliez savoir sa confessionconfession ne qui il est ?Sire, dist elle, si puet bienbien faire, 
      ce me semble. Mes je ne cuidoie pas qu’il i eust si grant confession 
      conme ilconme vous faites samblant qu'il y a. \pend
\pstart Pas de nouveau § BAinsi demoura atant li afaires. 
   Mes li deables, qui par le mauvés conseil conchie a la fois 
   leles bon, mist en la chastelainne 
   uneune male penssee dont moult fist puis d’anui, 
   car il li fu avis que cilz homs 
      qui la avoit esté mandez eust une acointe qui son seingneur vousist veoir avant que 
      ilele sceust qu’qui
      elle fust. 
   Et de ceste pensee ne la meist nul hors, tant estoit elle mal penssant. Et qu’en avint il ? 
   Il ne demoura mie ii jours aprés que la chastelene mist 
   son seingneur a reson, si li demanda :
   Sire, ne demandastes vous mie a cele honme qui 
      cele besongneces beles oevres avoit 
      faitfait que cele pucele aporte a vendre chascun joesdi ?DameCertes, dame, dist il, nennin, 
      je ne li demandai autre chose que ceste. Mes bien i cuit recouvrer dedens 
      leur termine qu’i m’a ditbrief terme.Sire, dist elle, bien vous en croi. \pend
\pstart Pas de nouveau § BCeste chose demoura ainsint. 
   Assez pou aprés anuia au chastelain 
      de ce que li emperieres demoura tant que il ne reperoit a lui, si conme il vousist. 
      Dont il avint qu’il ne se pot tenir qu’il ne venist la ou il ouvroit touz les jours. Lors si li dist qu’il ne s’en peust aparcevoir. 
      Sire, je sui a grant meschief que vous ne venez a moi parler.Chastelain, dist illi empereres, 
      je ne puis pas a vous parler a ma volenté, car j’ai si grant doute que je ne soie conneus que trop en sui en male penssee.Sire, dist illi chastelains, 
      n’enne doutez ja que par moi nulz en sache riens.
   Nouveau § BAdonc pristrent i jour entr’eus ii que il se mistrent ensemble secreement, 
      par quoi li chastelains sot la ou li emperieres demouroit. 
   Lors li dist li chastelains quique 
   mout avoit grant couvoitise qu’i peust estre ses compains en aucun lieu 
   ou il peustpour 
   ferefere cesfere les bonnes oeuvres, 
   aussi comque il fesoit.
   Chastelains, dist il, je 
      vueilvous acompaigne a moi bien que vous en soiez compainz en bonne maniere. 
      Et touz ceus entour quique je pourrai perseverer et fere chose quil plaise 
      a Nostre Seingneur.Sire, dist il, Deux le vous rende ! Mes je di encore en autre maniere que j’ai entendu. 
      Je couvoite que je peusse mener ma fame a ce qu’elle me vousist croire 
      en bonne maniere conme j’ai entendu que la vostre fait.Par Dieu, dist li empereres, aussi voudroie je de par 
         Nostre Seigneur !Certes, dist li chastelains, j’en seroie moult joians. 
      Mes ele m’est moult contraire en pluseurs manieres, si ne sé quel conseil j’en puisse avoir. \pend
\pstart Pas de nouveau § BLorsLors li 
   conta li chastelains 
   a l’empereeur 
   une partie de lasa maniere 
   de sa fenme,
   et li emperieres li dist :
   Dous amis, touzjours ai je oï dire que li preudoms si 
      fetfet volentiers la preudefenme, 
      et la preudefenme le prodome. Et se vous savez chose qui meilleur li soit a fere, 
      si li enseigniez debonnerement, et nous prierons a Dieu qu’i la convertise et li doint sens et entendement du retenir.Et Dieux vous en oïe, sire !, dist li chastelains.
   Nouveau § BLors se departirent atant li empereres et 
   li chastelains, et ala puis li chastelains souventes fois la ou 
   li emperieres estoitouvroit et traveilloit, 
      si comme j'ai devant dit pour savoir se par aventure li deist aucune chose qui bonne li fust. 
   Mais sanz faille il parloient si couvertement li uns a l’autre que nuls ne se peust apercevoir qu’il eussent riens a fere ensemble,
   pource qu’il ne vouloient pas que nuls peust cuidier que ce fust si grant chose 
   quecomme de lui connoistre.
   Et de ce li savoit li emperieres moult bon gré, car il veoit apertement que il n’estoit chose, 
   se il li requeist, qu’il ne fust appareilliezaprestez du faire.
   Et qu’en avint il ? La fame au chastelain,
   que li deables avoit en cure prise, La précision ainsi redoublée de l'emprise diabolique que subit
   la châtelaine ne peut laisser aucun doute sur l'issue des évènements, dans une portée très manichéenne de cet affrontement 
   entre inspiration divine et diabolique. qui ne pooit venir a chief de ce qu’ele avoit en propos, se ce n’estoit par la jalousee 
   i sien escuier qui trenchoit au mengier devant lui et li dist :
   IlAmis, il couvient que tu me faces 
      une chose que je te dirai, en maniere que il ne sera chose neïs de mon cors vergonder que je ne face por toi se tu m’en requiers, 
      et je te dirai quelequele chose tu me feras. Je sui certainne que 
      cilz paÿsans qui a mon seingneur parla l’autre 
         jourjour si longuement,
      si croi que c’est i truant et i 
      ypocritefaus ypocrite et 
      vientvient a mon seigneur atraire 
      pour je ne sé quele truande ou chanlande que il tient. 
      Je ne sé en quel lieulieu et est cele qui a fet cel ovrage 
      de quidont tu nous oïs parler, qui est si bien faite que nuls ne la pourroit miex 
      faire se ce n’estoit par l’art de l’anemy. Si vueil que tu te donnes garde se tu peusses savoir riens de leur couvine. 
      Par quoi il peust avoir aucune chose qui contre moi peust estreestre, 
         et se tu le pues en nule maniere savoir, si le me di, et je ferai du tout ta volenté. \pend
\pstart CilzQuant cil ot entendu sa dame qui ne fu pas 
   loiaus serjans, ainz fu couvoiteus de faire lasa volenté 
   sala dame, 
   qui estoit plus engrant de faire et d’oïr males nouvelles que bonnes. Li dist au conmencement :
   Dame, s’il estoit ainsi que je traïsisse mon seingnor 
      pourpar vostre volenté acomplir, 
      ce seroit chose moult estrange, car vous savez bien qu’ilque mon seigneur 
      n’a serjant en son hostel qu’il croie autant com il fet moi. 
      Et si ne cuidiez pas qu’il ne l’etlez trouvé en moi en aucune maniere. 
      Et s’il ne li a trouvé, si en ai je grace conment que ce soit, pour laquele chose 
      je vous diraidirai une chose, ma tres chiere dame, a coi vous vous acorderez, 
      et tout par couleur de verité. On dist em proverbe quequ‹i›[e] 
      quiconques ainme miex de mere, si est fausse nourrisce, 
   pour quoi je di que vous devez estre mere de mon seingneur amer. Si vous pri, quele que la vostre volenté soit, 
   que vous i gardiezgardez vostre honneur et la moie, car sachiez, 
      que qui m’en doie avenir, je ferai vostre volenté par ainsi  que je tout premierement 
      feraiaie ma volenté de vous par quoi je soie asseur de vous.FastreFrere, 
      dist la dame, sages estes et apenssez, et je vueil bien faire vostre volenté pource que vous 
      soiez mieusmieus apensez et miex entalentez de fere ma requeste.
   Le nom de ce personnage, adjuvant de celle qui se fera la plus grande et la plus notable ennemie de Cassidorus, 
   est évidemment symbolique : Fastre, associé, de manière explicite dans l'appellation que lui donne le châtelain ci-dessous, au fatras, 
   dénote ainsi les sens donnés à ce substantif en mfr : de ramassis d'idées confuses, de balivernes, mais aussi de tour, ruse, artifice 
   (DMF). L'onomastique se veut une fois encore révélatrice, comme pour pallier à l'absence de nom conféré au personnage de la châtelaine,
   dont l'anonymat n'est pas sans rappeler celui de la vile impératrice du roman souche bien sûr. Le nom de l'écuyer offre ainsi un indice
   supplémentaire, au-delà de la précision déjà donnée de son peu de loyauté et de sa prise de position bien affirmée dans 
   cet échange avec la châtelaine, de sa vilenie. \pend
\pstart Que vous iroiediroie je 
   plus celant de leur affaire ? 
   La dame 
   n’i regarda honneur ne foy ne loyauté, ainz se mist en servage envers celui a qui elle ne 
   vouloitdeveoitveoit chose qu’il 
   eustvousist avoir de lui, si que il avint que 
   li escuiers meist volentiers chose sus a 
   son seingneur par quoi il deservist a estre bienvenus de sa dame. 
   Si dist a son seingneur que moult avoit grant merveille de ce 
      qu’il se celoit envers lui de cel honme a qui il parloit si souvent et si ne savoit que ce 
      vouloit diredire. Et quant li chastelains ot entendu Fastre, si li dist :. 
   Fastras, dist 
      li chastellainsil, par temps le savrez.
   Lors ne demoura pas longuement qu’il s’en vint a l’empereeur 
   la ou il traveilloit a l’ueuvre, et dist tant que 
   illi empereres li otria a l’endemain de venir a lui en 
   son recet, car aussi estoit il dymenche, que on ne devoit pas fere labour de mains. 
   Si avint que li emperieres s’en vint oïr messe au 
   chastel et amena o soi le chastelain et 
   son escuier, en qui il se fioit moult. Lors n’ont finé, 
   et tant qu’il vindrent en son recet, qui estoit merveilleusement 
   bel com en tel lieu. \pend
\pstart Quant l’empereris vit le chastelain, 
   si mua une merveilleuse couleur, et nonpourquant l’apela elle moult courtoisement et l’embraça moult humblement.
   L'émotion de Fastige est doublement révélatrice : du danger de cette irruption du châtelain, mais surtout
   de l'écuyer qui l'accompagne, mais aussi de ses qualités, elle qui parvient à dépasser son étonnement pour faire preuve de la courtoisie 
   attendue.
   Et dist :
   Ma tres douce dame, loez soit li glorieus pere 
      du ciel qui tele honneur m’a faite quant ilqui vous a amené en ses parties 
      et me doint fere chose a vous et a vostre compaingnie qui vous puisse 
      plaireembelir.Sire, dist elle, de la vostre grant humilité vous sache Dieu gré !
   Lors se sont assis touz iii, l’un asez pres de l’autre, et tant que 
   li emperere fist signe au chastelain que il feist trere arriere 
   son escuier pour dire plus priveement ce qu’il leur plairoit, et il si 
   fistfu. Et quant il fu hors de la meson, 
   si vint aussi conme cilz quil ne s’en donnoit de garde. Se trait ensus 
   le lyon, qui senti que cilz n’estoit pas venu pour l’amor de 
   l’empereeur ne de l’empereris. Lors se dreça et 
   fist moult laide chiere, conme cilz qui l’eust devouré se ce n’eust esté pour l’amour de l’empereeur, 
   car il savoit de sa nature que il ne l’en savroit ja gré. Si conmença moult forment a grondir, et cil vint arrieres criant :
   Aide Deux ! Que ferai ?
   La réaction du lion est significative, plus encore que celle de l'impératrice, dans un jeu d'échos d'ailleurs
   notable, du caractère peu fiable de cet écuyer dont le châtelain ne se méfie pourtant en rien (Cassidorus non plus d'ailleurs, malgré
   toutes ses précautions antérieures) et vient s'ajouter aux effets d'annonce immanquables de l'issue de l'histoire. \pend
\pstart Pas de nouveau § BDont sailli 
   li empereres avanten piez et 
   vint en l’encontre du lyon et dist :
   Conment, sire compains ! Ja ne savez vous qu’il 
      est venusvint ceens avec moi ?
   Dont prist une vergette et le feri sus la croupe, mes onques pour ce n’enne lessa 
   lea grondir, 
   ainz pristne a faire moult
   laidemauvaise chiere. 
   Mes li emperieres li conmanda que il se tenist em pes, 
   et il se lessa cheoir a ses piez aussi debonnerement conme uns aniaus. Lors s’en vint li emperieres 
   raseoirarriere seoir delez 
   l’empereris et 
   le chastelain, et parlerent 
   longuement‹b›[l]onguement ensemble de leur affere. \pend
\pstart Pas de nouveau § BAtant prist le chastelain 
   congié a l’empereeur et a l’empereris et revint a 
   Gomor a heure de vespres, si trouva la chastelainne 
   qui l’atendoit et quiqui ne vouloit qu’il seust que elle 
   estoitfust engrant de savoir dont il venoit 
   et ou il avoit estécomme elle estoit. 
   Si avint que, quant il fu reperiez, elle dist :
   Il me semble a vostre aler que vous estessoiez 
      lassez.Dame, dist il, ce n’est pas merveille, car je n’alai mes pieça tant a pié conme je ai hui fait.Et dont venez vous ? ditce dit elle.Dame, dit il, couvient il que vous le sachiez ?sachiez ?"
   "Sire, dist ele, non s'il ne vous plast. Mais je vous pri tant com je puis que vous le me diez.
   Lors lala chastelaine prist par la main et l’enmena en sa chambre et li dist :
   Dame, on vous mande par moi salus, et si vous envoie l’en une aumosniere qui moult est de grant valeur.
   La chastelainne pristvit et prist 
   l’aumoniere et connut bien que ce estoit de l’ouvrage dont elle avoit tant couvoitié a 
   avoiravoir, si dist :.
   A non Dieu, siresire, dist, 
      cil ou cele pour qui amour 
      on lale m’envoie
         envoie, cil ou cele qui que ce soit ait bonne aventure ! Or me dites 
      doncdonc, dist la chastelaine, qui 
      est cil ou celec'est 
      qui lale 
      m'envoiem’a envoiee.Dame, dist illi chastelains, 
      ce est une dame que je couvoite moult que vous l’aiez veue et 
         que vous feussiez de sa maniere en touz biens et en touz bons poins.Et sanz faille, dist la dame, vous ne deistes hui parolle 
      dont si bien vous creusse. Ha ! sire, si vous semble ore cele dame de si noble maniere tant miex 
      que moi ? Certes, sire, or gardez l’aumosniere pour la seue maniere, puisquequi 
      tant vous plest, car ja la moie je‹n›[je] ne 
      quierquit changier pour chose que je 
      voiesache qu’il y ait a reprendre ! \pend
\pstart Atant rejeta la chastelainne l’aumosniere au 
   chastelain, et il en fu touz courouciez et dist :
   Dame, touzjors ferez vous des vostres !En non Dieu, sire, dist ele, mes vous, car onques foy ne honneur ne loyauté vous ne me portastes, ne ja ne ferez ! 
      Mesque une 
      chanlandechan‹d›[l]ande, quele que ele 
      soitfust, ne vous semblast touzjours 
      meilleurde meilleur maniere 
      deque moi !
   Lors ne se pot tenir le chastelain qu’il ne deist :
   Ma tres chiere dame, 
      cele qui l’aumosniere vous 
         envoienvoioite n’est pas chanlande, 
   car je ne cuit dame ne pucele de sa valeur en toute ceste contree. Et si me prissiez ore moult pou quant vous avez tele penssee envers moi, 
   qui cuidiez que d’une chanlande vous aportasse tel present.
   Lors jeta la chastelainnedame 
   une grant rafarde et dist :
   Or poez oïr du saint hermite qui hui a esté convertis en i pas d’asne.
   La réplique de la châtelaine nous laisserait presque attendre un récit enchâssé à la manière de cette 
   technique narrative bien cultivée dans les premiers temps du Cycle des Sept Sages surtout. Il offrirait un écho notable 
   au mauvais conte que sert également Pelyarmenus à son frère, dans une démonstration frappante de l'orientation prise par ces 
   discours à la portée sapientiale détournée. \pend
\pstart Quant le chastelain oÿ que 
   la chastelainne prisoit si poi ses dis et tournoit tout a truffle, 
   si sailli a l’uys de la chambre et le ferma et puis revint a la chastelainne et dist :
   Dame, aucune fois ai je oï dire que li preudoms si doit fere la preudefame, 
      et je ne cuide se touz biens non en ce que je vous ai fet et dit, et vouspuis 
      lechose que je puisse dire prisiez moult pou, 
   pour coi je vueil que vous sachiez que je sui chastelains et vous chastelainne, qui estes mal chastoiee. 
   Mes par lacele foy que je vous doi, se je 
      sé plusmieux que vous
       plus vous oy parler, je vous doi chastierchastieray 
      conme chastelains doit fairefet chastelainne.
   Lors la prist a une main, et a l’autre l’en donna tant et desus et desous que elle dist :
   Ha ! sire,voir se dit qui dist : 
      a tart bat cul qui ne bat culetCette 
         expression sentencieuse n'est pas répertoriée chez Morawski ni chez Le Roux de Lincy. à vérifier PD stp. !
   L'insertion de ce proverbe, vulgairement connoté, offre un contraste saisissant avec le propre jeu 
   de citation auquel recourt le châtelain, qui reprend la leçon de Cassidorus, via un plaidoyer remarquable sur le sème de châtelain.
   Atant la lessa li chastelains, et elle demoura en sa mauvesse pel, qui estoit moult esprise de mauvés 
   art. La mention de la peau et du mauvais art offre une référence assez explicite au héros éponyme 
   du Roman de Renart, réputé pour sa fourrure rousse et l'art auquel il a élevé la ruse. 
   Lors ne fun'ot onques em 
   pes ne ne finacesse 
   devanttant que elle ot parlé a celui 
   Fastre a qui elle enquist moult s’il avoit veu chose ou 
      il eust point de soupeçon. 
   Cilz, quil nese doutoit qu’il ne deist chose 
   dont il fust aperceus, dist :
   Dame, sus m’amel'ame de moi, 
      vous avez trop grant tort, qui mon seingneur avez en soupeçon, 
   car je ne vi onques en lui chose qui a reprendre feist. Et de l’autre part vous estes trop hastive, 
   car encore est la chose si nouvelle que trop seroit la fenme habandonnee ou il avroit si tost vilonnie.Par Dieu, tu dis voir, dist elle. Mes or me di ou 
      li paÿsansil 
      demeuredemeurent et quele fame 
      ceele est.Dame, dist il, foyfoy que que je 
      doi a vous, que je cuit que, 
   se toutes les dames et les damoiselles qui onques furent estoient toutes en une piece de terre et cele y feust dont vous 
      me demandez, que de valeur, que de sens, que de biauté, que de parler, 
      que de toutes graces elle en enporteroitavroit l’onneur 
      et toute la seingneurie.Je ne sé, dist la chastelainne, 
      que ce est, mes aucune chose vous donne elle qui si la loezavez loee.
   Lors li moustra cil l’aguilier que elle li ot donne et dist :
   Dame, voirement m’a ele donne cest aguillier, et si vous envoia une aumosniere 
      par 
         mon seingneuret dist que on le vous donnast, 
      mes je ne sé se vous l’avez eue.Et que donna elle a vostre seingneur ?Si m’eïst Dieux, dist il, neent. Mes son seingneur 
      li en donna une autre a merveilles riche, et a moi donna elle cestun 
      aguillier en amende de ce que leur lyon me dut avoir devouré. \pend
\pstart Quant la chastelainne oï ce, si dist :
   Ont il dont ii tel 
      lyon avec eulz qui les gens veult devourer ?Par foy, dame, oïl. Qui envers euls avroit male penssee aussi conme je oi.En non Dieu, dist elle, dont n’iroie je mie volentiers selonc ce que je n’ai mie bonne oppynion ! 
   Et nonpourquant, si me couvient il veoir cele dame que tu m’as tant prisiee.Je ne puis veoir, dist il, que ce puist estre tant que vous aiez la maniere 
      qui est en vousque vous avez.Or m’en lessel'ai covenir, dist elle. \pend
\pstart Pas de nouveau § BAprés ce passa li courrous du 
   chastelain, 
   qui moult volentiers queist tour par coi il peust tourner la chastelainne 
   dehors de sa male erreur. 
   Cele, d’autre part, ne chaçoit mesfors que elle le peust decevoir. 
   Et il estoit trop legier a fere, pour quoi il avint que 
   lale chastelainne vint au 
   chastelain et li fist i faus ris et dist :
   Sire, or sai je bien quili quelz a tort 
      de moi et de vous.Dame, dist il, il me plest moult bien que vous le sachiez.Aussi fet il a moi, dist elle. 
      FrastresCar Frastres m’en a mis 
      tout em pes, qui aussi a des joiaus com 
      vous avezles autres.Dame, dist li chastelains, 
      car il les al'ales  
      aussi bien deservi conme vous avez or fet.Taisiez vous ! dist elle. 
      Touzjours a on tort a vostre dit.
   Lors li mist la fausse chastelainne 
   lessus lui les bras au col et le cuida besier, 
   mes il se trait arrieres et dit :
   Dame, riens ne vous vaut, 
      car avant avrez vous conneu vostre fol cuidier que vous preingniez pes a moi !Sire, dist elle, en non Dieu, je le connois. 
      Mes que je aie l’aumosniere que ma bone amie m’envoia que je onques ne vi.Dame, dist il, elle ne vous est pas moult eslongniee. Mes que vous vueilliez croire bon conseil.En non Dieu, sire, de vostre conseil croire 
      que ne me quiercuit 
      je ja departir. \pend
\pstart Atant aporta li chastelains l’aumosniere et 
   li dist :
   Dame, ne cuidiez pas qu’eleque ceste 
      viengne de vilain lieu, 
      car je cuit que de meilleur dame ne fu onques aumosniere faite.Sire, dist ele, si bonne et si belle conme ele est me plaist il que vous l’aiez, 
      puisque vous n’en eustes nule.En non Dieu, dame, sauve soitsoit la 
      vostre grace, je n’i failli pas, car messires son mari m’en donna une qui n’est une mains 
      vaillant.Sire, distce dist 
      la chastelainedame, 
      en la bonne heure ce soit ! \pend
\pstart Pas de nouveau § BUne fois entre les autres avint que li chastelains fist venir 
   l’empereeur et l’empereris 
   en sa meson priveement, si qu’il ne fu nuls qui riens en seust 
   forsfors que 
   le chastelain et Fastres, sanz qui il ne pooit estre asouvis. 
   Et adonc fist li chastelains fermer la meson a l’empereeur. 
   La chastelainne, quil ne queroit avancement de nullui fors que por sa volenté acomplir, 
   sot et entendi que li ouvrier estoient en la forest pour fere 
   la cele a 
   l’empereeur. Dunt s’en vint a celui Fastres et li dist :
   Tres mal aventureus gars, ja ne m’avoies tu pas dit que li ouvrier fussent a fermer 
   Robur, la maison a cel hermite ? 
      Si me di ou il sont ore tant que on met a ce faire.Ha ! dame, souffrez vous tant que je sache a quel fin se pourra 
      venirtorner, 
   car je vous di que cele dame est ceens et 
      nene le 
      veultveult enseignier 
      monson seingneur 
      que nulson l'i sache por le meschief qu’il 
      en porroit avenir de 
      cel deable de lyon qui nulluy n’espargneroit tant que 
      son seingneur n’ine li 
      seroitfust present. 
      Et pour ce cuit je vraiement que on le vous cele, car vous i 
      voudriez aler et li lyons 
   vous voudroit devourer se vous aviez vers eulz nule male penssee.Voire, dist elle. Est ce dont la raison ?Certes, dist il, je n’i sé autreautre chose 
      que ceste, et ainsi le m’a dit mes sires.Ce ne vaut riens, dist elle, car je la verrai avant 
      queque je de ceens me departe.Dame, dist il, pour Dieu merci, ne faites nesun semblant que vous riens en sachiez, 
      car il seroit seu que je le vous avroie dit.N’enNe dotez ja, 
      sire chietis. 
      Je ne sui pas si nice ne si folle que je ne quiere bien achoison 
      quepar quoi je vendré a ce que je voudré
         . \pend
\pstart Lors vint la nuit que li chastelains vint de 
   RoburRobur, c'estoit li lieus ou il faisoit
   fermer la celle a l'empereourLa précision de B dénote de la forme de saturation des informations 
   liées aux noms de lieux et de personnages dont procède l'ensemble du Cycle. Ce manuscrit témoigne fréquemment du souci de venir expliciter 
   les identités et les lieux mis en présence., et la chastelainne si li vint a l’encontre 
   moult amiablement, etqui li dist :
   Sire, encore n’eussiez vous pas conmencié cest afere dont je cuit que vous venez se je ne fusse.Dame, dist il, vous avez voir dit.Et conment est, ce dist ele, que vous ne les avez ceens amenez ?Dame, dist, je le vous dirai.
   Lors li conta l’achoison du lyon, 
      conment il vouloit touz ceus devourer qui avoient vers eulz nule male penssee.
   Ha ! sire, dist ele, pour Dieu, truffles sont ! Autre chose y a. 
      Ne puet la damela dame et il 
      nul lieu aler sanz ce lyon, 
      en nul lieu de coi on me fet si grant mencion ?Dame, dist li chastelains, si feroit bien. 
   Mes cil qui touzjors se doute n’est pas merveille, si se tient volentiers sus sa garde. 
   Et pource que je veil que vous voiez la bonne dame et sa maniere, 
      je la vous ferai veoir‹s›[v]eoir, se je puis, sanz le lyon.Sire, dist ele, moult grans mercis.
   Nouveau § X2Atant se faintfenist 
   le chastelain et vintvint a l'empereour 
   en une chambre ou l’empereeur
      et l’empereris estoient loing de 
   son hostell'hostel et li dist :
   Sire, conment que ce soit, j’ai en couvent a la chastelainne 
      que je li ferai veoir 
      l’empererisma dame l'empereris. 
      Et vous savez bien que quanque ce soit li convient 
      ila veoir.Amis, dist li empereres, puisqu’il le couvient, si la verra,
      mes vous savez bien que ces fenmes sont si merveilleuses que, quant il sevent aucune chose qui face 
         a celer, que adonc seront elles plus engrant de la descouvrir que cele ne fesoit mie a celer. 
         Aucunes en i a, je ne di mie 
         toutes, car ce seroit meschiez se toutes estoient tochiees d’une tele tache, 
         mes je ameroie miex que 
      elleil la veist hors de ceens 
   en nostre ceule que ailleurs, et vous dirai la reson pour quoi. Il me semble que, se ele aloit ja en 
      vostre chambre ou en aucun lieu ou la chastelainne fust, que ce seroit une chose moult 
      diverse. Et je ne vueil que en nule maniere la dame 
      aillevoist en nul lieu 
      que mon lyon ne soit o lui, car je m’aseure de touz ceus a qui il fet bele chiere.
      Et vous lo queque se vous ne cuidiez en 
      bonne foy qu’ele soit en nulle soupeçon de nous ne de nostre affaire que vous ne vous penez d’amener 
      la ci, car je me doute que mal n’en venist.Sire, dist li chastelains, 
      foipar la foi que je vous doi, ja ça sus ne la ferai venir, 
   car il n’est nul qui sache mie bien que male fenme pensse.Ja ne set elle pas que nous soions ceens ?SireSire, dist il, sachiez que ci fait.Or vous soufrez, dist il, et je vous savrai a 
      diredire assez tost a dire  
      quant je voudrai que elle 
      lama dame 
      viengne veoirvoie. \pend
\pstart Atant se parti le chastellain de 
   l’empereeur et revint a la dame, 
   qui mout bien sotsavoit dont il venoit. Et ele li dist :
   Sire, j’ai grant merveille que vous alez chaçant, car vous me semblez aussi conme touz esfraez.Dame, dist il, si sui je, 
      car je nene vous 
      vueil ore plus celer que ces bonnes gens sont ceens en tele maniere 
      queque il ne vuelent que 
      nuls ne nule les i seustsachent.
   Lors"Non, sire, dist ele, et pour quoi le font il ?"
      "Dame, dist il, je le vous dirai." Lors li 
   contaconta li chastelains 
   pourquoila raison pourquoi c'estoit.
   Sire, dist elle, je m’en souffrerai, puisqu’il en font si grant hernois 
      aussi bien conme il font.Dame, dist il, et je vous em pri, car vous 
      lesle verrez ailleurs ou il ne vous coustera pas tant conme il feroit ci.En non Dieu, dist elle, bien pourra avenir. \pend
\pstart Moult cuida bien avoir li chastelains mis en bonne pais 
   la chastelainne, mes non fist, car elle se penssa qu’ele 
   les iroitl'iroit veoir, que qui 
   ilil li enil li deust 
   cousteravenir. 
   Adonc s’en embla de son seingneur et 
   fist semblant qu’ele feust i pou deshestiee. 
   Si fist venir Fastres a lui et li dist :
   Je ne puis faire en nule maniere que je 
      voiepuisse veoir cele dame que tu m’as tant prisiee. 
   Mes foy que doi a Nostre Seingneur, je la verrai avant que je dorme, mes que j’en deusse recevoir grant honte ! 
   Si couvient que tu viengnes avec moi, et je la verrai au dehors de la chambre par aucun lieu.Dame, dist il, je vous jure que, se vous i alez en tele maniere, que il sera seu. 
      Et se je i aloie eta 
      mon seingneur le savoit, je seroie deshonnorez.Va au deable ! dist elle. Tu ne vaus pie ne autre oissel ! Or saches que je irai avant toute 
      seuleseule que je ne la voie.
   Dont s’en vint la chastelainne en une garde robe, si s’apresta et s’escourça haut 
   pour miexmiex veoir et 
   atournerfaire son afaire qu’ele avoit empris, 
   puis prist une de ses chamberieres o lui et vint 
   lacele part ou 
   li emperieres estoit parmi i jardin, et n’arresta onques, 
   conme cele qui avoit le deable ou cors, tant que elle vint en la chambre ou li empereres 
   et l’empereris soupoient, et tant que ele vit parmi i pertuis 
   la dame, qui se seoit delez son seingneur 
   en signe de moult grant amour. Et ainsi conme ele guetoita ce, 
   li lyons senti 
   la chastelainne au dehors ou ele veoit l’empereris 
   moult volentiers. Lors sailli cele part, aussi conme touz hors du sens, et la chastelainne 
   ot si grant paour que ele cuida estre maintenant devoree, si se mist a la voie le plus tost que elle pot, 
   et tant qu’ele vint trebuchant en une fosse, le chief avant, si que l’yaue reclost desus lui. 
   Sa pucele, qui venoit 
   aprésaprés li, vit que 
   sa dame gisoit ou fossé, si conmença a crier :
   Haro ! ma dame noiese noie !
   Et Frastres, qui en aguet estoit si conme cil qui vouloit savoir conment li affaires iroit, 
   vint a ce copcop la que li deables 
   l’emportoitli aportoit et vit 
   sa dame ou fossé, si sailli aprés, 
   si cuida sa dame rescourre, mes assez 
   otot il avant beu. 
   Si les cuida uns deables noier 
   entr’eustous 
   ii quant i autre deable vint, qui leur aida, par quoi
   cilz Fastres vint a la chastelainne. Si l’embraça et 
   la mist hors du fossé au miex que il pot. 
   Lors cuidacuida il bien que sa dame 
   fust noiee, si ne sot que fere, car il n’osoit faire 
   noisechose dont noise fust oÿe pource que 
   li chastelains l’oïstoÿst et le seust. 
   Il cuidast estreet qu'il ne fust honnis, ainsi conme se ce feust par lui. \pend
\pstart La chastelainne, 
   aussi conme vous avez desus oï, avoit tant beu par bouche et par nés 
   de l’yaue 
   du fossé que elle eust esté noiee 
   se ce ne feust uns deables, qui avoit esperance que par lui avendroit encore 
   pispis se il pooit, 
   dont il avient a la fois que aucuns eschapent de aucun peril que il veulent dire que ce fet Dieux qui le 
   consentconsent et qu'il les en gete et sauve
   On note l'emphase mise sur l'emprise du diable, dont le nom apparaît à sept reprises sur ces deux paragraphes, 
      et ce en contraste avec la promesse que la châtelaine prononce au nom de Dieu. Son erreur d'interprétation paraît ainsi résumée,
      et éclairée par ce commentaire à portée morale. Cela joue aussi bien sûr de l'effet d'annonce, en soulignant la fatalité de la fin 
      de Cassidorus que cette curiosité excessive finit par occasionner.. 
   EtMais nous trouvons ci 
      endroitendroit en escript que ce fet aucunes fois le deable qui 
   les sequeurt pource que il onta esperance que il facent encore pis, 
   pourpar quoi il y a aucuns qui pas ne se chastient quant il eschapent 
   d’aucun peril, neent plus que fist la chastelainne, que li deables secourut 
   ainsi com vous avez oï, 
   quecar, quant cilz Fastres l’ot ainssi 
   sachiee du fosséfossé, si cuida qu’ele fust noiee, 
   elle s’en volt fouir. 
   QuantCar il senti en lui qu’ele ne 
   fuestoit pas morte, si en ot moult grant pitié, 
   conme cil qui moult l’amoit par naturenature moult grandement. 
   Dont s’apenssa d’une chose, qu’i li leva amont les rains et la teste contreval 
   et rendi l’yaue que elle avoit beuepar la bouche, si que elle revint 
   aussi conme a lui meismes. 
   Dont l’emporterent cilz Fastres et la pucele 
   en unela garde robe qui la iert, 
   si conme j’ai desus dit. \pend
            \pstart Aprés que ce fu avenu, les tables furent mises, et se devoit 
   li chastelains asseoir au souper. Si demanda la chastelainne, 
   et l’en li dist que elle estoitestoit un poi 
      deshestiee. Lors ne se volt 
   tenirlaissier que il ne venist a lui, 
   si la trouva tele conme cele qui ne pooit mot dire. SiLors li 
   demandademanda li chastelains 
   conment ce li pooit estreestoit avenu 
      en si pou d’eure. 
   Nuls neLa chastelaine ne sot que respondre et ne fu nulz qui li osa giehir 
   cel afaire, et il s’apenssa lorsmaintenant que 
   ce estoit de paour qu’ilelle avoit eu. 
   SiLors li enquist et demanda 
   se ele avoit esté le jour en nul lieu.
   Sire, dit une damoisele, ma dame 
      alafu anuit en ce jardin, 
      sisi en revint 
      moultsi atainte 
      de malet si malade, 
      siqu'ele se coucha ainsi conme vous poez veoir.
   Ha ! dist il, bien est !
      Quant li chastelains ot entendu la damoisele, si s'apensa et dist que bien savoit que ce pooit estre.
   Atant s’en vint li chastelains la ou li empereres 
   et l’empereris estoient et les trouva moult courrouciez, 
   etLors leur demanda 
   qu’il avoient et quele chiere il faisoient.
   Chastelains, dist illi empereres, 
      sije la fais tele conme li temps l’aporte, et dont venez vous ore a 
      ceste heure ?Sire, dist il, je vieng d’un lieu ou aucun avroient bien mestier de miex qu’il n’ont. \pend
\pstart Pas de nouveau § BLors ne volt 
   li chastelains celer conment il estoit a 
   sa fenmela chastelaine 
   etet dist qu’il cuidoit bien que ele eust la esté, par quoi ce li estoit 
   avenu.
   Chastelains, dist il, pour ce ai je aucune fois veu 
   qui conseil ne croit son conseil ne fait a resoingnierVariante du proverbe
      "Qui conseil ne croit dolent s'en voit" (Morawski 1872) avec une orientation intéressante sur les effets du conseil en question###PD ?. Encore me venist il assez miex a estre en ma petite 
      mesonnette et prendre ce que Deux m’aeust pourveu que ci feusse venus pour 
      nulmal destourbier.Ha ! sire, dist li chastelains, tout ce est 
      veritezveritez que vous dites ! 
   Et pour Dieu, puisque ainsi est que autre chose ne puet estre, dites moi ceste aventure, 
      car je ne truis chiez moi qui la veritéle voir m’en 
      dievueille dire.Sire, dist li empereres, du voir vous en dirai 
      je ce que j’en sé. \pend
\pstart Pas de nouveau § BVeritez 
   estil est que nous soupions anuit bien et em pais, 
   conme ceuscil qui ne se 
   donnoientdonnoit garde de nullui qui ceste part deust venir, 
   mais que nostre lyon sailli a cele fenestre et senti, 
      mes jemes mais je ne sé ce se fu 
      la chastelainne ou autre. 
   Mais tant vous di je bien, si m’eïst Dex, que je cuidai de la male vie que il menoit pour 
   issirissi hors qu’il nous deust ceens devourer, si que je ne soi que faire. 
   Sifors que je sailli a m’espee et 
   mem‹i›[e] mis entre lui et ma fenme 
   pour lui occirre, se je peusse. Et quant il vit ce, si se mist a merci 
   devantcontre moi, dont je lo le glorieus roy de paradis, qui 
   ainsianuit nous a garanti de mort. \pend
\pstart Quant li chastelains ot entendu 
   l’empereeur, si se mist a ses piez et dist :
   Ha ! sire, pour celui Dieu que vous 
      creezservez vous cri je merci.Chastelains, distce dist 
      illi empereres, levez sus et si ne le faites jamés.Sire, dist il, je ferai la vostre volenté. 
      Mes de ce qui est avenu a la chastelainne ne sui je pas dolens, car je li avoie conté toute
      la nature du lyon, conment il ne puet nullui soufrir qui male volenté ait vers vous. 
      OrMais or puet elle bien savoir la verité, 
      sipar quoi elle 
      s’ense puet chastier.Mais or me dites, dist li emperieres, 
      qui li avoit dit que nous estions ceens ?Ce vous dirai je, dist li chastelains. 
   Il est verité que elle savoit bien que li ouvrier estoient en vostre ceule. 
      EtEt que pour ce covenoit il que, tant que li ouvrier i estoient, 
      que vous feussiez ailleurs, si li dis que vous estiez ceens. Et pour ce que je li eusse celé et ele le seust par aucune 
      aventure qu’ele n’i eust nule male penssee, ainz li dis aussi conme en confession.Je m’i acort bien, 
      dist li empereres. \pend
\pstart Pas de nouveau § BAtant prist 
   li chastelains congié 
   et s’en est reperiez aussi conme s’il n’en seust riens et s’asist au souper. 
   Et fu mout tartavant en la nuit avant que il fust fais. 
   Et quant ce vint aprés souper, si vint a la chastelainne, 
   ausi conme s’il ne seust dont ce li fu venus. Si trouva qu’ele estoit revenue a lui, et li dist :
   Ha ! sire, conme j’ai esté malade et conme vous m’avriez tost pleuree se j’estoie morte !Dame, dist il, les mors aus mors, les vis aus vis ! 
      En non Dieu, mauvés homs, ainsi est il.Or me dites, 
      damedist li chastelains, que vous avez eu.Ha ! sire, je oiai tel paour de ce 
      lyon ou je m’estoie alee esbatre en ce jardin ! Si l’oÿ et cuidai qu’il me deust 
      devorer, pour quoi il me prist tel paour dont je cuidaiqu'il me couvint 
      perdre la parolle.Dame, dist il, ce poise moi ! Je amasse miex que celui qui y ala fust volez ou fossé, 
   si qu’il eust tant beu que deables l’eust emporté.
   Lors se parti atant li chastelains de la chastelainne 
   et vit qu’ele li mentoit apertement. \pend
\pstart Ainsi avint a l’empereeur 
   conmede ce que vous avez oï,
   car il demoura tant leens avec l’empereris que sa ceule 
   fu forment fermee de parfons fossez, si que l’yaue couroit entour d’un glaive de parfont et de xxv piez de lé, 
   et y ot haut palis, que nulz n’i pooitpeust monter sanz eschielle, 
   et une porte desfensable, que nul n’i peust entrer sanz congié. 
   EtEt ce quant ce fu fet, dont repera 
   li empereres et l’empereris, 
   qui moult furent asseuré quant leur cele fu atyree, 
   ainsi conme vous avez oï. 
      Si me vueil ore taire atant de eulz 
      et retourner a Celydus, qui de Rome 
      estoits'estoit partis 
      ainsi com j’ai ditfait mencion 
      ça en arriere 
      en l’ystoire. \pend
         
         
            Ci devise li contes 
               conment Celydus
               vint a Daphus aprés ce qu’il se fu partis de 
               Fastidorus, son frere, qui mors estoit, 
               et conment il vint a son pere l’emperere 
               ou il estoit avec le chastelain
               Comment li empereres se parti de Romme et comment la nouvele en vint au roy Celydus
               Ci vient li contes du roy Celydus de Jherusalem.
            
               Enluminure sur 1 colonne (e) et 11 UR. 
                  Celydus à genoux devant un chevalier en cotte et 
                  heaume et Helcanus couronné sur son 
                  trône après la mort de Fastidorus alors 
                  qu’il cherche Cassidorus. 
               
            
\pstart Or nous dist 
   ci endroit li contes que, ainsi 
   quecomme 
   li empereres fu partis de Rome, 
   si conme j’ai devant dit ou conte, 
   que la court fu moult esmeue ainsi conme il fu drois. 
   La nouvele en vint au damoisel Celydus, dont je fis devant mencion 
   conment il prist congié a son frere, Fastidorus, puis s’en vint a grant anui en 
   Espaingne, dont il avoit esté de par sa mere. 
   Daphus, 
   dont j’ai devant traitié, si avoit esté garde 
   de sa terreterre moult grant piece 
   et avoit eu grant discencion a la ligniee Calcas. Mes il avoit si mis au bas touz ceus qui grevé 
   l’avoient qu’il n’i avoit nul quil lile peust grever ne aidier. 
   Et de ce li estoit moult bien avenu. Mes une maladie li 
   estoit avenueprise en une de ses 
   mains, qu’il ne s’ense pooit mes aidier, 
   ainz estoit trop forment abatus. \pend
\pstart Pas de nouveau § BQuant Celydus 
   sot ceste aventureaventure dont je fis devant mention, 
   sisi en fu trop dolens. 
   Il vint la ou Daphus estoit et li dist :
   Ha ! sire, or puis je bien dire que je sui cil qui a perdu le destre poing et le senestre.
   Quant Daphus oï ce, si dist :
   DousBiaus dous amis 
      Celydus, vous soiez li bienvenus ! 
      ConmentOr me dites conment le fet 
      mon tres chier seigneur, 
         vostre pereli empereres de Romme ?Sire, fetdist 
      li damoisiaus, de luivous 
      nene vous sai je dire autre chose fors 
      qu’il s’est partis de Rome sans le seu de 
      nullui fors que de l’empereris, qu’il en a menee avec lui sanz nulle autre compaingnie fors 
   que de son lyon, qui le suivi l’endemain 
      que il s’en ala la nuit devantaprés. \pend
\pstart Quant Daphus oï ce, si fu moult esbahis et dist :
   Ha ! Celydus, or puis je bien dire 
      que je sé auques l’achoison pour quoi vousvous avez dit que vous 
      avez perdu le bras destre et 
      le senestre. Et vraiement, du destre ne sui je pas certains, mes du senestre 
      me poisepoise il que vous l’avez ainsi perdu.Sire, dist Celydus, pour ce dist voir qui dit que 
   
     qu'avientquant avient  
      uneune Fortune n’avient seule.
   Le proverbe référencé par Morawski, n 1732, est "Quant avient, n'avient sole", le premier ajout de V3 s'inscrit 
   donc probablement dans la volonté de correspondre à cette formule sentencieuse. L'ajout de "Fortune" tiendrait peut-être à une volonté d'en préciser le sens.
   Or ne vous esmaiez, dist Daphus, 
      car il ne puet estre que ceste chose ne tourttourt a honneur a 
      plusseurs, et vous diré en quele maniere. Touzjours avez oï dire : Qui soi oublie ne s’avance
      ###Cette expression sentencieuse n'est pas trouvée chez Morawski ni chez Le Roux de Lincy. à vérifier PD stp ; 
      et d'autrede l'autre part, li philosophe dist : 
      Fai si la besongne ton seingneur que tu ne mettes la teue en oubli
      ###Cette expression sentencieuse n'est pas trouvée chez Morawski ni chez Le Roux de Lincy. à vérifier PD stp, 
      por quoi je di par le phylosophephylosophe que : 
      Mesire n’a pas si faite la besongne son seigneur
      ###Cette expression sentencieuse, dans la lignée de la précédente, 
         n'est pas non plus trouvée chez Morawski ni chez Le Roux de Lincy. à vérifier PD stp, 
      c’est a dire fés faire droiture etet procurer et fere venir 
      le tort a droit que il meismes n’ait esgardéesgarde de 
      sa besongne a faire, par quoi il s’est mis a la penitance faire, ce est a 
      l’ame sauver, qui est tresors etet garde perpetuele joie sanz fin.
   L'interprétation de Daphus, ami cher à Cassidorus et témoin de l'un des précédents épisodes de souhait 
   de retrait du monde de Cassidorus, se voit évidemment approuvée par le jeune Celidus, au contraire de celles des ceux qui sont 
   ainsi associés aux médisants incapables de percevoir les louables intentions de l'empereur. L'insertion des formules sentencieuses 
   joue de l'esprit moralisateur ainsi conféré à cet échange pour éclairer et valoriser le départ de Cassidorus à rebours des 
   critiques proférées par ses détracteurs face au fait qu'il délaissait son trône et ses terres, voir §620. \pend
\pstart Pas de nouveau § BQuant 
   li joennes damoisiausLi joennes damoisiaus quant il ot entendu 
   de DaphusDadaphus (sic)Daphus, 
   il fu plus a sa bonne pes qu’il ne avoit esté devant. Lors dist :
   Sire, voirement se repose qui 
      a a preudonme 
      a faireparole. 
      Or sachiez de voir que tout cil qui ont oï parler de l’alee 
      l’empereeur n’ont pas espons 
   la seue emprise en aussi bonne maniere com vous avez. Et pour ce m’en anuie il, 
   si que je voudroie que tuit cil qui en dient chose qui a dire ne fet en eussent leur deserte.Ne vous chaut, amis ! dist Daphus. 
      Touzjours a on assezvolontiers parlé sus les preudeshonmes quant 
      il ne font au gré des mesdisans.
   Moult parlerent ensemble entre Daphus et Celydus. 
   Si avint aprésaprés ce qu’il ne demora guieres que  
   Celydus atourna son affaire et s’en vint a 
   Daphus et li dist qu’il voloit aler en la queste de 
      l’empereeur, son pere, car il estoit voir qu’il ne pooit estre chevalier d’autre que de 
      luicelui empereour son pere, et il en estoit bien temps d’ore en avant. 
   Lors li dist Daphus que a ce s’acordoit il bien. 
   Dont il li dist 
   que, 
   se Dieu ne li eust envoiee l’essoinne que il avoit, 
      il alastque il fust alés o lui pour savoir son estre et sa volenté. 
   Aprés ce avint que Daphus avoit i damoisel a filz, 
   qui avoit a non le non de Dyanor sa fenme, 
   aussi conme estrait du non sala
      et 
   meremere du damoisel.
   Cil, qui estoit pou mainsnez de Celydus et n’avoit pas mains de cuer, se pourpenssa 
   qu’il n’avroit pas congié de lal‹...›[a] 
      mere d’aler avec son cousin. 
   Si dist a lui meismes que ja pour ce ne demourroit. \pend
\pstart Quant Celydus ot pris congié a 
   Daphus, Dyannor fist aussi com s’il le convoiast. 
   Et quant Celydus vit 
   qu’il fu temps du retourner, si vint a lui et l’embraça etet puis li dist :
   DousHa ! Dous amis et biau cousin, 
      je vous conmant a celui qui donné adonne sens a moi et a vous et 
      entendement de lui croire et amer. SiPour quoi vous pri que vous priez 
      Dieu pour moi, et je de l’autre part li pri que, se je puis fere chose qui li plaise, que vous en soiez perçonniers. \pend
\pstart Pas de nouveau § BQuant 
   Dyanorcil, son cousin, 
   l’ot entendu, qui le cuer avoit debonnere, si ne se pot tenir qu’il ne 
   soupirast, et li vindrent les lermes aus yeux, qui li coulerent aval sa tendre face, 
   etet li dist :
   Conment, biau cousin Celydus ! 
      Vous cuidiez vous ainsi departir de moi en tele maniere ?Amis, dist il, quant fere le couvient, nul 
      nen'en puet aler au devant.Certes, biaus cousins, dist Dyanor, 
   je n’en veil autrement aler au devant fors que sanz moi vous n’irez en nul lieu que je ne voise 
      avecaprez vous.
   Quant Celydus oï ce, si ot moult grant joie a son cuer. Et nonpourquant li dist il :
   Ha ! Dyanor, biaus dous cousins, ce ne seroit pas 
      bienfaitbien que 
      vous sanz le 
      seu de 
   vostre pere et de vostre mere feissiez tel chose.
   Que vous iroie je racontantdiroye je toutes 
      leurs parolles ?
   Li dui damoiselsi furent qu'ilz se mistrent en leur 
   cheminchemin et vindrent a Rome. 
   Si ne vueil ore pas tenir conte de leur affaire de ci adonc qu’il 
      furent venus en la cité de Rome, 
   carque il 
   avoient entenduentendirent que li emperieres 
   Fastidorus avoit grant piece jeü d’une grant maladie et n’atentoit on se l’eure non que 
   l’ame s’en partist. Si avoit si grant pueple a Rome que il sembloit que touz li mondes y feust.
    \pendRubrique B : Comment Celidus et Dyanor son cousin vindrent a Rome pour savoir nouveles du pere Celydus.
\pstart LiOr dist li contes que quant Celidus et Dyanor, 
   son cousin, furent venu en la cité de Romme, liLa variante de B va de pair avec l'insertion
   d'une rubrique supplémentaire et vise à mettre en exergue l'entrelacement des aventures de Celydus avec Dyanor 
   damoisiaus Celydus, 
   qui n’estoit pas nices ne a aprendre, et d’autre part qui bien estoit conneus des grans seingneurs, vint entr’eus, 
   si fu moult conjoïs. Et fist on tant qu’il vint devant 
   Fastidorus, son frere. 
   Si le connut ilsi malades com 
   il estoitcil qui moroit et li dist :
   Ha ! Celydus, biau frere, je me muir !Sire, dist il, ja Dieu ne place que je la vostre mort puisse veoir !
   Atant conmanda li empereres que on feist vuidier la chambre fors 
   que de Celydus, 
   son frere, et on si fist. 
   Lors mist li empereres Fastidorus 
   son frere Celydus a raison et 
   li dist :
   Frere, or m’avez vous trouvé en tel painne conme Dieu veult. 
      Or me dites conment vous le faites et sese vous puis oïstes 
      nouvelles de nostre chier pere l’empereeur.
   Celydus dist a briez mos que 
   nennin, ainçois s’estoit esmeus de son païs pour lui querre. 
   Et lors li dist la raison que j’ai desus dite. 
   Et quant Celydus li ot tout dit, 
   siil li dist :
   Ha ! frere, distce dist 
      Fastidorus, conme vous estes entrez en une bonne painne ! 
      Or vous vueil je prier conme a mon chier frere et ami que, quant vous le 
      pourrez veoir, que vous le me saluez et li portez nouvelles de ma mort, en tel maniere qu’il ne plaise a Nostre Seingneur 
   que je ne soie en vie et que je n’ai nul hoir de ma char qui aprés moi doie la terre tenir, 
      ainz vendra la terre aa mon frere, Pelyarmenus, 
      duquel on me donnem'a fait a entendre que 
      il am'a ma mort hastee plus qu’il ne 
         deust. Et pource que jeje le doute qu’il ne face encore pis 
      de quele hore qu’il soit entrez en l’empire, si voudroie moult qu’il s’en reperast, et vousist mettre les choses qui par sa 
      defaute pourroient estre mal ordenees, car je sui tous certains, se il ce ne fet, il pourroit bien pis faire. \pend
\pstart Pas de nouveau § BQuant Celydus 
   vitoÿ ce,
   si distvit et sotvit bien 
   qu’il estoit prodoms et de bonne conscience. 
   Si ne vueil ore ci plus arrester pour chose que il 
   plus deissent, 
   mes moult apaisa Celydus son frere 
   et moult li ot en covenant dea 
   ferefere tot ce que il conmanda. Aprés ce avint que il morut dedenz 
   iii jours. Si fu mis en terre a si grant honneur conmeque 
   il apartenoit a lui. 
   Pelyarmenus saisi l’empire, mes ne cuidiez pas que Celydus 
   se lessast a acointier 
   dea lui de parolles. 
   Mes Peliarmenus li otria li et son conseil assez faussement, par coi il se departi de lui 
   au plus tost que il pot. Si ne demora pas que il se mistrent en lor chemin, lui et Dyanor, 
   ainsi conme il cuidierent que 
   li emperieres, son pere, 
   s'enen feust alé. \pend
\pstart Moult traverserent li dui cousin le paÿs en divers liex, qui grant desir avoient d’oïr nouvelles 
   de ce qu’il queroient, mes riens ne leur valutvalut, car il ne se mistrent pas en
      chemin ou il en peussent oÿr nouveles. 
   SiCar nous trouvons ou conte qu’il se mistrent 
   au repairier en Gresce pource qu’il autrefois avoit esté en 
   hermitage ou paÿs et en la terre. 
   Lors ont tant alé d’une part et d’autre qu’il vindrent en Constentinonble et trouverent 
   HelcanusHelcanus, l'emperaour, 
   qui les reçut a tres grant honneur, et avoit ja entendu la mort de son frere 
      Fastidorus, dont il avoit si grant anui, car il avoit ja entendu 
   que Pelyarmenus avoit mis hors de lui touz ceus qui avoient esté au conseil 
   et a l’ordenance de la pais de lui et de ceus de Costentinoble. 
   Si pria mout a Celydus et a 
   DyanorDyanor aussi 
   qu’il demorassent avec lui. \pend
\pstart Pas de nouveau § BLi damoisel qui s’estoient parti de leur paÿs pource 
   que vous avez oï s’escuserent en tele maniere que il ne pooient lessier ce pour coi il estoient meu. 
   Mes ja si tost ne savroient la fin de leur besongne qu’il repereroient 
   versa lui sanz contredit.
   Biaus seingneurs, dist Helcanus, je vous tenisse compaingnie, 
   mes que je le peusse fere aussi conme li uns de vous. Mes je sui touz certains que, se je lessoie ja l’empire en autel point conme 
      nostremon pere fist, 
      que je n’i rentreroievenroie jamés sanz plus grant donmage que nous 
      n’enne receusmes.
   La réflexion d'Helcanus prend une portée métanarrative à l'issue de ce long cycle de guerres intestines
   qui ont éprouvé le royaume de Constantinople, qui semblent en effet seulement reposer sur l'absence de Cassidorus sur son trône, 
   à sa place de souverain mais aussi de père affirmé de Pelyarmenus qui a trouvé, dans l'obstination de son père pour conserver 
   son anonymat, un argument de taille pour prétendre sa mort et convoiter son empire. Tirant leçon de ces écueils, Helcanus 
   se veut souverain plus avisé, par sa simple présence sur le trône. Une fois encore, le Roman de Pelyarmenus nous semble donc procéder
   plutôt du miroir aux princes en creux, au contraire de la tendance du reste du cycle à dresser avant tout le portrait du bon souverain.
   La précision "comme mon pere fist" ne laisse en effet aucun doute sur le lien de cause à effet entre cette absence de Cassidorus loin 
   de son empire et ce "grant dommage" que cela a occasionné.
   Sire, distrentfont li damoisel, 
      il vaut miex que vous demourez. Mes le plus tost que nous pourrons, 
      nous revendrons a vous et 
      feronsferons du tout vostre volenté. \pend
\pstart Lors pristrent congié 
   li dui damoisel, si se mistrent en leur chemin a aler cele part ou 
   li empereres avoit jadis conversé en l’ermitage. 
   Si n’arresterent tant qu’il vindrentvindrent en la forest qui moult estoit grant 
      et horriblevindrent lay vindrent. Cil qui joenne et de grant biauté 
   estoientestoient plain, se mistrent en 
   la forest a chevauchier, si conme il leur fu avis que il 
   trouverenttrouveroient 
   l’ermitage au 
   saint honme dont il avoient enquis. 
   Si alerent tant qu’il s’ambatirent en i val ou il courroit une riviere large et parfonde qui d’une montaingne descendoit, 
   et estoit si roide que nul carrel d’arbaleste ne se peust prendre a si tost aler. En ce qu’il estoient iluec, 
   si virent venir une nef ou il avoit marcheans de Perse. 
   Si firent tant par leurs prieres qu’il entrerent en la nefnef avec 
   et alerent tant que, par le miracle saint Nicholas, 
   ilque il arriverent au port, que on apeloit 
   Cleodor – ce fu a demie liue de Gomor le chastel 
   dont j’ai ci devant parlé 
      en monau conte.En dépit des difficultés posées par la forme d'agglutination des deux ermitages fréquentés par Cassidorus,
         il nous semble ici possible d'identifier cet ermitage comme étant celui d'Espiere, se distinguant donc de celui auquel 
         Cassidorus se trouve effectivement à ce moment du récit avec Fastige, celui de Robur. La proximité de leur présentation, 
         sur le même paragraphe, mais aussi la proximité ainsi supposé de leur situation géographique 
         (qui nous paraît presque condenser les deux lieux) a compliqué l'identification de l'ermitage en question ici.
         Mais la description donnée ici d'un ermitage autrefois occupé par Cassidorus, et surtout de la forêt effrayante et
         dans laquelle court une rivière descendant d'une montagne (telle que celle dont il est question ci-dessous)
         ne peut que faire allusion à l'ermitage d'Espiere. Il s'agit pour rappel de celui que Cassidorus finit par atteindre, 
         selon l'appel divin qui le fait quitter sa cour à cette fin, grâce au lion qu'il rencontre dans la forêt d'Espiere en effet dépeinte 
         comme terrifiante et dans laquelle il est éprouvé d'une forme de tentation qui ne fait qu'exacerber la crainte suscitée par cette forêt.
         Voir §509-511 l'arrivée finale de Cassidorus au recet dans ce val dans lequel court la rivière, pour éclairer le rapprochement 
         de ces descriptions. Il semblerait donc que la quête de Celydus et Dyanor vers l'empereur, 
         via l'ermitage qu'il a déjà occupé, les met efficacement sur la voie de l'ermitage, pourtant différent, auquel il réside à présent, 
         l'élément déterminant de leur recherche, la dévotion et le repentir qui anime le retrait de Cassidorus, semblant donc les guider 
         de manière plus assurée que la situation géographique même de cet ermitage. La dimension merveilleuse de la description de la forêt
         se trouve ainsi aussi renforcée, au-delà de l'épisode qu'elle accueillait déjà, de manière dédoublée d'ailleurs pour Cassidorus puis Daphus
         qui venait l'y retrouver, par le voyage qu'il permet, sur une nef qui navigue sur cette rivière, qui n'est donc pas sans rappeler 
         les espaces frontières des univers narratifs arthuriens notamment. L'origine perse des marchands qui permettent à Celydus et Dyanor
         de rejoindre Cleodor et surtout Gomor pourrait pour sa part jouer de la portée mystérieuse orientale qui subsiste et semble ainsi 
         réactivé dans ce roman pourtant déjà largement occupé par l'Orient byzantin. L'ensemble de l'épisode du voyage de Celydus et Dyanor
         vers Cassidorus et Fastige prend en effet des proportions proprement initiatiques, comme de bien entendu en regard de leur 
         objectif qu'est leur adoubement, par le meilleur chevalier du monde qu'incarne Cassidorus. Leur venue se fait en effet 
         l'occasion d'un énième rebondissement guerrier et d'une dernière démonstration des qualités chevaleresques, au-delà des qualités 
         spirituelles dont il est en train de faire la démonstration, de Cassidorus qui mène à la victoire du châtelain de Gomor contre le 
         marquis d'Ostrac. Mais Celydus et Dyanor jouent également un rôle crucial dans cette guerre, y gagnent ainsi leur statut de chevalier,
         conformément à un parcours tout à fait conforme au canevas en la matière, empreint tant d'éléments merveilleux que de démonstration
         de bravoure. \pend
\pstart Quant il furent la arrivé, si 
   entendirentdescendirent que 
   li marchis d’Ostrac avoit mandé au 
   chastelain de Gomor 
   qu’il venist en sa prison pour faire sa volenté, pource qu’il avoit saisi les biens d’un 
   sien chevalier que il ne vouloit delivrer. 
   EtEt que ce que 
   li chastelainsil 
   avoiten avoit fait, ce 
   fu pour le droit sauver de l’eglise de 
      saint Nicholas. 
   Lors n’i ot nul qui ne s’afichast que de la ne se partiroient jusques a tant qu’il avroient fet secours a 
   l’eglise. 
   Celydus et Dyanor, qui ceste chose oïrent, furent moult joiant, 
   si s’apenserent que il s’acointeroient du chastelain. Il vindrent a lui ou il estoit. 
   Et quant il les apercut, il vint contre eulz et les salua moult doucement. 
   Quant il se furent entresalué, 
   illi chastelains leur enquist 
   dont il furentestoient et. 
   Il distrent qu’il erent 
      d’Espaingne et s’estoient partis de leur paÿs pour une merveilleuse aventure, 
      et le benoit corsconfessor 
      saint Nicholas les avoit la arrivez par miracle. 
   Si distrent qu’il demourroient tant que l’eglise 
      en avroit sa raison. \pend
\pstart Pas de nouveau § BQuant li chastelains 
   oï ce, si sourrist de joie et lilor mist 
   sesles bras au col et dist :
   Dont vueil je que vous demourez o moi, 
      car encore ai je moult pou de confort de serjans que je prise 
      autant conme je fas vous.Sire, distrent li damoisel, la vostre merci. Et nous ferons volentiers vostre volenté.Or me dites, amis, conment vous estes apelez.
   Il respondirentont respondu quedistrent :
   Sire, j’ai nonil avoient non li uns 
      Celydus et cilzli autres 
      Dyanor.
   Adonc demanda li chastelains a Dyanor pourcoi il n’estoit chevalier.
   Sire, dist il, quant il plaira a Nostre Seingneur, 
      je troveraitrov‹i›[e]rai 
      celui qui fere le me doit, dont je pri au benoit confesseur 
      saint Nicholas qu’il me lesse tant vivre que je le puisse trouver sain et sauf 
      et entier.Celydus, biau sire, dist li chastelains, 
   se je pooie savoir par vostre volenté qui cilz est, je ne cuit pas que pis vous en fust.
   Dont li a reconnut tout leur afaireson afaire
      et le son compaignon aussi, 
   si com vous avez oï. \pend
\pstart Pas de nouveau § BQuant li chastelains 
   les ot entendusentendi ce, si dist :
   Ha ! beneure enfant, voirement vous a le benoit 
      corsconfessors saint Nicholas 
      ci amenez ! 
   Et croi tout apertement que ja la besongne de l’eglise ne venist au desus 
   se ce ne feust par vous et par celui que vous alez querant.Ha ! sire, distrent li damoisel, dont en savez vous aucune chose qu’il nous eleescera, 
      s’il plest a Dieu et a vous.Verité avez dist, dit li chastelains.
   Lors li demanderent pour Dieu que,
   se il en savoit riens, que il leur en deist aucune chose.
   Biaus seingneursseingneurs, dist il, 
      je ne vous en puis ore plus dire que vous en avez oï. Mes encore nuit parlerons nous ensemble, et 
      demourrezrevendrez o moi, se il vous plest, au souper, 
      et je vous en dirai ce que j’en 
      pourrai savoirscay. \pend
\pstart Pas de nouveau § BAtant ont pris congié li dui damoisel au 
   chastelain, et 
   li chastelainsil 
   s’en vint a 
   Robur, ou li emperieres et 
   l’empereris estoitestoient, 
   car li empereres ne fesoit oeuvre pour la discencion que vous avez oï, 
   car il se pourveoient de jour en jour pour le marchis, qu’il atendoient. 
   Quant li empereres vit le chastelain, si li demanda :
   Quelz nouvelles ?Sire, dist il, vostre honneur croist de jour en jor.
   Lors li conta l’aventure de Celydus et de Dyanor 
   et conment saint Nicholas les avoit avoiez aussi conme par miracle.
   Chastelains, dist li empereres, 
      cuia qui Deux veult aidier, nul ne li puet nuire. 
   Et sachiez que ceste aventure me plaist quant il plest a Nostre Seingneur, et vous di que je vueil armes 
   et veil fere Celydus et 
      DyanorDyanor, son cousin chevaliers.
   Quant li chastelains oï ce, 
   si fu moultsi ne fu onques mais si joians, 
   en tel maniere qu’il ne pot mot dire 
      de joie. 
   Et lorsli empereres 
   li dist :
   Chastelains, alez, si m’aprestezaprestez que j'aie 
      robe emperial. Et gardez que nul ne sache qui je sui fors que li pluseur cuideront que je soie issus 
      hors de la nef pour faire a vostre gent aide. \pend
\pstart Ainsi conme li empereres devisa fu fait. 
   Et Celydus, ainsi 
      conmeque j’ai desus dit, estoit reperiez au 
   chastelain et avoit soupé avec lui la nuit devant. Il vint a court, 
   entre lui et Dyanor, et li chastelains vint 
   a eulzlau ou il l'atendoient 
   et les prist par les mains et les mena devant l’empereeur en une chambre. 
   Quant Celydus vint devant son pere, si le connut. 
   Lors se lessa cheoircheoir a ses piez et li dist :
   Ha ! biau pere, recevez moi a amy, qui par tantes contrees vous ai quis.
   Dont le leva li empereres, 
   etet le baisa son cousin aussi. 
   Lors les trait d’une part et dist :
   Biau seingneur, pource que j’ai entendu que Nostre Sires vous a ci amenez aussi conme par miracle, 
   me sui je fet a vous connoistre pource que je vous vueil donner l’ordre de chevalierie et vueil que vous soiez chevalier Jhesucrist 
   pour essaucier son non et pour mettre foy et loyauté au deseure 
      et orgueil et felonnie au desous.Pere, dist Celydus, ainsint 
      l’ai jel'avons nous en 
      couvenantcouvenant a faire, 
      etet moi et mon cousin 
      aussint. \pend
\pstart Atant mist Celydus l’empereeur 
   a reson et li dist :
   Peres, mon frere Fastidorus, 
      li vostre filz, vous salue, 
      et ma dame l’empereris, 
      de qui je deusse avoir premierement parlé. Si 
      me dites conment il li est.Ha ! biau filz, dist 
      li empereres, ele est toute hestiee, grace 
      a Dieu avons iiii biau 
      filz,filz, ce distfait 
         li empereres,
      ilqui 
         aont ja i an a ceste Penthecouste 
      que je ne vous vi.Biau sire, Deux glorieusglorieus peres, 
      vous en soiezpuissiez estre aoez, dist Celydus, car bien em pourra estre encore l’empire de 
      Rome essauciee, qui aujourdui est povrement porveue !Conment se tient Fastidorus 
      donc ? dist li empereres.Sire, dist il, ainsi conme il plest a Dieu, car il est trespassez de ce siecle. 
      Et croi que Pelyarmenus hasta sa mort, 
   ainsi conme Fastidorus le me dist avant 
      qu’il fust mors.La collusion sur ce court paragraphe des nouvelles de 
         la naissance des quadruplés et de la mort de Fastidorus, ainsi que du tort que l'on peut prêter à Pelyarmenus à ce sujet, 
      paraît jouer les effets d'annonce quant au rôle des jeunes enfants pour rétablir l'ordre dans l'empire de Rome. \pend
\pstart Pas de nouveau § BQuant 
   li empereres entendi ce, si fu moult courrouciez, et li vindrent les 
   lermesarmes aus yeux et 
   distli distrent :
   Ha ! sire, vrais Deux, 
      conme il a de couvoitise en ce mondesiecle ! 
      Et quecom je cuit que Rome avra encore 
      a sousfrir se jeDix n’i met autre conseil !
   Lors enquist li emperieres se 
      Fastidorus avoit nul hoir de sa fenme. 
   Celydus dist que nanil."Certes, sire, non." 
   Et d’Elcanus li dist 
   tot l’afaire conment il fust volentiers venus o lui se ce ne fust pour son empire destourber. \pend
\pstart Moult parlerent moult longuement ensemble 
   li empereres 
   et li damoiselli doi damoiselCelydus. 
   Lors vindrentvint nouvelles que 
   li marchiz 
   estoit venus a meuz a tout x mile escus, et venoit sus 
   le chastelain de Gomor. 
   Si veillierent cele nuit xx damoisel a l’eglisse monseingneur 
      saint Nicholas, 
   qui l’endemain furent tuit chevalierchevalier de la main a l'empereour por l'amour
   de Celydus et Dyanor. Que vous feroie plus lonc conte ? Il avint que cil de la nef s'alierent a Celidus quant il sorent que il fust
   chevalier et furent en sa bataille .LXX. armeures de fer tout a cheval. Li chastelains en avoit .V. cent et li empereres
   qui ne vaut pas estre conneus ainz fu aussi comme chevaliers banerez issus des nez et fu seigneur du clergié et ot en sa baillie
   que prestres que chanoines que clers qui tous s'afichierent de morir avant que nuls d'eulz deust ja fuir, si furent .IIII. cent.
   Li empereres, quant il vit que il avoit tele maisnie a conduire, si se mist entre eulz et dist : "Biax seigneurs, ne quidiez pas
   que je n'aie autre fois esté en bataille et vous avez bien apris a boire et a mengier et estre bien a aise, si ne cuidiez pas 
   que il ne couviegne le cors traveillier. Et j'ai souvent oÿ dire que clerc sont hardi, or porra l'en savoir la verité. Veez ci
   le marcis qui vous vient tolir vostre vivre. Et ci a moult d'autre bonne gent qui vous viegnent aidier a deffendre. Vous avez droit
   et il ont tort. Si soit desordené qui aujourd'ui suivra ne pour mort ne pour vie." Adont distrent : "Sachiez pour voir, 
   nous sommes toz convertiz. Chevauchez, nous vous sievrronz et ne vous faudrons ja por mort ne pour vie.
      Cette variante de B équivaut à ce que donne V3, tout comme G et X2, à la fin du §687 ci-dessous.
      . 
   Atant sont issus de Gomor bien jusqu’a 
   iii mile, que a pié que a cheval, li marchis d’autre part, 
   tot aussi orgueillieuxfier conme lyon encontre 
   i mouton, 
   et venoit contre eulz conme ceuscil qui bien estoient 
   x mille contre iiim.
   Celydus, qui avoit veu la premiere bataille, trait avant sa gent, 
   si ne choisi pas le marchis, conme cil qui cuidoit avoir tout gaaingnié. \pend
\pstart Dyanor, qui avoit cuer volentif, 
   vint a Celydus et li dist :
   Biau cousin, ceste premiere jouste vous requier je.Et je la vous otroi, dist Celydus.
   Atant se desrenga Dyanor et vint, les saus menus aussi conme arondelle. 
   Et quant le filz au marchis le vit venir si noblement 
   qu’ilsi n’i ot nul qui osast habandonner 
   son cors fors il, qui point son bon cheval 
   d’Espaingne et li vint si aigrement qu'il ne feust nul qui 
   le veist venir qu'ilen son venir ne le deust doter, 
   Dyanor, qui a ce avoit mise s’entente, 
   li vint a l’encontre si acesmeement que tuit cil qui le virent venir 
   prisierent son afaire, car aussi conme dis me tu : Vous ne me poez 
      eschaper l’acueilli en tele maniere de l’espie, 
   que le fer tresperça le blazon et tout ce que il consuivi et que parmi le cors passa li fers, 
   si que l’arçon derriere ne pot le cuercop endurer, ainz rompi, 
   et cil cheï enmi le champ. Celydus feri 
   i autre aprés Dyanor 
   et toute sa compaingnie. Qui donc les veist tenir tres esforcieement ou Celydus se feri, 
   dire peust que tout li xm n’eussent pooir a lui. 
   Dyanor, qui avoit ja fet son poindre, vint ferant
      li vint a senestre, qui enchargié avoit li et son cors en tele maniere qu'il feroit 
   a destre et a senestre, que moult estoit grant merveille, car il estoit 
      trop bon chevalier. 
   Et pource que l’emperere 
      l’avoit fet chevalier en avoit il 
      trop bongreigneur cuer, 
      et tuit cil aussi qui chevalier avoient esté fet pour l’amour de Celydus et de 
      Dyanor. 
      Que vous feroie je lonc contediroie je ? 
      Il avint que cil de la nef s’alierent a Celydus 
   quant il le sorent si bon chevalier. Et furent de sa bataille lxx armeures de fer tout a cheval. 
   Li chastelains en avoit v cens d’esleus, qui bien le firent aus cops donner. 
   Et li empereres, qui ne volt pas estre conneus, 
      ainz fu aussi conme chevalier banerés issus des nés et 
      fu seingneur du clergié et ot en sa bataille prestres, 
   chanoinnes et clers, qui touz s’afichierent de mourir avant que nul d’eulz deust ja fouir. Et furent 
      iiiic, quique touz furent bonne gent 
      et qui bien aidierent a l’empereur et aus autres si conme 
      il aparut.Cette section est absente dans B, car donnée plus haut.
      On peut d'ailleurs noter que cette variante peut sembler plus logique, puisque seraient ainsi présentées les forces en présence
      en amont du combat relaté ensuite, sans que cette présentation soit intégrée comme une précision une fois le combat amorcé. \pend
\pstart Pas de nouveau § BQuant 
   li emperieres vit qu’il ot tel gent a conduire, si se mist entr’eus et dist :
   Biaus seingneurs, ne cuidiez pas que je n’aie autrefois esté em bataille. 
   Et vous avez apris a mengier et a boivre et a estre tout a aise, car ce n’est pas vostre mestier ne acoustumance de porter armes 
   ne d’estre es batailles. Si ne cuidiez pas qu’il ne conviengne le cors traveillier. Et j’ai souvent oï dire que clers sont hardi. 
   Or pourra on savoir la verité. Vez ci le marchis, qui vous vient tolir vostre vivre, 
      et ci a moult de bonnes gensgens 
      qui vous veulent aidier a desfendre. Vous avez droit et il ont tort. Si soit desordené qui aujourd’ui fuira ne pour mort ne pour 
      vie.
   Adonc distrent il :
   Sire, or sachiez, pour voir, nous‹...›[n]ous 
      sonmes convertis. Chevauchiez, et nous vous suivrons ne ja ne vous faudrons 
      ne pour mort ne pour vie.Cette section est absente dans B,
         car donnée auparavant. Voir note ci-dessus.
   Que vous feroie je lonc contediroye je ? 
   Li empereres 
   desrenga et ses clers ausitant i refist, a l'ayde des clers, 
   qui si bien le firent que 
   l’empereresil 
   prist le marchis
   et le livra au chastelain, qui maintenant l’envoia ou 
   chastel de Gomor. \pend
\pstart Celydus et Dyanor, 
   quant il sorent que li empereres avoit pris 
   le marchis a l’ayde des clers 
   qui si hardiement l’avoient fait, si en orent ausi conme i pou de 
   confusion. L'exploit réalisé par Cassidorus, menant la compagnie, toute symbolique, des clercs 
   paraît fonctionner comme un levier de sens important en cette fin de roman. Le cycle des sept sages procède en effet d'une réflexion
   importante sur les qualités clergiques des héros, qui se trouvent ici à nouveau valorisées, dans un roman qui semble pourtant 
   avoir longtemps mis de côté ces considérations, emporté par la nature très chevaleresque de son récit. Il paraît donc très 
   significatif que dans son ultime combat, Cassidorus remporte cette bataille en menant des clercs au combat. 
   L'épisode gagne ainsi en importance, pas tant, au contraire des nombreuses autres batailles du roman, en regard des enjeux de cette 
   bataille (quoique, il s'agit d'une bataille livrée pour la défense de saint Nicolas), mais de sa portée métanarrative.
   Il vient ainsi renouer avec les premiers chevaux de bataille du cycle pour louer un héros aux qualités avant tout clergiques,
   mais aussi chevaleresques en sa fin de vie (non sans faire co-exister avec cette valorisation les critiques qui peuvent être émises 
   face à ces qualités de gouvernant, comme on l'a vu notamment avec la réflexion d'Helcanus à ce sujet). 
   Lors se sont esvertué et euls mis en habandon de leurs anemis mettre au desous. Mais, tout aussi 
   conmeque 
      j’je vousavoie dit, tant en y avoit que, selonc ce que il estoient, 
   que, se leur pooir ne feust si grans et li benois sains monseingneur 
      saint Nicholas 
   ne les confortast, il n’eussent duree a eulz, qui estoient si grant plenté et avec ce chevaliers esleus et plains de grant volenté 
   pour confondre le chastelain, qui tel honte leur fesoit. Mes riens ne leur valut, 
   car ilil les 
   escouvint qu’il feussentestre 
   occisoccis et mehaigniez 
   et ne porent de nule part avoir duree. Et trovons que des x mille personnes que il estoient, il en i ot la moitié que ocis 
   que prispris que afolez, et le remanant se 
   mistrentmist a garant au miex qu’il porent. \pend
\pstart Pas de nouveau § BAinsi fu pris 
   li marchis et sa gent maumenee. 
   Lors s’en retourna arrieres li chastelains et touz ceus de sa compaingnie. 
   Et trouvons en l’ystoire queque, pour plus briesment
   outrepasser que li marchis fu touz liez 
   et touz joians quant il pot estre honme de l’eglisse et faire pes ferme et estable par touz les plus grans seingneurs du pays, 
   car il fu ainsi esgardé que ce avoit esté venjance de Nostre Seingneur et miracle 
   du bon saint monseingneurde 
   saint Nicholas, qui fu puis si essaucié ou paÿs et aillieurs que bien le pueent savoir li 
   pluseurs. Si me veil ore atant taire de ce et venir a Celydus 
   et a Dyanor, ainsi conme il se partirent de l’emperereeur.
   L'emphase mise sur la portée miraculeuse de, tour à tour, la venue de Cassidorus, puis celle de Celydus 
   et de Dyanor à Gomor, la victoire remportée contre le marquis d'Ostrace dénote de l'orientation très forte que connaît le récit au 
   contact de Cassidorus bien sûr, mais aussi encore ensuite de Celydus qui, comme en gommant les origines merveilleuses que lui confère
   sa figure maternelle féérique, va endosser une fonction presque célestielle dans le récit, plus encore dans celui du Roman 
   de Kanor qui suit. \pend 

         
            Ci devise conment li empereres
               mena Celydus et Dyanor 
               veoir l’empereris et ses iiii enfans, 
               et aprés conment il se partirent de lui pour aler en Jherusalem, 
               et devise conment il alerent par Antyoche 
               et le prince les detint
               Comment Celydus et Dyanor vindrent a l'empereor et comment il les mena en son recet
               Ci vient li contes a Celydus et a Dyanor, son cousin, qui se partirent de l'empereeur
               et se mistrent en leur voie.
            
               Enluminure sur 1 colonne (b) et 12 UR. Cassidorus mène Celydus et Dyanor devant Fastige et les 4 enfants.
            
\pstart En ceste partie dist li contes que, quant ceste bataille fu ainsint outree 
   et les choses mises a point ainsint conme j’ai dit, 
   Celydus 
   et Dyanor 
   vindrentvin-drent a l’empereeur vindrent a 
   l’empereeur,
   qui au plus priveement que il pot les mena en son recet et 
   et
      trouverent l’empereris ainsi com j’ai fet et troverent 
   l’empereris, ainsi conme j’ai devant fet mencion en 
      mon contel'ystoire conment 
   elle avoit esté delivree a grant pais 
   et a grant honneur i an avoitavoit pres 
   de iiii damoisiaus a filz, ainsi conme il nasquirent en iiii jours l’un aprés l’autre, dont li premiers ot a 
   non Canor, le secont Sycor, le tiers 
   Domor, le quart Rusticor. Et touz ces iiii nons 
   furent pris ou non et par la volentévolenté a l'empererys de l’emperere, 
   leur pere, qui avoit a non Cassydorus. 
   Ces damoissiaus nourrissoit l’empereris si amiablement que pour riens ne 
   vouloit‹p›[v]ouloit sousfrir que nulle autre fenme les alaitast 
   quefors elle, car 
   elleselle 
   dientdisoit que nul lait d’autre fenme 
      n’estoit si couvenable a la nature de l’enfant conme de la vraie mere.
   ###YFJ Cette remarque de Fastige dénote d'une réflexion intéressante sur le pouvoir maternel de l'allaitement,
   qui n'est pas sans annoncer aussi les enjeux de l'allaitement des quadruplés, mais aussi du fils de Nera, dans le Roman de Kanor.
 \pend 
\pstart Pas de nouveau § BQuant la dame vit 
   Celidus, si le connut et sailli em piez 
   et l’embraca par grant amour et dist :
   Ha ! tres dousdous filz et bon amis. 
      Jamés ne vous cuidai veoirveoir ne oyr. 
      ConmentDites moi conment vous est 
      ilil et que l'en fait ou pays dont vous venez ?Dame, dist ilil, moult vous ai a dire de 
      toutes choses, 
      assezassez tost 
      leen vous diré. Mes encore desirre je plus 
      a veoir et a 
      savoirsavoir de vostre estre se il pooit avenir.En non Dieu, biau filz, vez ci tout mon estre et le plus de mon desirrier 
      est 
      que je puisse nourrir vos iiii freresvos iiii
         freresces III enfans voz, 
      que vous poez ci veoir, 
   tant qu’il puissent vivre sanz moi et sanz le pere.Dame, dist il, nostre sires vous otroit fere tel chose et tel nourreture que Sainte Eglise en soit 
      essauciee et lor lignage en puisse estre honores.
   Lors les esgarda moult Celidus. Et tandis s’acointa l’empereris 
   de Dyanor que ele n’avoit onques mes veu et il li dist qui il 
      estoit et adonc li fist elle moult grant joie. \pend 
\pstart Celydus, quil ne se pooit rasazier des enfans regarder, 
   qui moult estoient gracieus a veoir que nul ne les veist qu’il n’en fust esjoïs, les esgarda tant que jamés d’eulz ne se vousist partir. 
   Et en ce qu’il s’esjoïssoit en eulz veoir 
   s'esjoui, carcar car il 
   choisi que chascun avoit a son col pendupendu aussi comme une filatiere de 
   fin or a une cordelettecordele de soie, dont il vit qu’en chascune estoit 
   entailliés le non de l’enfant ainsi conme je les ai desus nonmez. Et quant se 
   vit Celydus, 
   si s’esmerveilla pourcoipar quele raison ce fu et le demanda a 
   l’empereris.
   Biau filz, dit elle, 
      la reson pourp‹a›[o]ur quoi je l’ai fet si n’est autre 
      quecar, se vous ne 
      lela savez regardez l’un et puis l’autre, 
      et vous trouverez que tous sont pris a i regart l’un et l’autre.Certes, voir avez dit,dit, ce 
      distfait Celydus.Et pource, dist l’empereris, 
      que je ne puis pas bien connoistre l’un de l’autre ai je fait ceste chose pour toutes aventures 
      Etet si sachiez. Et 
      si sachiez que vez ci l’ainsné, qui a non Canor ; le secont aprés 
      SydorSycor ; 
      le tiers Domor ; le quart 
      RusticorRusticot. 
      Et ces iiii propres nons a il en mon seingneur ou il a iiii sillabes. 
      Li ainsné si a la premiere et li autres les autresl'autre a la seconde, 
         le tiers a la tierce et le quart la quarte. 
      Et ce n’ai je pas fet ensans senefiance,
      L'expression en senefiance que propose V3 et G à sa suite est habituelle suivie de l'objet de 
      l'interprétation ainsi proposée (en senefiance de/que), mais n'est pas inhabituelle (voir DMF). car li cuers me 
      dist que encore en aucun temps leur pourraporroit ce avoir mestier. \pend 
\pstart Quant Celydus ot entendu 
   l’empereris, si la tinttrait 
   a moult sage et dist :
   Dame, vostre sens encoreparra encore aus damoisiaus 
      pourraet leur pourra encore avoir mestier. Et je sé vraiement que Nostre Sires les 
      a fet naistre aussi conme pour lui soulacier. Tout soit il ainsi que il ne puet avoir plus de joie qu’il a. 
   Moult longuement se sont desresnié iluec en moultmaintes manieres 
   de parolles, dont je trop aroie a desresniertraitier se je 
   toutes les voloie raconter. 
   Mais a ce vient ore li contes que, quant il orent tant parlé et d’un et d’el, 
   li empereres 
   s’estoit pourveus tandis d’eulz donner a mengier et a boivre et i fu li chastelains, 
   sanz qui il ne volt pas estre a cele fois, et tant que parolles coururent entre les ii damoisiaus qu'il distrent 
   qu’il erent entrez en mer et qu’il avoientavoient en volenté 
   d’entreraler en Jherusalem 
   pour aourer le sepucre ou Deux avoit esté mors et vis. 
   Li empereres leur conmanda que ce ne lessassent il pas et 
      au plus tost qu’quantil 
      pourroient
         pourroientpourroient repairier, que il reperassent en l’empire 
      de Rome et de Coustentinoble et, s’il y avoit chose 
      descouvenable, que il aidassent au droit 
      maintenir a lor pooir, qui sovent a mestier que il soit sostenus. 
   Lors ont pris lili dui damoisel congié et sont venus a leurs compaingnons, 
   qui les atendoient aussi conme 
   li enfant font 
   leurle pere 
   quecar tout ensement li marcheant 
   dont j’ai devant parlé les avoient pris en amour pour la grant proesce 
   dont il estoient et pour le 
   grant sens 
   qui estoit ende eulzeulz,
   comme l'enfant fait la mere qui doucement le norrist. \pend 
\pstart Quant li dui damoisel et leur compaingnie furent ensemble, si se mistrent arrieres en mer au plus tost 
   que il porent, car il orent vent a leur devise. Et leur aida tant Dieux l’un jour aprés l’autre que il vindrent em brief terme la ou il 
   vodrent venir. Il descendirent a terre et se mistrent tot a unecom par 
   compaingnie ou chemin devers Antyoche, dont je ne veil ore ci faire 
      mencionconte de chose qui leur avenist en la voie, mais tant 
   esploitierentesploitierent que poi firent de sejor tant qu’il vindrent 
   devanten la cité 
   d’Antyoche. \pend
\pstart Pas de nouveau § BA icel temps dont je parolle avoit 
   i prince en la ville qui avoit a non Melchis et avoit le los d’estre le plus 
   richesage Sarrazin qu’il feust en la terre. 
   Il estoit joennes et biaus et plain de tres grant proesce. Li dui cousin Celydus et 
   Dyanor oïrent moult de bien retraire de lui et furent tempté de lui veoir, 
   si firent tant que il le virent i jour en sala sale 
   ou il parloit a i chevalier de la terre de 
   Jherusalem, qui avoit aporté nouvelles du roy, 
   qui avoit fet prendre de ses honmes pour aucunes raisons dont li contes ne fet ci mencion. 
   Li princes regarda et vit les ii cousins, qui bien 
   sembloientsemblerent estre ce qu’il estoient, 
   si lessa le chevalier a qui il parloit et vint a eulz aussi debonnerement conme s’il les conneust et les salua en françois, 
   conme cil qui bien le parloit. Cil, qui ne furent pas esbahy, li ont rendu son salu. 
   Lors parla a eulz si gracieusement qu’i leur enquist de leur estre, 
      siltant qu'il nel vodrent celer, 
   ainzqu'il ne 
   leur enl'en 
      conterentdeissent grant partie. \pend 
\pstart Quant li princes les ot entendus, si les detint, 
   vosissent ou non, mes ce fu ausi com par grant amourgrant signe d'amour, 
   et tant que li princes prisa moult les ii chevaliers pour la biauté et pour le sens 
   que il trouva en eulz et cuida bien que il fussent preus et hardis de leurs cors. Si s’apenssa 
   conmentque il trouveroit achoison par 
      quoi il les peust detenir, si leur dist quant il s’en cuidierent aler et prendre congié :
   Biau seingneurs, or ne vous anuit, car vostre gent ont i pou vers moi mespris. Et se j’en 
      pooie par vous avoir l’amende, de ce ne vous seroitseroit il mie pis.
   Quant Celydus ot entenduoÿ 
   le prince, si cuidasi cuida si cuida 
   que aucuns de sa compaingnie eust vers lui mespris, si dist :
   Sire, je ne di pas. Se nulz de nous a mespris vers vous, nous le vous ferons amender.
   Lors li dist le prince :
   Amis, je ne di pa que vous sachiez rienz de cest meffait, 
   ainz apartient au roy de Jherusalem, 
   qui a fet prendre de mes honmes sanz raison, si conme il m’est avis. Et pource que je n’en sé pas la verité me plest il que vous 
      demourezdemouriez avec moi jusques a tant que j’en savrai autre chose.SireBiax sire, distrent il, 
      de nous poez vous fere vostre volenté, car la force en est vostre.En non Dieu, biau seingneurs, voirs avez dit. Et ainsint est il de guerre. Mes ne cuidiez pas que je 
      vous doie fere tenir vilainne prison, mes vous m’avrez en couvent que vous 
      ne partirez de ma court pour nule chosemengier ne pour boire ne pour dormir 
      jusqu’a tant que je vous donrai congié.Sire, distrent il, et nous l’otrions, puisque faire le couvient. 
   Mes pour Dieu, que nostre compaingnie puisse aler quele part qu’ele voudra.Et je l’otroi, dist il. \pend 
\pstart Ainsi furent retenu li dui damoisel, et tant qu’il reperierent a leur compaingnons 
   et leur conterent de ceste chose la verité. Quant il orent ce oï, si furent moult esbahi 
   et distrent :
   Biaus seingneurs, ne cuidiez pas que pour avoir que nous puissions esligier 
      vous doiez demourer en la prison du prince.
   Lors les en ont mout merciez, puis vindrent li 
   païsantmarcheans au 
   prince et li ont prié que pour 
      amouravoir 
      lessastil lessast 
      aler leurs ii compaingnonsles.
   Biaus seingneurs, dist li prince, ne cuidiez 
      vous pas 
      que pour avoir le face et que 
      je n’aie avoir assez. Mes je ai pou de tiex prisons conme je cuide que vous 
      estes. Si ne m’en offrez nulautre avoir que 
      levos cors 
      quecar je n’em prendroie autre avoir ne autre tresor.
   Rien ne valoitvalut chose qu’il deissent, car demourer couvint les ii 
   chevaliers devers le prince. 
   Et avint ainsint conme l’estoire le conte que tout li grant seingneur sarrazin 
   si ont plusseurs fenmes et touzjours en ainmentainment il plus l’une 
   que toutes les autres, et de cele sont il trop volentiers jalous. 
   Dont il en avoitavroit
   iii entre les autres qu’il avoit fet mettre en une tour, et li autre, dont il en y avoit xxx, estoient en 
   i hostel appareilliees, et la estoient li chastre qui les servoient 
   et gardoient. Aucunes fois 
   li princessarrazins 
   venoit et menoit les ii chevaliers veoir ses fenmes et leur 
   conmandoit a toutes qu’eles leur feissent 
      feste, et elles leurle 
   fesoientfirent de tout leur pooir.
   Et dist li contes que, s’ilse toutes peussent fere 
   toute lor volenté, qu’il feussent tout autrement conjoui qu’il n’estoient. Mais il conmença a anuier aus damoisiaus, qu’il n’avoient 
   cure de ce qu’eles couvoitoient. Et li prince s’en apercut moult bien et dist :
   Ha ! biau seingneurs, je me painne de tout mon 
      pooir de vous fere joie et chose quil vous peust plaire, mes je ne puis, 
      car il me semble qu’il ne vous est de nul deduit dont il est a moult d’autres. 
      Si vous pri que vous me dites aucun lieugiu qu’il vous puist esleescier, 
      car, s’il vous plest a avoiravoir de dames ou pucelles, je vous feraicuit faire 
      avoir des plus nobles de cest paÿs et, se il vous plaist, deduis de chiens ne de oissiaus ne de chevaus courre ne berser, traire 
      ne lancier, ne chevaliers verser, ne lances brisier, escus fraindre et troer, quintainnes rompre, haubers despaner, 
      je le vous ferai avoir, ne sé que plus vous die. \pend
 \pstart Quant li dui chevalier oïrent ainsi 
    le prince deviserparler pour eulz 
    atraire a ce que vous avez oï, si l’en mercierent moult 
    durement et distrent :
    Sire, voirs est que moult avez quis de nostre deduit. Mes encore n’avons nous riens trové de ce que 
      nous couvoitons, car nous savons de verité que nostre loy est contraire a la vostre. Et pour ce ne couvoitons nous pas que 
      nous aillonsvous aillies sus 
      vostrenostre gent, 
      car nous n’irionsirons mie 
      sus la vostreen vostre aide moult volentiers. 
      Et s’il avenoit que nous ailleurs eussions a fere, volentiers nous essaierons d’armes en ce paÿs. \pend
\pstart Pas de nouveau § BLi prince, 
   qui ce entendi, si sot maintenant qu’i furent chevalier adroit. Et que fist il ? Il 
   vintvint maintenant aussi conme d’aventure en 
   la tour ou lesles .III. dames 
      estoient, dont j’ai desus parlé, si dist l’une d’elles 
   iii :
   Sire, il nous est avis qui vous nous avez 
      oublieesmises en oubli pour ii 
      vassausmusars que vous avez 
      arresté aussi com pour leur biauté, dont nous nous doutons moult que vous n’en soiez entrez en male 
      boisdievoye.
   Quant li princes oï ce, si conmença a sousrire et dist :
   Moult avez ore fole cuidanceaventure, 
      qui cuidiez que je aie tel avis et qu’il d’autre part soient tel com vous donnez a entendre.Sire, sire, dist l’autre, je ne sé pas ne 
      ne pourroie croire que vostre penssee fust male. Mes ne nous creez jamés s’il ne sont de male nature, et 
      lele vous prouverai.EtOr me dites dont conment ? 
      dist li princes.Non ferai, dist elle, que vous n’avez que fere du savoir. \pend 
\pstart Or entendez du deable conment il cuida fere les ii chevaliers destruire. 
   Li princes, qui 
   estoitestoit preudoms en sa loy, fu temptez du savoir se li chevalier 
   estoient tel conme cele li donnoit a entendre, etque il volt savoir 
   seque‹...›se la dame l’en pourroit fere sage et li dist :
   Dame, riens ne vous vaut, car savoir 
      vueil des ii chevaliers la verité.Sire, dist elle, par amoursDieu souffrez 
      vous, carque je me doubte que vous ne le sachiez plus tost que 
      vous ne voudriez. Et d’autre part je croi que vous ne m’en leriez pas couvenir, car vous estes trop merveillieus.Dame, dist il, je vous en lerai bien couvenir, sauve mon honneur et la leur, mes que ce soit de leur 
      droit.Or les nous feitesf‹r›[e]ites venir amont, 
      dist elle. Et quant vous les avrez amenez, si vous destournez de ceens, mes que vous veez que nous ferons, car autrement ne seriez 
      vous pas em pes, ce savons nous bien.Dame, dit li princes, je ferai vostre volenté.
   Atant s’en vint aus ii chevaliers et leur dist :
   Je vous vueil mener la ou vous ne feustes 
      onquesonques." "Et ou est ce ?, distrent il..
   Lors les mena la ou les dames estoient, qui de biauté avoient le pris. 
   Et ceus, qui furent esbahis de leur acesmement et du seurplus, saluerent les dames et elles eulz, et ne cuidiez pas que toutes 
   iii ne prissassent les chevaliers moult et leur noble contenances. Mes fenme qui 
   fiene fu faite que pour honme decevoir fist asavoir a chascun 
   en esgardant que moult leur plaisoit lor estre. \pend 
\pstart Pas de nouveau § BLi prince, 
   qui volt acomplir la volenté de la fame et sa musardie confermer, 
   se trait ensus et se mist en une garde robe et veoit auques ce que elles pooient fere. Mes il ne pooit pas oïr leurs parolles. 
   Lors prist cele qui plus ovroit souvent de tricherie 
   et qui plus en savoit les ii chevaliers et 
   sesses .II. compaingnes, et elle fu la quinte. 
   Si mist leurs v testes ensemble et dist :
   Voirs est que sens de fenme est moult pou prisiez. Et nonpourquant, trop pou de choses entreprent 
      elle a fere qu’ele n’enne viengne au desus, car nous avons tant fet par nostre 
      art que nous vous avons fet venir ça sus, car 
      nousvous veions 
      que vous estiezpour tiex comme vous et bel et gent et couvoitiez de 
      dames et de damoiselles. Si vous prions de conmun absent toutesainsi comme nous ci 
         sommes, dont il n'y a nule dont vous n'eussiez tost vos volentez se il puet estre 
      que vous nous moustrez s’il a tant de bienproesce 
         en vous et d’onneur conme il 
         aperty a de biaute. \pend
\pstart Dame, dist Celydus, 
   vous parlez a loy de dame preuspreus et cortoise et 
   sagesage et bien apense. Et se tout ce i failloit, si y voi je tele chose 
   qui moult fait a couvoitier. Si ne cuidiez pas que nous lessissions en nule guise a fere chose qui vous peust tourner a honneur ne a 
   joie de cuer. Si conmandez sus nous, et se il puet estre acompli par le cors de 
   ii chevaliers, nousnous nous 
   ien metronsmetrons en 
   painne.
   Lors esgarda l’une l’autre et furent aussi conme toutes prises, que l’une pour l’autre n’osoit dire sa volenté. Et nonpourquant dist une :
   Sire, il ne vousvouss 
      anuie pas se nous avons amusé mon seingneur, 
   car il nous fet ça sus garder ceste tour dont nous nous souffrissions bien se il vosist, car li homs qui fenme 
      enferme ne destraint : il tient le cuer et le cors enferme.
      SL/PD : proverbe, vraiment ? J'ai des doutes au final après avoir cherché, en vain, 
         chez Morawski et Le Roux de Lincy, et la formule ne me semble pas si forte et surtout très liée au propos narratif ici.
      Et tot en tele maniereainsi est il de 
      mon seingneur, car il tient nos cors et met nos cuers a meschief, pour 
      laquele chosequoi il n’en est pas amez de touz, car nous vous 
      couvoiterions plus a avoir i jor en nostre recreacion que 
      nous ne ferions lui cent.
   Iluecques avoient les dames i pou dede leur deduit a parler 
   a Celydus et a Dyanor, 
      son cousinaus chevaliers. Et li prince, 
         leur seingneur, si musoit 
   au tro conme musart et abotoit 
   par i petit pertuis etou il veoit les dames d’eures en autres 
   rire moult debonnerement. \pend 
\pstart Pas de nouveau § BLors 
   quant li sires vit ce, si penssa 
   qu’il seroitert escharnis et que il estoit ja 
      qui en tel point creoit celes qui ne faisoient que moquier le, 
      ausi conme maintes fames font maint preudonme. 
   Lors vint le prince et dist aussi conme por feste :
   Biaus seingneurs, assez en eussiez de ii, mes que l’autre eust son pareil.Sire, dist l’une, or venez avant, si avra chascun sa chascune.
   Dont se lessa li princes cheoir entr’elles, si en bleça une i pou, dont elle dist :
   Je amasse encore miex que vous feussiez dont vous venez.
   De ceste parolle que la dame distce se courrouça moult 
   durementforment li princes, 
   et li donna une moult grant pasmee et li dist :
   Dame, touzjours ne fait il pas bon voir dire !
   Atant sailliil sailli em piez et prist les 
   ii chevaliers 
   Celydus et Dyanor 
   par les mains, si les mena aval et puis si leur dist :
   Biau seingneurs, que vous semble il de moi ? 
      Sui je i 
      folsot ?Sire, dist Celydus, pourquoi dites 
      vous ce ? 
      Telz est li usages de ce paÿs.
   Moult le paia bien Celydus et il dist :
   Celydus, par foy, se vous saviez pourcoi je vous menai a elles, 
   vous en avriez grant merveille.Sire, distce dist 
      Celydus, moult i puet avoir de raisons, mes cele ne cuit je pas savoir.Assez a temps, dist 
      li princesil, 
      lela savrez.
   Adonc ont lessié de 
   celeceles
      ceceles 
   a parler et fist tant li prince 
   que il vint en la tour sanz le seu d’eulz. \pend 
\pstart Les dames, ainsi com jeje vous 
   avoie dit que li princes
   s’estoit partis de elles et en avoit donné a une une buffe, cele conmença a faire son duel et dist a ses compaignes :
   Se vous ne metez painne que jeje ne 
      soiepuisse estre 
      vengiez des oeuvres de ce jalous, 
      je n’avrai jamés joie a mon cuer !Certes, distce respont chascune 
      de elles, serions nous moult liees se nous poyons 
      eschaperissir 
      de ses lasdes las de lui 
   par quoi il ne nous tenist prises conme il fet.
   Atant vint li princes amont et dist :
   Or vueil je fere la pais de moi et de 
      mama chiere damefemme.
   Cele qui savoit du jeu dist :
   Sire, de nostre pes avez pou a fere, et bien parut a vostre courrous que pou prisiez amour de fenme ! 
      Mes je nene le sé pas 
      seque cele d’onme fait 
      aussivous.
   Lors embraça le prince celle et se pena mout de dire et de fere chose qui li 
   peust plerepleust. Aprés volt 
   li princes savoir que elles avoient trouvé es 
   ii chevaliers 
   et pourcoi il lesa les 
   avoit fet venir a elles. Cele, dont j’ai par desus fait mencion, respondi :
   Par foy, sire, se vous ne feussiez ore si tost venus, tost les 
     eussionseusse pris de ce que je vousisse. 
   Mes vous estes si hastis que on ne vous puet chose dire ne fere que vous 
     n’entrezne venres maintenant en male voisdie.
   Li princes se penssa que voir se dissoit,
   et tant que il dist :
   Foy que je doi a vous iii, je vous en lerai couvenir que qu’il m’en doie avenir.Bien avez distdites, font elles. \pend
\pstart Ainsi demoura li affaires a cele fois, car il fu tart et 
   fu heure de souper. 
   Si revint li princes aus compaingnons pour 
   eulzleur tenir compaingnie, conme cil quil ne 
   pooitpooit fairepooit faire 
   croire qu’il ne fussent de trop grant pris. 
   Dont il avint que Dyanor ne pot mettre les dames en oubli que il avoit veues en 
   la tour, dont la plus joenne l’avoit si detenu 
   par la main, tant conme il avoit la esté, que il estoit 
   ausiauques disposez en amer dames et damoiselles. 
   Et tout aussi iert Celydus, mes il estoit plus avisez, et d’autre part il avoit un cuer 
   si orgueilliex que il pas ne meist 
   son cuersa cure en 
   autreautrui fenme. ###Motif très assumé
   de l'orgueilleux d'amour qu'il incarne en effet ensuite avec Alerie ? 
   Et Dyanor ne s’apenssa de riensce, 
   conme cilz qui dist en son cuer que ja cele n’espargneroit se il la tenoit en lieu ou 
      ilil il eust pooir. 
   Cele de l’autre part penssa a lui en tele maniere que elle dist a lui meismes que 
   avant feroit elle merveilles de son cors que elle ne parlast a lui en lieu et en temps. 
   Les autres ii pensserent a Celydus. Et avoit a non l’une 
   Alyenor et l’autre Leodore, 
   et la tierce Fauque. Ces iii dames furent compaignes de pignons. 
   Si pensserent a ce qu’eles ameroient 
      lesles .II. chevaliers, 
   mes pource que elles furent iii et ileulz ne furent que 
   ii si distrent que ja nulles d’elles ne giehiroit a l’autre lequel 
      elle ameroit. \pend
            \pstart Pas de nouveau § BAinsi avint ce qu’il ne pot estre destourné par l’anemi, 
   qui cuida par ce destruire les jouvenciaus par les jouvencelles. Lors demourerent tant chiez le prince 
   que la novelle desdes .II. chevaliers vint en 
   Jherusalem, et cuittant
               
   qu’il s’acointierent au roy. Et ceus qui aporterent la nouvele distrent que 
   ii chevaliers erent venus en Antyoche 
   qui erent li plus soufissant d’armes qui fussent en nulle terre. 
   Quant li roys oï ce, si demanda conment li chevalier 
      avoient nonestoient apelez et dont il estoient né. 
   Cil distrent leur non et dont il estoient. 
   Li rois, qui engrant fu des chevaliers a voir, 
   manda en Antyoche parlement au prince et 
   qu’il vouloit a eulzlui parler aussi 
      com cil ne seust riens d’eulz. 
   Li princes de l’autre partpart li 
   remanda au roy
   qu’il vouloit bien prendreprendre le 
      parlement a lui. 
   Si assemblent li roy et li prince 
   en la cité de Rohais. \pend 
\pstart Celydus et Dyanor 
   demorerent en la cité d’Anthyoche, car li princes 
   ne volt soufrir que il alassent avec lui. Et quant les dames desus dites le sorent, si pensserent que 
   mal iroit leur afaires se ellesil 
      ne pooient parler aus chevaliers. Et lors firent fere une lettre et 
   l'enveloperentla geterent en i luiselet de soie retors en 
   i vergier, ou elles virent Celydus et 
   Dyanor. Le lieu seul de cette rencontre qui s'organise entre 
   les trois dames et Celydus et Dyanor suffit bien sûr tout à fait à indiquer la dimension amoureuse des aventures qui vont 
   nous être relatées. Il s'agit en effet d'un locus amoenus tout à fait typique dans la littérature médiévale. 
   Il vindrent cele part et pristrent la lettre et trouverent 
   ens escript ce que vous pourrez savoir. 
   La nuit vint, et li dui 
   damoiselcosins firent 
   tant que il entrerent en 
   ce vergier. 
   Si ne demoura pas que les dames ne meissent leur compaingne en une corbeille, puis l’ont avalee jus de 
   la tour en tele maniere com 
   il furent tout apresté, siqui troverent que ce fu 
   Fauquedame Fauque, que 
   Dyanor couvoitoit. Et ce il la vit volentiers, ce ne fait pas a demander. 
   EntreAdontA Dyanor et 
   la joenne dame ce sont traittraistrent 
   a une part, si trouvons ou conte que 
   moult anuia a Celydus ceste aventure. Mes il en vit Dyanor 
   tel atourné que il ne li osa escondire a fere sa volenté. Et tandis comque 
   il menerent leur joie et leur volenté ensemble, Celydus se mist ou 
   vergier a cueillir roses et 
   fleursfleurs yndes et bloies, et tant en cueilli que, quant il fu temps 
   que il orent assez esté ensemble, que cele Fauque 
   se remist arriere en sala corbeille et 
   Celydus i mist tant de roses avec que ele fu toute plainne. Et celes qui furent en 
   la tour amont l’ont resachiee 
   a eulzamont. 
   Et quant ele fu amont, si en menerent grant joie ensemble, conme celes qui bien 
   cuidierentcuidoient recouvrer a leur volenté faire chascune 
   a son toursa fois. \pend
\pstart Celydus, qui estoit mout sages, 
   blasma Dyanor de ceste chose et dist que mout les metoit en grant 
      aventure des cors et des amesarmes se il estoient aperceu.
   Cousins,Cousins, ce 
      distfait Dyanor, je n’en puis mes, 
      car se on m’en deust pis fere que je n’en oÿ, si ne m’en fusse je pas tenus.Or ne vous aviengne plus ! dist il.
   Il respondi que non feroit il. Que vous diroie je de 
   Dyanor ? Il se cuida garder de ceste chose, mes fenmes, qui du tout ne couvoitent pas 
   la pais de chascun, ainz veultveulent acomplir 
   salor volenté, pour 
   coicoi a l’autre nuit aprés 
   la seconde des dames se mist jus de la tour 
   en tele maniere conme l’autre avoit fet la nuit devant. 
   Dyanor s'embla de Celydus et vint ou 
   vergier et trouva Alyennor, 
   qui cuida trouver Celydus. Dyanor, qui pas ne fu a aprendre, 
   escusa son cousincompaignon et se 
   contint vers cele en autele maniere conme il fist vers 
   la premiere.la premiere. Que vous feroie je lonc 
      conte ? 
   Ausi conmeCe que 
   il fist des iipareillement fist il de la tierce. \pend 
\pstart Pas de nouveau § BAinsi se contint 
   Dyanor tant qu’il fu aperceu de 
   la guaite qui guaitoit par nuit chiez le prince, 
   et tant que Dyanoril 
   l’ocist et le gita en une yaue parfonde. Et ne pot estre celé que l’en ne 
   s’en aperceust de celui qui mors estoit, 
   que on ne sot qu’il fu devenus. Si en fu grant parole aval la meson. Mes de ce me couvient taire et venir a ce que 
   li princes avoit une sereur, 
   qui a celui temps estoit la plus requise de grans seingneurs que pucele de toute payennime, 
   car tant estoit bele et gentegente et sage que on ne trouvast sa pareille, 
   et si n’avoit encore pas plus de xvxvi ans. 
   Si estoit a ce menee qu’ele avoit ja tant de sens en lui qu’ele estoit en sa loy 
   une des plus sages pucelles d’ilec environ, 
   et avoit oï parler de ces ii chevaliers. Si fist tant que il vindrent devant lui a i chastel 
   assez pres d’Antyoche ou ele sejournoit. Et quant ele les ot acointiez et lor nons demandez, 
   si les prisa moult en son cuer, carconnut en son cuer que 
   Celydus fuqui fu sages 
   et bien apenssez, si li enquist de la loy crestienne. Cil, qui bien estoit 
   endoctrinez, li respondi si sagement que ele dit :
   Celydus, je vueil de vous fere mon 
      grant amy. Et ne cuidiez 
      pas que ce soit pour 
      nul male volenté 
         nefors pour 
      l’amour de 
      nulle male couvoitise que l’amour de Jhesucrist 
      ne soit la principal. \pend 
\pstart Pas de nouveau § B et GQuant Celydus 
   ot oÿ 
   la pucele ainsi parler, 
   si ne fust mie siplus liez 
   qu’i l’eust fetpour estre roy de 
   Jherusalem. Lors se mist 
   a jenous devant luielle et joint ses mains 
   etet li dist :
   Pucele, mes que ce soit voirs, j’en merci moult hautement celui qui pour 
      nousvous com pour moi mourust 
      en la crois. 
      Et vous doint perseverer en cest estat en tele maniere que Sainte Eglisse en puisse estre essauciee !Amis, dist elle, a mon pooir vostre requeste iert 
      essauciee et
      asouvie, mes il couvient ceste chose 
      celerceler forment jusqu’a tant que Nostre Sires 
      nous donra maniere par quoi il en soitsera autre chose faite.Damoisele, dist ilCelidus, 
      jeje ai entendu et voi que vous estes sage, 
         et j’en feréserai 
      du touten fait et en dit a vostre volenté.Bien avez dit, dist ele. Mon frere si repairera de ce parlement, 
      si savrons que il avra fet. Et selon ce 
      qu’i fera, je vous ferai savoir ma volenté.
   Ainsi prist Celydus congié a la pucele, qui avoit a non 
   Alerie. En ceste acointanceAinsi 
   furent amdui li uns a l’autre loyaus amans. 
   Dont il avint puis grant honneur a toute Sainte Eglise et a toute la contree de Jherusalem. 
   Si me couvient ore atant 
      tairesouffrir de 
   la pucele et des ii chevaliers et venir au prince, 
   dont li contes a ci devant conmencié a conterdevant dit. \pend 

         
            Ci devise conment le prince 
               oï nouvelles de sa guaite, 
               que Dyanor avoit ocis, 
               et en fu mis en prison, et Celydus l’en delivra
                  delivra si comme s'ensuit
               Comment liCi vient li contes au princes d'Antioche 
                  alacomment il ala 
                  contre le roy de Jerusalem 
                  au parlementparlement a Rohais.
            
            
               Enluminure sur 1 colonne (a) et 15 UR. 
                  Dyanor mis en prison pour avoir tué 
                  le garde, 
                  délivré par Celydus.
               
            
\pstart Or dist li contes ci endroit que, quant 
   le prince ce fu partis d’Antyoche, que il ne fina d’errer, 
   sitant que il fu a Rohais. 
   Et l’endemain vint li rois de Jherusalem 
   de l’autre part. Si avint ainsint que lor parlement ne se prist pas, si qu’ilil ne 
   se departirentdepartissent 
   aineusementanuieusement
   li uns de l’autre, par quoi li i desfia l’autre. Si s’en repaira 
   le prince en Antyoche au plus tost que il pot. 
   Si avint ainsi que li filz a la guaite vint a lui et li dist :
   Sire, malement va li affaires, car vostre gaite est perdue, 
   et cuidons vraiement que uns de vos François l’ait occis pour aucune male volenté, 
      car moult volentiers aloient joant avalavant 
      lecel vergier 
      aude costé le pié de 
   cele tour en regardant contremont d’eures en autres. 
   Mes je ne sé pas por quoi ce fu, ou pour bien ou pour mal. \pend
\pstart Pas de nouveau § BA cest mot fu 
   li princes espris de jalousie, si vint en 
   sa tour amontamour (sic) 
   La leçon de V3 est à coup sûr symbolique de la fonction de cette tour qui abrite les 3 favorites de Melchis.
   et trova les dames taintes et pales et leur demandaenquist 
   conment il leur estoit. Celles, qui bien furent apenssees 
   de respondre, si ontli ont 
   distrespondu :
   SiSire ne 
      nousvous doit il pas moult anuier de ce que vous avez tant demouré 
   que n’eusmes puis recreacion de vous ne d’autres ?D’autres ? dist il. De qui avriez vous eu 
      recreacionrecreacion ? De mes François ?De vous, sire. De vos François, sire, 
      n’avons nous pas eu moult de recreacion quefors tant qu’il ont esté en 
   ce vergier ou il nous ont tramis des roses.Conment ! dist il. Et en quele maniere ?
   Lors li ont aportee la corbeille, mes ilil li 
   ontavoient osté la fort corde 
   de soie par coi elles s’estoient avalees et i avoient mise une 
   plus foible corde qui en nule maniere ne peust avoir soustenu nules d’elles. 
   Si li distrent :
   Sire, par ci avons nous eu des roses de 
      ce vergier, 
      et nous a a toutes sembléensemble que elles eussent moult perdu 
      de leur coleur avant que elles venissent amont, si que ce est le plus bon 
      exempleexperiment par quoi nous cuidons que 
      vosvostre François si soient de male nature. \pend 
\pstart Pas de nouveau § BAdonc ot 
   le prince moult grant joie et se mist hors de la soupeçon ou cilz l’avoit mis, car, 
   s’il eussent mostreese il eust trove la corde par ou elles 
   estoients’estoient devalees, touz li mondes ne li feist mie 
      croire que cil ne fustfussent montez amont ou 
      celles devalees aval. 
   Nouveau § BDont fu li deables dolens que il ne pot trover art 
   par quoique il fussent destruit. Si penssa que 
   en aucune maniere diroient les fames aucunes choses par quoi elles s’empescheroient. 
   Et nonpourquant avint il que i peschierres trova la guaite en l’yaue ou 
   Dyanor l’avoit gité. Quant ce vit li princes, 
   si fu moult courrouciez, car il ne le sot sus qui mettre, carque il 
   n’i ot nul qui li seust a dire 
   quique nul le heist 
   ne qui mal li vousist, 
   et si estoit il tout pourfendu jusques ou pis. Or avint une 
   avision que li princes en fist 
   d’une chose 
   qui li fu loee a fere, car il fu mis en i lieu secré, et fist on aler chascun de la 
   mesnieemeismes 
   la ou il estoitestoit pour  
   savoirsavoir mon (sic) 
   ### 
   se il seingneroit, mes il n’i ot nul pour qui il 
   seingnast que pour Dyanor. \pend
\pstart Pas de nouveau § BQuant li princes 
   vit ce, si dist :
   Conment ! Dyanor, 
      pourcoi avez vous occis ma gaite ?Pource, sire, qu’il me desfendoit les roses a 
      cueillircueillir." "Queles roses, dist il, furent ce ?" 
         "Sire, dist il, ce furent des roses que je cueilli en cest
      vergier por tramettre aus dames de vostre tour dedenz une corbeille 
      que elles avaloient ou vergier.
   Nouveau § BLi princes s’apenssa a ce que il pooit dire voir 
   et de l’autre part mentir. Si dist :
   Sire vassal, 
      sisi il vous est moult mal avenu de 
      ce que vous onques l’occeistes sanz autre courpe, car je ne puis pas veoir legierement que il ait esté occis pour 
      cestetele reson 
      quecom vous m’avez ci dite.
   Et lors fu pris Dyanor et mis em prison. \pend
\pstart Pas de nouveau § BCelydus, 
   quant il vit ce, si ne sot que faire, car il savoit bien que, se 
   li princes savoit la verité, que donc seroient il destruit. 
   Si s’apenssa que il couvenoit painne mettre a ce qu’il peust plaire au 
      prince. Lors vint a lui et li dist :
   Sire, mal nous est avenu de vostre 
      guaiteguerre, 
   qui nous a mis en tel dangier conme nous sonmes pour pou d’achoison !Vassal, distce dist 
      li princesil, 
      riens ne vous vaut, 
      carque autrement est alez li affaires que vous ne me dites. 
      Si ne cuidiez pas que je autre chose n’en sache avant que il m’eschape.Sire, dist il, vous poez de nous fere vostre volenté, conme cilz qui bien 
      eny a le pooir.Je n’en ferai riens, dist 
      li siresprinces, 
      fors que toute raison. \pend
\pstart Pas de nouveau § BEndementresEn ce qu’il parloient ainsi, estes vous i mesage qui vint de 
   son chastelprevost Dyerlo 
   – c’estoit i chastel qui marchissoit en la terre au
   roy de Jherusalem. 
   Et li aportoit cistcist de nouvelles que 
   li roys avoit fet siege devant 
   cest chastel et fait grant donmage 
   a merveilles entour. Quant li princes oï ce, si fu 
   a merveillesmerveilleusement moult courrouciez et esmeus en ire. 
   Lors conmanda aa plusors a envoier querre gens pour fere secors au 
   chastel. EndementresEn ce vint 
   Celydus au prince et li dist :
   Sire, se il vous plaisoit que l’amende de vostre guaite 
   peust estre restoree par ce que nous vous aiderions a avoir vostre reson du roy, 
      et nous nous em peneronspenerions.Et conment, dist li princes, m’i pourroie je fïer ?Sire, dist il, le dangier ou mon cousin est entrez aussi conme 
      par mescheance nous mainne a ce que nous facionsfaçons ceste chose.Voire, dist le prince, se il vous plaisoit 
      a faire.Par foi, dist Celydus, 
      voirement nous pleroit il avant a faire que nous feissions pis.
   Que vous diroie je plus 
      lonc conte ? 
   Ainsi fu que li princes se fia en Celydus et en 
   Dyanor et leur charga vc Sarrazins. 
   Et il meismes d’autre part en ot x mille. 
   Si se mistrent a chevauchier devers le chastel devant nonmé. 
   Novelles vindrent en l’ost le roy que li princes 
   venoit pour li lever du siege. Lors s’appareillierent de eulz armer, et ne demoura pas moult qu’il n’asemblassent ensemble 
   a moult grant effort. \pend
\pstart Celydus, qui pas ne volt lessier que il ne tenist son couvent, 
   conduit sa gent d’une part pour miex soi fere connoistre, si que li princes seust 
   qu’il feist d’armes pour la seue amour. Si vint Dyanor a lui et li dist :
   Conment, sire ! Nous estuet il fere contre 
      nostre loy chose qui moult nous est descouvenable ?
   Dyanor distdist il :
   ###Comme cela arrive assez fréquemment, les insertions de discours se font de manière variée,
      souvent plus marquée dans B, ici suivi par X2, que dans V3.
   Contre ce ne poons nous aler se nous ne voulons fausser nostre couvenant.Par mon chief, dist Dyanor, ja n’avendra 
      a fairepour tel affaire que 
      nous soions contre nostre 
      loy se vous me voulez croire.Ja par Dieu, dist Celydus, ne m’avendra que je fausse 
      lemon couvennant 
      que j’ai fet pour donmage que je y voie que je 
      nene le doie restorer en autre maniere.
   Quant Dyanor oï ce, si ne dit mot, ainz tourna le destrier cele part ou il choisi la greingneur 
   presse. Si ne cuidiez pas selonc ce que l’estoire le devise que tant ne feist a l’aide de ceus qui alerent 
   aprés que par force trespassa les rens de batailles a nos crestiens et assembla au 
   roy desus nonmé. Celydus ne volt pas lessier seul 
   son cousin couvenir, ainz s’en tourna 
   aprés lui, et tuit cil de sa partie. 
   Qui donc veist conment li dui chevalier se contindrent en la bataille, des autres 
   eust lessié le regart et le resort, car il se pristrent si aigrement
   a nostre gent que mal de celi qui desfensse meist en lui qu’il ne couvenist reculer 
   lenostre roy et 
   sanostre gent, si que tel cuer et 
   telesi grant force donnerent aus Sarrazins 
   qu’il ocistrent tant des nos que li remanans se mist a la voie. 
   Et trouvons en la fin que li roys eust esté detenu 
   se Dyanor n’eust esté, qui pas ne s’ense 
   mistvolt metre en painne. \pend
\pstart Pas de nouveau § GLi princes, 
   qui ceste besongne vit et sot, vint aus ii cousins et leur dist :
   Biaus seingneurs, j’ai eu victoire par vostre esfort. 
      Et li guerredonsla guerre est moult grant qui a rendre vous en est.Sire, ce dist 
      Celydus, em pou de donmage gist a la fois grant conquest. 
      ###L'expression sentencieuse n'apparaît pas chez Morawski ni chez De Roux de Ligny. PD stpEt en celui 
      conquest parfaite confusion i puet 
      on avoir.
   Ore dist li princes :
   Il n’est pas poins que vous m’en diez ore la reson. Mes or nous en retraions, et puis si 
      parleronsparlezparlons 
      ensemble d’autre chosed'el. \pend 
\pstart Pas de nouveau § BAtant se retraitrent li 
   Sarrazin, 
   et li prince se mist ou chastel, o lui 
   sesles ii cousins et plusseurs autres barons. 
   Quant touz furent desarmez, si en i ot de bleciez et de navrez, et meismement Celydus et 
   Dyanor le furent si conme cil qui pas n’avoient esté oisseus 
   touz ii, selonc ce que j’ai fet mencion. 
   Li mire, qui pas ne furent a aprendre, les mistrent en tel point que il furent 
   mis a couvenable garison. Quant il furent a lor volenté et il 
   furent emparlezorent parléparlerent 
   de plusseurs aventures, si mist li princes 
   Celydus a reson et dist :
   Vassal, or est il poins que vous me dites en quele maniere ne pourquoi vous me deistes au departir 
      de la bataille qu’en pou de donmage jesoit a la fois grant conquest et en celui conquest parfaite confusion.Sire, dist il, je le pooie dire par plusieurs raisons, et vous en dirai mon avis. 
      Nous sonmes entrez en vostre dangier pour acomplir la volenté d’aucunes 
      de vos fenmes, lesqueles mistrent en volentévolenté a mon cousin 
      qu’il occeist un musart a pou de donmage, pour lequel nous vous avons aidié a sormonter 
      lenostre roi de 
         Jherusalem. 
      Et de ceste victoire qui a conquest vous‹n›[v]ous tourne doit estre tournee 
      d’autre part a parfaite confusion.Ha ! Celydus, biau sire, ce ne voi je pas que il n’ait reson 
      en ce que vous avez dit. Mes ce ne me doit pas tourner a donmage ne a confusion. Se je pooie trouver 
      maniere par quoi je vous peusse tourner a mon confort et en m’aide, 
      en maniere que vous o moi vousissiez demourer, et je vous ferai seigneur de moi et 
   de toute la princee d’Antyoche.Sire, dist il, la vostre grant mercis. Mes ne cuidiez pas que, pour estre 
      princesire de toute paiennime, 
      je ne seroieserai demain contre nostre gent, aussi conme je fui hier, 
      et je n’en deusse fere l’amende.
   Li princes, qui moult iert preudoms en sa loy, penssa que cil 
      n’iert pas en volenté de demourer en lui tant qu'ilcom il en peust 
      partir. De l’autre part, il se penssa que, se crestien l’avoient par devers eulz, 
      que moult li pourroient fere de donmage. Si penssa une chose qui bele fu, en ce que il 
         le detint en tele maniere conme je vous dirai. \pend
\pstart Celydus, qui sages et apensez estoit, 
   vintvint en cest afaire au 
   prince et li dist :
   Sire, la renonmee et le sens dont j’avoie oï parler de vous me fist venir en vostre presence par si que 
      mon cousin et moije avons esté 
      detenu par devers vous. Or en soit ainsi conme il plaira a Nostre Seingneur 
      que vous aiez em partie fet la vostre volenté, que 
      vostre courtoisie s’estende a ce que nous 
      puissions aler la ou nous avons voué et promis.Celydus, 
      chierbiaus chier sire, vostre requeste ne quier je pas refuser. 
   Mes de ce ne me porroie je pas apaisier que vous partissiez de moi fors sain et sauf et entier, pour coi il vous couvient entendre a garir, 
   et puis pourrez alerirez quel part il vous plaira. \pend
\pstart Pas de nouveau § B
   Que vous diroie je ? 
   Ainsi demouraala que couvint 
      sejornerdemourer 
   Celydus et 
      Dyanorles .II. cousins avec le prince, quil ne faisoit 
   quefors pensser 
   en quele maniere il peust ceus prendre par quoi il demourassent 
      avec lui et n’alassent en Jherusalem. Si avint i jour ainsi 
   conme aventure l’aporte que li princes 
   manda sa suer, 
   de qui je ai fet devant mencion, qu’ele ne lessast pour nule riens 
   qu’ele ne venist en Antyoche, le plus honorablement de dames et de damoiselles que ele pourroit. 
   La pucele, qui bien avoit oï et entendu conment son frere 
   avoit mis arriere le roy de Jherusalem 
   par la proesce de Celydus et de Dyanor, 
   son cousin, ele entendi 
   au mant que son frere li avoit fait. Et que fist elle ? 
   Elle fist maintenant son atrait et se mist dedenz iii jour a la voie, et ot avec lui toute la fleur de la biauté des dames 
   et des pucelles du païs. Et quant ce sot li princes, si fu moult esmeus de joie fere, 
   conme cilz qui grant los avoient conquis en ce qu’il avoit levé le 
   roy de Jherusalem du siege. \pend
\pstart Nouvelles vindrent en Antyoche que 
   la pucelle venoit. Li princes avoit fet crier parmi 
   la ville qu’il n’i eust honme qui a cheval peust monter quil n’isist aus chans en aucuns 
   acesmemens encontre sala suer, 
   qui n’avoit esté en la ville grant temps avoit. Adonc n’i ot nul qui cheval eust qui n’ait fet 
   son pooir de joie fere. Et mistrent Sarrazin tel painne a ce que on parlast d’eulz de joie faire 
   que il en durent estre a touzjours grevé, si conme de richesces et de nobles paremens. Celydus, 
   de qui je tieng mon conte,
   La leçon de G nous paraît dénoter d'un certain malaise face à cette affirmation d'ordre tout à fait narratif 
      mais assez peu crédible bien sûr.###
   ne se mist pas en oubli, 
   entre lui et son compaingnon, car il estoient 
   montez es chevaus couvers de paremens a la guise françoise.
   Si ne furent onques ii chevaliers acesmez si tres noblement conme il furent. Et sachiez que 
   li princes l’avoit fet faire pour plaire a touz, et meismement pour 
   sa sereur mettre en voie de conjoïr 
   leles 
   ainsicom ele fist, ainsi conme vous 
      le pourrez oïr ou conte. \pend
\pstart Pas de nouveau § BEn ce que li baron 
   d’Antyoche vindrent encontre 
   la pucelle, elle vit que touz se penerent de li fere honneur, car si merveilleusement se penerent 
   en leur guise de fere divers esbatemens qu’il ne fust nulz quil n’en deust avoir grant joie. Et meismement 
   la pucelle avoit grant merveille de son frere, 
   qui tel chosejoie avoit emprise pour lui conjoïr. 
   Si cuida vraiement que il le feist pour lui decevoir et pour lui donner 
   a aucun prince de qui il eust confort et aide, pour destruire et mettre au 
   desous le royaume de Jherusalem. Si s’aficha en lui meismes que pou li 
   valutvaloit a ferefere ceste 
   feste et quecar ja jour de sa vie n’avroit honme a seingneur se il n’estoit 
   honme de la loy crestienne et de tres grant valeur. En ceste penssee ou elle estoit vint 
   li princes a son encontre et li dui cousins, li uns d’une part et l’autre d’autre, 
   si l’ont moult gentement saluee. 
   Elle choisi Celydus en ces acesmemens françois, si se dreca en son estant et dist :
   Encore puisse je veoir en joie et en grant honneur tant de valeur et de confort 
      conme mon frere a eu en victoire, 
      si conme j’ai entenduoÿ envers 
   le roy de Jherusalem. \pend
\pstart Quant ce entendi li princes, si ot moult grant joie 
   et s’aclina vers lui et dist :
   TresHa ! Tres chiere suer, or voi je bien 
      que li vostre Dieu a seurmonté la loy aus François en ce que je vous dirai. 
   Car pour cestui atraire et lui honnorer a faire vostre volenté vous ai je mandee en tele maniere que j’ai fait issir contre vous 
   touz ceus de la cité d’Antyoche et pour vous moustrer cestui ou il a plus de valeur 
      et de proesce que en nul qui aujourd’ui vivesoit en vie.
   Quant la pucele oï ceentendi ce et l'oÿ, 
   si ot moult grant joie et loa Jhesucrist en son cuer 
   et sot bien que tout ce estoit miracle de Dieu que tout ce avenoit, car or vit elle bien qu'ele pourroit venir a tout ce 
   que ele couvoitoit et dist :
   Vous estes bien apensez de tout ce que vous avez a ffaire, et bien me plaist ore tout ce que vous en 
      avez faitdit, mes que vous me lessiez des ore en avant couvenir.NeN'en doutez ja, dit il, que autrement 
      sai je bien que la terre d’Antyoche seroit perdue se il nous eschapoient. \pend
\pstart Pas de nouveau § B
   Li princes
      Atant ont laissie ester. Et li princes 
   vint aus ii chevaliers et leur dist en grant signe d’amour :
   Par foy, biaus seingneurs, je vueil bien que vous sachiez que c’est ici 
      ma suer que vous avez veue, que on tient a la plus 
      sagesage que 
      fenme qui soit en la 
      superiour Galille. Et se elle n’est bele et franche, dont n’en sé je nule, 
      et vraiement que je veil que vous sachiez que pour l’amour que j’ai en vous 
   et pour vous plus honnorer ai je fet ceste chose que vous poez ci veoir. 
      Si vous vodroie prier en aucune maniere que vous vousissiez 
   honnorer la pucele, qui bien le vaut, et pour vous, de l’autre part, faire miex connoistre. \pend
\pstart Celydus, qui moult fu joiant de ceste requeste, dist :
   Biau sire, ne cuidiez mie que, en toutes les manieres que nous pourrions fere chose qui 
      peust plaire a la pucele, qui tant a de valeur en soi selonc Nature, qui son pooir a fet 
      de lui ordener de ce que elle a grace de biauté et de sens selonc l’avis de ceus qui la connoissent, que 
      nousnous ne 
      ferionsfeissons volentiers 
      chose qui a son gré li venist.
   Lors vint li princes a lui et li dist :
   Ma suer, ne vous plairoit pas chose que li 
      François feistfeissent qui vous venist a gré ?Frere, dist ele, je ne voi ore pas qu’il 
      en soit mestiers selonc ce que j’ai entendu qu’il ont 
      esté navreznavrez et blecie en la bataille que vous avez eue. 
      Et d’autre part, nul jeu ne nul esbatement si ne vaut riens a ce que il sont crestien. 
      Et d’autre part, nostre 
      gentgent sarrazin 
      si ne pourroientpourroit tourner a amor et a joie les uns contre les autres.Ha ! suer, dist il, voirement ne puet li cuers, qui bien n’est a lui dire sens qu’il n’i ait 
      aucunaucunes regart de 
      folie. Sachiez que encore ne sont il pas bien gari de la bataille, mes il ne m’en souvenoit. Et nonporquant les ai je tenus pource 
      que je ne vouloie pas qu’il se departisissent de moi fors que sain et sauf et entier. Et pour ce vous ai 
      je 
      mandeemandee avant, que je vueil que vous les m’aidiez a detenir.A ce m’acorde je bien, dit elle. Mes je vueil que vous les me faciez venir touz ii.
   Et il si fist. Et lors s’enclina la pucele devers Celydus, 
   qui moult mist grant entende ad'oÿr ce que elle 
   diroitvoloit dire, et elle 
   li dist :
   Biau sire Celydus, 
      quantgrans mercis quant pour l’amour de moi, 
   vous et vostre cousin voulez fere chose qui me tourt a 
      joiejoie et a honneur ! Mais sachiez de voir que 
      encoreoncore (sic) 
         en aucun tempsen meilleur point et quant greigneur mestier en sera 
      me feré vous chose dont il me sera plus grant mestier et pourfit. \pend 
\pstart Quant Celydus ot ce, si li plut moult ceste parolle et dist :
   Damoiselle, se Dieu me dointdoint ja 
      bonne aventure et tant d’eur que je peusse pour vous fere chose 
   qui vous tournast au salu de l’ame, moult en seroie desirrans et en vendroie plus seurement a ma fin.
   CesteVeez ci une 
   chosechose qui tourna moult a pris et en honneur a celui qui ne chaçoit 
   se l’onneur non de Dieu, car il avint que 
   la pucele, qui ert entree en la voie que ele peust venir a parfaite honneur que elle 
   et celui dont j’ai desus parlé 
   tant qu’ilsans faillefirent tant qu'il 
   vindrentvindrent arriere en la cité. 
   Quant la pucele fu de son char descendue, si ne volt onques souffrir que 
   son frere ne autre la destrassent fors Celydus 
   et Dyanor, qui en furentfu moult esgardé 
   pour leurs nobles contenances, et meismement 
   la pucele, qui bien se contint a loy de fenme qui bien sembloit qu’ele se meist en 
   painnepainne d'aprendre les a faire sa volente et il d'autre part se remetoient en paine 
   qu’ele leur pleustplaisoit. Et estoit avis a chascun de ceus qui les 
   esgardoient que leur Dieu eust fet ses ii personnes 
   pour touz eulz aourer. Si n’i ot nul qui bien ne se paiast de ceste chose, et ot chascun la pensse au 
   prince, de coi li contes est assez plus biaus. 
   Si vueil a ce venir que li dui chevalier enmenerent ainsi la pucele 
   en la plus haute chambre du palais, qui moult estoit noble 
   et bien atournez, car li princes l'avoit a merveilles hautement fait amender. \pend 
\pstart Les iii dames de 
   la tour, 
   dont j’ai desus fet mencion, pour plus honnorer la pucelle vindrent 
   en grans acesmemens contre lui. Et quant elles ont choisi les ii chevaliers 
   aprés, si furent moult esbahies, et nonpourquant saluerent eles mout
   gentement la pucele et tout aussi les .II. cousinschevaliers aprés, 
   et eulz d’autre part ne se maintindrent pas nicement, car il reconnurent les dames, si que il firent grant joie 
   quequi pas ne le vodrent lessier pour 
   le prince, qui fu pres d’eulz, qu’il ne 
   s’ense tenist pas mal a paié pource 
      qu’il ne firent pas les mesconneus pour nule male soupecon. \pend
\pstart Pas de nouveau § BLa pucele Alerie, 
   qui moult estoit bien apriseapensee, 
   se prist garde de ce que li dui chevalier connurent les dames. Lors trait a une part Celydus et li dist :
   Conment, chevaliersire chevalier ! 
      Connoissiez vous le grant tresor de mon chier frere le prince ?Damoiselle, dist il, or sachiez que je n’ai pas mis moult grant painne en elles connoistre, 
   mes autres fois les avons nous veues. Si nous en couvient contenir a la guise du païs.Par mon chief, dist elle, je ne cuidoie mie que vous feusiez si creables !Damoiselle, dist il, si sonmes encore plus.Je ne le cuidoie pas, dist elle, a ce que je ai veu a telle y a.Conment ! dist il. A quoi est ce ?Je le sé bien, dist elle. Mais je ne vous en 
      vueil ore pas ci fere conte : assez ai a parler d’autre chose.
   Lors fu heure que l’en se deust asseoir au disner, et fu l’yaue cornee. Li baron et 
   la pucelle meismes et tout 
      li autrecil qui durent mengier 
   se sont assiss'assistrent a la table 
   et mengierent a la guise du paÿs 
   si ordeneement conme merveilles. Aprés mengier se mistrent ensemble 
   les pucelesli princes et 
   lili autre baron pour parler de plusseurs choses. 
   Si avint que li baron d’Antyoche se 
   mistrenttraistrent d’une part et conmencierent une raison 
   a mettre avant, car li plusseur distrent que 
   li roys de Jherusalem 
   estoit moult au desous de sa force de ce que il avoit perdu moult de ses mieudres chevaliers, et il meismes estoit navrez en tele 
   maniere que ill'en 
      cuidoientcuidoit vraiement que il deust morir et que bonne chose seroit 
      que on alast sus lui tandis conme il estoit en povre pointsi bas. \pend
\pstart Pas de nouveau § B
   EndementresEn ce qu’il estoient en ceste matire, 
   la pucelle tenoit Celydus a conseil et desfesoit bien ce que 
   son frere et li autre 
   faisoientcuidoient faire, et vous dirai conment. 
   Nouveau § BCelydus, 
   qui estoit moult sages de la loy Jhesucrist, avoit a ce mis la pucelle que l’amour Jhesucrist 
   estoits'estoit conjointe en l’amour 
   humaine, et non pas que ce fust 
   enpour nul 
   mauvés vice ne vilain 
   que li unsuns d'euls ne li autres vousist 
   acomplir chose qui tournast a nul conchiement d’amede dame
   ###La leçon de G dénote peut-être une mécompréhension ou une sortie de la dimension morale chrétienne 
   qui émerge des échanges de Celydus et Alerie.. 
   La pucelle tenoit 
   Celydus en ce que il disoit :
   Tres chiere damoiselle, ne cuidiez pas que la vostre grant valeur, qui a ce ataint, 
      me fet en vous connoistre parfaite honneur de science, grace de 
      biautébiauté de valeur, valeur de franchise 
      permenablepermenableté de ferme creance 
   et fin desirrier de perseverer en toutes bonnes vertus ?Ha ! Celydus, biau dous amis, 
      neq ne cuidiez pas que je 
      ne covoite moult l’eure que je vous puisse tenir en lieu ou conscience me reprengne de ce que j’ai tant atendu de moi mettre a fere 
      ce qui tournast aau profit de ferme creance et que 
      je puisse fere encore 
      conscience et penitance qui me tournast au proufist 
      dede l'ame et de salut ? \pend
\pstart Pas de nouveau § BDamoiselle, dist il, 
   pour Dieu merci, puisque il est ainsi que il plest 
   a Jhesucrist que vous soiez de ceus que il racheta de son precieus sanc quant il fu estendus en la Sainte Crois, 
   si pourchaciez tant que nous puissons venir ou lieu ou il morut a si grant honte com Sainte Eglise le tesmoingne.Or regardons donc, dist elle, maniere 
      conment nous en puissons mieux a chief venir, 
   car voirs est que mon frere le prince, cuide vraiement et pour ce 
      m’a ilmal envoïe 
      querre queci endroit que 
   je vous mette a ce que pour l’amour de moi doiez o lui demourer, et encore plus, que il s’acorderoit a ce volentiers que vous 
   me preissiezeussies a fenme avant 
      queque vous de lui departissiez.En non Dieu, dist Celidus, et j’en loe et regraci le benoit filz 
      de Dieu quant j’ai de lui tele grace que je en sui a ce venus ! \pend
\pstart Pas de nouveau § BA ce, dist 
   la puceleele, 
   estesen estes vous venus. 
   Si conmandez et en dites vostre avis, et je du tout leen ferai.En non Dieu, dist il, je 
      lel'aila pensse
      bien fere selonc ce que j’ai entendu de vous. 
   Il couvient que nous faingnons nostre volenté a ce que nous puissons venir a chief de ce que nous voulons fere. 
   Je vous mousterrai grant signe d’amour et vueil que li princes sache que il ne soit riens 
      que vous conmandez sus moi que je nen'en soie appareilliez du faire, 
      pour quoi vousvous li 
      pouezporrez dire que je o vous 
      m’en irai de quele heure que vous 
      vous en ailliez. 
      Et quant vous m’avrez pris ainsi conme a vostre loy, si me porrez conmander vostre bonne volenté. 
      Aprés si querrons et lieu et maniere de faire nostrevostre volenté.Ainsi l’otroie je, dist la pucele. \pend
\pstart Ainsi comEn ce que il 
   avoientont li uns et li autres 
   devisié, Dyanor si avoit bien eu son temps, car 
   les iii dames qui orent bien lesir de dire ce que il leur plaisoit, 
   pour quoi il n’i ot nule quil ne li deist en l’oreille :
   Sire, je sui ençainte de vous. Et bien sachiez que, de quele heure que 
      nous departironsvous partirez de 
      vous et vous vous en voudrez aler au gré de 
      mon seingneur, sachiez 
      que nous irons aprés vous en quel lieu que nous vous sachionssachons.En non Dieu, dist il, ce me pleraplaist 
      moult bien, mes que vous le puissiez fere sanz moivous grever.Oïl, distrent elles, mes 
      que vous nous diezdites nous en quel lieu vous irez quant vous 
      departirez de ci.En Jherusalem, dit il, 
      nous couvient traire avant que nous partons de ceste terre. \pend
\pstart Pas de nouveau § BAtant c’est 
   li princes embatus esen lor parolles 
   des iii dames et de Dyanor, 
   qui dist aussi com par jalousie :
   Dyanor, je vueil que 
      vousvous me dites de quoi vous parliez quant je m’embati sus
      vous.Sire, dist il, ce estoit de vostre guaite, qui me ledenga 
      de ce que je cueilli les rouges roses du 
      rosiervergier sanz 
      sonvostre congié.Amis, dist il, or ne vous en chaut, 
      car il a chier comparé et nous en avons eu riche marchié !
   AprésEn ce revint li princes a 
   Celydus et a sa sereur, et il se drecierent contre lui, 
   et il dist :
   Or voi je bien que ma suer 
      voudroit avoir i tel chevalier de mesniee.En non Dieu, frere, si l’avrai se il veult, et vous 
      aussiaussit (sic).En moi, dist li princes, ne demourra il pas.Non fera il en moi, distce dist 
   Celidus. \pend
\pstart Par mon chief, dist Celydus, 
   bien va li affaires et hautement, et j’en doi bien loer celui qui tous nous fist a s’ymage.
   Moult ont parlé et avant et arriere, mes en la fin prist 
   le prince la pucele 
   et la mist a conseil et dist :
   Ma tres chiere suer, je me fie moult en vous, car ces ii 
      François vous ai mis en main pour ce que je vous ai dit.Frere, dist elle, ce est si grant chose d’eulz. 
      CeCe me semble qu'il 
      m’est avis que trop avroit a faire qui de leur propre volenté les voudroit jeter, pour quoi il me semble que ce seroit 
      bonbon que ce seroit bon que vous lor 
      donnissiez franchement congié d’aler en quel lieu que il 
      voudrontlor pleroit, et vous dirai pourquoi. 
      Il est voir que mon cuer s’adonne a ce queque il m'est avis que, 
      se moije et Celydus feussions 
   tout d’une creance, que nous serions moult tost acordé li uns a l’autre. Et pource que li homs et la fame si ont naturel entendement 
   de couvoitier ce qu’il ne pueent pas avoir legierement m’est il avis que, se li 
      unsuns de nous 
      se departoitdepartoit ja de l’autre, 
   que em brief terme seroient moult engrant l’un de l’autre veoir son pareil. Et pource que reson demande toute ordenance 
      de ceste honnestéde chose honneste ai je esperance que 
      Celydus revendroit vers moi au plus tost que il pourroit selonc le semblant que il me moustre 
      et je a lui. Et vez ci la raison pour quoi je voudroie miex que nous 
      peussionspeussions miex a ce venir que nous couvoitons.Par mon chief, dist li princes, et je l’otroi ainsi, 
   car moult y a belle raison et bonne, mes que vous vueilliez venir a ce que mon cuer desirre.Ja Dieu, dist elle, ne place, qui tout fourma, que pour amor que je puisse avoir en vous 
   ne envers lui puisse fausser celui envers qui je me sui vouee et promise.Je ne le voudroie pas, dist li princes, 
   qui n’entendoit pas a ce qu’elle entendoit. \pend
\pstart Ainsi fu la chose prise en tele maniere entre 
   le prince et la pucele. Et aprés ce revindrent ensemble 
   la pucele et Celidus, et li dist 
   la damoiselle qu’ele avoit ainsi dist a 
      son frerefrere, com vous avez oÿ. 
   CelydusCele respondi :
   Ha ! conme ce est une voie, tres douce amie, par quoi li royaumes de 
      Jherusalem pourroit estre essauciez !Ce sai je bien, dist elle. Et pour ce vueil je que ce soit au plus tost que on pourra.
   Lors dist Celydus :
   Deque de quele heure que vous 
      departiezdepartezdepartirez de ci voudrai je prendre 
      congié d’aler en Jherusalem, et aprés ferai tant que je reviengne veoir 
      le princele prince et vous
      a plus d’amisanemis et a mains 
      d’anemisamis. Se 
      vousvous et lui 
      ainsi ne voulez croire mon conseil, je feré du miex que je pourré.
   Adonc estoient cil dui amant en tel lieu que la pucele li mist ses bras au col et l’eust, 
   ce cuit, baisié se elle ne l’eust lessié pour ii choses : l’une fu pource que confusion 
   dede ce que fenme ne doit pas estre 
   si habandonnee, et l’autre pource que elle n’estoit pas encore baptiziee en humanité, tout le feust ele esperituelment. Si dist :
   Dous amis, et je vous otroi par ceste couvenance mon cuer et toute ma penssee, 
      car ce sera miex que se je m’en aloie avec vous ausi conme 
      en larrecina larron." "Et je l'aime assez miex, dist il, donques.a larron. \pend
\pstart Moult parlerent longuement ensemble 
   entre Celydus et 
   la pucele. Aprés ce demoura la pucelle en 
   Antyoche iii jours et puis prist congié au prince 
   et aus ii cousins et s’en ala de la ou ele estoit venue. Celydus demoura avec 
   le prince tant que il fu touz garis, puis s’en vint a lui et li dist :
   Sire, a vostre congié nous couvient paier nostre veu, ainsi conme il est a faire a tote maniere 
      de gent.
   Quant li princes oï ce, si respondi moult debonerement et dist :
   Celydus, biau sire, quant il ne vous plera plus a demourer 
      avec moi, vous irez quel part que il vous plaira. Et sachiez que, se je ne cuidoie que vous prochainnement ne deussiez 
      revenirrepairier, que moult a envis vous lerai de moi departir.SireSire, 
      ce dist il, ne cuidiez pas que 
      je feusse si engrans de la departiedepartie de vous se je n’eusse 
      vraie esperance de retourner en cest païs ou je lessié mon cuer et toute ma penssee.
   Et quant li princes 
   entendi ceste parolle de Celydusde ceste chose, 
   sachiez que mout en ot grant joie et 
   li dist :
   Biau sire Celydus, 
      moultdist li princes, moult 
      vous devez ore prisier et amer, carque je sé bien que d’autrui 
      l’estes vous, dont je ne vueil pas tenir conte devant ce que je verrai que vous ferez. \pend
\pstart Pas de nouveau § BAtant ont leurs parolles a ce menees que 
   Celydus atournaatira son affaire, 
   et prist congié il et Dyanor et orent sauf conduit du prince 
   parmi la terre des Sarrazins jusques a tant que il 
      vindrentfurent en la terre des Crestiens. 
   Si me vueil ore atant taire d’eulz et 
      vueil venir a 
   Pelyarmenus, 
   dont li contes fet 
      icici endroit mencyondessus dit. \pend
         
         
            Ci devise li contes 
               conment Pelyarmenus
               tint l’empire de Rome aprés la mort de Fastidorus, 
               son frere, pource qu’il n’estoit demouré nul hoir de lui, 
               et devise 
               conment Pelyarmenus ala veoir Helcanus 
               et Dorus, son frereses freres, 
               en leur paÿs
               Ci vient li contes a Peliarmenus, nouvel empereour de Romme
               Ci se taist li contes de Celydus et parole de Pelyarmenus.
            
               Enluminure sur 1 colonne (f) et 14 UR.
                  Pelyarmenus, empereur de Rome, 
                  voyage couronné à cheval suivi de trois personnages vers 
                  Constantinople pour voir 
                  Helcanus et Dorus. 
               
            
            \pstart Ci endroit distrepaire 
   li contes, ainsi com vous avez devant oÿ, si conmeque il plut a 
   Nostre Seingneur, que li preudonme vont plus tost a la fois a la mort que ne font cil 
      de qui on ce soufferroit moult volentiersbien###L'expression 
         sentencieuse n'apparaît pas chez Morawski ni chez De Roux de Ligny. PD stp, 
   tout aussi conme je vous ai conté de 
   Fastidorus, qui assez tost morut. Aprés ce que 
   li bons emperieres ce fu hors mis de son empire, 
   Pelyarmenus, ainsi conme je vous 
      ai conmenciécommençai a dire, 
   vint a l’empire pource que Fastidorus n’avoit nul hoir de sa char. Cilz, qui n’estoit pas de mal 
   vuidiez, penssa a la male racine dont onquesonques il ne se 
   voltpot ou ne volt retraire.
               Comme toujours, la narration est explicite quant à la vilennie de Pelyarmenus, 
               lui qui est plein de mal et dédie même ses réflexions à ce qui nous est présenté comme sa mauvaise racine. 
               Sa mauvaiseté, répétée, se fait ainsi proprement organique.
               Dont il avint que dedenz le moys 
   qu’il fu asseurez de ses 
   barons de l’empire et osta tous ceus que il avoit trouvez en office, 
   et ily mist autres tieux conme il cuida qu’il fussent 
   miexne miex 
   a son acorttort soutenir. Et qu’en avint ? Si 
   li empereresempires 
   ### fu a ce menez 
   que touz les baillifs, en quele maniere qu’il peussentporent avoir deniers, 
   fust a tort, fustou a droit, 
   siils aportoient tout au tresor 
   l’empereeur. Vez ici empire malmené, car nul 
   nene se pooit 
   marcheander ne 
   chevirchevir dedens Romme 
   de loial marcheandise ne de loyal aquest, si que barat et tricherie fu partout ci au desus que droiture et loiauté aloient mendiant 
   tout aussi conme elle fet maintenant en aucun lieu que je bien diroie, et tout par defaute de bon seingneur 
   et de ses mauvés menistres. Et pource que 
      chascunchascuns preudoms puet bien savoir 
      se je di verité m’en vueil jeje maintenant taire atant. \pend
\pstart Pas de nouveau § BLors avint en celui termine a 
   Rome que aus preudonmes anciens 
   et aus dames de bone vie et aus pucelles de valeur, qui cure n’avoient de vilonnie, li jovencel, qui de naturel entendement 
   se devoient mettre au bien fere, et li mal ordenez qui ne chaçoient fors hustin et laidir ceus dont j’ai dessus 
   dit et se prenoient a eulz pour  
   avoiravoir Et se prenoient a eulx pour
   achoison de malfaire. Cil estoient batu et encusélaidengié et trait en
   causecause-et trait en cause devant baillis et devant prevos,
   et devant touz ceus qui estoient establi pour emplir la bourse a l’empereeur, 
   si que li mesdisant et ceus qui mellee fesoient dont deniers pooient naistre estoient bienvenus 
   entre les baillis et entre les prevos a l’empereeur. 
   Et ceus apele je ahaniers au deableOn note une fois encore l'association
   diabolique de Pelyarmenus, ses prévôts étant qualifiés de cultivateurs du diable, soit ses propres ouvriers en effet., 
   c’est a dire a touz mauvés princes pour laquele chose cil de Rome quil ne voloient 
   que pais et droiture se metoient en leur chambre priveement quant on leur avoit tolu le leur ou batus ou malmenez 
   et disoient en lermes et en pleurs tres piteusement :
   Ha ! sire emperieres Fastidorus, 
   de la vostre mort nous est il mal avenu ! Sire, petit avez regné aprés celui qui ne cuidoit pas 
      que vous deussiez si pou vivre. 
   Sire, a mort ne vous tenonsveons 
      pasnous pas, mes encore atendons nous de vous 
      aucun secours et aucun recouvrier. 
      Sire, voirementvoirement se 
      disoientdistrent voir li pluseur qui 
      distrent qu’en bon hermitage entre cil qui en loyauté de justice demeure###L'expression 
         sentencieuse n'apparaît pas chez Morawski ni chez De Roux de Ligny. PD stp, 
      carsire encore ne poons nous veoir que, se vous feussiez demouré tant que Nostre Sire vous vousist donner santé, 
      que ja pour ce mains de bien feust demourez a faire. \pend 
\pstart Ainsi plaingnoit en Rome chascun son donmage 
   endroit soi de leur nouvel emperere, 
   qui a nule droiture ne vouloit venir por tant que deniers li peussent estendre.Il s'agit 
   d'une orientation intéressante de la vilennie de Pelyarmenus une fois sur le trône, son vice s'incarnant cette fois dans ces penchants
   matérialistes qui viennent incarner son mauvais gouvernement. 
   Et qu’en avint il ? Il 
   aünaamassa si grant avoir et 
   mistsi grant 
   en tresorensembletresor que 
   il couvoita l’empire de 
   Costentinoble a joindre avec le sien. Mes tant doutoit ses freres 
   Helcanus et Dorus qu’il ne les osoit envahir sanz pensser traïson, 
   dont il avint que il fist sa voie aprester au plus noblement que il pot et vint en Gresce 
   et fist savoir a son frere sa venue. Cilz, quil ne se doutoient de nule traïson 
   qui nuire li peust, vint contre lui a moult grant joie et conjouy l’un l’autre par grant signe d’amour. 
   Pelyarmenus, qui savoit 
   moultplus que nul autre couvenablement parler, li dist :
   Frere, je vieng a vous em partie de grant amour, et pource que vous estes mon ainsné frere le dis je.
      La nuance que pose Pelyarmenus dans ce "grant amour" porte à sourire. 
      Il n'hésite d'ailleurs pas à exprimer la caractère attendu de cette affection, mais c'est surtout la minimisation du
     moteur que devrait constituer cet amour dans sa venue à Constantinople qui s'avère la plus ironique. On observe une tension 
      entre la trahison que Pelyarmenus est présenté, de manière tout aussi ironique, de pas oser envisager, celle qu'Helcanus ne peut 
      imaginer, de manière assez naïve en regard de leurs précédentes confrontations, et le jeu sur le semblant d'amour qu'ils se renvoient,
      que Pelyarmenus souhaite clamer, celui qu'Helcanus prête à son demi-frère (de manière cependant là aussi orientée, puisque seule la 
      couleur d'amour est évoquée), et la parfaite compréhension, au final, des mauvaises intentions de Pelyarmenus pour Helcanus 
      qui n'en affiche rien pour autant.
   Et d’autre part, j’ai moult grant desir de savoir novelles de nostre bon pere, 
   qui se parti de Rome si conme vous savez.Biau frere, dist Helcanus, vous m’estes moult honnestement venus 
      veoir. Que vous soiez li bienvenus ! Et les raisons por quoi vous l’avez fet si 
      aont couleur d’amour. Et une autre fois vous irai ausi veoir 
      quant je savrai que poins et lieus 
      en sera. Si me dites conment il vous est, que a moi, Dieu merci, 
      n’est il se tout bien non.Tout aussi vous puis je dire, distde 
      Pelyarmenus. 
   Et de nostre chier frere Dorus, quant en oïstes vous nouvelles ?Il n’a pas iii jours, 
      dist Helcanus, que je le vi. 
      Il est touz sains et touz hestiez. Mes 
      li dux, ses sires, est mort, dont il est moult courrouciez.Ainsi avient il, 
      Pelyarmenus dist, 
      de ceus qui plus ne pueent vivre !Ainsi va, dist Helcanus, 
      le mort au mort, le vif au vif. 
   Et pour ce feroit bon bien fere qui bon seroit, car aprés nostre mort trouverons nous qui pou de bien fera pour nous. 
   Et pource que je ai encore eu pou d’espace de respondre vous de monseingneur vostre 
      perepere et le nostre
   vous puis je dire que j’ai envoié iiii mesages par toutes les iiii parties du monde por savoir et enquerre de lui, 
      mes onques puis n’en oï nouvelles. Et de Celydus, 
      nostrevostre frere, 
   dist Helcanus, en oïstes vous puis nule chose ?Certes, dist il, nenil.
   Que vous iroiediroie je 
      faisant plus lonc conteconte de chose ou il
         n'avroit deduit a l'escouter ? 
   Helcanus conjoÿ son frere et le mena de lieu en lieu, 
   et il fu soutilz et vit bien que Pelyarmenus n’estoit venus pour nul 
   bienbien ne pour nul avancement de lui en son empire. 
   Et nonpourquant, il n’en daingna onques semblant fere. Et li traitres si sot 
   si couvertement parler de ce que il volt que ja 
   n’en fust aperceus dene s'en aperceust nuls qui bien ne le seust.
La relation des deux frères prend des proportions proprement guerrières, dans une perspective rusée qui 
débute donc au niveau des émotions qu'ils choisissent de se manifester ou se dissimuler. À la prétendue joie et au prétendu amour 
qu'ils expriment de prime abord vient donc s'opposer cette dissimulation de part et d'autre, ce qui fait ressortir surtout la prise 
de conscience d'Helcanus, sa subtilité, mais aussi la noirceur de Pelyarmenus qui ne peut plus leurrer que celui qui
"bien ne le seust". \pend
\pstart Cil qui demouré estoient hoir de ceus qui 
   la bonne dame, mere Helcanus, avoient traye, 
   lor estoit encore une estincele demouré dedenz leou cuer qui bien voloit 
   el tyson ardoir de celui qu’il ne queroit el fors que il trouvast par quoi il fust alumez. Et cel tyson estoit 
   Pelyarmenus, qui acquist ce que cil li orent en couvent que, se il savoit nul droit 
   enen l'empire de Constentinoble, 
   que il y demandast son droit et que ja por eulz ne le lessast, car il amoient miex son droit que le tort 
   leur seingneur.
   Par foy, dist il, biaus seingneurs, vous 
      avez moult sagement respondu. 
   Et je ne voudroie chose demander a mon frere ou je n’eusse bonne reson, 
   et vous dirai quele. Il est voir que anciennement Rome si a esté souverainne de toutes villes 
   que on doit apeler cités, et bien le vous moustre nostre pere esperituel, ce est li pappes, 
   qui a Rome demeure, a qui nous devons touz reverence. Et a ce s’ensuit 
      il que quiconques soit emperere de 
      RomeLa leçon de G semble 
      vouloir gommer, ou ne pas comprendre, les arguments séditieux de Pelyarmenus en ne conservant que la supériorité papale, 
      et non celle que Pelyarmenus semble vouloir tirer, de sa position de souverain de Rome, sur le reste du monde., 
      tout autre seingneur terrien doivent estre a celui aclin.
   Chascuns dist que bien y aavoit 
      reson pour quoi. Aussi ot 
   Pelyarmenus achoison de soi meller a son frere, 
   si conme vous orrez avant
      avant mes 
      que seque je laisse 
      monle conte. \pend
\pstart Quant Helcanus ot conjouy 
   Pelyarmenus, si ne pot plus demourer ou paÿs, 
   ainz prist congié, si repaira a Rome, qui moult estoit vuide de loyauté, de droiture, 
   si conme j’ai devant dittouchié, 
   pour quoi il ne demoura se pou non que Pelyarmenus 
   prist i chevalier, qui avoit non Mainfroy 
   – celui avoit fausseté et traïson en soy plus que nul autre –, si li dist Pelyarmenus que 
   il alast en Constentinonble 
   et deist a son frere ce que vous pourrez oïr. 
   Mainfroy, qui ne mist pas en oubli ce qui li fu enchargié, ala tant, l’un jour
   plus, 
   l’autre mains, que il vint en Constentinonble a i jour de Pasques 
   que Helcanus avoit compaingnie o soi des barons de la terre. Il vint devant lui, 
   et fu ainsint comque li baron avoient mengié. Lors li bailla une lettre 
   en sa main de creance. 
   Helcanus la lut, puis esgarda celui, mes il ne le connut pas, si dist aussi conme celui 
   que li cuers dist que il n’aportoit pasnule bonnes 
   nouvellesnouvelles a celui :
   Que veulz tu dire ?, aussi conme en courrous. 
   Cilz, qui l’esgarda ou vis, ot grant paour, car moult avoit Helcanus fier regart quant il estoit 
   courrouciez. Et de l’autre part, il n’aportoit pas nouvelles quiqui li 
   deussent plaire. Si dist en tele maniere que mesage doit fere le conmandement son seingneur 
      ne ne doit mal reçoivre ne pis oïr. Et quant il ot ce dit, si conmença a dire :
   Par foy, tres chier sires, voirs est que monseingneur vostre frere 
      si m’envoie a vous pour vous fere asavoir une chose que miex amasse qu’i la vous eust mandee par 
      escript que ce qu’i lale 
      m'escouvenist dire.Ces scrupules, dans la bouche d'un messager dépeint d'emblée pour ses vices, 
      dénotent combien sont vils les plans de Pelyarmenus.
   Dis hardiement ! dist Helcanus.
   Sire, dist il, volentiers, 
      puisque vous le voulez. \pend 
\pstart Pas de nouveau § BOn a fait entendant a 
   mon seingneur 
   que anciennement Constentinonble 
   estoita esté subgiete a Rome 
   et que li emperieres de Costentinoble 
   fesoitfait honmage a 
   l’empireempereour de 
   Rome. Et pour ceste reson, vostre frere vous mande que 
   vous ne le tenieztenez mie a orgueil ne a 
      felonnie, ainz vous fet en guerredon et en amour savoir que vous 
      veingniez a Rome aussi conme pour lui veoir, et la li pourrez faire honmage. \pend
\pstart Quant Helcanus ot 
   ceceli entendu, si ne respondi pas moult hastivement, 
   ainz se refraint en son ire ou il estoit et puis conmença a faire aussi conme 
   i faus ris et dist a celui :
   Veulz tu plus nule chose dire ?Sire, dist il, nenil, jusques a tant que je aie 
      oïoï autre response de vous.
   Lors fist Helcanus venir touz les barons qui la furent entour soi et fist a celui redire 
   tout ce que vous avez oï. \pend
\pstart Pas de nouveau § BQuant chascun ot ce entendu, si dist 
   Helcanus :
   Ore, biaus seingneurs, avez vous oïentendu 
      dede mon tres chier frere Pelyarmenus. 
      Vous semble il que il ait reson en ce que il ditvous ayez oÿ ?
   Aucuns y ot quil ne se porent taire qu’il ne deissent :
   Sire, il nous semble que nous aions une terre couvoitiee du deable 
      et que nous ne puissons jamés vivre em pes !Biau seingneurs, ce dist il, couvoitise et envie ne 
      peutpeuvent jamés 
      assez avoirL'association, une fois encore, de Pelyarmenus au diable, résume le personnage 
         au moment de le voir se lancer dans cet énième temps de guerre qui le confronte à sa fratrie. En belle démonstration des 
         qualités qui lui ont maintenant été investies avec son intronisation, Helcanus se montre tout à fait lucide, capable d'identifier
         les deux vices que son demi-frère vient incarner, en symbole du mauvais gouvernement.. Et ne vous merveilliez pas de Pelyarmenus, car il resemble l’ueil que j’ai oï 
      conter queque le couvoiteus 
      Alixandre trouva souz le perron de marbre, quil li 
      moustroitmoustre par senefiance qu’il ne seroit ja plains, 
      tout fust il ainsi que nuls ne peust contre lui contrester. 
   Ausi ne puet sousfire a Pelyarmenus chose ou il ne doit riens avoir. 
   Et pource que je cuit que Pelyarmenus ait pris cuer en aucun de vous
      me couvient il que sanz le conseil de mes anemis li remande tele response. 
      Os tu, vassal qui m'es ci envoiez de par Pelyarmenus de 
      Rome ? Di a ton seingneur que je li mande que, 
   de l’empire dont mon pere fu heritez et reçut les honmages, 
      que il ne soit tant osez que, 
      puisquepuis l'eure que 
      tu li avras dit, 
      demoure en lieu ou je sache que il tiengne denué de terreseignorie. 
   Et se il ce ne fait et il vient en lieu ou je le puisse trouver, la pourra recevoir l’onmage de moi tel 
   conme je li voudrai faire !Cette intervention d'Helcanus n'est pas sans rappeler 
      celles qu'il opposait aux sages dans le Roman de Cassidorus (###CC). Elle renoue en effet, sous forme condensée, 
   avec le motif du conte enchâssé avec cette référence à Alexandre le Grand, qui n'est cependant ici pas développée (les seuls 
   contes proprement insérés dans ce roman dédié à Pelyarmenus n'étant plus qu'associés à cette vile figure) et l'interprétation 
   conclusive qu'en offre Helcanus. \pend
\pstart Pas de nouveau § BAtant fu cil touz liez 
   quant il se pot partir de Helcanus 
   et vint a Rome et conta a son seingneur 
   de que il avoit trouvé. Pelyarmenus
   la response Helcanus, qui ne s’effrea pas, moult manda partout a 
   sesses a ses amis ce qu’i li plot, et non pas du 
   touttout que l'en li aidast par amors ne par lignage, mes pour le sien.
   Cette précision quant aux raisons qui animent l'ensemble des traîtres, passés et présents, à se rallier 
   à Pelyarmenus est révélatrice du renversement que vient opérer Pelyarmenus, notamment en regard de la logique lignagère posée 
   au coeur du récit du cycle. C'est d'ailleurs cependant cette argument lignagier qui perdure, au contraire du moteur affectif 
   qui se voit renversé en peur, comme pour bien manifester le régime de terreur qu'instaure Pelyarmenus.
   Dont il avint que li couvoiteus couvoitierent l’avoir et li autre le vodrent faire par lignage et li autre par cremeur. 
   Si n’avint onques a nul empereeur qui fust a Rome qu’i meist tant de gent ensemble 
   conme fist Pelyarmenus pour aler en Gresce. \pend
\pstart Helcanus, d’autre part, avoit mandé a son frere 
   Dorus ceste chose et a touz ceus ou il cuidoit avoir secours que il li vousissent aidier son regne 
   a desfendre. Nouveau § X2Dorus, qui grant despit avoit de celui 
   que il avoit jadis mis en telle merci, conme devant 
      a esté ditest contenu, jura que, sese il jamés 
   puet avenir que il en soitfust au desus, que il ne conchieroit jamés ne lui ne 
   autreautre ne conchieroit. Il ne fu pas pereceus 
   que partout la ou il cuidoit avoir secours et aide 
   il envoia. 
   A celui temps estoit mors le bon conte Robert de Flandres, 
   qui la concorde avoit faite, si conme devant a esté dit
      il est contenu en cest livre. Mes i autre prince qui avoit 
   la fille au bon conte, qui Mardocheus estoit apelez, 
   regnoit a celui temps en Flandres et estoit filz au 
   bonpreus 
      ducduc, Karus de Nise, 
   qui encore regnoit en Grece. Cestui Mardocheus 
   si fu moult couvoiteus de faire secours au bon Helcanus, si prist tant pou de bonne chevalerie 
   conme il avoit et vintvint sans sejor en Grece. \pend
\pstart DaphusJaphus 
   li FrizonsLa leçon de G est la plus cohérente : 
   c'est Japhus qui tient le trône de Frise, et non Daphus, associé aux terres espagnoles du Chastel Mignot de Celydoine.
   C'est d'ailleurs de Japhus et non de Daphus dont il est question plus bas dans la liste des combattants en présence (§751). 
   et maint autre baron que il amena o lui 
   ne se mistrent pas derriere. Josyas d’Espaingne, 
   qui autre fois avoit esté ou paÿs, si avoit mis son cuer en la femne 
   LeusLeris, 
   fille a l’empereeur Cassydorus, qui avoit a non Cassydore
   Il n'est plus question ensuite de cet amour, Cassydore étant remariée, contre son gré, à Raimfort, 
   fils de Dyomarques d'Aragon, sans que ne soit mentionnée une relation avec le fils du roi d'Espagne.. 
   Cil n’avoit voulu aidier au couvoiteus Pelyarmenus pour nul denier, 
   ainz vint en l’ayde Helcanus moult esforcieement. \pend
\pstart Pas de nouveau § BMoult ot Helcanus 
   grant confort et d’uns et d’autres et assembla ses oz es plains de Tabour, 
   ouou-ou il avoit par conte 
   cm combatans. 
   Pelyarmenus, qui d’autre part revint, en avoit trop plus, car, 
   ainsi conme l’ystoire le tesmoingne, il en avoit c et l mille, que a pié que a cheval. 
   Ci endroit puet on auques savoir que, selonc ce qui a esté ailleurs contenu, que moult y avoit de bons chevaliers, 
   et ne cuidoient ja assez venir a temps ensemble, pour coi il chevauchoient si asprement 
   que il sembloit a chascun que il eust tout gaaingnié. \pend 
\pstart Pas de nouveau § BPelyarmenus, 
   qui a ce avoit mise sa cure que il ne li chaloit que la chose coustast mes qu’il peust mettre ses ii freres a 
   mortmort, car le cuers y avoit mis, si fist faire trente paire de lettres, 
   que il tramist a divers princes que, qui li pourroit rendre Helcanus 
   et Dorus mors ou vis, il li donroit l’empire de Costentinoble 
   etet li donroit s'amor et 
   devendroit son honme tout ligement. De ces lettres ot il x paire en l’ost Helcanus 
   et ou sien xx paire. Ha ! Deux, conme ci avoit cruel traïson quant li vaillans princes ne savoit le mortel encombrier qui 
   leur estoit a avenir de ceus meismes qui plus pres lilor estoient ! 
   Ha ! traïson, conme tu es invisible 
   quant il n’est nuls qui de toi se puisse garder ! Et vraiement, ci le puet on bien veoir, car li faus 
   Pelyarmenus ne cuida ja assez 
   avoir neveoir 
   ale temps venir 
   a sa grant confusion ne l’autre partie a son grant martire. Lors ont tant chevauchié li uns sus les autres que les os ce sont 
   entraprochié, et virent l’un l’autre qui a merveilles faisoient a 
   redouter. \pend
\pstart Dorus, qui la rage avoit dedens soi et ne li chaloit quel 
   painnepainne ne quel ahaut 
   il souffrist, mes qu’il peust acomplir sa volenté, pour quoi il mist sa gent a une part ; 
   Mardocheus, d’autre part, et 
   Japhus aussi, Josyas 
   d’Espaingne, 
   Karus de Nise, Mirrus et 
   Heleas, Nestor et Lycorus 
   et maint autre baron, dont li contes fet assez pou de mencion. 
   Pelyarmenus, d’autred'autre d'autre 
   part, avoit mainte bataille, 
   dont li contes ne touche pas moult, fors d’aucuns de qui li contes s’est teus de ci a ore. 
      Et vous dirai desquelz :le premier 
   li roys d’Arragon 
   i fu tres efforcieement ; 
   li marchis de Fabonne ; 
   li vassaus de Terre Labour ; 
   li princes d’Aquilee ; 
   le duc de Puille ; 
   le marchis de Montyr ; 
   le roy de Sezille ; 
   li quens de Pyse ; 
   li viquens de Naples ; 
   li princes de Lamoree ; 
   et tant d’autres que nulz ne le pourroitpeust pensser. \pend
         
         
            Ci devise 
               conment Pelyarmenus 
               assembla ses oz pour aler en Costentinoble 
               et mettre le preus Helcanus et 
               Dorus, son frere, 
               a mort par traÿson. Et aprés devisse conment Mirrus 
               ala en mesage a Pelyarmenus de par 
                  Helcanus.
            
               Enluminure sur 1 colonne (c) et 12 UR.
                  L’armée de Pelyarmenus attaque 
                  Constantinople pour mettre à mort 
                  Helcanus et Dorus. 
               
            
\pstart Ci endroit dist li contes que
   quanten ce que les ii parties 
   se furent 
   aprestees de ferir ensemble vint i sage prince de Grece qui avoit a non 
   Karus, leestoit 
      duc de Nise, qui 
   avoit servi d’armesarmes a l’empereeur 
   Cassydorus. Cil vint a Helcanus et li dist :
   Sire, si vraiement m’eïst li Sires qui tout fourma que moult devez grant paour avoir.
      Je nel leré ja 
      que je ne le vous die. Il est voir que la vostre force si est moult grant selonc ce que nous cuidons bien avoir droit, 
      ainsi comme on le nous 
      donnea donné a entendre 
      ici. Mes les aventures si sont moult merveilleuses, si ne vous 
      doit pas anuier. Et se vous metez vostre frere 
   de vostre droit en son tort le plus tost que vous poez, je irai parler a vostre frere 
      et savrai quel onmage il veult avoir de vous. Il puet bien avenir que on li a 
      donné afait entendre chose ou il 
      aait droit a ce qu’il a entrepris. 
   Si l’en blasmerai de par vous de ce que je cuiderai ou il ait reson, et du seurplus vous en ouverrez par conseil.Ha ! sire dux, de conseil ne me verrez vous ja retraire. Vous m’avez maintes fois bien servi, 
      et ferez encore tant com il vous plaira. Si en feraifaites vostre 
      volenté, mes que Dorus, mon frere, s’i acort. 
   Il li fu mandez, si li dist on ceste chose.
   Fi ! dist il. Quant ja a ceste chose ne 
      m’acorderai que li sires requiere son serf ja 
   par celui Dieu qui de ma mere me fist nestre ! Se vous premiers envoiez a lui que il a vous, ja puis n’i ferrez cop en vostre avancement.Ha ! chevalier de tres grant cuer et de parfaite puissance, dist Helcanus, 
      ne cuidiez pas que je sanzsanz vous et vostre conseil en oeuvre. 
      Mes Karus ne le dit a mon avis 
   que pour droiture, et pource que qui s’umelie, il se souhauceVariante
      du proverbe "Qui s'abesse eus l'essauce" (Morawski 2118) ###PD ?.Voire, dist Dorus, vers honme de value. 
      Voire, mes ci ne voi je 
      quefors orgueil et felonnie. 
   Et quant plus s’umelie le preudonme vers le felon et plus li donne cuer de confondre celui que il veult destruire !Frere, dist Helcanus, il avient souvent.
   Quant Karus oï ce, si ot grant despit de ce que son conseil ne fu creus et dist :
   Sire, ne cuidiezcuidiez vous pas que 
      pour nulle male couvoitise je vous aie dit ce que vous avez oï, 
   car je ne me doute pas que, quant ce vendra as cops donner, que je m’en traie ensus.Biau sire Karus, 
      distce dist Dorus, de ce n’avons nous nule 
      doubte, mes je m’apensse d’une chose que j’ai entendu : que il ont la 
      l mile honmes plus que 
      nousnous n'avons. Et pource que le fol serf en qui 
      n’a nulefoi ne merci set que la force en est seue me doubteroie 
   je qu’il ne nous eust en despit.Sire, dist Karus, je vous tieng a moult sage 
      et non pas si que jeje ne prise plus la vostre chevalerie 
      et le hardement qui en vous est que nulle autre chose. Et ne vous anuit i mot que je vous vueil encore dire, 
      et puis ira chascun de nous a sa bataille conme ceus qui n’avons que ester. Il est voir tout ce 
      qu’que vousavez enpenssé et dist endroit de ce que 
   quiconques s’umelie envers felon, que il li donne cuer de soi confondre.En non Dieu, ce dist 
      Dorus, ce ne di je pas, 
   ainz dis que il li donne cuer de confondre celui que il veult destruire.Sire, dist il, ainsi l’ai je entendu. Mes je le vouloie dire par autres parolles. 
   Et pour ce n’ai je pas oublié ce qui contraire est a celui qui cuer prendroit contre douce requeste. 
   Les mescheances aviennent par orgueil. Et vous ne verrez ja venir honme au dessus de refuser reson et droiture que, 
   s’il en vient au desus, que ce ne soit victoire au deable. Et pour ceste reson ai je souvent oï dire que plusseurs en ont esté 
   maumis quant il avoient bon droit et il se metoient hors de conseil qui a humilité et raison s’acordoit.
   Une fois encore, le récit se fait sapiential et se rapproche de la veine des miroirs aux princes 
   au sujet de ces critères de bon gouvernement, éclairé par les conseils dispensés et par ces vertus de raison et de droiture,
   par opposition à l'orgueil qui équivaut à la "victoire au deable", qui paraît ainsi à la fois associée à celle des vices et à 
   celle de Pelyarmenus qui vient incarner ce mauvais gouvernement. \pend
\pstart Quant Karus ot ce dit, si n’i ot nul qui ne 
   dieait dit a lui meismes que 
   moult est sages et parfés en chevalerie. Mes il n’i ot nul si hardi qui osast respondre pour 
   Dorus, qui dist aprés le duc :
   Il me poise du serement que j’ai fet, car je ne me donnoie garde de ce point, 
      ne je ne vueil pas que ce conseil demeure pour moi, car ilqui est 
      droituriers. Et Deux le me pardoint, si vraiement quecom 
   je n’i gardoie que droiture a mon avis !
   Lors s’est trais Karus a sa bataille sanz plus dire. 
   Et i autre princeprin-prince, qui 
   ot non Myrrus, 
   si vint a Dorus et li dist :
   Sire, ne vous anuit de vostre serement, que sachiez 
      NostreMon Seingneur 
      connoist bien vostre maniere, 
      car on ditje li oÿ dire que serement qui 
      n’estestoit fet par conseil n’estoit pas a tenir a celui 
      qui veoit qu’il n’i avoit bonne reson. Et d’autre part, monseingneur vostre frere n’i 
      envoiera pas, ainz irons et moije 
      etou 
      Karus de nostre plainne volenté.Amis, dist il, je vois a ma gent, car il m’est avis que je les voi ci venir sus nous. Si alez quel 
      part qu’il vous plera, car je m’otroie a touz conseus, mes qu’il ne soient 
      de felon ne de traïteur. \pend
\pstart Pas de nouveau § BAtant vint Mirrus a Karus et li a dit :
   Sire, de vostre conseil ne se doit nuls retraire. Aler 
      nousvous couvient el mesage.MyrrusMirrus, Mirrus, 
      dist Karus, ce est a tart ! 
   Vez ci nos anemis ou il viennent fier et orgueillieux pource que il sevent lor victoire. Mes nous, qui avons a souffrir ce que vous 
   pourrez encore anuit savoir, nous dotons. Et i seul jour de respit vaut cent mars. Et a la journee d’ui, perdre nous 
      couvient, mes je ne sé pas combien. Sire, tournez a vostre gent, 
      car je me doute, se vous outre passez, que vous ne soiez retenus. \pend
\pstart Pas de nouveau § BAtant se mist Mirrus 
   a la voie en guise d’onme qui vouloit aler enconter mesage, 
   et on li fist voie jusques a Pelyarmenus, qu’il conmanda 
   a sa gent 
   a
   eulz ester jusques a tant que il eust dit son 
   mesage,mesage, et on li fist voie jusques a Pelyarmenus. 
   Nouveau § BMirrus, qui sages homs estoit, vint a 
      Pelyarmenus et li dist :
   SireSire, dist il, je vieng a 
      vous de par Helcanus, vostre frere, 
   qui est moult esbahis de ce que vous venez en tel maniere sor lui.Myrrus, dist il, bien savez la raison, 
      et encore me plaist il que vous miex le sachiez. 
   Il est voir que je mandai a vostre seingneur par grant signe d’amour qu’il me venist veoir a 
      Rome aussi conmeque j’estoie lui, 
      et il me remanda et orgueil et felonnie, 
   en tele maniere que je sa terre et l’empire, dont son pere avoit esté herité et receu les 
      hommagesdonmages, vuidace ou je venisse en lieu que je et 
      liil 
      lele le peussions desresnier, car je n’i avoie droit, 
   conme cil qui pas n’avoie esté engendrez de son pere. 
      Et de ce ne me doi je pas mout courroucier, car on set bien que sa mere 
      en fu chaciee de l’empire pource qu’elle s’abandonnoit aus garçons et 
      ailleurs, dont il fu engendrez. Et cele, qui tant sot de l’art de nigromance, 
      fist tant que le deable la gari a ce que elle s’acorda puis a mon pere, qui 
      ma mere en a menee, ainsi 
      com l'encom 
      setle set bien. 
   Si le comparra vostre seingneur, si m’eïst 
      DeuxNostre Seigneur, avant que 
      je retournerepaire 
      jamés en mon paÿs ne en l’empire de 
      Rome !Il s'agit là d'une déformation importante des faits 
      bien sûr, en ce qui concerne le début de ce nouvel épisode de guerre en tous cas. Ce n'est évidemment pas une invitation 
      de courtoisie qu'avait fait parvenir Pelyarmenus à son demi-frère et Helcanus ne l'a en aucun cas accusé de n'être pas de son sang.
      Pelyarmenus semble une fois encore animé du ressentiment que lui a inspiré la dénégation de Cassidorus, alors sous des traits 
      (pseudo)anonymes, d'être son père. Or, si la position bâtarde de Pelyarmenus a en effet été avancée par Helcanus, ce n'est en 
      rien en regard de la paternité de Cassidorus, mais plutôt de la légitimité de son union avec Fastige tandis que sa mère, Helcana, 
      n'était en réalité qu'exilée et non décédée. Non content de prêter une vile accusation à Helcanus pour légitimer ses envies guerrières, 
      Pelyarmenus avance pour sa part des accusations très déshonnêtes à l'égard d'Helcana et des raisons de son exil, qu'il associe
      aux mauvaises moeurs d'Helcana et son retour seulement à une origine diabolique, qui aurait favorisé son "art de nigromance" 
      (qui ne peut pour sa part être réellement remis en question, Helcana ayant orchestré son union avec Cassidorus par le biais de
      visites qu'elle lui rendait en rêve pour le convaincre de se marier). Cette forme de distorsion des faits nous paraît relever 
      de l'esprit de ruse qu'incarne Pelyarmenus tout au long de ses aventures. \pend
\pstart Pas de nouveau § BSire, 
   dist Mirrus, je ne sui pas pour ce venus 
   ci, mes s’il vous plaisoit je respondroie a ce que 
   l’enele vous a fet entendant.Dites tost, dist il, quecar 
      je n’ai que fere de votre sermon !Sire, dist Mirrus, 
      ainzmais pas je ne vueil que 
      vousvous ne sachiez la verité 
   de ce que je vous vueil dire. Il est voir, aussi conme vous avez dit, que vous li mandastes assez courtoisement que il venist a 
   Rome et que vous aviez entendu que li emperieres de 
   Constentinonble devoit estre honme a l’empereeur de 
   Rome et que vous vouliez que il fust vostre honme. Mes sires, 
      qui pas ne cuidoit que vous ce li deussiez mander que autres n’avoit fet a ces devanciers, 
      li dist et respondi 
      aussia ce 
   conme homs corrouciez que vous avant tenriez l’empire de lui que il de vous le sien, 
      et s’il estoit ainsi que vous ces choses vousissiezvousissez mettre a 
      point, si feissiez respitier ceste journee, et mon seingneur le feroit de l’autre partie.Myrrus, dist il, vous dites merveilles ! 
      SeSire, se vostre seingneur se veult 
      mettre devers moi tant que jl’en sache la 
      verité, je le sousferrai venir a amende, et se ce non, 
      si l’enait qui bien l’a deservi !
   Myrrus, qui bien fu apenssez, dit :
   SireSire, bien avez dit, je le ferai a 
      mon seingnor ainsintainsint savoir.Alez, dist il. \pend
\pstart Ainsi se parti sagement Mirrus du 
   traitre, car autrement l’eust il detenu. Et quant il revint arriere, il leur conta 
   la felonnie que il avoit trouvé, et dont il avint que chascun preudonme deot 
   cuer de lyon et li autre orent paour, et li tiers couvoitierent la destruicion de leur seingneur droiturier. Hé ! las, com grant pitié
   de si tres noble vassal ! Et Nostre Sires, qui ne volt fors les siens, 
   apar glaive
   souffrirsouffri que ses ii os s’esmurent a venir ensemble, 
   et vous dirai conment. Dorus, qui cuidoit bien sormonter a 
   ferefere d'armes cel 
   conmencement, ne volt sousfrir que nuls eust la premiere bataille se il non ; Mirrus, la seconde ; 
   Karus, la tierce ; Josyas, la quarte ; 
   Japhus, la quinte ; et Helcanus fu li estandars. \pend
\pstart Pas de nouveau § BMardocheus 
   fist l’arriere garde, qui ot maint vaillant chevalier 
   en sa compaingnie. De l’autre part ot li marchis de 
      Fabonne la premiere ; 
   li vassaus de Terre de Labour, 
   la seconde ; le duc de Puille, la tierce ; 
   li roys de Sezille, la quarte; 
   li vassausvisquens de 
      Naples, la quinte ; 
   li marchis de Montyr, la sixte ; 
   li quens de Pyse, la septiesme ; 
   li princes d’Aquilee, la huitiesme ; Pelyarmenus 
   fu li estandarten l'estandart, 
   et li roys d’Arragon 
   l’arrierefu en l'arrierefu l'arriere garde, 
   qui moult s’aati de mettre au desous ceus qui jadis l’avoient mis au bas. \pend
\pstart Lettrine manquante BLi marchis de Fabonne, 
   qui joennes et plains de chevalerie estoit, vit venir Dorus, o lui grant esfort de 
   chevaleriebaronnie, et non pas a loy de gent esperdue, et il meismes 
   d’autre part ne se contint pas a loy d’onme marri, ainz envoia au vassal 
   de Terre de Labour que il ne lessast pas, conment que la chose preist, 
   que il n’assemblast assez tost aprésaprés ce que il verroit qu’il 
   ce seroit mis en la bataille, car 
   sese il iceulz pooient outrer aus autres 
   n’avroient il que fere. Lors ne demoura pas que il s’entraprochierent d’ambeii pars. Qui donques oïst cors, arainnes sonner, 
   tabours et tymbres tentir et freteler, ces armes resplendir, enseingnes venteler,
   le soleil ou fin or reluire et alumer, ces destriers en argueilpar orgueil 
   brandirbraidir et traverser, ces chevaliers en armes, ces escus acoler, ces fors espiez estraindre, brandir et aviser, 
   souvent d’eures en autres sousfauchier et combrer, ne fust coart el monde qui ne deust recouvrer cuer 
   et force et vigor 
   et soi asseureren soi. \pend
\pstart En ceste maniere assemblerent Dorus et 
   le marchisMarchis de Fabonne 
   si aigrement que le plus tost qu’il porent venir ensemble s’entreferirent 
   des espiez en telle maniere qu’il n’i ot si fort qu’il ne volast em pieces, dont du plus foible deust estre mis i au macour
   
   a terre. Et pour ce ne demoura qu’en celui poindre ne meissent touz mains aus espees, et s’entrevindrent de si merveilleus aïr 
   qu’il n’i ot celui escu qu’il ne 
   fendisistfendissent jusqu’en la boucle, 
   et descendirent les cox sus les hyaumes, qui furent aduré. Et nonpourquant, cil qui le plus fort 
   fu ne pot le cop soustenir qu’il ne 
   soit entrez en la coiffe et le bacin faussé. 
   Aprés ces coups ont les autres referus, si que en la fin Dorus avoit si mis 
   le marchis au sousfrir. Mes ce ne li valut riens, car 
   le vassal de Terre de Labour 
   revint de l’autre part, qui touz iriez feri sus. Et cil iert troptrop grant et 
   fort et le cuida prendre aus mains, etmais cil, qui en proesce florissoit, 
   li vint en telle maniere que la ou le vassal sele cuida ferir se hasta 
   Dorus et li lança l’espee si a point que il li mist par dessous l’esselle ou corps. En ceste maniere 
   recouvra Dorus au 
   marchis, qui l’eust mort se Dieux le vousist avoir souffert. 
   Mais quant li vassaus 
   fu cheus, de si tres grant force vint sus 
   lui que Dorus n’ot pooir de plus faire et dit a soi meismes que 
   il se trairoit ensus en desfendant soi de l’espee en si ruiste maniere qu’il n’i avoit celui quil 
      ne feust touz liez quant il fuporent eschapez. \pend
\pstart Li marchis, qui ert venus de mort a vie, 
   si vit que le vassal fu occis, si dist :
   Biaus seingneurs, moult a si grant meschief. Alez, et si emportez 
      le vassal, et si ne cuidiez pas que devant que 
      Dorus iert lassezlassez si-iert lassez 
   ne pourroit avoir a lui nulz duree. Si le me couvient tant gaitier que je le prendrai a point, se je onques puis.
   Ainsi comque Dorus aloit les grans presses
   cerchant, si ne cuidiez pas que ses chevaliers feussent oisseus :oisseus, car 
   sachiez que non, car li contes dist que li 
   uns en avoit a ffaire aussi conme a ii. Mes Dorus, qui par la bataille 
   regardoitgardoit les siens 
   et reconfortoit en decoupant et en abatant ses anemis, leur donnoit cuer et hardement de fere ce que il 
   devoientfaisoient. La gent au 
   vassal 
   emporterent leur seingneur devant Pelyarmenus, 
   qui conmanda au duc de Puille que il s’abandonnast, et il si fist, entalentez de chevalerie faire, 
   si que li Grieu en orent le pyeur en pou d’eure. 
   Quant Mirrus vint, il et li sien, qui ne furent pas a aprendre a qui il devoient avoir le tournoy, 
   car nous trouvons ou conte, qui assez briefment em parolle pource que il ne puet pas raconter le fet de chascun, 
   que Mirrus et li sien mistrent ci au desous en leur 
   premiere venuepremieres venues 
   le duc de Puille 
   et le marchis qu’en pou d’eure resortirent de ci a leurs gens et perdirent terre. \pend
\pstart Quant se vit Pelyarmenus, si ot paour de son sort 
   et conmanda que le roy de Sezille et 
   le marchis de Montyr se meissent en 
   la bataille. Cil, qui grant force avoient, le firent si aigrement que bien le 
   pot on veoir, car li marchis et li sien estoient une gent de 
   tres grant vertu. Et avec ce, il avoit eu 
   proescepromesse de bien fere la besongne, 
   pourpar coi il se mistrent tant avant conme 
   ilil y parut. Car li contes dist que 
   li marchismarchis dessus dit 
   avoit choisi Dorus ou il avoit occis i sien ami. Si tourna le destrier sus quoi il sist 
   et vint a lui de si tres grant vertu que, vousist ou non, le preus Dorus 
   ferile feri 
   le marchisle marchis
      avant que il s'en fust donne garde i si grant cop que il 
   li fistguerpi ambedeus 
   les estriers vuidier, 
   et s’aclina sus le col de son destrier, si que avant que il feust redreciez li revint 
   li marchis de si pres 
   queque il, 
   vousist ou non, le mist a terre du cheval. Iluec fu mis a terre li bons chevaliers, qui onques mes 
   jour de sa vie n’i avoit esté mismis a terre pour coup 
   qu’il eust receu. Lors fu il si courrouciez que la ou il vit 
   le marchis li courut sus conme lyon et li pourfendi son destrier 
   dud'un grant coup dont il cuidoit ataindre le marchis 
      que il li mist l’espee jusques en la coraille. \pend
\pstart Li marchis, quant il se vit a pié d’encoste 
   le preus Dorus, ne s’esmaia pas, 
   ainz fist semblant quesi que se 
   ilil ne venist en sa presence. 
   Et la s’entracointierent en tele maniere que ce sembleroit fable du dire et longue chose du recorder, 
   mes en cele escremie l’eust maté Dorus. Quant li marchis de 
      Fabonne, qui estoit en aguait de lui sorprendre, li vint tout a demis, et Nostre Sires, 
   qui pitié ot du bonloial chevalier, 
   lili fist tant de remede que il li 
   envoiaenvoia Mirrus en tele maniere que, quant 
   le marchis cuida ferir sus Dorus, 
   Mirrus li donna i cop si grant qu’i l’abati du cheval 
   a terrejus, entre Dorus et 
   le marchis. Et quant ce virent li uns 
   et li autres, si furent aussi conme tout esbahi. Lors descendi Mirrus a pié et vint au 
   marchismarchis de Montir 
   et li dist en reprouche :
   Mal vous est avenu quantquant vous 
      onquesonques vous fustes si hardis que vous 
      au meilleuraus meilleurs des bons vous osastes emprendre !
   A ce mot sesi Dorus 
   leun Fabonnois 
   et li dux Mirrus, 
   marchisle marchis de 
      Montir, si firent 
   d’entr’eus ii la dessevree. 
   Lors monta Dorus ou cheval au Fabonnois et 
   Mirrus ou sien. Meismes ainsint n’eussent eu garde nos 
   fransfrançois chevaliers tant 
      conmeque li autre se fussent ainsi prouvé. \pend
\pstart Quant Fabonnois et cil de Montyr 
   sorent que leurs seingneurs furent ainsi mis a mortoccis, 
   si furent moult esbahy et n’orent cuer de puis faire bien. Adonc ont tourné les dos, 
   et ceus les ont menez jusques a leur gent, qui a ce ont recouvré, carque 
   li quens de Pise et 
   li quens de Naples et meismes l’estandart 
   Pelyarmenus se sont ferus entreentre les 
   Grejois, 
   si a une envaïe queenot
   moult en y ot 
   d’occis et d’asfolez. Que vous diroie je ? Je 
      nene vous puis pas raconter touz les fais de chascun, 
      mes tant que des miex faisanz me couvient fere mencion, car en ce vint Karus 
   en la bataille ausi fier conme i lyon, conme cil qui mil chevalier n’encontroit qu’il ne meist jus du cheval ou mort ou afollé.
   Josyas d’Espaingne revint de l’autre part, 
   qui moult gentement s’ise prouva. \pend
\pstart Pas de nouveau § BHelcanus, quant il sot que 
   Pelyarmenus estoit en la bataille, ne se volt targier que il ne venist tout aussi 
   conme honme sanz mesure et a qui il ne chaille que il deviengneCette précision se veut presque 
   prophétique, le manque de mesure manifesté par Helcanus à cet instant annonçant l'aveuglement de son opposition avec Pelyarmenus
   et sa perte qu'elle vient signer, dans une nouvelle portée réflexion quant au bon comportement du souverain., 
   si feri le destrier par ambedeus les costez si aigrement 
   que touz ceus qui entour lui furent en orent grant merveille. Et le cheval, qui grant et 
   fortfort et roide estoit et avec tout ce li 
   mieudres du monde, si se mist a la voie en tele maniere que tuit cil qui voie ne li firent porent bien mal encontre faire. 
   Et quant cil en qui il se fioit ont ce veu, si ne vodrent demourer que il ne le 
   suivissentsuissent. Et cil, qui avoit ou corps l’ardure 
   de la honte et du lait que on li fesoit, s’abandonna trop a ce qu’il estoit guaitiez de traistres qui le vouloient sorprendre, 
   dont il avint que, quant il vint entre ses anemis, qu’il se feri entr’eus tout ausi conme 
   uns espevriersl'espevriers entre les menus oissillons, 
   et il le vont fuiant ; autresi li fesoient voie li pluseur. Et cil qui contre lui se metoient n’i avoient duree. 
   Si avint que tuit li firent 
   voiejoie
   parpartout la ou il vouloit aler. 
   La fesoit li gentilz empereres 
   merveilles si tres grans que tuit cil qui le veoient, fust des siens ou des autres, en trembloient de paour, 
   et il le fesoit pource que il li estoit bien avis qu’il n’eschaperoit ja de la journee sanz mort, et pour ce 
      se vengoit il tant com 
   il avoit d’espace sa honte afin que l’en n’en peust recorder males 
      nouvelles, car il amoit miex mourir a honneur que vivre a honte.La leçon de B
      semble presque induire un refus d'annoncer ainsi la fin d'Helcanus avant qu'elle ne survienne, à préserver le suspense de cette bataille,
      dont l'issue est ainsi directement indiquée, après les précisions déjà données quant à l'entrée en bataille d'Helcanus ci-dessus. \pend
\pstart Pas de nouveau § BEn ce que il se contenoit en tele maniere, 
   si esgarda i chevalier cest afaire. Si en vint au plus tost que il 
   pot a Mirrus, le fier, et li dist :
   Ha ! sire, tout avons perdu se vous ne secourez mon seingneur !
   Adonc li conta conment li emperieres se contenoit folement.
   Après la mention du manque de mesure d'Helcanus, la qualification de son comportement comme "fol" ne laisse
   aucun doute sur sa destinée dans cette bataille, ni sur la cause exacte de sa disparition surtout.
   AdoncAtant guenchi Mirrus cele part 
   et trouva que li preus Helcanus estoit ja entrepris en tele maniere que 
   auquesil s’estoit 
   mis au souffrir. Quant 
   Myrrus et plusseurs des siens vindrent, qui derompirent la presse, la endroit fu bien esprouvee 
   la force et la bontéla proesce et l'amor que 
   Myrrus ot o soi, car li contes dist que, avant que 
   Helcanus se peust percevoir de secours, en mist il 
   xxiix a mort. Et quant il se senti lasche et il 
   otot aussi come un pou repris s’alainne, si s’aficha ou destrier de si 
   grant force qu’il le fist archoier desous lui et ala fichier que il fist estraint 
   ls’espee et feri a destre et a senestre de si grant force que nulz ne pooit 
   sesses pesans cox 
   endurerendurer ne souffrir. \pend
\pstart Myrrus, qui vit ce, ot grant merveille de ce que il fesoit, 
   si li fu avis a sa contenance que il n’estoit pas bien en son sens. Lors li vint au devant et le cuida 
   prendre par le frain, 
   mes Helcanus li donna un cop si grant et si pesant que il l’abati sus le col de son destrier, 
   vousist ou non, et l’eust mis a terre se il ne se feust pris aus bras au col de son cheval. 
   Lors l’eust referu quant i chevalier se mist au devant, qui dist :
   Ha ! sire, ne veez vous goute, qui Myrrus, vostre bon ami, 
      voulez mettre aa la mort ! \pend
\pstart Pas de nouveau § BQuant Helcanus 
   entendi celi, si se retrait et ne pot mot dire de courrous. Lors vint Myrrus a lui tout estourdi 
   du cop qu’il ot eu, si li dit :
   Conment, sire, vous habandonnez vous 
      ne en quelle maniere, meismement honme qui a ce a fere que vous avez ?Ha ! Myrrus, dous amis et loiaus compains, je vous ai blecié et non 
      pas a escient. Si vous pri que vous le me pardongniezpardongnez avant 
      que je muire, car je voi bien que ma mort est 
   venue et que il plaist au RoyRoy de paradis que je muire a la journee 
      d’ui par les mains de ses mauvés traistres.Ha ! sire, distce dist 
      Myrrus, que est ce que vous dites ? Voirement pert il bien que vous n’estes pas en vostre sens !Non sui je, dist il.L'observation de Mirrus, et la confirmation 
   par Helcanus, vient une fois encore insister sur le manque de mesure et de raison, de sens, dont il fait preuve dans cet affrontement 
   qu'il pressent final, jusqu'à manquer de tuer l'un de ses plus proches alliés, dans une dramatisation importante de cette scène bien sûr. 
   Lors n’orent pas bon lesir de dire ce que il vodrent, 
   car la bataille assembla de toutes pars. Si ne vit onques nulz tant de gent mourir d’une part et d’autre conme l’en peust la faire. 
   Mes pource que je ne puis du tout recorder conment li affaires ala de l’autre part, tout le seusse je 
   – si sembleroit ce oiseuse aus miex entendans –, si m’en passeré outre au miex que je pourrai et vueil venir a ce que,
   quant il furent pelle et melleperle et merle, si 
   y ot moult plus des uns que des autres 
   et ne fu ore pas li giex a droit partis, mes ce n’eust esté le droit aus Grejois qu’il avoient, 
   et d’autre part il estoient trop 
   bonnemieudre chevalierie par quoi, se Dieu leur eust destiné, 
   Ronmain n’i eussent eu duree. Et nonpourquant 
   n’en orent il mie en la fin le meilleur, car li contes dist que, quant il furent tuit assemblé, la partie aus 
   Griex si avoit cuer recouvré de ce que li frans Mardocheus, 
   qui avoit fet l’arriere garde, tresperça les rens et vint a toute sa gent en la bataille 
   Pelyarmenus, qui tant se fesoit garder curieusement que nul ne le peust aprochier 
   sese ce ne fust de trop loing. Et nonpourquant, a cele envaïe 
   Mardocheus si esforça si merveilleusement que par force de chevalerie se prist a lui, 
   vousist ou non, et fu mis jus du cheval a terre en tele maniere 
   que ce fu merveilles qu’il pot estre resqueus. Mes il fu 
   remisresqueus et mis arriere ou cheval, vousissent 
   cil ou non qui entre piez l’avoient mis. Et quant il 
   fufurentfurent 
   arriere ou cheval, un cuer et un hardement et une force vint en lui si merveilleuse que nul ne pooit 
   avoirdurer a lui duree. \pend
\pstart Sodores, 
   deun chevalier de la meismes marche et de la mesniee 
   Mardocheus, vit que cil pour qui il avoient les rens 
   vuidiezcerchiez 
   fulor estoit resqueus. Si n’en fu pas bien paiez. Et que fist il ? 
   Onques nuls homs ne se mist en plus grant painneperil ne ne fit plus 
   degrant proesce fors que i chevalier qui ot a non 
   Eleazar, qui fu des Macabiex, qui ocist l’olifant entre ceus qui le 
   gardoient.###L'intégration d'un Macchabée parmi les troupes de Mardocheus nous semble relever 
   de la dramatisation déjà notée de cette scène de combat et de la dimension sacrificielle des affrontements pour le camp 
   d'Helcanus en résistance aux viles prétentions au trône de Pelyarmenus. Tout 
   aussiaussi le fist Sodores
   parmi touz ceus quique Pelyarmenus 
   avoient a gardergardoient, vint a lui et li donna i cop en tel 
   manere que i chevalier gita son bras au devant. 
   Mes onques ne li valutle secorut que il ne le pourfendist jusques en la 
   teste, et le hyaume et la 
   coiffecoiffe et le bacinet, 
   si qu’i l’eust du chevalcheval jus 
   abatumis se 
   ce n’eusteussent 
   esté ceus qui entour lui estoient. Et quant le chevalier ot feru si a droit, si cuida recouvrer, 
   mes il ne pot, conme cil qui n’ot point de suite. Lors fu de toutes pars entrepris, si 
   qu’il ne pot en la fin repairier aus siens, ainz fu iluec detrenchiez et occis. \pend
\pstart Pelyarmenus, qui bien cuida estre mors, si se fist mettre hors 
   de la presse et conmanda que il feissent bellement par bataille leur gent retraire, conme cil qui bien cuidoit avoir failli a son sort, 
   qu’il avoit sorti que a cele journee cuidadevoit il avoir victoire. 
   Si se conmença a desesperer. Mes aussi conme le proverbe dit : 
   Aler ne puet qui les piez a cuisCette référence sentencieuse 
   n'est pas répertoriée chez Morawski ni chez Le Roux de Lincy. PD vérifie stp, 
   tout en tele maniere avint il de soncestui sort. Car quant 
   Grieu virent que ilRommain resortissoient petit a 
   petit, si cueillirent si grant orgueil em pou de temps qu’il cuidierent avoir du tout seurmonté 
   lalor partie adverse. Adonc se sont esforcié d’eulz mettre a destruiction, 
   si le firent si aigrement que en ce se conmencierent si fort a haster qu’il se departirent de leur conrois et 
   firentfist aussi conme dis me tu :
   Je enporterai le pris et serai le mieux faisant. \pend
\pstart Pas de nouveau § BDorus, 
   qui moult avoit enduré le jour en tres parfaite chevalerie, s’abandonna a ce que il peust prendre 
   Pelyarmenus ou lui mettre au desous. Si le suivoit a coite d’esperon en tele maniere 
   qu’il avoit moult pou de suite, et il s’en aperçut moult bien, si conmanda a ceus qui le suivoient qu’i ne retournassent pas, 
   ainz le suivissentsuissent touzjours. Lors avint que 
   Pelyarmenus se mist a la voie a moult grant esfort, o lui cil qui miex l’amoient. 
   Si furent bien pourveu de fere ce que il vouloient, mes il n’orent pas voie a leur volenté, car il les 
   en couvint aler le costé d’une montaingne a moult grant meschief. 
   Il entrerents'embatirent en une forest grant et large, dont les arbres 
   furent merveilleusement largeshaus et anciens. Mes elle n’estoit pas espesse 
   d’arbres par quoi l’en i pooit chevauchier em plusseurs lieus aisieement. 
   Dorus, qui aprés eulz se metoit les esclos, si ne cuidoit pas venir a ce qu’i les peust metre 
   au desous ne vengier son grant courrous. Lors mist moult grande paine de soi esforcier, mes trop avoit le cheval sus coi il 
   sist traveillié et ses compaingnons aussi les leur, si que li plusseurs de sa compaingnie demourerent iluec estraier en cele chace, 
   dont il furent si dolent que bien i parut, car li auquant se mistrent a pié aprés leur seingnor, 
   qui en nulle maniere ne voudrentvolt la chace lessier. Et qu’en avint il ? 
   Il chevauchachaca tant que il ne fu que li xxviie 
   de xxx chevaliers qu’il cuidoit bien avoir a son besoing. \pend
\pstart Pas de nouveau § BEn ces 
   vint.VII. 
   chevaliers qui estoient avec Dorus en i 
   avoit i traitre, qui moult metoit grant painne a ce que 
   Dorus fu trahis etmors ou pris. 
   Cilz se hastoithasta plus que nuls des autres, car il avoit cheval a sa volenté. Si fist tant que il se mist au devant de sa compaingnie, 
   aussi com s’il vosist premierement ferir sus eus. Et quant ce vit Dorus, si s’aperçut et 
   li dist le cuer que il enchauçoit folement. Lors sacha ensus son cheval et dist :
   Ha ! biaux seingneurs, nous sonmes traÿs !Conment, sire ! dist un chevalier de grant cuer. De qui vous alez vous doutant ?De celui que vousvous en veez la aler, 
      dist il.Par mon chief, dist uns autres, je ne m’en douterai jusqu’a tant 
      queque je autre chose en verrai.La 
      réflexion de Dorus dénote de sa lucidité vis-à-vis du traître, tout comme vis-à-vis de Pelyarmenus, de manière plus générale.
      L'introduction de ce nouveau traître procède de l'omniprésence de la félonie qu'a instillée Pelyarmenus et exacerbe encore
      son propre vice dans ces derniers épisodes. \pend
\pstart Pas de nouveau § BAtant se mistrent en i 
   assensassens si comme pour savoir se nulz d’eulz les suivoit. 
   Celui dont j’aiavoie 
      devant parlé avoit aconsuivi Pelyarmenus et fet 
   signe aussi conme se il eust a eulz feru et qu’il l’eussent pris et mis a merci, 
   mes il avoit d’autre chose parlé, et dist a Pelyarmenus 
   conment il pooit avoir tant atendu qu’il n’aloit encontre celui qui eschaper ne li pooit.
   Certes, dist Pelyarmenus, 
      j’ai tel paour qu’il ne m’encontre que je suis tout asseur que, se il metoit main a 
      moi,moi que 
      je seroie tost vaincus.
   Lors respondi i chevaliers et dist :
   Qu’est ce, sire ? Que
      vous avez ore dit ? 
      Voirement sont il aucuns qui sont moult preus a une grant besongne 
      faire, c’est a conmencier 
      la. 
      Et quant ce vient au besoing, si s’en puet on 
      pou aidier.
   Dont fu Pelyarmenus moult courrociez et dist :
   Voirement dites vous verité.
   Et puis aprés si dist :
   Que sont ore devenu cil chaceeur, que il ne sont pieça avant venu ?
   Lors orent moult grant merveille que Dorus pot estre devenus, si retournerent aucuns 
   qui avoient moult grant desir de parassouvir leur volenté. Dont il avint une merveilleuse aventure, 
   ainsi conme li contes le devise. \pend
\pstart Ainsint conme j’je vousavoie 
   dit de desus conment Dorus et les siens avoient sus sachié 
   lorles frains de la chace, car il se mistrent aussi conme 
   au retour et s’estoient torné moult soutivement au travers d’une roche, et cil qui se furent partis de 
   Pelyarmenus, qui s’estoient mis au retour aprés Dorus, 
   cuidierent vraiement qu’il retournast pour paour d’eulz, si se tindrent moult a deceu quant il leur fu ainsi eschapé, 
   si ne sorent que faire de lui suivre.
   Par mon chief, dist li uns d’eulz, 
      ja ainsi ne m’eschaperademorra, car il a le cuer perdu or aprés !
   Lors se mistrent es esclos aussi conme il porent miex savoir qu’il avoient esté chacié, si n’orent pas alé moult longuement 
   qu’il encontrerentorent encontré x des chevaliers 
   Dorus qui venoient tout a pié en haste aprés 
   leur seingneur droiturier, conme a cil a qui il ne 
   vouloientvouloient faillir ne a mort ne a vie,
   ainssi conme vous poez oïr. \pend
\pstart Pas de nouveau § BCil dont j’avoie dit desus, 
   qui parti s’estoient de 
   Pelyarmenus, virent ceus venir. Si leur corurent sus, mes il n’estoient pas garçons, 
   ainz leur furent moult contraire, car, ainsi conme je vous 
      avoieai devant 
      fet menciondit du chevalier qui povres 
   estoit d’avoir et riche de chevalerie qui avoit non 
   Hainaut, cil amoit Dorus de grant amour pour la grant 
   chevalierie qui en lui estoit, et d’autre part ilil en atendoit grant bien. 
   Il vit ceus venir qui venoientvenoient sor eulz moult tost, si cuida 
   vraiement que Dorus feust mors ou pris. Si dist a sa compaingnie :
   Biaus seingneurs, or n’i a fors que du bien 
      faire.faire, car Vez 
      cici venir ceus qui ont 
      Dorus, nostre bon seigneur, 
      occisoutré ! \pend
\pstart Lors sont cil a eulz assemblez, si distrent :
   Trop estesa avant alé ! Mors estes avec 
      vostre seingneur se vous ne vous metez a merci !
   Cil n’orent cuer de fere leur requeste, ainz se trestrent encontre une roche, 
   ainsi conme Deux leur avoit pourveu. 
   Il leur ont livré estal tout aussi com li sengliers fet aus menus chiens quant il ne veult fouir devant eulz. 
   Que vous feroiediroie je 
      i grant conte et une riote de 
   cel povre chevalierce ? 
   IlPour ce que il sembleroit a aucun que ce feust chose impossible. 
   Cil qui preudonme furent et avoient a leur avis bon droit 
   – et si avoit encore cuer de lyon chascun en soi, pourpar quoi, 
   a l’aide du vaillant chevalier, 
   il outrerent les x qui se furent partis de Pelyarmenus, 
   que onques i tout seul n’en eschappa, et n’i ot celui qui eust pooir de respondre chose 
   qu’ilil leur vousissent enquerre ne demander, pour quoi il leur avint si 
   glorieusement que les chevaus a ceus qu’il avoient occisoutré leur furent 
   touz vis demourez –, dontdont leur firent une chose qui 
   bien leur doit 
      estre mise en conte. Car par le conseil Hainau osterent leurs connoissances et pristrent 
   celles a ceus qui iluec gisoient mort, si ses'en sont isnellement garni, 
   et puis si se mistrent es chevaus que il avoient conquis. Mes il furent moult durement navré et traveillié, mes pour ce ne se vodrent 
   il pas mettre a repos, ainz se mistrent a la voie pour savoir et 
   entendreenquerre la 
   veritévenue de 
   leur seingneur. Mes le conte se taist
      a parler de 
      eulzeulz ci endroit et 
      retournerepaire a Helcanus, 
   qui d’autre part fesoit sa chace, ainsi conme j’aiavoie 
      fet mencion devant ou conte. \pend
 
         
            Ci revient li contes a Helcanus,
               qui s’est remis en sa queste
                  queste, ainsi comme s'ensuit
                  comment il morurent entre lui et Mirrus.La leçon de 
                  B est plus cohérente sur le plan narratif : Helcanus n'entreprend plus aucune quête, puisqu'il meurt.
            
               Enluminure sur 1 colonne (c) et 12 UR. 
                  Helcanus à cheval en armes dans la forêt, 
                  suivi d’un homme.
               
            
\pstart Pource que je ne puis pas mener mon conte a ce qu’il ne couviengne savoir l’une chace 
   aprés l’autre, ainsi conme il m’est avis, Myrrus, qui toute autre chose avoit mise arriere 
   pour garder de pres Helcanus, si conme je ai desus dit qu’il s’estoit 
   habandonnez a fere ce qu’il couvenist a lessier, 
   car il aloient Ronmains chastoiant en tele maniere que mal de celui qu’il ne fouist devant eulz, 
   dont il avint que tant y ot des occis que nuls n’en cuidoit vif eschaper. Li traitres, de qui li contes ne fait ore 
      pas cici endroit mencion 
   qu’il firentqui il furent pource que trop vilz chose 
   sembleroit a aucuns – qui dolens sont de ceus qui par lignage et parce que bons arbres si doit par reson porter bon fruit, si avient 
   aucunes fois qu’il couvient qu’en cest arbre faillefaille par aucunes 
   resons aussi conme plusseurs sevent, cilz maleoit fruit fu en agait tres bien pourveus de force –, si vindrent de si 
   tres parfaite vertupres que Envie, qui nul donmage qu’il puist avenir ne redoute a nule povre gent, donna cuer et hardement a ceus qui 
   reconurent que il vindrent tout a i fes sus les ii preudonmes qui n’avoient pas entour eulz leurs amis et 
   furent aussi tout entrepris conme l’en conte du bon Judas MacabeusLa
   référence à Judas Macchabée exacerbe la portée presque sacrificielle de la mort d'Helcanus et de Mirrus entourés de leurs ennemis, 
   qualifiés de manière insistante de traîtres d'ailleurs, qu'ils ont pourtant si bien affrontés jusqu'à se voir surmenés., mes en la fin 
   vindrentmirent leurs anemis arrieres dos et ne leur en sot tant venir de 
   toutes pars que touz ne les meissent a mort l’un sus l’autre, si que parmi les mors leur revenoient les autres touz erragiez, 
   si que leurs chevaus leur furent occis, et furent mis a terre, vosissent ou non. Qui dont veist 
   les uns les autresl'un et l'autre leurs vies chalengier et a la fois 
   les durs coxcuer 
   recevoir,recevoir, a la fois 
   l’un trebuchier adens et l’autre resaillir pour lui aidier et recouvrer a celui qui plus s’abandonnoit d’eulz envaïr, 
   grant pitie peust avoir d’eulz qui en lui eust raison. \pend
\pstart En cestui enchaus ne demoura pas que la nouvelle n’en 
   alastvint a Mardocheus qu’il vint 
   cele part au plus tost qu’il pot, o lui grant plenté de sa baronnie. Mes Deux, qui a ce les avoit pourveus, ne volt souffrir que il 
   feussent resqueus, car il furent iluec conquis et mort conme ceus qui ne 
   pooientporent durer. Mardocheus, 
   qui la force avoit amenee, mes ce fu a tart, assailli ceus qui demorez estoient de 
   l’celuienchaus en tel maniere que mal fust de celui qui i demourast que 
   touz ne feussent decoupez et occis. Mais atant demoura lors 
   lalor chace, conme cilz qui bien sorent que 
   Mirrus et Helcanus furent mis a mort. \pend
            \pstart Pas de nouveau § BQui donc veist 
   Mardocheus saillir entre les occis 
   et tourner l’un desus l’autre et conment il huchoit a haute vois em pleurs et en lermes :
   Ha ! biau sire Helcanus, quel part vous pourrai je mes trouver ? 
      Sire, quel avis avez vous eu, qui sanz moi gisiez entre ses autrestraytres 
      que je ne vous puis trouvertrouver ne encercher ? Sire, responnez moi ! 
      Se je ne vous puis baillier vif, je ne cuitquier jamés nul jour avoir 
      joie ! 
   Moult peustpeust on avoir grant pitié de Mardocheus conment il se dementoit 
   de la mort Helcanus et si ne le pooit trouver, tant en y avoit d’uns et d’autres. 
   Et sanz doute la bataille avoit moult duré, etquar il estoit si tart que 
   l’en ne pooit savoir quilequel 
   estoientestoit li i 
   ne quiliquel estoient li autre
      et ly autres, par quoi il couvint que Mardocheus conmanda que on 
   alumast torches, et l’en si fist en pou d’eure, par quoi les 
   ii chevaliers furent trouvez a trop grant painne et trais hors des autres. Lors ne fu onc si grant duel mené conme l’en 
   peust iluec veoir de plusseurs chevaliers. 
   Mes je ne veil pas ci du tout demourer a deviser conment li afaires ala, mes atant vous puis je dire qu’en 
   pou d’eure fu partout seue la mortl'aventure du 
   vaillantvaillant chevalier 
   Helcanus. Si se trait chascune partie au miex que il porent les uns avec les autres. Ha ! Deux, 
   si tres grant duel demenerent li baron de leur seingneur qu’il avoient si mauvessement gardé 
               et secouru que il n’estne fu nul
   que, s’ilqui les veist, 
   quequi grant pitié n’en deust avoir, 
   meismes Josyas d’Espaingnes 
   et Japhus le Frizon : cil dui s’en desesperent. Tous li autre baron enqueroient de 
   Dorus, qu’il nen'en savoient nulles 
   nouvelles, si cuidierentcuidierent veraiement qu’il jeust mort par la 
   champaingne ou avoit tant d’uns et d’autres que on ne savoit qui estoie i ne qui 
   autreautre estoit. \pend
\pstart Cele nuit couvint la baronniebaronnie venir 
   jesir sanz loges et sanz trez tendre toute armee. Mes du boivre ne du mengier ne fet pas li contes mencion moult 
      grant, car tel anui orent li auquant que moult furent d’autre chose esmeu. A lendemain oïrent nouvelles que 
   Dorus s’en aloit aprés Pelyarmenus, qui s’en aloit sanz lui 
   asseurer. Lors li dist chascun que c’estoit grant merveille conment il pooit tant durer de fere ce 
      que il fesoit et qu’il avoit fet a la journee d’ier.
   Par foy, disoient les autres, je ne sé, car nous ne cuidons pas que iii des miex faisanz 
   peussent tant fere conme il fist ier tout seul.
   Moult alerent loant la chevalerie de 
      Dorusle chevalier. Mes de ce qu’il ne sorent qu’il furent devenus furent merveilleusement esbahy. 
   Lors se mistrent ensemble li grant seingneur pour avoir conseil qu’il avoient trouvé d’aler avant pour oïr nouvelles de lui. 
   Cil qui miex l’amoient erent engrant de lui suivre, et li autre disoient que ce n’estoit nul esploit 
      de folement enchaucier. Et d’autre part, il nel savoient
   en quel lieu querre.
   Avoi ! dist Mardocheus. De quoi sonmes nous en debat ? 
      QueCar chascun voit bien que nous n’avons aujourd’ui plus de chief que 
      lui. Et quant li membre faillent au chief, ce n’est mie merveille ce tout va male part, pour quoi je vous pri a touz ensemble 
      que vous me suivez au plus tost et au plus ordeneement que vous pourrez, 
      car je ne fineré jamés devant que je savré nouvelles de li.
   Lors a conmandé lili preus et le vaillans 
   Mardocheus que sa gent se meissent d’une part et feussent 
      apresté, et d’autre part il conmanda que Helcanus et 
         Mirrus fussent aportez iluec pres en une abaÿe et fussent ouvert et embasmé, car, 
         se il reperoit jamés, ilil s'en iroit et les en feroit 
         porter en Coustentinoble et 
         mettre a grant honneur en terre. 
   Lors se mist Mardocheus a la voie aprés Dorus, 
   et les autres qui miex miex aprés lui. Si me vueil ore taire i pou d’eulz et venir a l’averse partie 
   pour miex metre mala matire a point et 
      a droit. \pend
\pstart Aussi conme vous avez 
   oïoï dessus ou conte conment 
   Helcanus fu outrez de ceus meismes qui fere ne le deussent selonc droiture et par lignage, 
   li rois d’Arragon, qui toute s’entente avoit 
   mise a ce que Envie n’est liee de nul avancement de nul preudonme, mes qu’ele 
   puist venir a chief de ce qu’ele couvoite, car, aussi conme dist li philosophes, Envie est touzjours liee 
      quant ele voit preudonme mourirCette référence sentencieuse 
         n'est pas répertoriée chez Morawski ni chez Le Roux de Lincy. PD vérifie stp, mais cela me semble plutôt être une référence
      à Aristote, non ?, et en dist i moult haut mot : 
   Voirs est, distfait elle, que si est noble 
      chose de preudonme que parfaiteparfaire honneur, 
   si dit qu’ele le fet vivre d’une seconde glorieuse vie, car il est escript que, quant li homs muert, 
      son non ne puet mourir, ainz couvient que son non demeure quel que il soit, 
   si que Envie est si joians de ce que elle voit preudonme partir de cest siecle que ele a son non li couvient touz mautalens pardonner. \pend
\pstart Pas de nouveau § BTout aussi fist 
   li rois 
   d’Arragon, car onques, puis que li preus Helcanus ot aquis 
   l’onneur au roy d’Espaingne, en qui servise il avoit 
   jadis esté encontre lui, ne le pot amerIl s'agit là d'une référence à la guerre qui oppose 
   Theodore, le roi d'Espagne, et Dyomarques, le roi d'Aragon, que viennent soutenir d'une part Helcanus et de l'autre Pelyarmenus 
   et Fastidorus. Voir le Roman d'Helcanus, en part. §192-218.. Et ici revient a moralité li proverbes 
   de quique je avoie devant traitié, 
   que moult vilainne chose est de ferefere de 
      bonne semence mauves fruitVariante du proverbe "Bon fruit vient de bonne semence" (Morawski 289), 
      avec une insistance notable sur le vice dans l'inversion qui est ainsi proposée dans la présentation de la sentence.. 
   Ha ! seneschal de Rome, qui si vilainne semence 
   engendrasengendrastes en 
   si vaillant dame, 
   conme vous a envis eussiez daingnié souffrir faire de vostre char et de vostre sanc si vilain meffet 
   conme fist vostre filz de ce qu’il i deust tourner a honneur et a loenge ! 
   MaisMais il me couvient ore de ceste chose 
      taire et venir a ce que tuit qui demourez estoient se ralierent ensemble et se mistrent a lui et puis orent leur espie de 
   touz sens, qui leur ont dit conment leur 
      averse partie se maintint.
   Or va bien, dist li rois. Or vous metez aprés, 
      quecar je vous asseur que ja pié n’en eschapera.
   Lors se sont mis aprés eulz, mes ne demoura gueres que il les trouverent. 
   Si couvient aci tout lessier et venir a Pelyarmenus, 
   qui ot grans merveilles puisque ses chevaliers se furent partis de lui, ainsi conme il est contenu ou conte. \pend
\pstart Voirs fu que Pelyarmenus se douta moult de ce 
   que il avoit esté bleciez, si conme il a esté dit, dont il avint que, quant il vit que il n’orroit 
   autres noveles de sa gent, si se mistrent x a la voie pour aler aprés ceus 
   quequi Haynau et ses compaingnons avoient 
   detrenchiez et occis. Mais il n’orent pas moult alé 
   quantqueLa leçon de B est révélatrice du flou
   qui imprègne tout cet épisode, fait d'imbroglios identitaires en tous genres. Mais c'est bien Hayno et ses hommes qui ont décimé 
   les troupes de Pelyarmenus et non l'inverse, voir §776. il virent celui 
   Haynau et ses compaingnons qui venoient contre eulz, si conme 
      j’aiavoie devant dit. 
   Si cuidierent bien que ce fussent leur compaingnons, mes em petit d’eure furent contraires, conme ceus qui les envaïrent 
   et leur coururent sus si aigrement que mal de celui qui conseil peust mettre a soi deffendre, tant furent il esbahi. 
   Si ne demoura pas que il furent touz mis a mortmort en poi d'eure. Si en i 
   otorent de tiex 
   qui orent grant avantage de ce que il connurent meudres chevaus que il n’avoient 
   fet. Lors se mistrent a chevauchier avant, si n’ont finé tant qu’il choisirent 
   Pelyarmenus, lui trentiesme, qui tuit estoient descendus de leurs chevaus pour miex reposer eulz 
   et leurs chevausdestriers. Et que firent les x compaingnons ? 
   Maintenant vindrent sus eulz, aussi conme pour dire nouvelles. 
   Mes em petit d’eure furent d’eulz assailli, de quoi il ne se donnoient garde. 
   Si avint que si aida deables a Pelyarmenus que il monta en son cheval avant 
   que nuls desdes .X. compaingnons, 
   qui estoient x, 
   fussentfust venus a 
   lui, et des autres y ot occis ne sé quans avant que 
      l’en feustil fussent 
      esen lor chevaus montezpluseurs. 
   Iluec ot une bataille si merveilleuse que Pelyarmenus cuida estre 
   occistrays de sa gent meismes. Lors se 
   mistferi en la forest, lui tiers 
   sans plus. \pend
\pstart Dorus, de qui j’avoie fet devant mencion conment 
   il avoit sus sachié son frain ausi conme sus l’oÿroÿ le chaple 
   des espees qui 
   retentissoientretentissoit de loing en la forest, 
   lors se mist a venir cele part, si ne demoura gueres qu’il s’embati en une lande ou cil se combatoient, mes nul 
   n'iqui vit des siens, 
   ce li fu avis. Et nonpourquant vint il la le plus tost qu’il pot, si trouva les x contre les xxvii 
   qui mout grant painne avoient mis a ceus desconfire, qui si pou furent contre eulz. Il ne connut pas les siens 
   pource qu’caril s’estoient desconneudesconneu, ainsi comme je ai devant dit, 
   et cil qui contre lui erent l’ont aperceu. Si orent tel paour de lui qu’i se partirent d’iluec au miex 
   que il porent et se ferirent en la forest aussi conme tout ensemble, li i et li autre ça et la. 
   Et de cestui afaire avint il i mortel encombrier. 
   EtCar ainsi conme je truis el 
      conteconte, cil qui s'enfuioient se mistrent en la forest li uns ca 
      et li autres la. Dont il avint que, le plus si emporta le mains. Et il est bien voir que tieux a bien pooir de 
   fouir quil n’a pas cuer de plus atendre. 
   Tout aussi avint il ade la mesnie 
   Pelyarmenus, quequi de la paor que 
   il orent du chevalierchevalier doute orent 
   il cuer de fouir et non pas de desfendre, si qu’il avint, 
   ainsi conme je vous ai dit, que de la grant paour que il orent li uns se mist 
   amontavant et li autres aval et couvoita chascun le sien a 
   chacier. Si le firent folement, car, aussi conme chascun puet 
   savoir,savoir que plus furent xxvii 
   queque ne furent xvi, pour quoi chascun des xvi 
   firentfist chace par 
   eulzsoi. Vez ci une autre aventure assez perilleuse, car cil qui 
   en tel guise se 
   connoissoientconnoissent et 
   en tel gieu sevent bien que plus fort et mains de peril eust il es xvi se il fussent assemblé que chascun par soi, 
   si avint qu’il en y ot iii quil se mistrent en i absens sanz eulz asseurer Dorus, 
   si conme il en vint tout seul, et chascuns des autres, li uns ça et li autres la, ainsi conme Fortune les menoit. 
   Mes le conte repaire a Dorus, quil ne volt pas retourner pour 
   paour de ceus qu’il eust d’eulz suivre. Si nous dist ci 
      endroit li contes que cil devant leli fuioient 
   et n’orent cuer de retourner vers lui, tout seussent il que il fust seul. Mes il 
   n’orent cuer de fouir si parfaitement 
   qu’il ne les poistpot a ataindre jusques a la nuit 
   que ilque se cuida esforcier, 
   etmes son cheval par mescheance 
   adreças'adreça a i arbre, et la se creva il le cuer de 
   lasa force, de lasa 
   roideur que il ot en lui, et Dorus meismes en 
   fu bleciez. \pend
\pstart Quant les iii, qui ainsi s’enfuioient, virent ce, 
   iceulz qui se regardoient de fois a autre 
   si sachierent leurs frains et regarderent ceste grant mesaventure, 
   ainsi comque vous avez oï. Si dist li uns d’eulz :
   Or affiert le recoursretor !
   Lors vindrent touz iii sus le noble chevalier, 
   qui pas ne s’esbahi aussi conme moult d’autres eussent fait. Si fist une chose que moult d’autres n’eussent pas fait, 
   car chascun puet bien savoir qu’il avoit tant le jour feru et on sus lui en tantes manieres que escu ne 
      targe qu’il eust aporté emen la bataille ne li eust pas duré jusques la. Et que fist il ? Il prist la selle de son cheval 
   et coupa les cengles et les mistprist en la main senestre et la gita amont 
   sus son chief, et a la destre tint l’espee, dont il avoitavoit le jor 
   maint honme abatu 
   Ronmain etot mis a mort. Quant cil virent que il ot ce fet, si ont ensus sachiez leurs frains, 
   et n’i ot nulnul d'euls si hardi qui l’osast aprochier, dont li un d’eulz dist i mot moult covenable :
   Ha ! chevalier de parfaite honneur et de tres haute grace, je me rendisse a vous, 
   mes que vous me vousissiez recevoir a merci.
   Et quant li autre dui compaingnon 
   oïrentl'oïrent le 
   leurleur compaignon ainsi parler, si n’i ot nul quil ne feust meu a ce, et distrent que 
   tout aussi feroient il volentiers. Lors respondi Dorus et dist :
   Ha ! seingneur ronmain, autrefoisautrefois ja 
      m’avez vous menti ! Je ne vous celeroie pas plus que vostre seingneur, qui mon frere deust 
      estre, et si nous veult tolir nostre droit heritageheritage, ainsi come vous le poez savoir. 
   Mes se vous cuidiez avoir droit envers moi de moi metre a mort ou en vostre merci, vez ci moi tout seul encontre vous trois !Par mon chief, dist li uns, de moi n’avez vous huimés garde, 
   car je voi et sé que nous n’i avrions duree que de nous et de nos chevaus fere occirre. \pend
\pstart Quant Dorus ot ce entendu, si se penssa 
   qu’il estoit a pié et ne savoit en quel lieu. Et d’autre part, il estoit tout seul 
   et en assez grant peril. Dont il leur dist :
   Biau seingneurs, selonc l’aventure qui estm'est 
      avenue, je vous merci de vostre present. 
   Mes ce me dites se vous savez qui je sui.
   Lors dist cil qui premiers s’estoit offers de lui mettre a merci :
   Je sé de voir que vous estes Dorus, frere 
      Helcanus, l’emperere de 
      Constentinonble.Par foy, dist il, il i pert mal que vous le sachiez quant ci entre vous me veez a pié ! 
      Et s’il n’i a nul de vous qui autre chose en face ne qui autre 
      chosecourtoisie ne vilenie 
      enme vueille faire, que je tout seul me 
      mettemetre 
      entre vos mains pouraussi comme pour 
      fairefaire la vostre volenté !
   Lors n’ot pas ce dit quant l’autreil si mist pié a terre et dist :
   Sire, je me met en vostre merci. Vez ci mon cheval, si montez en la selle pour faire 
      la vostre plainne volenté. \pend
\pstart Pas de nouveau § BAtant monta Dorus 
   ou destrier saus touz drois, 
   et li autre dui ne s’asseurerent pas, ainz se mistrent a la voie mout isnellement. Et Dorus si dist a 
   celui qui descendusdemorez estoit
      et qui demoroit sanz son cheval :
   Amis, or sui je miex asseur de vous que vous n’estes de moi. 
   Mes pour la courtoisie que vous faite m’avez vous asseure je que, se il pooit jamés avenir que je feusse en lieu ou 
      jamésje le vous 
      peusse rendre, la courtoisie que vous m’avez ci faite, tenus seroie du rendre. 
      Et de ci endroit ne me quier je partir sanz vous 
      jusqu’a tant que vous serez en nostrevostre compaignie.Sire, dist il, la vostre merci. Et 
      sisi veraiement m’eïst 
      DeuxDeux que, il ne fu onques 
      heurel'heure que mi ancisseur n’amassent miex 
      la vostre partie que la nostre en bonne maniere, et bien i parut aucunes fois.
   Or li dist Dorus :
   QuelOr me dites quel part vous 
      pleroitplaist il 
      miex tourner en ceste forest ? Et sien ce me dites vostre non et 
      qui vous estes, car, si m’eïst Deux, je ne sé ou nous sonmes. Et d’autre part, il couvient que vous 
      montiezmontez derriere moi, 
   car il n’afiert pas que chevalier voist a pié aprés autre en tel point conme nous sommes.Sire, dist cil, sauve soit vostre grace ! Car neent plus que vous savez 
      ououi nous sonmes ne 
      le sé je pas. Et d’autre part, alez quel part qu’il vous plera et que 
      aventure vous voudra mener, carque sus cheval ou vous soiez montez ne 
      cuit je monter huimés, car trop ai estéesté hui traveilliez. 
      SiSi je ainme miex a aler a pié que a cheval pour toutes aventures. \pend
\pstart Dorus, quant il entendi celui, si li sot merveilles bon gré de 
   soncest servise. Lors ne volt iluec arrester, ainz se mist au retour, 
   ainsi comque il cuidoit estre venus. Mais il n’ot pas moult alé quant il en fist le contraire, et en ce le 
   suivoitsuivoient cil a pié a qui il enquist de son non et aprés de son 
   lignage.
   Sire, dist il, voirs est que j’ai nonje sui apelé 
      Abylon. Et fu Gasus, mon oncle, qui jadis au 
      tempstemps que Pelyarmenus 
      entraregna en Gresce 
      et i regna qu’il en fu garde et conmandierres de l’empire de 
   Constentinonble.Et vraiement, dist Dorus, je ne cuit pas que je peusse huimés 
      trouver chose quil me fust contraires quecar vous m’avez ci amenteu 
      le plus loial chevalier qui onc çainsist espee, ne je ne cuit pas que, tant 
      qu’commeil vous souvenist de lui, eussiez nule volenté de fere nule 
      traÿson.Sire, dist il, ja Dieu ne place que je soie en lieu ou traïson soit faite 
      pour quoi je la puisse destourner.Or menous dites dont, 
      ditfait Dorus, conment il vous est 
      avenu entre nousvous de ce nous vous trouvasmes jehui 
      matin si entrepris conme vous estiez xxvii meslez a autre 
      x de vostre gent.Par foy, sire, de ce vous puis je conter la plus merveilleuse aventure que onques mes 
      oissiezosasse conter pour verité.
   Lors li conta, tout ainsi conme desus est ditvous avez oÿ el conte, 
   conment les x.X. premiers 
      chevaliers furent retourné pour savoir quique il estoit devenus qui si 
      esforcieement les avoit enchauciez.Et quant nous veismesil virent que ces 
      x n’estoient reperié, si se mistrent autres x a la voie 
      pour savoir ce que pooitpeust 
      estreestre que il n'estoient repairié. Lors ne demora pas grant piece
      aprés quant les premiers qui de nous s'estoient departi repairierent a nous et nous corurent sus en tele maniere 
      que Pelyarmenus se parti de nous lui tiers comme cil qui bien cuida estre traÿs et nous nous deffendismes contre ceuls 
      dusques atant que nous vous eusmes choisi qui fusmes com tous certains que nostre deffense ne nous eust riens valu fors de nous
      faire occirre. Et ne fust ore se por ce nous que cil .x. qui des nos devoient estre que il nous aloient destraignant
      que je ne cuit pas que au loing aler peussons avoir duree contre euls. Et veez ci que je vous puis faire savoir.
   Et li conta Abylon tout ainsi conme devant est dit 
      ou conte. \pend
\pstart Quant Dorus ot oï 
   Abylon 
   retraire ceste aventurechose, si 
   en ot moult grant 
   merveillemerveille de ceste aventure, 
   car jamés ne s’apenssast que Haynau eust ainsi esploitié. En cestui conte chevaucha 
   Dorus aval la forest, et cilAbylon le 
   suivoit le miex que il pooit, conme cilz qui a moult grant painne et a moult grant travail le fesoit, et avec ce il aloient le 
   contraire de leur retour. Si leur avint une perilleuse aventure, car il oïrent une clochette sonner auques 
   pres en la partie ou il s’estoient mis. 
   Et lors n’ont finé d’errer tant que il ont trouvé aussi conme une lande. Et leur fu avis qu’il avoit une 
   abaÿe ou milieu.  Il firent tant qu’il vindrent a la porte, si la trouverent fermee. 
   Et Dorus i hurta, conme cilz qui entrer y vouloit. Lors parla 
   li portiers et demanda 
   quique s’ilestoit.
   Amis, dist Dorus, di nous s’il entra hui ceens nule gent armee.Sire, dist cil, il en y a et entre et issu. Dites moi qu’il vous plaist.Amis, distce dist 
      Dorus, je vueil que tu me dies se tu sez a qui cil estoient que tu dis.Ha ! sire, dist cil, je ne 
      vous en savroie dire verité fors que tant 
      que ilil que il queroient i chevalier a unes armes vermeilles a i lyon d’or billeté.Or me di donques, dist Dorus, queles connoissances cil avoient.Sire, dist cilcil, ce vous sai je bien dire, 
      il estoient trois, dont li uns ot une cote de noir cendal a trois coronnes d’or devant 
      etet autre tant derrieres ; le secont avoit une cote de cendalynde 
      a couronnes d’argent devant et derrieres ; li tiers avoit une cote blanche a i lyon vermeil.Par mon chief, dist Dorus, celui au lyon vermeil cuit je bien 
      connoistre ! 
   Mes les autres ii ne sé je qui il sont.Sire, dist Abylon, je cuit et croi que 
      il sontce est de vostre gent qui sont retourné de 
      nostrevostre partie. Et celui que il dist a la cote de noir 
      cendal est le plus aspre chevalier qui onques montast en cheval.###Ces références 
      aux armes de ces chevaliers ne nous semblent pas offrir des indices formels de l'identité desdits chevaliers, aucune mention
      dans ce sens n'ayant été donnée auparavant. Il est d'ailleurs intéressant que Dorus et Abylon croient chacun reconnaître un 
      chevalier sous ces armes, alors que les deux compagnons de Pelyarmenus ne sont à aucun moment identifiés dans la suite de cet 
      épisode.### \pend
\pstart Amis, 
   distce dist Dorus au 
   portier, 
   lesse moi leens entrer, car j’ai a merveilles grant fain, et mon cheval est traveillié, et d’autre part il a ci i chevalier 
   a pié qui m’a suivi grant piece, si n’est pas droiture ne usage que il voist plus a pié en 
   cetel point.Ha, sireHa ! dist 
      li portiers, sanz congié ne l’oseroie 
      je faire, car il m’est deffendu. Mes dites moi qui vous estes et je l’irai 
      diredire volentiers a mon seingneur.VaVas i, 
      li dis, dist il, que je sui de la mesniee a 
      l’empereeur de Costentinoble 
   et sui i sien grant amy.
   Cil se parti de la et vint en la chambre a l’abbé, qui soupé avoit, et li dist son 
   mesagemesage son mesage.
   Ore, dist li abbés, que 
      malli mal 
      il puissentli puisse estre venu ! Se on li osast escondire, 
      il n’i entrast mes hui ! Va et si le lés ceens entrer et nene li 
      dis pas que ilil y ait nullui avec moi 
      etains dis que je sui deshestiez et que je me gis en mon lit.
   Cil vint arrierearrie-arriere 
      et 
   lessa Dorus entrer ens et 
   son compaingnon, puis mena son cheval en l’estable. Et aprés mena 
   lesles .II. chevaliers a l’ostel moult convenablement. Si avint ainsi que 
   l’en fist de lui moult grant feste. Mes encoreoncore (sic) li eust 
   on faite plus grant 
   semais que on seust qui il fust. Mes de 
   l’abbé, qui avoit ainsi respondu au portier 
   com je vous ai dit, 
      me couvient que je vous diedire pourquoi il respondi en tele maniere. \pend
\pstart Pas de nouveau § BVeritez fu que 
   cilz abbés estoit de l’ordre saint Beneoit, 
   i grant gentil honme qui avoit esté estraiz de ceus qui furent destruis pour l’empereris 
   Helcana, si conme il a esté devant dit. Si avint 
   ainsiainsi par aventure que 
   Pelyarmenus et si dui chevalier estoient avec 
   l’abbél'abbé aussi comme par aventure
   et par amisté que il li volt faire, si volt Pelyarmenus savoir qui ceus estoient 
   qui a cele heure s’estoient embatu leens. Si dist a l’un de ses chevaliers :
   Il vous couvient savoir qui cil sont dont nous avons entendu ceens leur venue.
   Li uns d’eulz se leva et prist leun des serjant a 
   l’abbé et vint a Dorus 
   la ou il estoit voianto lui Abylon, 
   qui estoient seant a la table, touz garnis de leurs armes fors que chascun d’eulz avoit son hyaume osté et mis d’encoste soi sus la 
   table et sl’espee toute nue desus couchiee, 
   qui estoit moult estrange chose a veoir. \pend
\pstart Le chevalier qui estoit a 
   Pelyarmenus esgardaesgarde 
   Dorus, si ne le connut pas, mes il connut Abylon. 
   Lors vint arriere a son seingneur et li conta ceste merveille. 
   Et quant Pelyarmenus oÿ ce, si sailli em piez et vint cele part, 
   si connut son frere Dorus et vint a l’abbé au plus tost 
   qu’il pot et l’a embracié moult en haste et li dist :
   Ha ! gentilz sire, merci, je sui au desus de ma besongne, mais que vous me vueilliez aidier !
   Lors li conta l’aventure de son frere, 
   aussi conme vous avez oÿ. Et lors li dist 
   li abbés :
   Conment voulez vous, sire, que j’en esploite ?Sire, dit il, je vous pri que vous me 
      faciezfaites tant d’onneur que je puisse fere ma volenté.
      Ha ! sire, dist li abbés, 
         mercipour Dieu, , 
         merci, je ne soufferroie por nulle chosechose qui fust que il mourust ne que 
      j’en feusse retez de traïson, car je en seroie destruis et nostre abbaÿe d’autre part.Sire, dist Pelyarmenus, merci, je ne vous requier chose que 
      vous ne puissiez bien faire tout em pais, et si vous dirai conment. Il est ainsi qu’il a i 
      mien chevalier o lui, et je ne sai 
      conment. Si voudroie savoir en quel maniere nous pourrions parler a lui sanz ce que Dorus le seust.Par foy, dist li abbés, je ne sai conment nous em puissons ouvrer.Je le vous dirai, dist il. Il vous couvient aler a eulz parler et faindre cest afaire et 
      savoir conment il est avec DorusDorus, mon frere.Et j’en ferai la besongne, dist il. 
   Lors s’en vint li abbés la ou Dorus estoit. 
   Et quant Dorus sot que ce fu 
   li abbésli abbés de laiens, si s’umelia moult contre lui, 
   et il s’asist d’encoste lui et li dist :
   Sire, vous soiez li bienvenus !Sire, distce dist 
      Dorus, et vous 
      aiezaiez bonne a-iez bonne aventure !
   Adonc a demandédemanda a l’abbé :
   Sire, savez vous qui nous sonmes ?Sire, dist il, voirement lequi vous estes 
      sai je bien. Mes je cuit que vous ne voulez pas que l’en le sache.Bien puet avenir, dist Dorus, que je ne voudroie pas que tuit le 
      seussent.Sire, dist li abbés, et je ne volz pas lessier quant je soi 
      que vous feustesestiez ceens que je ne vennisse a vous pour savoir 
      conment il vous a esté en la journee d’ui.Moult noblement, dist Dorus. Or dites avant ce que il vous plaira.Je vueil, dist il, pourcepource que je avoie oÿ dire 
      qu’il avoit ceens i des grans amis 
   mon seingneur de Constentinonble 
      que je ne vouloie pas lessier, mesque je 
      sui venusne venisse a vous pour conmander a fere toute vostre volenté.Sire, dist Dorus, la vostre grant mercis. \pend
\pstart Moult fu Dorus conjouys de l’abbé 
   et de touz ceus de leens, et leur fu aporté des meilleurs vins et des plus fors de 
   leensleens, ne que l'en pooit trouver en tout le pays. Dont il avint que il, 
   si em burent a grant plenté, et lorlors dist 
   li abbés :
   Sire, je vous asseurasseur huimais 
      de touz vos anemisanemis ceens.Sire, dist Dorus, grans mercis.Or faites oster vos hyaumes, dist li abbés, 
      car onques mes n’oï parler d’onme quiqui tant se doutast qui ce feist 
      que vousvous en avez fet.Non, sire ? dist Dorus. Par 
      foymon chief, dans abes, dist il, bien vous en 
      croicroi. Tout autresi n'en oÿ je onques mais parler que ore mais or le poez veoir
      et oÿr. Si vous en convendra faire a l'avenant.
   AdoncLors commanda l'abé que l'en ostast les hiaumes. Adont mist chascun 
   la main au sien, et dist Dorus a l’abbé :
   Sire, vostre conmandement ne s’estent pas ci endroit ne si avant 
      que je vueille que vous nousvous desgarnissiez de 
      nosvos armes pour chose qui pas vous peust porter grant conquest.
   Dont fu li abbés esbahis, si ne se sot de quoi couvrir fors que il dist :
   Sire, Deux moie courpe, je cuidoie bien dire.Ha ! sire, pour Dieu merci, je ne me savroie en qui fïer, et je vous dirai conment, 
   et si ne vous anuit se je vous ai repris, car je vous en conterai une merveilleuse aventure pour ce que vous en soiez en bonne pais. 
   Vez ci cest chevalier qui siet delez moi que je trouvai hier a mon anemi, 
   et je estoie le sien. Si ne sui pas du tout asseuré de lui, conment que il soit de moi.
   Lors li conta a briez parolles conment il s’estoient entracointié et venu jusques la. \pend
\pstart Quant li abbés 
   ot ce oÿ, si ot moult grant merveille de ceste chose 
   et penssa en son cuersoi meismes que 
   moult pou se devoit prisier honme qui n’avoit en soi 
      aucuneaucun poi de franchise, si dist oiant touz :
   Ja Dieu ne me doint s’amour se, a ce que j’ai entendu de vous ii, 
   s’il n’a plus de droiture et de franchise en vous qu’il n’a en la moitié des moinnes de ceens.
   Lors n’i ot nul d’eulzde ceulz qui la furent quil n’ait eu grant joie 
   de ce que li abbés a ditdit. Or vous feroie je
   ore plus de mencion de chose qui ne fait a metre en conte. 
   Dont fistcommanda Dorus 
   osterque l'en ostast 
   la tablede devant lui, et l'en si fist, 
   siet quant il ontot 
   lavé, puiss'aclina devers Abylon et li dist Dorus 
   a Abylon 
   en l’oreilleoreille quoiement :
   Se Deux me doint joie, 
      cilz abbés me semble uns faus varlés.Tout aussi fet il moi, dit Abylon.Cette
   réflexion de Dorus, tout comme son refus au préalable de se désarmer, démontre une fois encore sa grande lucidité. \pend
\pstart Pas de nouveau § BAtant saisi Dorus 
   son hyaumehyaume et s'espee et puis sailli em piez et dist :
   Or aus chevaus !ConmentComment, sire ! dist 
      li abbés. Vous ne vous partirez huimés de ceens, se Dieu plest !Sire, dist Dorus, ja Dieu ne place que je ja aie repos 
   en hostel ne en meson devant que je sache conment il est a monseingneur mon frere, 
   que je lessai jehui entre xxm combatans.Ha ! sire, dist li abbés, 
      pour Dieupour
      G qui copiait à partir d’une leçon telle que celle que nous avons en V3, 
         aurait décidé d’enlever l’exclamation je ne 
      le vous osoie demander.Vous avez droit, dist Dorus, 
      car assez a temps en pourrez oïr ce qu’il en serasavra.
   Atant se trait Dorus vers Abylon et li dist :
   Sire compains, je vous doi moult mercier de la vostre grant franchise. 
      Si sachiez que je mene vueil de ceens departir conment qu’il 
      en aviengne, 
      car aucuns de mes compaingnons mene quierent, 
   qui sont de moi en doubte. Et vous demourrez, se il vous plaist, car je ne vous vueil chose requerre qui soit contre 
      vostre volentévous.Ha ! sire, distce dist 
      Abylon, je ne vous leroieloeroie 
      aler sanz moi tout 
      seul pour nule rien ne pour amour que je aie a mon seingneur. Mes prions a 
      cest abbé que il nous preste i cheval, et il li ert moult richement 
      guerredonné. \pend
\pstart Atant vint Dorus a l’abbé et 
   tint a une main son hyaume et s’espee 
   a l’autre et laleet l'autre main 
   gita 
   au colcol de l'abé et dist :
   Sire, je vous requier pour Dieu et pour 
      humilitéhumanité i cheval qui me peust porter de ci a ma gent. 
      Et sachiez que il vous 
      seraiert moult 
      bienhautementrichement
      renduguerredonné, se je jamés vous 
      revoirevoi, car je n'en ai point ainsi comme je vous ai conté.Ha ! sire, dist li abbés, ja 
      savezsavez vous bien que 
   quanque je ai ceens est vostre frere. Mes, foy que doi saint Benoist, mon patron, 
      que je n’ai ceens cheval qui vous feust couvenable. Mes demorez ceens huimés, et le matin je vous en querrai un, que 
      quique il me doie couster.
   Lors fu Dorus esbahis, et parla Abylon et dist :
   Sire, vous ne poez ne devez mon seingneur 
      escondireescondire de ce que il vous requiert.
   Dont prist li abbés Abylon 
   etet le traist d'une part et dist :
   Conment, sire chevalier ! Estes vous ore si tost acordé a vostre anemy ? Cuidiez vous que, 
      se je eusseseusse cheval ceens qui li fust 
      couvenablecouvenable que il couvenist, 
      que vous fussiez moyen de ceste requeste ?Ha ! sire abbez, n'estesde n'estes vous honme 
      de religion ? Conment parlez vous ore si desordeneement ? Ne cuidiez que il 
      couviengnecouviengne que, se il a 
      aucuneaucune chose discencion entre 
      mon seingneur et son frere, que ce ne soit 
      convenable chose de porter bonnes parolles de l’un et de l’autre ? Si ne couvient pas pource que il soit 
      monde moy anemi ne je le sien. 
      Par mon chief, je ne sui mie ne si ne vueil estre sison malveillant, 
      car je cuit 
      quequant vous et je pourrions estre mal de lui. Si pourroient 
      il estre bon ami ensemble, et pour ce ne le di je pas que, 
      ce vous nelm'i confortez de ce 
      quedont il vous requiert, il n’ira pas a pié, car il enmenra mon cheval 
      et je demourrai ceens jusqu’a tant qu’i le me renvoiera ou que je pourrai avoir autre. Et ceste 
         courtoisie li voudroievoudroie je 
      ferefere pour l'amour de monseigneur 
         monvostreDe manière intéressante, c'est la leçon de X2 
         qui semble plus cohérente que celle de B. frere. \pend
\pstart Quant li abbés ot entendu le chevalier,
   si sot qu’i fu sages et preus et dist :
   Abylon, 
      ainsi conme je vous ai 
         oï nonmer, 
   il a ceens chevaliers qui m’ont envoié a vous conme en confession et vous mandent que vous vous partez de 
   Dorus au plus secreement que vous pourrez, tant qu’il aient parlé a vous.
   Quant Abylon oï ce, si mua coleur et vint a Dorus et li dist em bas :
   Sire, il a ceens de nostrevostre gent, 
      et je ne voudroie pour nul avoir queque nul 
   mal vous em peust avenir pour que je le peusse trestourner. Il m’ont mandé que je aille a eulz parler. Alez vous ent, 
   aussi conme se de ce ne seussiez riens, et montez sus vostre cheval et issiez la hors, 
   car cilz abbés vous est contraires. Et se je puis par nul sens qui en moi soit, je revenrai a vous. 
      Si tenez mon moiennel et leque vous 
      sonnezsonnerez se il vous semble que je ne reviengne a vostre volenté. \pend
\pstart Pas de nouveau § BMoult ot Dorus 
   grant joie de la loyauté Abylon et li mist amdeus ses bras au col et dist :
   Amis dous, je vous pri, se vous poez plus 
      reperier a moi, que vous reveingniez. 
      Et je vous jur conme chevalier que je sui que je vous ferai 
      seingneur de moi, mes que je puisse de ceste bataille eschaper vif.Sire, dist Abylon, alez au Saint Esperit, et j’en ferai mon pooir.
   Atant se departirent les uns des autresli uns de l'autre, et 
   Dorus dist que il voloit aler 
      veoirveoir comment il estoit a son cheval. Lors li 
   dist onfu dit 
   qu’i li estoit bienbien et a aise et 
      que de ce ne li convenoit il ja pensser.
   Si fet, dist il. Je ne m’en fieroie en nul honme 
      devanttant que je l’avrai veu.
   AtantAtant en vint en la 
   sallesalle d'une chambre ou il orent soupé, 
   mes il vit un grant conseil de gent qui parloient ensemble, mes il fist aussi 
   conme se il ne li en chausist, 
   ainz en vint a l’estable et trouva le cheval sus quoi il estoit venu gisant malade, a son avis. Lors esgarda aval et 
   vit trois chevaus grans et gras et en bon point, si demanda a qui il estoient, 
   et uns garçons li dist qu’il estoient a 
      i chevalier du paÿs.
   Amis, dist ilDorus, il n’a honme en 
      ceste terre qui m’escondisist son cheval, 
      ce dist Dorus. Va et si mes ma selle sus cestui et si n’en fais ja dangier ne noise, 
      carque je me courrouceroie a toi !
   Cil n’osa escondirefaire contredit de ceste chose, 
   ainz a fet son conmandement. Et puis a fet mettre le cheval hors enmi la court. 
   Dont monta en la sele au plus tost que il pot, si vint a la porte et se fist hors mettre. Et quant 
   le garçon vit que il enmenoit 
   sonle cheval 
   sanzoultre son gré, il 
   conmença a crier, et en vint a l’abbé la novelle, 
   ainsi conme il iert devant Pelyarmenus, qui 
   parloitpalloit (sic) a Abylon.
   Sire, dist li abbés a 
   Pelyarmenus, par foy, onques mes tele chose n’oÿ 
      quecar vostre frere 
      Dorus enmainne vostre bon cheval et a bien 
      pou mon garçon occis.Par mon chief, dans abbés, il a bon droit, du mien se puet il bien tant 
      aidierfier conme i cheval vaut. 
      Mes or me dites conment il sot qu’il estoit miens.
   De ce, dist li abbés, ne sé je 
      riensriens, mais ce sai je bien que il l'emmaine.
   Dont a on fet venir le garçon avant, et li demanda on 
   conment la chose estoit alee, et cil leur conta tout en tele maniere 
   conme il estoitavoit alé, mes de ce que le cheval feust a 
   Pelyarmenus ne set il riens.
   Bien va dont ! dist Pelyarmenus. 
      Or le lessiez aler, quecar de plus apenssé chevalier n’iert il jamés nouvelle oÿe. \pend
\pstart Ainsi conme je vous ai conté ça en arriere avoit 
   Abylon parlé a Pelyarmenus et raconté 
   l’aventure et la novelle de 
      Dorus conment il s’estoit acointié a lui.
   Mes il lile couvint 
   a mentir en ceste maniere que je vous dirai, 
   aussi conme il li couvient a la fois 
   ferefere a maint preudonmeDans la lignée de
   tout cet épisode porté par une logique sapientiale qui vient soutenir la conclusion de cette guerre fratricide, cette assertation 
   n'est pas sans évoquer le proverbe "Bons mentirs a la foie aiue." (Morawski, n 291)..
   Sire, dist Abylon, il est voir que je me sui acointié a 
      vostre frere en la maniere que vous avez oÿ. Mes sachiez que la 
      causecolor por coi je l’ai fait 
      n’est autre fors que je lel'en cuidoie amener a ce que je le seusse en 
      aucun lieu ou je le peusse amener et que vous le peussiez saisir 
   et prendre, tant que vous eussiez pes envers eulz couvenable par quoi vous feussiez bon ami 
      ensemble, car vraiement il n’est pas preudoms qui ce ne couvoite. \pend
\pstart Pas de nouveau § BQuant 
   Pelyarmenus ot ce entenduAbylon ot ce dit, 
   si gita a Abylonli mist Pelyarmenus 
   les bras au col et li dist :
   Vraiement, Abylon, vous estes preudoms ! 
   Et se vous peussiez avoir ce fet, vous feussiez mes amis et si en eusse ouvré par vostre conseil.Sire, dist il, encore cuit je que je le feroie, mes que vous 
      m’en lessissiezme vosissies laissier couvenir.Oïl, dist ilPelyarmenus. Or me 
      dites,dites, dist Abylon, 
      conment vostre gent s’est maintenue puisque vous partistes de la bataille.
      Je n’oïen oï onques nouvelles, dist il, 
      ne je ne me quier partir de ceens devant que je 
      sachesache devant que je savrai 
      conment nostre besongne va, car je me sens moult blecié et navré, 
   et nulz ne me set ceens fors vous et ceus que vous savez.Sire, dist Abylon, je lo que vous soiez ceens 
      tant que vous sachiez conmentde la 
      besongnebesongne comment ele va, car il m’est avis que 
      cist abbez vous est 
      aucunsanemisLa leçon de G ### 
      amis.Vous dites voir, dist il, car son lignage fu destruis par 
      la fausse empereris dont vous avez oui, 
      par quique tant de maus sont venus. \pend
\pstart Pas de nouveau § BEn ceste parolle ne demoura pas 
   queque Dorus que Dorus
   ne sonnast ii mos, et Abylon 
   l’entendi, si dist :
   Sire, je oï vostre frere qui m’apelle. 
      Or esgardez se 
      vous voulezil vous plaist que je voise a lui.Et conment ? dist Pelyarmenus. 
      Li eustes vous en couvent que 
      vousvour (sic) iriez aprés lui ?Sire, dist ilAbylon, voirs est que, 
      quant l’abbé me dist que il avoit de vostre gent ceens 
         quilquil estoient et vouloient parler a moi, je li dis 
      qu’il demourast huimés, et il me dist que 
      il pour riens neil ne demorroit. Dont li dis je que, 
   se je pooie, je iroie aprés lui au plus tost que je pourroie. Et s’il vous plaist, je irai, 
   se ce non je m’en deporterai et ferai envers vous ce que je devrai.Alez i, dist Pelyarmenus, et faites tant de ceste chose 
      et d’autre que je m’en aperçoivepuisse apercevoir, et que vous soiez 
      preudoms et loyaus.De ce ne doutez ja, dist il. \pend
\pstart Pas de nouveau § BAtant s’en vint 
   Abylon 
   en l’estable et prist i autre cheval 
   que le sien et se mist aprés Dorus au plus tost que il pot et le trouva en la lande ou il 
   l’atendoitert en aguait de lui. Et quant li uns vit l’autre, si furent 
   moult joiant. Dont se sunt entresalué et enquist 
   Dorus a Abylon et qui cil furent 
   qu’il avoientavoit trouvé en 
      l’abaÿe.
   Sire, dist il, vous n’en avez pas moult a faire ne plus ne vous en dirai. 
      Mes gardez quel part vous vodrezil vous plaira a aler.Et conment, dist Dorus, le 
      pourroie je 
      savoir ?savoir quand
   Je ne sai ou nous sonmes ne de quel part nous sonmes ci venus.Je le vous dirai, distce dist 
      Abylon. Veez a la lune quele part elle doit esconser.Voir dites, dist Dorus. 
   Atant se sunt mis vers Oriant en la forest. Il estoitestoit aussi comme vers la 
   mie nuit, et fu aussi conme vers la Toussains. 
   Si avint que il prist a Dorus i si grant sonmeil que il dist :
   Ha ! Abylon, 
      sire compains, a dormir me couvient.Sire, dist Abylon, fere le poez.Sire, dist il, si ferai je hardiement 
      a la foy Dieu et a la vostre. \pend
\pstart AdoncAtant descendi 
   Dorus, et Abylon 
   prist le cheval en sa main et demoura ou sien. Dorus se coucha sous i arbre 
   ou ilet dist li contes que se fu moult tost endormis. Et en ce avint que 
   les ii chevaliers qui s’estoient partis 
   d’Abylond'Abylon ainssi comme je ai dit dessus quant il presenta son cheval a 
   Dorus vindrent cele partie, 
   aussi conme par aventure, qu’il avoient ouÿ le son du moiennel que Dorus avoit sonné. 
   Et quant Abylon les oï venir, si 
   est traitse traits'est trait 
   ensus de Dorus. Et quant i 
   l’ont aperceu, 
   ilen ce que la lune luisoit clere, sont venus a lui, vousist ou non. 
   Quant Abylon lesce 
   vitvit, si les connut, si vint contre eulz aussi com s’il 
   en eust grant joie 
   que il les reconnut moult bien. 
   LorsQuant il l'ont veu, lors a enquis li i a l’autre 
   conment il avoient esploitié. Cil distrent que toute la 
      nuit il avoient estéalé 
      sus et jus ne onques puis il n’avoient trouvé 
      rien qui lor eust aidiéaidié ne veu fors ce qu’il avoient oï 
      leson moyennel a qui son il estoient venus.Alez, dist Abylon, par cest assens, 
      et la trouvereztrouverez vous 
      une abbaÿe ou mon seingneur 
      Pelyarmenus 
      estest, aussi comme se il ne vosist pas que l'en li seust, 
      car si n’avez vous queque faire pource que cest 
      deableIl est évidemment intéressant que l'étiquette de diable habituellement associée à Pelyarmenus 
      soit ici, pour les chevaliers de Pelyarmenus, justement accolée à Dorus. 
      Dorus est iluecici pres, lui tiers, 
      et s’i nous trouvoit ja ci endroit, il nous seroit mal encontre.Et conment dont ! dist l’un des ii. 
   Dont vous vient ore ici le cheval mon seingneur ?Trop atendriez l’aventure du savoir, dist il. Mes tant dites a 
      mon seingneur que je sui auques en la voie de la besongne 
         asouvir que il m’avoit dite. \pend
\pstart Pas de nouveau § BLors n’ont plus cil iluecques atendu, 
   ainz se sont mis a la voiepartis d'ileuc au plus tost que il porent. 
   Si devezpoez savoir que moult est l’onme plain de 
   grant vertu qui a loiauté en soy. Or sont ore assez de chevaliers qui voudroient dire qu’il seroient 
   pou d’onmesprodommes qui eussent convoitié la destruiction d’un tel honme 
   pour aquerre un pou d’avoiramor a leur seingneur. Mes cilz ne couvoita fors 
   que il peust mettremettre en aucun lieu le preudonme a sauveté. 
   Si fu iluec tant que Dorus ot tant dormi et si forment que i li fu avis en son dormant que 
   Helcanus, son frere, li estoit venusvenoit 
   devant, aussi conme vestu d’uneune robe de pourpre d’or estelee, 
   et li sembloit qu’il estoit en une compaingnie d’angels et estoit plus bel que il ne l’ot onques veu. Si li disoit :
   Frere, or vous coviengnesouviengne de 
      ceste guerre, carque 
      j’en ai tant fait que je n’en puis plus, car il me couvient aler en une region dont jamés ne me verrez devant 
      aule grant jour 
      espoentable. A cel saint jour 
      quequant le Grant Mestre tendra Son jugement, et la 
      iertsera rendu le 
      droit jugement de nostre droit contre le 
      droittort Pelyarmenus, 
      nostrevostre frere.
   Et atant se partpartoit Helcanus de 
   Dorus, son frere, si conme il i ert avis en son dormant. \pend
\pstart En ceste visionraison s’esfrea moult 
   Dorus. 
   Si s’est esveilliez moult tristres et moult dolens. Lors si sailli sus, si vit Abylon, si le hucha, 
   et il vint a lui. Et Dorus li demanda se il avoit guieres dormy, et 
   Abylon li dist :
   Sire,que nennil. 
      Qu’avez vous, qui me semblez effreez ?Amis, dist il, je monterai, et puis si savrez que je vous dirai.
   AdontAtant monta Dorus sus son cheval. 
   LorsLors ne se sont iluec arresté, 
   ainz se sont mis a la voie. 
   Et dont dist Dorus :
   Amis, je ai veu une avision en mon dormant 
      qui m’a mis en moultmon grant esfroy.Ce n’est que tout bien, dist Abylon, se Dieu plest, 
      car nul ne doit avoir en songe que seurté de tout bien.
   Lors li a conté Dorus ainsi conme i li estoit avenu en 
      son dormant. \pend
\pstart Pas de nouveau § BQuant Abylon 
   l’ot entenduentendi, si 
   vitvit et sot par sonau 
   songe et par experiment de la lune que Helcanus estoit occis et mis a mort, 
   mes il ne li volt pas iluec despondre, ainz li dist :
   Sire, ne doutez ja de ce, car il ne puet avenir que vous 
      n’aiezoïez par temps bonnes novelles.Par foyfoy, amis, dist 
      Dorus, onques mes n’oï parler de tele aventure, jour de ma vie, 
      conme de la 
      battaille quecomme nous eusmes a la journee d’ier. Car je n’oï de nulle 
      partie nouvelles que mes compaignons sontsoient 
      devenusdevenus et d'autre part de roi 
         n'oÿ je nule nouvele.
   Lors li conta Abylon conment cil dui chevalier furent venus a lui 
      et conment il s’ense delivra pour toutes aventures.
   Par foy, dist Dorus, Abylon, 
   biaus dous amis, je sé bien, se vous me vousissiez avoir trahy, que je ne fusse pas ici. Et sachiez que je seroie moult liez 
   se je ja jour de ma vie vous pooie rendre le guerredon de la compaingnie et du servise que vous m’avez fait.Sire, pour Dieu, dist Abylon, ice me dites 
      pourcoicomment ce est que 
   je n’ai oï nullui de vostre compaingnie sonner ne cor ne moyennel par coi aucun de vous feussiez ensemble ravoié.Abylon, dist Dorus, 
      de ce n’enne poez vous mes se vous le dites. Mes de ce soiez certains 
      que, a l’eure que je me mis ier aprés vous, je avoie le meilleur cor 
      du paÿspaÿs, mais une branche le me toli et je ne vueil le vostre sonner 
      pource que il me sembleroit aucune raison estre. Car se aucun de vostre gent venoient au son, je ne me pourroie tenir qu’en aucune 
      maniere ne me preisse a eulz ou eulz a moi, par quoi il vous anuiast.Sire, distce dist 
      Abylon, celveraiement cel a esté le 
      mien avis, car encoreoncore (sic) ne vousisse 
      je pour grant chose que ces ii 
      chevaliers vous eussent seu en tel point conme vous estiez.Il m’eussent tué, dist Dorus, a leur pooir !Ce ne dis je pas, dist Abylon, 
      ainz en eust tel chose esté dite dont nul autre que je n’enne 
      savroitsavra parler. \pend
\pstart Moult longuement ont chevauchié les ii 
   chevaliers, qu’il n’ontont oÿ ne trouvé chose qui 
   leslor embelisist. Et ce ne fu pas 
   moult grant merveille, car, aussi conme vous avez oÿoÿ devant 
      el conte, cil qui faisoient leur chace, ainsi conme je 
         vous ai conté, il en y ot de tieux qui folement 
   la firent. Et parpar lor 
      folement enchaucier il en furent mort, et 
   parpar leur maleur les autres 
   lela firent en telle maniere que il les mistrent a mort, 
   vousissent ou non. Mais li contes n’en fait ore pas autre mencion pource que trop y avroit a dire 
   avant que je venissel'en peust venir 
      au principal de la bataille. 
         Si m’en vueil ore outre passer au plus briefment que je pourrai sanz corrompre la matire 
   et vueil reperierrepaire au plus souffisant de ceus qui avoient fet la chace aprés Dorus. \pend
         
         
            Ci endroit revient le conte a 
               Hayno, le bon chevalier qui si bien le fist conme nulz le pot faire, 
               conme cil qui aloit querantaprés 
               Dorus.            
            
            
               Enluminure sur 1 colonne (d) et 12 UR.
                  Hayno, à cheval en armes rouges à motifs blancs, 
                  à la recherche de Dorus.
               
            
\pstart En ceste partie
   Or dist li contes que 
   Hayno, dont vous avez oï desus, ne chaça pas les siens en vain, car, 
   aussi conme je truis el conte, le levrier qui est afaitié 
   au lievre prendre ne se preuve mie miex quant il voit que sa beste est pres de la garenne 
   et il a paour que elle ne li eschappe : adont s’esforce tant que il li tolst son refui et le demainne, si que par force le prent. 
   Tout aussi fist le bon chevalier, car avant que il tirast son frain en 
   otoccist il vi, 
   et ses compaingnons xii.XII., ainsi en occist il .XVIII. des .XXVII., 
   et les autres s’en eschaperent, qui occistrentoccistrent de leurs 
   anemisanemis .III., et les autres se ralierent au mieux 
   que il porent et par le son de lor moyenniaus. \pend
\pstart Pas de nouveau § BHayno, 
   qui la seue chace avoit faite et qui avoit le jour moult pené a ce qu’il peust sormonter ses anemis, fu moult ataint du grant travail 
   que il avoit le jour eus, mes il ot greingneur pitié de son cheval que de lui. Adonc descendi a pié et trouva son cheval moult 
   formentmalement navré, si que il vit bien qu’il ne 
   potpeust pas longuement durer. Lors conmença a sonner i cor 
   moult aigrement, si que aucuns de ceus de sa compaingnie l’oïrentoïrent et vindrent
      a lui li uns de la, si que desd'euls x 
   en vindrent les vi a lui. Quant il furent ensemble, si ont 
   demandé de Dorus, que il avoient veu venir a leur secours, mes il n’i ot nul qui nouvelles 
   en seust. Dont ont sonné et resonné tant que pour leur compaingnon 
   que pour Dorus que, se il les 
   oïssentoÿst en aucun lieu ou il 
   fussentfust, que 
   ilil ne 
   venissentvenist a eulz, fust pour bataille ou pour aucune chose. 
   Mes riens ne leur vautvalut, car en ce point de adonc li estoit son cheval mort et avoit fet escu et targe de sa selle, 
   si conme vous avez oï el conte. Et d’autre part, il estoit trop loing d’eulz. Et avecques tout ce avoit il 
   assez d’autres ententes. Et encore d’autre part li estoit la voieli vens 
   contraire, si que il furent iluec moult longuement, dont il avint que les autres v qui estoient venu en la bataille avec 
   Dorus se rassemblerent jusques a iii et li autre dui furent mort en la chace. Et des autres
   x aussi conme par aventure vindrent a eulz les ii, et li tiers fu ocis par moult grant male aventure, 
   quequar, quant il cuida estre 
   mismiex au desus de sa besongne, si li failli son cheval, et iluecques 
   fu occis aussi conme par defaute, car il ne pot recevoir les cox que il 
   gitoitgeterent a lui. \pend
\pstart Cil dui et les autres iii se 
   rassemblerentrassemblerent ensemble aussi conme par aventure. 
   Et ainsi conme je vous ai conté desus conment Hayno avoit fet a soi 
   et a ses compaingnons changier leur connoissance qu’il ne feussent conneu de lor anemis, si que, quant ces ii virent 
   cesles iii
      .III. de la compagnie Dorus, il sachierent leurs frains et osterent leurs hyaumes, 
   car li autre estoient venu avec Dorus. Dont 
   distrent li dui :
   Biau seingneurs, n’aiez doute de nous ne nous de vous, car nous sonmes touz a i 
      seingneur.
   Lors se sont asseuré les uns des autres et ont 
   contécontré 
   les ii conment Hayno leur compaingnon avoit fet prendre les 
   des connoissances a ceus que il avoient premiers occis. \pend
\pstart Pas de nouveau § BAinssi se mistrent li v en une compaingnie 
   et puis quistrent lor compaingnons, et meismes Dorus queroient il sus touz les autres. Si se 
   departidepartirent 
   lesdes v 
   desles iii 
   des autres ii et vindrent aussi conme par cheance a l’abeÿe 
   dont j’ai desus parlédit. Et cilz iii 
   furent cil qui avoient enquis du chevalier vermeil au lyon d’or billeté, si que, quant il n’oïrent en 
   l’abeÿe fors que l’en lor dist aussi conme pour delivrance de 
      eulz que il avoit passé parmi la forest a moult grant haste. 
   Et quant il oïrent ce, si cuidierent que ce fust verité et se mistrent arriere hors de l’abbaÿe 
   et vindrent a leur ii compaingnons pource que il vodrent chevauchier plus 
   seurement en la queste de Dorus. \pend
\pstart Pas de nouveau § BCeste queste fu moult contrere, car, 
   aussi conme il avint, il alerent contre poil et contre lainne, si que toute la nuit furent cil 
   quede qui vous oez en tel 
   painneanui de leurs amis que il ne se porent autrement ravoier que vous 
   oez de ci tant que ce vint aussi conme sus le jour que Dorus chevauchoit moult penssis 
   et avoit le cuer a moult grant meschief de l’avision qu’il avoit veue en son dormant. Et si ne s’en osoit pas bien descouvrir 
   a son compaingnon. Et en cel anuy dist il :
   Ha ! Abylon, biau 
      dous amis 
      et dous compains, il ne plest pas a Nostre Seingneur que nous 
      puissonspuissons anuit venir en lieu ou nous puissons oïr novelles 
      de ce que nous voudrions oïr. Et vraiement, il n’est si grant folye conme de folement enchaucier et conme de lessier sa compaingnie 
      en tel affaire. Mes nul cuer courroucié si ne puet estre bien senéconseilliés 
      et l’un jour estre pere et l’autre parrastre.Sire, dist Abylon, ainsint est 
      il. Mes il me semble ainsint conme je puis veoir que encore n’avez vous 
      trouvé nul parrastre en lieu ou j’aie esté.Par mon chief, dist Dorus, vous avez voir dist. 
      Mes je ne sé conmentque il m’en avendra d’ore en avant. Je vueil sonner 
      i mot, ou vous sonnerez.Metez ça, dist Abylon. Je sonnerai.
   Atant pristil prist le cor et le mist a sa bouche et a sonné si haut que 
   uneune grant luye en touz sens en ala l’oÿe. 
   Dayno \pend
\pstart HaynoDayno 
   et ses compaingnons ont entendu le son, 
   si mist le sien cor a sa bouche et a sonné de grant vertu ii mos que bien le pot on oïr de loing, 
   car avis lor fu que cil que il avoient oï estoient moult loing, et sanz faille si estoient il. 
   Mes a leur son ne connurent il 
   les uns des autres. Et nonporquant n’i ot celui qui moult ne couvoitast a 
   venir li uns a l’autre tant plus pour oïr nouvelles que por bataille avoir. Ceus de la partie 
   Pelyarmenus avoient oï et l’un et l’autre son. Si connurent les ii parties et sorent 
   bien que le grosgros son iert de la partie 
   Dorus et li autre d’Abylon. Lors se mistrent vers le son 
   Abylon moult esforcieement, et li jours conmença a aparoir. Si se sont moult efforcié de venir a 
   luil'un, mes ce leur fu contraire que li dui se sont entraprochié, 
   car il venoient cele part ou il 
   avoientavoient premierement oï 
   Abylon, si que par ceste reson s’eslongnoientesloignierent 
      cil d’Abylon et les autres se sont entraprochié. 
   Si avint quequant l’une partie et l’autre orent tant alé que 
   Hayno a resonné ii mos, et Dorus enquist a 
   Abylon que il ne li celast pas se il connoissoit 
      cele son. 
   Cil dist que nennil.
   Et tout aussi ne le connois je pas, dist Dorus. \pend
\pstart Dont s’apenssa Abylon d’un grant sens et d’une parfaite cortoisie et 
   dist :
   Sire, moult plus de gent mevous 
      connoissent que il ne font vousmoi. Et d’autre part, 
      je me puis miex moustrer a eulz sanz peril que vous ne faitespoez. Si 
      demourrez ci endroit tant que je aie veu qui il sont.Abylon, dist Dorus, 
      moult faites pour moi !
   Atant se mist Abylon devers le son Hayno, 
   mes il estoit encore moult matin. Si n’arresta tant qu’il vit ceusceus venir 
   parmi une bele lande. Si li fu avis et bien li sembloit 
   que ce fussent cil qui contre eulz avoient esté, ainsi conme vous avez oÿ. 
   Mes toutes voies ne savoientsot il que pensser se ce furent il ou non, 
   car tant estoient leur cotes harigotees et despeciees qu’il n’i 
   avoitpooit connoistre se pou non de connoissance 
   par coi on les peust connoistre. Mais nonpourquant vint il vers eulz, 
   aussi conme se il cuidassent :
   Je ne vueil pas que vous cuidiez je ne vueil autre chose que bien.
   Cil de l’autre part virent Abylon venir, si ne se vodrent pas mettre contre lui pource 
   que il fu tout seul. Mais il sorent bien que ce fu cilcil tot seul qui 
   avoit corné, conme cil qui encore tenoit le cor en sa main. Et quant il vint a eulz, si les a saluez, mais il ne li rendirent pas son 
   salu, ainz parla cil Hayno et dist :
   Quelles novelles, biau sire ? \pend
\pstart Abylon, qui maintenant connut leur volenté, 
   dist aussi conme pour eulz tempter :
   Sire, je ne cuide savoir autres nouvelles que, se vous 
      les saviez, que vous en 
      deussiezdeussiez ja pis valoir.De ce poez vous estre liez, dist Hayno. Or ça ! Dites nous queles 
      nouvelles vous savez par quoi nostrevostre vie puist estre sauvee.Et conment, biau sire, queles nouvelles 
      devezvolez vous oÿr ? 
   Sachiez que je voudroie mourir en la piece de terre avant que je ne vous feusse aidant a vostre honneur sauver !C’est faus ! dist Hayno. 
   Car tu es des chevaliers Pelyarmenus.Autresi estes vous, dist Abilon, a mon avis.Tu mens ! dist Hayno. Ja verras tu conment il est amez de nous !
   Lors vint vers lui et l’eust occis quant il dist :
   Vassal, ne me touchieztouches tant que 
      vous sachiezsacheztu saches miex qui je sui.Or di donc ce que tu veuls dire, car nous sonmes chevaliers au preus 
      Helcanus et alons querant son frere, Dorus. Et se il estoit 
      ainsi que tu nous seusses dire bonnes nouvelles, ta mort seroit respitiee ; et se ce non, il te couvient tout maintenant morir.Biaus seingneurs, dit il, ja orrez telz nouvelles conme je vous mosterrai.
   Atant a mis son cor a sa boche et a sonné ii mos aussi conme pour Dorus apeler, 
   mes il n’ot pooir de ouirvenirouir le son, 
   ainsi com je 
      le truis el conte. \pend
\pstart Vous avez bien entendu, ainsi conme je 
   vous aiavoie 
   dit desus, 
   conment les vi chevalierschevaliers qui avoient entendu 
   Abylon et venoient au son du moyennel, et tant qu’il choisirent la trace de leurs chevaus, ainsi 
   conme il chevauchoientmarchoient par la forest, car il estoit ja jour 
   par quoi il aperçurent qu’ilil y avoient nouvellement passé. Si se mistrent 
   aprés eulz a esperon brochant et se hasterent tant 
   qu’il troverent Dorus qui atendoit Abylon, 
   si conme j’ai dit desus. Quant il ontil l'ont 
   choisiveu seul, si 
   ont veuchoisirent le cheval 
   PelyarmenusPelyarmenus, lor seignor. Il vindrent vers lui, 
   et il les vit venir, 
   si cuida maintenant que ce fussent de ceus qui eussent sonné. Et que fist il ? Il sot bien que ce n’estoit pas des siens, 
   si vint conme senglier vers eulz, et il l'ontil ont receu 
   conme senglier devant et derrieres et encoste 
   neismais cilcil en 
   quiqui il avoit cuer et hardement et valeur ne leur volt pas eschaper, 
   ainz s’est a eulz acointiez en tele maniere que il enles mist 
   a terreterre jus .III. et iii, dont le plus hestié n’ot pooir 
   que il montast desusen son cheval. Et les autres iii 
   s’efforcierent a ce que il le cuidierent mettre a merci sanz mettre a mort le cheval sus quoi il seoit. Mes cil, qui avoit esté en 
   maint autre peril, s’efforcaesvertua en tele maniere que maint aineus cox 
   recut et donna, si que en la fin occist le quart. Et lors virent li autre dui que il n’i avroient duree, si se retrairent 
   arriere petit et petit et tornerent en fuie, li i ça, li autre la. \pend
\pstart Quant Dorus vit que il jouoient de tel gieu, 
   si ne volt pas faillir a l’un. IlLors se mist aprés 
   le miex monté 
   et se mist a la voie de merveilleuse vertu, et cil s’embati aussi com par aventure en i chemin par la forest qui estoit et 
   grant et large et bien couvenable a ce que il ne grevoit riens as chevaus 
   qu’il eussentil n'eussent lor cors a lor volenté. 
   Cil qui devant s’en aloit ne s’aseure pas de 
   celui qui le chaçoit, l’espee ou poing toute senglente du sanc de ses compaingnons qu’il avoit 
   lessiez, ainsi conme vous avez ouy. 
   Dorus le huchoit de fois en autres :
   Retourne toi, fuiant a honte, et si reviens derriere 
      a ton honneur !
   Cil qui n’avoit cure de son sermon aloit touzjours devant lui, 
   conme cil qui avoit cheval a sa volenté. Et le Dorus ne fu pas a 
      cece tour tourné, car trop avoit 
      esté grevé. Si le conmença a eslongniereslongnier Dorus, 
   dont Dorusil issoit a bien pou 
   de sondu sens. Mes il ne volt lessier, conment que il alast, que touzjours 
   n’alast aprés. Dont i li avint de ceste chose une aventure 
   telle conme je vous dirai. Cil qui ainsi s’en aloit devant 
   vit un chastel ou costé d’une montaingne, aussi conme dehors le chemin. Il entra en une voie 
   de cel assens et tourna cele part, et Dorus 
   aprésaprés tous jours. 
   Si leur avint que, quant il orent celuile 
   chastel aprouchié, que 
   une pucele iert aus fenestres de la haute tour, 
   qui vit le chevalier venir fuiant aussi conme vous 
      avez oïoez. 
   CeleCele pucele, 
   de si loing conmeque 
   elle le vit venir, 
      lile hucha en haut :
   Ha ! chevalier qui mestier as d’aide, viens a moi a garant, 
      que ja cil qui te chace ne sera ja si hardi qu’il te face ceens nul mal !
   Ceste parolle oï li uns et li autres, car celece si ot une si haute vois 
   que bien la pot on oïr de loing. \pend
\pstart Quant Dorus 
   ot oÿ et veu 
   la pucele, si sot bien que 
   li chevaliers iert venus a garant. 
   Si ne sot que faire de plus avant enchaucier. Lors sacha son frain et volt retourner quant cele hucha en haut :
   Et conment, sire chevalier ! Avant vous couvient venir por avoir droit du 
      chevalier, se il vous a aucune chose mesfet ou vous a lui.
   Adonc ne pot Dorus retourner por nule honneur, ainz s’en vint vers 
   le chastel aussi conme en doubte. 
   Cil qui devant vint ne sot si tost venir 
   que il ne trouvast la pucele enmi la court, et avec lui mainte belle damoiselle. 
   Cil descendi a si grant haste de son cheval et vint a la pucelle et li dist :
   Damoisele, je vieng a vous a garant, car vez ci Dorus de 
      Costentinoble, qui m’ocirra se vous ne me garantisiez contre lui. \pend
\pstart Pas de nouveau § BQuant 
   la pucele entendi cest afaire, 
   si ne fu onques mes si liee dame ne damoisele, et ne pot respondre de joie, ainz s’en vint encontre 
   DorusDorus de joie, qu’ele vit entrer 
   en la porte, et li dist si haut conme ele pot :
   Biau sire Dorus, je vous desirroie a veoir ceens ! 
      Or ai je tant atendu que aventure vous y a amené. Et de ce loe je Celui qui 
      nousmoy et vous a fet 
      naistrenaistre d'umaine lignie.
   En ceste parolle descendi Dorus de son cheval au plus tost qu’il pot 
   et osta son hyaume de son chief et vint a la damoiselle, en l’une main son hyaume 
   et en l’autre s’espee toute nue, et li dist :
   Damoiselle, je vous salu. Et ne me sachiez nul mal gré du 
      chevalier se je le chaçoie, 
   car il m’eust pieça afolé ou mort se je ne me fusse deffendu. Mes il m’est avis, puisqu’il est venus a garant, 
   que bien le poezavez pooir du 
   garentir.Sire, dist elle, ainz vous ferai avoir vostre droit de lui quant j’en savré la verité. \pend
\pstart Damoiselle, dist Dorus, 
   or ne vous anuit pas, car par lesir le vous pourrai je bien dire ou le chevalier 
   qui ci vous demourra, 
   car aler me convient a la bataille ou je lessai mon frere des ier entour heure de nonne.Sire, distce dist ele, puisque je vous tieng ceens,
      vous n’en avez pooir du departir sanz mon gré !
   Atant conmanda la pucele le pont a lever et conmanda 
   a chascun de touz ceus de leenz 
   sus quanque ilchascun pooitpuet 
   meffaire que nul n’en issist 
   horsde laiens ne 
   n’entrast se ele ne le conmandoit. \pend
\pstart Pas de nouveau § BQuant Dorus entendi ce, 
   si fu tout esbahy et dist :
   Ha ! franche pucele, gardez que vous ferez, car se ne pourroit estre pour 
      tottot l'onneur de l’empire 
      que vousvous ceens me retenissiez outre mon gré !Sire, dist ele, n’aiez doubte quecar 
      vostre frere a la bataille vaincue 
      et mis au desousdesous de 
      sestous ses anemis.
   Quant Dorus oï ce, si cuida que elle deist voir et fu mout joians et dist :
   Ma douce damoiselle, dont poez vous fere de moi vostre volenté.Si ferai je, dist elle. Atant prist 
   la pucele Dorus et l’enmena ou palais amont et 
   d’iluec en sa chambre, qui estoit si belle et si noble que il fu avis a Dorus que 
   il en jour de sa vie n’eust esté en si biau lieu. Iluec le fist 
   la pucelle desarmer, vousist ou non, et le trouva 
   moultmoult bleciez et moult navré en plusseurs lieus. 
   Si dist laHa ! pucele 
   pour lui plus decevoir :
   Biau sire Dorus, ne cuidiez pas que, se Dieu ne vous eust ceens 
      amené, que vous ne feussiez mors. De telz plaies truis je sus vous, se vous eussiez 
      plus atargié i jour.Damoisele, dist il, dont m’amena Nostre Sire ceens.Voir avez dit, dist ele. Et puis a dit em bas, si que 
   nulsil ne l’oÿ :
   Ce ne direz vous pas avant que vous m’eschapez !
   Lors conmanda a une seue sereur qui estoit 
      mainsnee de lui qu’elle li aportast un ongnement que ele li avoit baillié l’autre jour. 
   La pucelette entendi sa sereur, 
   si fu moult tristre et moult dolente, mes elle n’osa contredire ce que ele 
   demandali commanda, ainz vint en une garde robe, 
   si ouvri i escrin, si en traïst hors une 
   boistebeste de voirre ou il ot pendu i brief qui disoit :
   « Ce est ongnement a ami ». 
   Ha ! fausse mauvesse, dist elle, pourcoi voulez vous fere le meilleur 
      chevalier du monde et le plus redoutez vivre en languissant ? 
      JaJeEn n’a il courpes en la mort 
      mon aÿeul, ainz est grant traÿson que vous voulez faire. Et ja Dieu ne m’eïst quant pour moi avra 
      ja mal tant que je l’en puisse garder !
   Et lors prist le brievet et le destacha, puis prist une autre boiste la 
   ou il avoit i brievet pendu qui disoit : « Icest ongnement est 
   a compaingnon ». Lors a ces ii briez changiez et 
   transporteztransposez 
   de l’une boiste en l’autre, car pour la reson de ce que les boistes estoient semblables i furent les briez mis. \pend
\pstart Dont revint la pucele arriere et aporta la boiste 
   et la bailla a sa suer, qui la pristprist et 
   et la moustra a Dorus et dist :
   Or esgardez a cest escript et veez que il a dedenz.
   Dont vit Dorus que il ot ou brief escript : 
   Cest ongnementong-ongnement a ami. Si dist :
   Damoiselle, je voi bien que il me covient de vous faire ma grant amie.Sire, dist elle, pour ce vueil je que vous m’enme couvenanciez avant 
      que je ien vous mette cure que vous ne vous departirez de moi sanz mon 
      congié.Damoiselle, dist il, mes que vostre congié ne me tourne a confusion, 
      et je le vous ai en couvent.Sire, dist elle, vous tendrez a confusion ce qu’il vous plera, mes vous m’en couvenancerez ma 
      requeste. Et ce se non, je ne vous prendré pas en cure. Et d’autre part, vous ne vous
      poezavez pooir de partir de 
      moiceeens sanz mon gré.Damoisele, dist il, et je l’otroi.
   Atant reversa la pucele 
   sesles plaies d’une part et d’autre et li oint 
   si bien conme il couvint, puis referma la boiste et dist :
   Sire, or soiez asseur, car avant tiers jour savrez vous se 
      jeje vous sui amie.
   Lors rebailla la boiste a sa sereur et li dist :
   Alez et si m’aportez l’autre. Et nonpourquant, metez ça ceste et si 
      m'aportezm'aportezde l’autre, 
      que vous trouverez semblable a 
      cesteceste boiste, car li dui chevalier ne sont pas touz 
   d’une complexion, si leur couvient avoir divers ongnemens.
   Cele s’en revint en la garde robe, moult corroucieedolente, et li couvint 
   prendre la boiste que elle avoit changiee, vousist ou non, et vint devant sa sereur 
   et li bailla.
   Sire, dist elle a l’autre chevalier, 
      savez vous lire ? Tenez et regardez qu’il a ici escript.
   Cist regarda le brief et distvit qu’il y avoit escript : 
   Ce est ongnement a compaingnon. Et lors li dist cilz :
   Damoiselle, moult vous doi amer et faire pour vous se mestier en aviez.Mestier en ai je, dist elle, car il vous couvendra fere ma requeste, 
      se vous voulez que je mette cure en vous.Et je lela ferai, dist 
      ilele, carque 
      je le doi bien faire.
   Lors li a cele ses plaies ointes au miex que ele pot et puis revint a sa sereur et li dist : 
   Va et reporte ces boistes en sauf.
   Et elle si fist et remist arrieres les boistesbrievez 
      si conme ellesil devoient aler. Lors revint arriere 
   et trova que les chevaliers estoient vestus de robes a leur mesure. Dont s’en vint a Dorus et li dist, 
   si que sa sereur ne s’en apercut pas :
   Sire, je vous pri que vous faciez la plus mate chiere que vous 
      pourrezpourrez par reson, 
   car je vous en dirai bien encore anuit le pourquoi.
   Lors ot Dorus merveille de ceste chose, mes il ne volt pas lessier que il ne feist le conmandement de 
   la pucelette, que il vit moult bele et moult gente 
   a touz endroisen tous drois, 
   et sa suer parloit a 
   l’autre chevalier, qui fesoit moult mate chiere, mes il se reprist 
   du plusde ce que il pot pour 
   la pucelle, qui voloit savoir 
   conment n’en quele maniere Dorus s’estoit a lui pris. 
   Cil li respondi et dist :
   Ma damoiselle, il n’afiert pas ore que vous le sachiez de par moi, car vous le savrez 
      bienmiex de Dorus, qui ceens vous 
      demourra. Si l’en metez a raison.Si ferai jevueil je faire, dist elle. 
   Lors les mist entreus ii ensemble a reson et leur dist :
   Il me couvient savoir l’aventure de vous ii conment vous estes ainsi ceens ensemble.Par ma foy, dist dont Dorus, de ma partie vueil je bien que 
      vous le sachiez.
   Lors li conta tout ainsi conme il li estoitfu 
      avenu de la compaingnie Abylon et conment 
      il avoitl'avoient 
      esté trouvé seul, ainsi conme 
         je aiil a esté dit 
         ou contedessus. 
   L’autre chevalier respondi ainsi conme je ai desus dit 
   conment il estoient venus a la vois du cors qu’il avoient 
      oÿoÿ et suivi tant que il 
      lala le trouverent. \pend
\pstart Pas de nouveau § BQuant Dorus ot celui oÿ, si fu moult joiant pource qu’il n’ot 
   plus Abylon en soupeçon, car devantdevant ce 
   avoit il doutéeu doute que il ne le vousist traÿr. Si dist :
   Jamés n’avrai joie jusques adonc que je savrai nouvelles 
      de lui, quecar il estdu 
      loial chevalier.Par foi, dist il, loial chevalier est il voirement.En non Dieu, dist la pucele, je le vueil veoir.
   Lors conmanda que les tables fussent mises, et on si fist. Si ont beu et mengié tout a leur volenté. 
   Et aprés mist la pucelle 
   le chevalier a raison et li dist :
   Il vous couvient aler en la queste d’Abylon, et l’amenez 
      a mon seingneur, qui ci est, et li distes 
      qu’il n’a pooir de soi partir de ceens jusques a tant qu’il i vendra. \pend
\pstart Dorus, 
   li hardis chevaliers, 
   si respondi a ce et dist :
   Ha ! franche damoiselle, conment porroit a ce venir ? Car espoir ne pourroit il estre trouvé. 
   Et ainsint cuideroient mes amis que je feusse mort et perdu, conme celui qui n’avroit riens valu en cest afaire.Sire, dit elle, souffrez vous, car, s’il ne vient anuit ou le matin, 
      aler vous em pourrez quele part qu’il vous plera.
   Lors ne li valut riens chose que il peust dire. 
   SiLors prist la pucele 
   le chevalier et le fist armer moult tost et li demanda :
   Sire, conment pourrons nous esploitier de Dorus, que je tieng ceens, 
      quicar jeje ne 
      vueil que jamés ne s’ense 
      puisse partir sain ne hestiéhestié de ceens, conme cil qui 
      sa mere fist 
      mon aÿel destruiremourir ? Mes pour ce ne vueil je pas 
      que l’en sache qu’il soit ceensceens entrez, car la chose si pourroit 
      si aler que, se l’en lile savoit, que je ne le voudroie pas. 
      Et pource que je vueil la chose desvoier, vous emporterez sa cote a armer et la metrez en aucun lieu ou il ait eu mortalité de 
      gentchevaliers toute despanee et mal atyree, par quoi l’en 
      cuideroitcuidecuidoit qu’il ait esté entre ceus qui 
      l’aient ainsi atyrié.On perçoit une fois encore l'importance de la lignée dans le cycle,
      les choix et actions des parents venant influencer le sort des descendants, dans les décisions des alliés dans les guerres 
      fratricides, et encore ici dans l'enfermement de Dorus par Cyboule. \pend
\pstart Pas de nouveau § BAinsi conme cele le devisa le fist 
   le chevalier 
   en tele maniere que, quant il ce fu partis du chastel et mis en son chemin arriere, 
   aussi conme il li plut miex, la pucelle fist une 
   merveilleuse chosechose
   dont vous n'orrez jamais parler de tele, car ele fistfist maintenant 
   la pierre forbatre du grant chemin qui venoit a son chastel et planter de bois, 
   si que par nule aventure yne peust jamés revenir que cil y peust rassener 
   qui a telle heure s’en estoit partis que il morut dedenz le tiers jour aprés.La portée inquiétante 
   de ce personnage est ainsi encore réactivée par les pouvoirs magiques qui lui sont prêtés.
   Mes li contes repaire 
      cici endroit a Abylon, qui avoit 
   sonné le cor, si conme vous avez oÿ devant, pour Dorus apeler, qui en tele 
   maniere se deffendoit, conme desus estil a esté dit. \pend
         
         
            Ci devise 
               conment Abylon 
               perdi Dorus et ainsint que Hayno
               le retint por savoir qu’il estoit devenus, 
               et devise conment la damoiselle 
               avoitl'avoit detenu 
               Dorus 
               en son chastel.
            
               Enluminure sur 1 colonne (e) et 12 UR.
                  Dorus mené par la main par une
                  damoiselle dans 
                  son château 
                  pour l’y détenir. 
               
            
\pstart 
   Ci endroit dist li contes que, quant 
   Abylon vit 
   queque quant-Abylon vit que 
   Dorus ne venoit, si en ot moult grant merveille, 
   et tout aussi ot Hayno et ses compaingnons, qui ont maintenant cuidié que cil n’apelast 
   gent fors que pour eulz mal mettre et trayir. Si ont maintenant choisi l’un de ceus qui demorez estoit des vi chevaliers 
   dont Dorus chaçoit l’autre, 
   ainsi conme vous avez oÿ. Cil, qui tout estoit espoentez de la doubte que il avoit que 
   Dorus ne le suivist, en vint par sa bonne aventure fuiant entre ceus qui iluec estoient. 
   Si l’ont saisisaisi tout maintenant de toutes pars et li ont enquis avant 
   que il liil feissent pis qu’il avoit, qui si fuioit 
   etet si nulz ne le chaçoit.
   Ha ! biaus seingneurs, pour Dieu merci, j’ai eu doubte et paour 
      d’undu 
      chevalier qui nous vi a mis a destruicion qu’il n’en y a nul eschapé fors 
      moique moi.Conment ! dist Hayno. Di nous en quel lieu tu
      le lessas 
      celui de qui tu parolles et se tu sez qui il est.Ha ! biau seingneurs, dist Abylon, 
      ce est Dorus, que je vous ai apelé. Et je cuidoie qu’il deust estre venus a mon apel. \pend
\pstart Pas de nouveau § BQuant Hayno et ses 
   compaingnons ont celui entendu, si furent tuit esbahi, 
   si sont venus cele part ou Abylon avoit lessié Dorus et 
   ont trouvé le martyre qu’il avoit fet, ainsi conme j’ai desus dit. Ceste aventure fu auques merveilleuse 
   aus compaingnons, si ne sorent a quoi entendre pour savoir premierement nouvelles de leur seingneur, 
   tant que Abylon les apela et dist :
   Ne cuidiez pas, biaus seingneurs, que je ne vueille mourir entour vous, 
   mes que vous sachiez avant ce que je vous conterai, conment n’en quele maniere je sui venus a vous, ainsi conme vous avez veu.Il n’est pas temps, dist Hayno, de lonc conte oïr, ainz vueil aler 
      avecaprés 
      celui qui n’a guieres d’amis avec lui, ainsi conme j’entens. 
      EtEt vous biaus seingneurs, soiez sesis d’eulz tant que vous aiez oï 
      autres nouvelles de ce que vousnous 
         alezalons querant. \pend
            \pstart Atant se parti d’eulz Hayno, lui tiers, et cil sont iluec demouré 
   qui ne vodrent oïr Abylon, qui tout ainsi leur 
   conta conme il a esté ditcontenu devant ou conte, 
   conment il s’estoitestoit premierement 
      acointié de Dorus et li avoit tenu compaingnie jusqu’a tant qu’il estoit venus a eulz. 
   Quant il ont entendu Abylon, si n’i ot nul quil ne deist :
   Voirement se moustrent li bon gentil cuer 
      oupartout en quel point que il 
      soientsoient de franchise envers lor seigneur !
   Lors li offrirent touz honneur et servise pour l’amour de leur seingneur, 
   que il moult devroitdevroient 
   estre son anemy qui vilonnie li feroit. Et a 
   l’autre chevalierchevalier 
   qui s'estoit embatus sor eulz 
   ont il enquis conment il s’estoient pris a Dorus en tele maniere. Il dist :
   Biau seigneurs, pour Dieu merci, chascun doute la mort, et la doit on fouir tant conme l’em puet. 
      Voirs est que je sui eschapez de Dorus, lese 
      chevalier doubté, et m’en ving a garant a vous pource que je cuidai que vous feussiez de nostre gent. Or est ainsi que je me sui 
      embatus entre vos mains, si poez de moi fere vostre volenté, car voirs est que entre vous vi chevaliers a la vois 
      et au son de vostre compaingnon qui ci est nous meismes a la trace 
         de lui et de son compaingnondu cor. Et quant nous eusmes tant suivi que nous trouvasmes 
      le chevalier douté, il nous 
      courutcouvit (sic) 
      ensus en 
      autelle maniere que, se nous ne feussions que nous dui, si nous fu avis qu’il n’avroit vers nous duree. Si nous combatismes tant 
      que les iiii des nostresde nous si furent mors avant que moi 
      ne mon compaingnon departissions de la bataille, ainsint conme vous poez veoir. \pend
\pstart Quant ceus orent celui entendu, si distrent :
   Sire, por l’amour de ce que vous venistes a la vois de 
      vostre compaingnon qui ci est voulons nous ore que il soit 
      nostrevostre garant. Et pour ce que vous dites que 
      le chevalier douté 
   vous couru sus premierement que vous a lui ne fu ce pas merveilles se vous vous meistes a desfense.
   Ainsi eschappa cilz que il n’ot garde de mort, ainz li distrent que il alast quel part que il 
      vousistvorroit, et tot autresi distrent il a 
   Abylon fors que il leur die son non avant 
   que ilque 
   de eulz se departe pour l’amour et pour la courtoisie qu’il avoit fait a leur seingneur. 
   Lors leur dist a touz que il avoit non 
      Abylon pource que son pere avoit esté ou chastel d’Abylon.
   Abylon, dist chascuns, alez quel part qu’il vous plaira, 
   car de nous n’avez vous garde.Seingneurs, dist il, grans mercis. Mes sachiez que je de ci ne me partirai 
      jamés jusqu’a tant que je du chevalier douté 
      savrai autres nouvelles, pource que je ne vueil pas qu’il me puisse reter 
      du fort encombrier qu’il a eu ne 
      de nule vilenie. \pend
\pstart Pas de nouveau § BEn ces parolles ont entendu grans sons de 
   cornes sarrazinois et de trompes et de buisinnes, 
   et leur fu avis que leurs gens fussent auques pres d’eulz et vousissent assembler a leur adverse partie. Dont dist li uns d’eulz :
   Nous sonmessonmes aussi comme cil qui 
      atendent l’abee. Nous ne savons nouvelles de vostre gent autrement que 
      ce chevalier nous a conté une grant courtoisie qu’il doit avoir 
         faite a Dorus. Et se ce est voir, si s’en voise aprés lui et face ce qu’il li plera. 
      Se il treuvetreuvent nos compaingnons qui de ci se partent, si leur die 
      que nous sonmes reperié a nostre gent, qui sont assemblé a noslor anemis.Ainsint, dist Abylon, le vueil je fere.
   Atant se sont d’iluec departidepartistes, si c’est misse mist 
   Abylon a la trace de 
   Dorus. 
   Mes atant lesse a aparler d’eulz et 
      retournerepaire ci 
      li contes a 
   Mardocheus, qui s’estoit mis aprés Dorus, 
   ainsi conme j’aiest devant 
      ditdit el conte. \pend
         
Ci vient li contes a Mardocheus, si conme il se 
   mistmist a aler aprés Dorus aprés ce que 
   l’en sot que Helcanus fu 
   morsmors. Comme s'ensuit.

   Enluminure sur 1 colonne (c) et 12 UR.
      Mardocheus et un compagnon à cheval dans la forêt à 
      la recherche de Dorus à la mort de Helcanus.
   

\pstart Ci endroit dist li contes, si conme vous avez oï devant, que 
   Mardocheus s’estoit mis les esclos aprés Dorus, 
   si que il chevaucha tanttant que il et sa gent qu’il encontrerent aucuns 
   qui estoient retournez de la chace pource que ilqui ne porent pas suivre 
   Dorus pour lor chevaus, quiqui lor 
   estoient estanchiez, si qu’il vindrent a lui et li ont dit que 
   moult folement enchauçoit Dorus et a pou de gent, 
      et d’autre part s’estoit grant folie que il s’entrebatoient entre ces montaingnes, 
      dont il ne pourroient retourner fors parmi leurs anemis. 
   Lors s’arresta Mardocheus, et li vindrent nouvelles que 
   li rois d’Arragon 
   le faisoit suivresuite 
   aprés lui. Adont prist Mardocheus xx chevaliers et leur conmanda que 
   il alassent aprés Dorus 
      etet lor commanda qu’il ne reperassent 
      devant ce qu’il oïssenttant que il savroient de lui novelles, queles 
      que elles fussent. Cil ont dit que ainsint le 
         feroientferont il. Et 
   Mardocheus se tint d’autre part a ce 
   qu’il savroit volentiers la volenté du 
   roy d’Arragon. Lors se mist au repaire, lui 
   tiers de chevaliers, si perça toute leur gent et la fist tenir coie tant qu’il 
   eust parlé a l’arriere garde, et il ainsint l’ont fait. \pend
\pstart Pas de nouveau § BDont ne demoura pas que 
   Mardocheus ne 
   venistvint au 
   duc de Nice, qui estoit son pere, et meismes a 
   Japhus le Frison, car il dui fesoient l’arriere garde, qui furent moult 
   joiant et moult lié quant i l’ont veu. Lors li dist 
   le duc, son pere :
   Biau filz, de sagement esploitier d’ore en avant ne seroit 
      ce pas folie ?Non, sire, dist il. Et qui savroit bon conseil donner, si le donnast !Je le donrai, dist le duc. 
   Car vez ci le roy d’Arragon qui nous suit. 
   Et quant il verra que nous serons entrez en ces montaingnes, il se ferra en la queue de nostre gent, si n’avrons pooir d’eulz aidier. 
   Si avrons ainsi perdu nostre honnor et lale vostre.
   Lors respondi Japhus le Frison et dist :
   Conment ! N’est il nulz qui sache que Dorus 
      soitest devenus ?Nennil, dist Mardocheus, 
   fors qu’il se mist a aler aprés Pelyarmenus, si qu’il n’est nul qui m’en sache dire autre chose 
   fors que je ai envoié xx chevaliers en la queste.Bien avez fait, dist chascuns. 
   Nouveau § BAdonc orent conseil qu’il feroient leur gent retraire en lieu ou il se pourroient deffendre 
   et faire aide les uns aus autres se mestier leur couroit seureseure. Et adont
   ont fait sonner la retraite. Si fist aussi 
   li roys 
      d’Arragon de l’autre part sa gent arrester tant que il savroit a quoi il beoient, car il vit bien que a ce qu’il 
   cuidoit fere avoit il failli. \pend
\pstart Pas de nouveau § BEn ce qu’il estoient en ceste chose vint 
   li rois de Sezille et 
   le duc de Puille au 
   roy d’Arragon et li distrent :
   Biau sire roys, nous sonmes venus en l’aide de 
   l’emperere de Rome, nostre seingneur, 
   et si ne savons nulles nouvelles de lui n’en quel point il est. 
   De l’empereeur de Constentinonble cuidons nous bien 
   qu’il soit mors, dont s’est pitié et duel a touz ses amis, et moult est laide chose d’amy et meismement de frere a autre d’entreprendre 
   tel chose conme de l’un et de l’autre occirre et mettre en prisonmort et 
      du tout mettre au desous. Encore y a autre chose, car maint 
      preudonmepreudonme sont qui aident a leur seingneur lor droit a 
      maintenir, et encore plus, que il mettent leurs cors et leurs avoirs en 
      aventure pour acomplir leur volenté, quele qu’ele soit. Et mal aient tout cil 
      qui pour acomplir leur male volenté font a preudonme vilonnie ! \pend
\pstart Pas de nouveau § BQuant il orent ce dist, si distrent :
         
      Or, lero]‹y›[L]e 
      faisons bien, etUne croix, de la main du copiste, est inscrite à cet endroit. ### 
      si metons nostre gent a ce que nous envoions a l’autre partie pour avoir i pou de respit, tant que 
      Pelyarmenus soit trouvez et que 
      l’en sache qu’ilPelyarmenus 
      soitest devenus, et puis si le metons a ce que reson si soit faite a 
      ce chietif pueple qui ci est assemblé d’une partie et d’autre qui se veult entrocirre, et 
      estest fait pour l’orgueil et pour l’envie d’aucun felon ou il n’avra 
      jamésjamés faite 
      droiture nedroite verité a son vivant. \pend
\pstart Quant le roy d’Arragon 
   oïvit ce, si vit bien qu’il avoit reson en ce qu’il disoient. Si 
   ne volt ne n’osa aler a l’encontre. Lors ont eu conseil qu’il 
   envoieroient a l’autre partie pour ceste chose. Lors fu pris respit les uns contre les autres jusques a iii jours, 
   et d’iluec en avantavant le prendroient plus lonc, se mestier en 
   avoientestoit. Lors avint que en cestui point vindrent les chevaliers 
   qui partis se furent d’Abylon, 
   conme j’ai devant dit, et ont fet asavoir 
   a touz lesses se 
   grans barons de l’ost l’aventure qu’il avoient entendueattendue 
   de celui Abylon, conment il avoit tenu compaingnie a 
   Dorus. Quant il orentoïrent ce 
   oui, si orent conseil et 
   orent grant merveille qu’il ne plaisoit a Dieu que il fust reperiez vers eus. Que vous 
      iroiediroie je 
      plus le conte alongnant ? Car je y metroie trop se de toutes les aventures qui 
         avindrent vouloie fere mencion. Si vueil a ce venir que leur conseil fu tel d’une part 
   qu’il se mistrent en la queste les plusseurs 
   de trouver Pelyarmenus et Dorus. 
   Si nous en fet li contes tant de mencion que il dit que 
   lesdes xx chevaliers que 
   Mardocheus envoia en la queste de Dorus trouverent toutes 
   les aventures dont li contes fet ci devant mencion 
      en tele maniereaussi conme je vous dirai. \pend
\pstart Pas de nouveau § B
   VeritéVoir fu que, entre toutes les aventures, 
   il trouverent le lieu et le cheval mort, 
   dont Dorus avoit de sala selle copees 
   les cengles, et la connurent il le cheval et cuidierent tout vraiement que la eust il esté mort ou 
      mené em prison. Aprés ce 
   ontil ont tant suivi la trace que il vindrent en 
   l’abbaÿe. Et quant il i furent venus, si ont enquis et demandé nouvelles de celui,
   et il leur en fu dite toute la verité, mes il n’en vodrent riens croire, si qu’il escovint a l’abbé 
   que il en demourast en grant anuy pource que il avoit la esté, et nul n’i avoit qui puis seussent nouvelles de lui. 
   LorsLors il avint que cil
   se mistrent arrieres en la forest et s’en partirent de lala et alerent 
   ainsi conme il entendoiententendirent qu’il 
   s’estoientestoit mis a la voie 
   entre lui et Abylon. Si ont tant alé et sus et jus qu’il encontrerent 
   Hayno, lui tiers, tout ainsint conme j’ai dit desus qu’il 
   s’estoientestoit mis 
   aprés Dorus. Cilz Hayno, quant il vit ceus, si osta son hyaume et 
   vint a eulz et lessaluaet les salua et ce fist connoistre 
   et enquist de leurs nouvelles, etmais pou en porent 
   savoirdire de bonnes. Et nonporquant li ont dit ce qu’i volt savoir d’eulz. 
   Et il d’autre part leurle leur dit ce qu’il avoit 
      de Dorus entenduoÿ. \pend
\pstart Pas de nouveau § BDe leurs parolles ne fet ci endroit 
   li contes plus de mencion, fors tant qu’il se sont d’iluec departis touz ensemble et alerent puis tant sus et jus qu’il 
   trouverent la cotte Dorus a armer entre un grant abateis de chevaliers. Si ont cuidié vraiement que 
   la eusteust il esté mort et trais 
      hors des armes. Si ne fu onques veuveu nul si grant duel conme 
   l’en peut la veoir. Et quant cil orent leur duel demené ainsi conme cil qui ne se porent reconforter, si se sont mis au 
   cheminretour vers leur gent et ont tant alé deça et dela que il sont 
   repairié. Et quant l’en leur demanda qu’il avoient trouvé, si distrent la verité. \pend
\pstart Quant li baron ont ce oÿ, si ne fu onques mes si grant destourbier veu conme il orent, et encore eust 
   il esté greingneur ce ne fust ce qu’il ont conmande com se sueffre pour lel'ont celé au 
   conmun. Et de l’autre part, Pelyarmenus fu tant quis et d’une part et d’autre qu’il fu trouvé en 
   l’abaÿe ou il estoit, ainsi conme j’ai desus dit. 
   Lors li fu contee la mort de son frere Helcanus, et de Dorus 
   n’estoit il nuls qui riens en seust. Lors fu li deables 
   Pelyarmenus si joians de la mort 
   son frere Helcanus et de 
      Dorus, qui pardonnoit touz mautalens a touz ceus qui 
   riens 
   li avoient forfet, et tant que il se parti en cele heurejoie de 
   l’abbaÿe et vint arriere a sa gent. Mes il fu tel mené du travail qu’il avoit eu du 
   chevauchierchevalier en ce qu’il avoit esté navrez ou chief que il plut a 
   Nostre Seingnor, aussi com par venjanceaventure, qu’il issi du sens, et le couvint lier aussi 
   conmeconme .I. desvé. Ceste aventure fu seue des 
   noblesnobles et barons d’une partie 
   et d’autre, si orent moult grant doute chascun de soi, car il sorent vraiement que ce fu 
   miracle etUne croix, de la main du copiste, est inscrite à cet endroit. ###tabourvengance 
   La leçon de V3 et G paraît plus obscure : le tabour, l'instrument ou le vacarme (DMF), semble difficilement
   associable au miracle divin avec lequel il est présenté. Plus encore, la croix semblerait, comme plus haut, indiquer l'erreur commise.de 
   Nostre Seingneur, dont il avint aussi conme je vous conterai a pou de parolles que 
   le roy d’Arragon et 
   le roy de Sezille et 
   le duc de Puille s’apensserent 
   qu’il s’acorderoient aus barons de l’autre partie en tele maniere que il se retrairoient chascun 
      en sa contree et feroient i parlement de pespes prendre 
      li uns d’eulz contre i autre de l’autre partie a jour couvenable, 
   et dedens ce avenroit aucune chose qui pourroit valoir a l’un ou a l’autre.De manière significative, 
   la paix est aussitôt possible que Pelyarmenus se retrouve hors-course. \pend
\pstart Pas de nouveau § B
   ToutDont ainsi fu le jour pris 
   conme j’ai devisé. Si avint que 
   Japhus le Frison 
   et le roy d’Arragon 
   pristrentpristrent le jour de parlement de la endroit au jour saint Jehan 
   aprés en suivantsuivant a venir, et parmi grans asseurances faites. 
   Lors se partirent atant de la et revint la partie Pelyarmenus 
   arriere a Rome et en leur contrees, 
   et l’autre partiepartie revint en Constentinonble. Et fu Helcanus 
   mis en terre a grant douleur de cris et de pleurs. Mes il couvint mettre cest donmage et cest anui a la plus grant pes que l’en pot. 
   Si ne me vueil ore pas ci arrester, ains vueil venir 
   au duel de l’empereris, qui estoit demouree ençainte, et meismement au duel de 
   maintemainte autre dame dont li contes fera mencion ainz 
      que je voisedie plus avant en ma matire, dont cil qui 
   estoientn'estoient encore 
   encore a naistrené en furent puis 
   eulztel que mout serons lor nons et leurs fais moult essauciez, 
   avant que je aie finé ceste hystoire. Si asfiert ci endroit que je lesse a parler des aventures devant dites, 
      car bien i cuide reperier en lieu et en temps selonc ce que mestier en sera, et repaire li contes a 
      parler de l’empereeur 
   Cassydorus, ainsi conme j’en lessai 
      ale parler ça devant ou conte. \pend
         
         
Ci devise conment la chastelaine 
   pourchaça la mort l’emperere en tele maniere com vous orrez 
   et devise conment il fu mis en terre et conment l’empereris revint en sa ceulle avec 
   le chastelainle chastelain. 
      Comme s'ensuit ça aprésCi vient li contes a l'empereror Cassidorus
         comment il morutmorut par la traison a la chastelaine.
            
               Enluminure sur 1 colonne (c) et 12 UR.
                  Cassidorus accompagné du châtelain 
                  se rendant vers la cellule où se trouve l’ermite.
               
            
\pstart Ci endroit reconmence 
   li contes que veritéVoirs fu, 
   ainsi com vous trouverez en l’ystoire, que ainsi conme je lessai mon conte de l’empereeur 
   Cassydorus quant il avint que Celydus 
   et Dyanor se furent departi de lui, ainsi com il est 
      contenucontenu devant, 
   que il se maintint moult noblementbonnement en 
   sonson travail et en son labour, ainsi conme il 
      esta esté desus 
      ditcontenu. 
   Car li chastelains meismes, qui esgarda a ce que i tel honme, 
   qui touzjours avoit esté veritables en ses affaires, avoit toute seingneurie mise arriere et s’estoit mis du haut au bas, 
   ne se pot tenir que il ne venist a lui et li dist :
   Sire, voirs est que le sage dist que qui ne s’ainme n’est mie digne de moisson 
      avoirVariante du proverbe "Qui ne seme ne cuilt" (Morawski 2043).. Et pource que je voi que vous semez en terre gaaingnable, se seroit contre reson 
      queque vous 
      feissiezfussies a recevoir. Et pource que je voi que j’ai pou cultivé et ahané, si sera ma moisson petite, 
      par quoi il couvient que je me mette a ce queque je voi que 
      vous estes fais. \pend
            \pstart Pas de nouveau § BQuant 
   li empereres entendi le chastelain, 
               si li dist :
   AmisAmis, dist li empereres, 
      DeuxNostre Sires vous doint tel cultivement faire que vous ne soiez a 
      la senestresestre partie du 
      jugementjuge, 
      oula ou cil seront qui mauvessement avront cultivé ce qui leur avra 
      esté enchargié.Sire, dist li chastelains, sachiez vraiement 
      et si ne cuidiez pas que je ne doie ouvrer par vostre conseil,conseil
      endroit de ce que je vueil estre aussi conme vostre bon compaingnon, 
   en tant conme de fere penitance et de Dieu Nostre Seingneur servir.Amis, dist li empereres, ceste compaingnie ne vueil je ore 
      pas mettre arriere, car j’ai aucune fois oï diredire que compaingnie 
      Dex la fist. Or me dites en quele maniere vous me voudriez tenir compaingnie, car ce me couvient il savoir.Sire, distce dist 
      li chastelaincil, je voudroie 
      volentiers amener la chastelainne a ce que elle me vousist croire en autele maniere conme 
      ma dame l’empereris fet 
      vous. \pend
\pstart Quant li empereres 
   entendioÿ 
   ceste parolle du chastelaince, 
   si gita aussi conme i faus ris et dist :
   Par ma foy, 
      dist li empereres, sire chastelains, 
      aussi voudroie je ! Mes il m’est avis que vous em pourriez assez povrement 
      chievira chief venir. Et nonpourquant, Nostre Sires l’avra 
      assez tost convertie quant il li plaira. Si vous en metez en painne, 
      si ferez que preudonme.
   Ainsi ont moult longuement parlé 
   li chastelains et 
      l’emperere ensemble de ceste chose. 
   Si prist li chastelains congié et s’en vint a 
   la chastelainne et la mist a reson et li dist :
   Dame, je voudroie a vous parler.De quoi, sire ?, dist elle. 
   Lors ne demoura pas que li chastelains mist la chastelainne 
   a ce que elle vit bien que li emperieres li toudroit 
   son seingneur se il demouroit auques entour lui. Et que fist ele ? Elle 
   s’esta apenssees'appensa 
   detoute 
   la greingneur traïson qui onques mes fust penssee, car tout maintenant elle li otria a 
   asouvirfere sa volenté et si li dist pour voir que, 
   tout ne li eust il requis a fere ce qu’il li requeroit, si 
      l’enli eust elle requis, et que tout vraiement elle vouloit mettre son 
      cors a penitance faire, car li temps estoit venus que faire 
      l'escouvenoitli escouvenoit.
   Dame, dist il, voirement a tost Dieu conversé celui 
      quique ilqui il veult mettre a bonnes 
      oeuvres faireCette expression sentencieuse 
         n'est pas répertoriée chez Morawski ni chez Le Roux de Lincy. PD vérifie stp. Mais est-ce vraiment un proverbe ?, et 
      l’unen doit couvoitier le preu de l’autre, car nous 
      devons estre en Nostre Seingneur i cuer, une volenté et une penssee.Sire, distce dist 
      la chastelainne, ce sé je bien, et je ferai vostre volenté, 
      quecar ce est raison et drois. \pend
\pstart Lors s’en vint li chastelains a 
   l’empereeur et li dist que 
   sa fenmne estoit toute convertie de fere le bien 
      qu’i li conseilleroit.
   OrCe est bien, or 
      ne voi je, 
      dist l’empereeur, 
      fors que vous et lielle 
      vous metezmetiez a penitance faire par le conseil de vostre confesseur.Sire, dist li chastelains, 
      je me vouloie mettre conme vous estesestes fait en aucun lieu, 
      
         jemoi et 
         ma fenme, 
      la ou 
      nous ne feussionsje ne fusse pas conneus plus que vous estes ici. 
      Et lors me pleroit miex mon estre  et avroie plus grant 
      entenciondevocion en ce qu’il me pleroit a fere.Par ma foy, dist li empereres, de ce ne sai je que dire, 
   car je ne vous voudroie conseillier chose dont vous feussiez tart au repentir, que 
   cilz muert a envis qui apris ne l’a. La chastelainne, 
      par aventure, souferroitsouferroit ja a envis ce que 
      l’empereris fet ici. 
      Et je m’en sui ci endroit a ce mis que vous veez que je travaille a ceste eglisse 
      edefier. Et vous n’i poez pas entendre aussi conme je fas, car ce seroit ausi conme desordenance, 
      et si n’est pas usage de fere ce en son lieu.Tout ce est voirsverité, dist 
      li chastelains. Et pour ce me voudroie je mettre hors de mon lieu aussi conme vous avez fet.Amis, dist li empereres, 
   je loeroie que vous atendissiez encore i pou pour savoir conment elle se prendra a fere ce que vous dites, 
   car vous pourriez tost emprendre une chose qui ne seroit pas remisse a point quant vous voudriez. \pend
\pstart Pas de nouveau § BAinsi avint de ceste chose que 
   li chastelains et sa fenme devindrent de bele maniere et 
   conmencierent a fere aumosnes, et fist moult la chastelainne 
   lale simple, conme cele qui en ce pourchace une traïson moult cruelle, 
   quequar elle vit bien que son seingneur 
   li requeroit a fere moult de choses qui n’estoient pas de sa volenté. Si fist tant 
   la mauvesse para 
   i traitre maçon qu’il emporta amont i hourdeis qui par traÿson estoit fes de pierre.La
      vilenie de la châtelaine, plus encore en regard des efforts du châtelain de Gomor pour remettre son épouse sur la droite voie (qui laissent, 
      dans les échanges avec Cassidorus, presqu'entendre que de tels efforts sont vains, seule l'intervention divine pouvant y subvenir) n'est pas sans renouer avec la veine misogyne si prégnante au début du Cycle. 
      Surtout, cette jalousie envers le conseiller, la figure de sagesse qui vient entourer son époux, qui se fait source de tous les vices 
      (curiosité maladive, trahisons en tous genres) et plus encore avec cette question de sa réluctance à la charité à laquelle incite ledit 
      conseiller, rappelle celle de la vile impératrice qui s'opposait à Marques dans le roman qui lui était dédié. De manière intéressante donc, 
      la seule mort relatée dans le récit fait dans le Cycle (celles de Marques comme de Laurin n'étant jamais proprement évoquées) vient ainsi 
      rappeler les menaces qui pesaient sur le tout premier de ses héros. 
   Et il li faussa, par quoi il fu touz derompus et despeciez sitost que 
      li emperieres fu sus montez, qu’i le couvint 
   cheoircheoir aval en 
   unetele maniere que il le couvint mourir a grant hachiee conme vrai martir.
   La précipitation de la narration semble aller de pair avec sa précision, son emphase sur les forces en présence,
   la châtelaine et ce maçon rangés du côté de la trahison, répétée à plusieurs reprises, et Cassidorus définitivement sanctifié par cette 
   mort de martyre.
   Mes tant li fist Deux d’onneur que il fu vrais confés et parla aule 
   chastelain et li dist en mourant :
   Ha ! dous amis, je me muir, ce qu’il couvient a chascun faire au jour que Deux li a 
      destiné et pourveu. Je vous pri pour humanité que vous aiez 
      pitiépitié et merci de 
      ma fenme l’empereris, par quoi sa volenté soit fete en tele maniere que, se il li plaist a 
      reperier a Rome, que vous faciezfaites 
      tant que ele i soit, et ses enfanz aussi ; et se il ne li plaist, que vous vueilliez fere vostre aumosne en lui. \pend
\pstart Quant li chastelains vit que 
   li empereres mouroit 
   et il entendi ce qu’il dist, si dist :
   Ha ! siresire, de ce, 
      dist li chastelains, 
      aiez vostre cuer em pais, car je ferai vostre requeste a mon pooir !
   Atant devia li sains 
      empereresempereres Cassidorus et 
   rendi l’ame voiant maint preudonme qui virent et oïrent une grant compaingnie 
   d’angels qui emporterent l’ame du bon martir en loanges et en chans el regne du ciel. \pend
\pstart Pas de nouveau § BLi chastelains, 
   qui ne volt pas que le saint honmepreudoms 
   n’eust loenges et honneur a sa mort de la grant humilité qu’il avoit fait, 
   ainsi conme il est contenu, fist asavoir a tout 
   le pueple qui il estoit. Et quant il sorent ce, 
   si furent tuit esbahi. Li chastelains le fist ensevelir au plus noblement que il pot, 
   et puis si vint a lui maint preudonme a l’ermitage ou l’empereris 
   estoit, qui nulles nouvelles n’avoit encoreoncore (sic) 
   oÿ de son grant destourbier. Si fu moult esbahie quant elle vit ceus venir a lui 
   en tele maniereainsi. 
   Lors li dist li chastelains :
   Dame, il vous couvient venir a mon seingneur, 
      a qui Deux a fet si apert miracle que tuit sevent que il 
      estesti
   empereeur de Rome et de Costentinoble.Sire, distce dist 
      la dame, de ceste honneur ne sé je pas que
      mon seingneur en ait moult a faire. Mes ce me dites pourquoi il n’est venus avecques vous 
      et m’eust conmandé sa volenté a faire, carque il m’a conmandé que 
      je ne me parte de ceens s’il n’est presens.
   Adonc vit bien li chastelains que dire li couvendroit tout 
   el avant que la dame osast contredire le conmandement 
      son seingnor et dist :
   DameDame, dist il, riens n’i vaut li 
      celers :celers que 
      mesiremesire est si malades que il m’envoie a vous que il a paour que 
      vous n’i puissiez venir a temps.
   Lors sailli la dame em piez et dist :
   Ha ! vrais Deux debonnaires, soiez li aidant et moi n’oubliez ! \pend
\pstart Pas de nouveau § BAtant se mist 
   la dame a la voie avec 
   le chastelain et avec la bonne gent qui erent o lui venus. Si ne finerent 
   jusqu’a tant qu’il vindrent ou moustier ou 
   li empereres gisoitgisoit mort 
   honnorablement, ainsi conme je avoie desus dit. 
   Nouveau § BQuant l’empereris sot que ce fu 
   son seingneur, si ne fu pas merveille se elle fist chiere marrie. Que vous diroie je 
   el ? 
   Il ne fufu nul cuer d’onme ne de fame, 
   tant fust durs, quil n’eust grant pitiépitié et grant doleur 
   desdes grans complaintes que elle fesoit. 
   Mes ce vueil je lessier a raconter, car chascuns puet bien savoir selonc ce que dit 
      estest devant qu’ele fist tant que tuit firent duel avec lui, 
   car meismes la chastelainne, qui sa mort avoit pourchaciee, 
   fist un si grant duel que mal feist a croire que sese li eust elle fait. En celui duel avint que un chanoinne se vost revestir 
   pour dire la messe de l’empereeur quant tuit cil qui la furent 
   distrent qu’il virent entrer en 
   l’eglise i evesque, tout revestu pour 
   la messe dire, et avecques lui grant plenté 
   de prestres et de clers si noblement appareilliez 
   qu’il sembloient angels de paradis. Et lors dist li evesques tout ce qu’il 
   escouvientescouvint 
   dire a tele personne mettre en terre 
   et fist tout sus le servise. \pend
\pstart Pas de nouveau § BAinsi fu 
   li emperieresemperieres Cassidorus 
   mis en terre devant le mestre autel de saint Nicholas. 
   Et quant l’empereris vit ce, si s’est apaisiee et 
   vint a l’evesque devant tout le pueple, et il s’umelia vers li et la tint moult longuement 
   en maniere de confession. Aprés se departi l’empereris de lui aussi debonnerement 
   quecomme s’il ne fust riens a lui de 
   l’empereeur fors tant que elle dist :
   Sire, drois emperieres et amis, 
      il m’est moult tart que je puisse aler aprés vous en vostre compaingnie ! 
      Car je sévoi bien que aprés moi ne venrez vous pas. 
      Si pri a Jhesucrist qu’il ait pitié et merci de l’ame de vous se il li vient a gré. Ainsi soit 
         ilil, amen !
 \pend
         
            Ci 
               fenistfine le livre de 
               Cassydorus, 
               empereeur de Rome et 
               de Constentinoble, 
               et aprés conmence li derreains de ses 
               enfansenfans. Explicit.
            Ci devise l’ystoire que, 
               aprés ce que li empereres fu mis en terre, 
               avint que cele qui gardoit les 
                  iiii enfans a l’empereris 
               ala ou chastel pour savoir que 
               il demouroient tant. Et endementres, il avint que 
               li lyons emporta les iiii enfans l’un aprés l’autre a 
               tous les berceus en i hermitage la ou 
               li hermites les berçoit et 
               nourrissoitnourrissoit, si comme il est contenu en l'istoire
               .
            
               Enluminure sur 2 colonnes (b-c) et 14 UR.
                  Les 4 enfants emmenés deux à deux par le lion
                  chez l’ermite. 
               
            
         
      
  

\end{pages}
\end{document}
        