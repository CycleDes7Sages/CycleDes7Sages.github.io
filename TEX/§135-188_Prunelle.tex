\documentclass{article}
\usepackage[T1]{fontenc}
\usepackage{microtype}
\usepackage[pdfusetitle,hidelinks]{hyperref}
\usepackage[english]{babel}
\usepackage[series={},nocritical,noend,noeledsec,nofamiliar,noledgroup]{reledmac}
\usepackage{reledpar}


\begin{document}

\date{}
        \title{Roman de Pelyarmenus}
\maketitle

\begin{abstract}
Abstract to be added
\end{abstract}
\begin{pages}
\beginnumbering
        
   
   
      
         
            \pstart Or dist li
                  comptes que moult ouvra la pucelle
               courtoisement envers Myrus et ses compaignons. Il
               ne finerent 
                  puis quedespuis quepuis il se furent 
                  partis de lui de chevauchier tant qu’il oïrent nouveles de Pelyarmenuz. Dit leur fu que il
                  estoit a i sien manoir qui 
                     estoit enassez pres estoit de
                  Constentinoble. Il n’ont finé,
               l’un jour plus, l’autre mainz, si 
                  sont venu enont aprochié
               la cité. Il 
                  ont eueu conseil d’eulz meismes que il yroient en Constentinoble
               
                  pour euls reposer et la 
                  avroientavroient il conseil comment il porroient mieus besoingnier. Lors 
                  vont en la ville
                     et ont leuront ostel pris 

                  ou maint autre l’avoient pris qui 

                        avoient avoloient besoingnier aussi comme 
                        eus meismes avoientil meismes faisoient. Il fu tart,
                  si fu tempset tanz de souper. Myrus a fait venir son hoste devant lui et li enquist ou li empereres estoit. Sire, dist il, lequel empereour demandez vous ?
               Amis, dist Myrus,
                  ja n’a il en ceste ville que
                     i empereour.
               Sire, dist il, non que je s
                     a‹a›[c]he. Celui est a Bel Manoir, a une
                  lieue pres de ci.
               Et quant vendra il, 
                     biaus hostes ?biaus ostes, en ceste vile
               
                     Par foy, 
                     siresire, dist cil, ce ne vous sai je mie a dire, quar il vient quant il veult, et quant il
                  veult il s’en va.
               Or me dites, biaus hostes, foy que vous me devez : 
                     savez voussavez nulles nouvelles de Gasus, le baillif
                     Pelyarmenus ? 
               Sire, dist il, hui le vi ci devant passer et bien
                  croi que il soit en la ville.
               Et ou demeure il ? dist Myrus.
               Sire, 
                     il demeure ou 
                     chastelchastel lassus. Quant il oÿ ce, si mist son
                  hoste a conseil et dist : Amis, se vous nous
                  vouliez aidier que nous 
                     parlissons a luia lui eussons parlé de conseil, moult y avriez grant preu.
               Nouveau § BSire, dist
                  il, dont me dites qui vous estes et se vous de riens le counoissiez.
               Amis, dist Myrus,
                  nous sommes au roy de Frise, mais nous fumes nés en Bretaigne.
               Ha ! sire, dist li
                     hostes, n’a pas lonctemps que li roys de
                        Frise fu ci devant a hostel, et le duc de Lembourc fu çaiens.
               Par foi, sire, dit Myrus, bien puet estre. Et je trop mieus vous en aime, et 
                     aussitout aussi fais je l’ostel.
               Sire, dist il , granz mercis! Et sachiez que je 
                     pensoie moult bienbien pensoie que vous estiés de leur gent. Et
                  sachiez que en moi vous poez bien fier et commander sor moi, 
                     et sachiez quecar je en ferai 
                     tout monmon bon pooir.
               Biaus hostes,
                     dist il, granz 
                     mercismercis, dist Myrus. Je vueil que vous faciez tant qu’il sache qu’il a chaiens i
                  chevalier qui est au roy de Frise, qui moult volentiers parleroit a lui 
                     anuitou anuit ou demain, de quele heure qu’il en seroit aisiez. 
               Sire, dist il,
                     je n’arresterai si

                           avraien avrai faitje ferai la besoingne a mon pooir. Dont fist mestre li hostes la selle 
                  seur son palefroy, 
                        si monta et ne demoura gaires queIl monta et ne fist demeure, sia son cheval et vint au chastel, puis hucha. Li
               portiers, qui bien le connoissoit, 
                  li dist : Sire, vous soiez
                        li bienvenuz !
               Amis, dist il, Diex vous beneye ! Est li baillis
                  çaiens ?
               Ouil, dist il. Dont descendi li bourgois et vint la ou il estoit. Et quant li baillis
               
                  lileMénard relève la substitution de "lor" à "les" et
                  vice versa par analogie avec "nos" et "vos" qui sont à la fois régimes directs et
                  indirects et en raison de l'emploi tonique de "lor" équivalant à "eus, eles"
                  (§39). Le cas de figure n'est cependant pas le même, mais la confusion entre
                  régime direct et indirect est courante chez V2. vit, si est venus a lui et
               li dist : Geffroy, que vous plaist ?
               Sire, dist cil, je sui ci venus 
                     pour parler a vousa vous parler pour i chevalier qui ci
                     m’envoie, 
                     quisi est au roy de Frise, si voudroit a vous parler moult volentiers, mais
                  que vous en fussiés aaisiez. Quant il l’entendi, dont li mist la main seur
               l’espaule et l'a trait d’une part et li dist en bas : Quele gent a il avec lui ?
               Par foy, dist il, v chevalier 
                     et leur maisniesa leur mesniees y sont.
               
                     Alez vous entAlez, dist il, 
                     tosttost chiez vous
                  
                     et li dites que je yrai maintenant, mais 
                     nen'en faites ja 
                     mentionmencion a nului qui il sont ne que 
                     yje la doie aler se a euls non.
               Sire, dist il, n’aiés garde. Il s’en est
               moult tost repairiez et compta a Myrus
               comment il avoit 
                     esploitiébesoignié. Biaus hostes, dist il, granz
                  mercis !
               
                  Lors ne demoura gaires que 
                  li baillis vint, 
                        oet o lui i escuierli vallés vint et le baillif sanz plus. Il monta amont et demanda l’oste, 
                  et il vint, et l’amene en une chambre ou li chevalier estoient. Sire, dist il a Myrus, vez ci le baillif que vous 
                     demandiezdemandez. Quant il le virent, dont 
                  sont sailli en piez et l’ont salué, et il leur rendi leur salu et 
                  puis regardaDont a esgardé l’un et puis l’autre, savoir se 
                  nulil nul en connoistroit, mais il n’en y 
                  otvit nul que il 
                  eust onques maisonques mais ot veu. Si dist ore : Biaus seigneurs, vous
                  soiez 
                     lesle bienvenus ! Je sui venus a vostre mandement.
               Sire, dist Myrus,
                  la vostre grant merciz !
               Dont l’a pris par la main Myrus et 
                  l’a traitle tret
               
                  a une part etd'une part. Dont li dist : Sire, vostre bonne renommee me fait
                  en vous fier.
               Sire, dist il, se je estoie pires que je ne soie, si
                  vueil je estre tenus de vous aidier pour la raison de ce que vous avez dit.
               Sire, dist il, granz merciz !
               
                  Dont li dist :
               Sire, veritez est que nous sommes ci envoiez pour
                  une grant besoingne. Si avriensSur l'emploi de finales
                     en -ens pour la P5 du subjonctif, cf. analyse linguistique. bien 
                     mestiergrant mestier de vostre aide et de vostre conseil. \pend
            \pstart 
               
                     Sire, veritez est que nous sommes a Helcanus, qui est filz a l’empereour Cassidorus de Coustentinoble, et cil est en Frise, 
                     aveco lui
                  le duc de Lembourc. Si savez bien que li roys et li dus furent l’autre
                  jour en ceste ville pour moustrer a
                     Pelyarmenus, vostre seigneur, que cilz
                     Helcanus estoit repairiez de la ou il avoit
                  conversé, quar les nouveles estoient hors getees partout que il et son pere et 
                     avecencore
                  ii
                  
                     petiz enfans estoient mort. Ore, sus ce, Pelyarmenus est entrez en la terre et se fait apeler seigneur du
                  paÿs, ce que il ne doit faire tant que il sache que cilz 
                     soit en vievivent que je vous ai 
                     compténommez. Ces nouveles aporta li roys de Frise
                  et li dus de Lembourc a Pelyarmenus, et 
                     illiPeliarmenusLe sujet du verbe change ici : il s'agit
                     maintenant de Pelyarmenus. La forme li résultant probablement d'une confusion
                     entre "il" et "li, nous préférons la leçon de X2. respondi assez
                  sagement, quar il dist venist Helcanus ou autres qui droit eust en la terre, 
                        et il le recevroit, si que ja n’en seroit repris ne blasmez. Li
                  conseuls Helcanus si 
                     etestSur l'emploi de "et" pour "est" voir XXX Mélanges
                     pour Simon Gaunt par Simone? tiex que il dist que il n’enterra 
                        jamaisjajamais jamais
                     
                        en cest paÿs devant ce qu’il sera asseurez des barons meismes de son frere qui les homages en a receuz
                     autrement qu’il ne deust,
                     ce dist. Et de 
                     ceuz qui l’ont receu en cest paÿscil meismes du paÿs qui l'ont leceu (sic) autrement qu’il ne deussent si en 
                     veultveuil l’amende avoir, se ainsi est que il soient trouvé en leur tort, et 
                     estc'est la raison pour quoi il nous a ci envoiez. Si 
                     voudrionsen voudrions volentiers avoir conseil comment nous porrions 
                     lemiex parler a Pelyarmenus par quoi il
                  ne pechast en nous que la besoingne ne fust bien faite et reportons arriere la
                  vraie responsse de son conseil que nous 
                     avonsavrionsa‹r›[u]rons trouvé. \pend
            \pstart 
                  Quant Gasus entendiGasus, quant il ot entendu
               Myrus, si ot grant joie a son cuer et dist : Puet 
                     ce estre verité, ce que vous m’avez ci compté ?
               Sire, dist Myrus, 
                     aussitout ausi voir comme Diex est, quar Clyodorus,
                  que vous bien counoissiez, est avec Helcanus
                  chiés le roy de Frise.
               Clyodorus ?
                  dist il.
               Voire, sire.
               Ha ! Clyodorus,
                  sergant adroit, tant avez alé que vous avez trouvé ce que vous voliez ! Par foy,
                  biaus seigneurs, dist GasusLa présence de l'incise "dist ..." se trouve habituellement
                     au début de la première prise de parole d'un personnage, pour souligner les
                     échanges dans le dialogue, qui est pour ce passage plus dur à suivre.,
                  maint hustin et mainte grevance ay eu pour 
                     verité dire et fairele droit dire et por faire le. Or n’est il mie poins de plus parler, et vous estes lassez de chevauchier
                  : le matin revendrai a vous priveement et penserai a vostre besoingne tout aussi
                  soingneusement comme se mon cors meismes 
                     estoity estoit a vostre point.
               Sire, 
                     ce dist Myrus, la vostre grant merciz ! Lors 
                  ont prisprist congié 
                  a lui et li baillis a euls, quar plus n’osa demourer que il ne fust aperceuz. \pend
            \pstart Pas de nouveau § B
               Li compaignon souperent et puis alerent dormir. Gasus, qui ne dormi pas toute la nuit, commença a penser a
               la besoigne de ceuz qui prié l’en avoient. Il sot bien que Pelyarmenus ne renderoit point la terre tant comme il
               la 
                  peustporroit tenir, et bien 
                  sotsot il que moult longuement 
                  y porroit estrela porroit il tenir ainz que autres y 
                  peust metremeist le pié tant comme cil qui l’avoient asseuré li vousissent aidier. Et il savoit
               bien que voirement li aideroient il, quar il se sentoient mesfait envers leur seigneur droiturier, et pour ce aideroient il
                  cestui au miex que il porroient. Aucuns en
               y avoit qui volentiers revendroient a 
                  voie de veritéverité, mais leur force n’est mie 
                  grantmoult grant, se ce n’estoit 
                  duli
               duc d’Athaines, mais cil estoit trop hors 
                  voiede voie por lui aidier. \pend
            \pstart En tele maniere se dementoit Gasus tout par lui. 
                  Lorset tant que il commença a penser comment il porroient
                  partir de 
                        leurson seigneur sanz destourbierLa
                  référence du pronom personnel "il" n'est pas évidente. D'après le contexte, "il"
                  référerait aux barons ralliés à Pelyarmenus, susceptibles de rejoindre Helcanus.
                  La leçon de B "son" indique que "il" renvoie à Gasus et aux hommes d'Helcanus qui
                  sont venus à sa rencontre.. Il sot bien qu’il estoit si plains de mal que a
               paines s’en porroient partir sanz destourbier que ce ne fust trop grant merveille.
               Dont s’apenssa qu’il les feroit aler au Biau Manoir, et la porroient il parler 
                     de la chose et faireet estre aussi comme se ce fust noient. Et se aucuns y metoit mal, il les 
                     deffenderoiten destorneroit, se il pooit ; et se il ne le pooit faire, si n’aroient il ja
                  mal sanz lui. A l’endemain assez matin 
                  se levas'est levez. Dont s’en vint 
                  par devers les messagiers, avec lui i escuier, moult priveement , sanz plus de 
                  compaignonscompaignie, et leur 
                  a ditdist qu’il 
                     s’apareillassents'aprestassent et 
                     fussentfust chascuns garniz priveement dessous 
                     leur chapessa chape
                  
                     pourpar toutes aventures et venissent au Biau
                     Manoir et 
                     que la porroit onla porroient il miex parler et plus a pays de leur besoingne, et il yroit devant aussi,
                  comme se il ne seust riens de leur venue, et si 
                     y seroit presens quant il parleroient 
                     de la besoingne que il avoient a faire a son
                     seigneur. En tele maniere comme il 
                  l’a ditleur dista dit l’ont fait. Il se sont de leur armes garniz et appareilliez et puis pristrent
               les bons branz d’achier dessouz leur chapes et sont montez sus 
                  leurlesleurs chevaus. Gasus pria a l’oste par amours que il les
                  feist mener au Biau Manoir ou il meismes
                  les y menast. Il dist que 
                     ja autres
                  
                     
                           de
                           que lui meismes ne les y menroitnes y menroit que il meismes. Dont 
                  ne finerentse sont mis a la voie et n'ont finé si vindrent a la porte a heure de tierce. Pelyarmenus estoit repairiez de son oratoire, et estoit 
                  Gasuso lui Gasus enmi la court. Myrus et si compaignon
               vindrent a la porte et distrent que 
                     on les laissast enslaienz les lessast entrer. Dont leur demanda li portiers qui il estoient. Il distrent 
                  que il estoient a Helcanus, filz a l’empereour Cassydorus  de Coustentinoble. \pend
            \pstart 
               Pas de nouveau § BQuant cil
               entendi ce, si fu touz esbahiz et dist : Biaus
                  seigneurs, or vous souffrez tant que 
                     j’aie parlé aje sache la volenté de
                  mon seigneur ! Cil en
               vint a i chevalier et li dist que tel
                  gent estoient la et que il le deist a son seigneur, se il vouloit 
                        qu'ilque les laissast 
                        ens entrer. Encore fu plus esbahiz li chevaliers. Il 
                  ens'en vint a Pelyarmenus et li dist que tel gent estoient venu a la porte, 
                     se il vouloit qu’il 
                           y entrassent. Mal dehais ait, dist il, 
                     quiqui onques la porte leur vea ! 
                     JaIci est Helcanus mes freres et sires et
                  souverains de moi.
               Certes, distrent cil qui l’oïrent, c’est 
                     moult bien 
                     distditSur "dist" pour "dit" voir l'analyse
                     linguistique. !
               Nouveau § BDont fu la porte ouverte aus messages, et descendirent maintenant. 
                  Quant Pelyarmenus les
                     vit, si ne se pot tenir qu’il n’alast Peliarmenus, qui les vit, ne volt laissier que il ne soit
                     venus encontre euls et les a saluez. Cil qui furent 
                  esbahituit esbahi de la belle chiere de lui li ont rendu son salu moult 
                  affaitiementhumblement. Maintenant a demandé Pelyarmenus
               
                  a ceuz que son chier frere
               
                  faisoitfaisoit et ou il ert. Sire, dist Myrus, messires vostre freres
                  
                     
                           lel'a fait biense bien non, 
                     la merci DieuDieu merci ! Si nous envoie a vous et vous fait sa volenté savoir.
               Et ou est il ? dist Pelyarmenus.
               
                     Avec son oncleSire, distrent il, avec son oncle
                  le roy de Frise
                     , dist Myrus .
               Voire ? dist Pelyarmenus. Je cuidoie qu’il fust en cest paÿs.
               Sire, dist Myrus,
                  non est, ainçois vous fait asavoir sa volenté par 
                     letressa lettre et par nous la raison por quoi.
               Par foi, dist il, et je 
                     moult volentiers 
                     savrai sa volentéle savroie. Dont a Myrus sachiee une 
                  lettrelettee qui estoit seellee du seel 
                  auel au (sic)
               roy de Frise
               et tesmoingnoit que ce estoit la volenté 
                  de
               Helcanus
               
                  tot ce qui dedenz la letre estoit contenu. 
                  Dont l’a Pelyarmenus prise et a rompu la cyre et puis 
                        si a leutrouva ainsiLors a pris Peliarmenus la letre et a froissiee la cire et leu  l’escript, qui dist : \pend
            \pstart Pas de nouveau § BHelcanusE (sic) Helcanus, premiers engendrez 
                     des filz Cassydorus, empereour de Coustentinoble et de Rommedes filz a l'empereur Cassidorus de Romme et de
                        Constentinnoble, damoisiaus des ii citez, fais savoir a Pelyarmenus de Romme, qui mon frere deust estre, qu’il viengne a moi
                  en quelque lieu que il me sache pour 
                     luilui lui (sic) escondire ou amender des vilonnies 
                     ou entraitesLe nom "entraite" provient du
                     vocabulaire médical (FEW IV 772a) ici pris dans le sens second de "mauvaise
                     farce malveillante" (TL III 620) ou plutôt de "promesse trompeuse", attestés
                     jusqu'au XIIIe siècle. Le copiste de B n'a pas reconnu le mot comme un
                     substantif mais comme le participe passé du verbe "entraire" dont le sens
                     d'"entraîner" (FEW IV 772a) n'est pas très satisfaisant ici.  que il a 
                     faitesfet envers moi et les miens. 
                     Et bien sache que je le deffi en quelque lieuou, se ce non, je le deffi de moi et des miens ou que je le truisse, mais que pooir y puisse avoir,
                      et bien sache que je en ferai autant de lui comme je
                        deveroie faireJe autant en ferai com je feroie de celui qui moi 
                     avroitairoit (sic) mis a son pooir em peril de mort 
                     par pluiseurs foiz. Et ainsi en a il fait, si comme il le
                        set bien, et a donné conseil et mis grant paine de mon petit frere et ma sereur
                  
                     fairefet destruire sans nule bonne raison qui y peust estre fors que de traïteur et
                  de homme 
                     mauvais et desnaturé. \pend
            \pstart 
               Quant Pelyarmenus
               
                  aot ce leu, si commença a sousrire et dist : Biaus seigneurs, mes freres ait bonne
                  aventure et ceuz aient honte par cui il m’a fait tel mandement, quar je sai bien
                  que onques de son cuer premierement n’issi.
               Sire, dist Myrus,
                  ne cuidiez 
                     miemie que
                  Helcanus soit tiex que
                  pour 
                     hommehomme nul qui vive il vous eust 
                     mandéenvoié chose qui devers le cuer ne li venist. Et d’autre part sachiez qu’il n’a conseil de nul traïteur ne 
                     d’ommed'ome nul qui ne soit 
                     couvenablescouve-bles et preudom en touz 
                     droizendroiz et des meilleurs chevaliers du monde. \pend
            \pstart Pas de nouveau § BPelyarmenus esgarda lors Myrus et dist : Biaus sire, 
                     jebien croi que de bons chevaliers ait il en sa compaignie. Mais ce vous veul je
                  bien dire que, de 
                        quiconquesquiconques conseil ceste letre viengne a moi, 
                        queje di queDans B, la reprise du verbe "dire" après
                        incidente est répétitive. Elle souligne la détermination farouche de
                        Pelyarmenus. elle vient de traïtour et de 
                        mauvaisfelon mauvais !
               Sire, dist Myrus,
                  pource sui je ci venuz que, s’il estoit nus qui vousist dire que mes sires
                     n’eust droit ou mandement que
                         mandé vous ail fait vous a, que, je le feroie estable mon cors contre le sienLa même expression "faire estable qch. (ici un mandement)"
                     se retrouve quelques lignes après. Nous la comprenons ainsi "Je ferais
                     respecter ce message dans un corps à corps". Dans sa deuxième occurence, la
                     leçon de B omet l'adjectif, le remplaçant par la reprise implicite de
                     l'adjectif "convenable" ("il fera couvenable le mandement" repris en "il le
                     feist")..
               De ceste parole fu Pelyarmenus iriez et
               regarda environ soi et vit 
                  i
                        chevalieri sagent (sic) chevalier qui sailli avant et dist : Et je sui cil qui
                  le mandement ferai descouvenable et de mauvais frere contre frere, ne que mes sires, qui ci est, ne fist onques
                  envers lui ne envers les siens 
                     choseonques par quoi Helcanus li deust 
                     avoir mandé 
                           telde tel mant ne onques ne li fist chose que il ne deust avoir faitefaire tel mandement que il ne deust faire a son bon ami et frere. Dont se mist avant Gasus ou Pelyarmenus recevoit
               les gages 
                  et dist : Ha ! 
                     siresire, dist il, ne faites pas tel vilonie ! Laissiez moi parler a vous.
               Que voulez vous 
                     dire ? dist Pelyarmenus.
               Je vueil dire, dist il, que messages qui est envoiez
                  d’omme qui vaille doit dire son message en la maniere qu’il li est enchargiez sanz
                  lui 
                     empeeschier de nul homme vivantestre empeeschiez d'omme soufisant.
               Gasus, Gasus,La répétition du prénom Gasus est maintenue, étant commune
                     à tous les témoins et ne posant pas de problème syntaxique. Elle peut résulter
                     d'une erreur de l'archétype mais également participer de la rhétorique de
                     Pelyarmenus qui feint l'indignation. dist 
                     Pelyarmenusil, 
                     dont n’affiert il mie a lui ne a autre 
                     de dire vilonnie ne outrage ! Foy que je doy a Fastidorus mon frere 
                     ou il fera couvenable le mandementce que il a dit, ou il 
                     penderaen p. par la gorge !
               Et s’il estoit, dist Gasus, qu’il le feist 
                     estable contre celui qui 
                     le desdiroitl'en desdit, 
                     que en voudriez vous dire ?
               
                     En non Dieu, dist Pelyarmenus, que que il en aviengne, l’aventure si en soit 
                     seuesienne, 
                     quar je vueil que il soit tout ainssi sans faille.dist Peliarmenus \pend
            \pstart Pas de nouveau § BQuant
                  Gasus
               
                  entendioÿ ce, si esgarda Myrus et li fist 
                  signesemblant qu’il parlast, et si fist il moult 
                  efforcieementappareilliement et 

                  li dist : Sire Pelyarmenuz, vez ci cest
                     chevalier qui mon gage a receu. Je vous demant, 
                     lase la deffense que il veult faire, se vous la tenez a vostre"Je vous demande si vous considérez comme la vôtre cette
                     défense qu'il veut faire". Sur ce type de dislocation, cf. analyse
                     linguistique..
               Biaus sire, dist Pelyarmenus, quant je vous respons, 
                     je vous faisne faiz je respondre. Si soiez sages de vostre besoingne"Avisez-vous de ce que vous faites".
               Et je vous 
                     fais, dist Myrus, ausi, dist Peliarmenus, faisLe copiste B ne confond pas forcément les
                     personnages et les tours de parole dans cette leçon. Au contraire, le fait que
                     Pelyarmenus maintienne la parole, même si la présence de l'adverbe "ausi" dans
                     l'incidente est rare et signale peut-être une perturbation, atteste de la
                     fermeté de la position de notre
                  anti-héros".
                  une demande : 
                     se il est ainsi que vous soiez convaincus, le seront
                        aussi cil pour qui vous le faites, quar sachiez que je voudroie bien que
                        ceste chose fiablement fust mise en respit"Sire, dist il, de ce vous respondrai je foiablement." "Veilliez
                        ceste chose metre en respitDans B, Mirus prend la parole s'accordant à la
                     demande de Pelyarmenus, puis Pelyarmenus lui répond. Le passage est aussi
                     difficile dans V2GX2, notamment par la présence de la proposition "quar...". Il
                     est possible que le rapport de cause introduit ici se rattache à la demande:
                     "Et je vous fais, dit Myrus, une demande: car sachez que je voudrais bien que
                     cette affaire fût loyalement retardée; ainsi si vous êtes déclarés coupables,
                     ceux pour qui vous agissez le seront aussi. tant
                  qu’il 
                     fussentsoient en lieu, et vous 
                     d’autrede l'autre part pour quoi l’une partie et l’autre soit seure 
                     de celui qui seraquiconques soit vaincus de ceste emprise 
                     fairefacefaite, 
                     eten l'amende a la volenté de celui qui au desseure en sera.
               Ce ne ferai je mie orendroit, 
                     dist Pelyarmenus, quar mes conseuls ne le 
                     m’aporteme porte
                  
                     miemie, dist Pelyarmenus.
               Nouveau § B
               Dont vueil je savoir, dist Mirus, de combien je serai avanciez se je fais estable de ma partie
                  ceste chose.
               Par foy, dist Pelyarmenus, de ma part ne serez vous avanciez fors de tant que
                  vous desmentistes i de mes
                     chevaliers, qui dist que vous conseil de 
                     folfelon et de losengier aviez creuz quant 
                     vousvous onques tel letre m’aportastes, et puis si la voulez faire couvenable 
                     sanz raisonet estable.
               
                  Quant Myrus
                     
                        ententoÿ que il n’en avroit 
                        autre choseel, si 
                        distli dist
                  : 
                  
                     Sire, puisque vous si couvenables estes que on
                  
                     i tel mant 
                     on ne vous osastne vous deust l'en envoier, ne souffrez mie que on me face tort ne vilonnie, que, se ainssi
                  est que je puisse ceste chose aventurer comme dit est, que je et mes compaignons
                  en puissons partir sans domage. \pend
            \pstart Quant Pelyarmenus s'oÿ ainsi 
                  aprochierreprochierLa variante de G "reprochier" explicite l'emploi
                  spécifique d'"aprochier" au sens d'"accuser"., si 
                  respondidist : Par Dieu, si 
                     ferezserez (sic) vous ! Et je commant 
                     aule ballif, qui ci est, et 
                     ilqui volentiers le fera, que il ne sueffre que on vous face 
                     nul tort.
               Sire, dist il, grant merciz. Dont fu 
                  il ordené que la bataille devoit estre a l’endemain. Gasus, qui moult liés estoit selonc 
                  l'aventuretoutes aventuresl’aventu-ture (sic) du commandement son seigneur qu’il
               avoit eu de lui, vint aus messages et leur dist : Biaus seigneurs, vous 
                     repairerezrepairez
                     Nous corrigeons ce qui résulte d'une
                        haplographie.
                   arrieres en Constentinoble, et
                  je meismes vous convoierai, car je me doute de vous. Et nonpourquant, n’avez vous garde se bonne non ne 
                     n'i avrezn'avrez
                  
                     ja mal sanz moi, de 
                     ce vous fai je seurstout ce soiez asseur. Il ont 
                  respondurespondu tuit ensamble que :
                  Sire, granz merciz ! Et
                        sachiez que moult en faites a amer !de ce faisoit il moult a mercier
               
                  DontAdont vindrent 
                  a leuras chevaus et monterent et se mistrent 
                  en leur chemin et 
                        n’ont finé si vindrent a leur hostiexvers Costentinnoble. Il n'ont atargié si sont venus a leur
                     ostel. Il fu temps de disner et les tables furent mises. Biaux sire, dist Mirus, il vous
                  couvient 
                     dignerSur la forme digner, cf.
                           analyse linguistique. avecques nousdemorer avec nous au mengier.
               Certes, 
                     sire, dist il, je en seroie blasmez d’aucuns. 
                     Et bien sachiez certainnementCar sachiez que, se 
                     ne fustce n'estoit pour ceste bataille, aussi volentiers 
                     com vous le voudriez en seroie jeque vous meismes en seriez liez. Et encore vous di je que pour l'amour de vous ferai je 
                     pour vous cetele choseque je 
                     
                     porrai ne feroie pour homme qui 
                     soit eny soit de cest 
                     
                     mon empire, quar je vous lairai çaiens avec vos compaignons en la garde de vostre hoste que je ne feroie pas
                     i autre, ainçois seroie saisiz de lui et le menroie au chastel amont tant que il avroit achevé ce 
                     qu’il aroitque vous avez empris. Quant Myrus oÿ ce, si dist
               : Ha ! sire, ja de ceste chose blasmez ne serez, foy
                  que je doi 
                     touza touz gentis hommes, ains 
                     m’enen yrai avecques vous, quar assez plus m’i plaist que ci a demourer.
               Et je l’otroy, dist Gasus.
               
                  AtantDont se sont mis a la voie parmi la
                  cité, qui ja estoit toute esmeue des messagiers de quoi li uns avoit pris
               bataille, si comme 
                     devantde-devant (sic) est dit. Moult furent esgardez de dames et de bourgois,
               et tant qu’il sont venu au chastel. Dont demoura avec Myrus
               i de ses compaignons, et li 
                  autreautre iiii sont 
                  repairiérepairié a l'ostel avec leur hoste. \pend
            \pstart 
                  GasusCasus, qui moult engranz estoit 
                  durement de faire courtoisie aus compaignons, prist Myrus et son compaignon et les
               mena 
                  maintenant au chastel amont. Li mengiers fu touz
               pres et les tables 
                  furent mises. Il 
                  ontl'ontLa référence du pronom régime dans B n'est pas
                  limpide dans la mesure où on attendrait un pronom pluriel, renvoyant à Mirus et
                  son compagnon. Nous pensons que B fait ici une erreur due à une hésitation entre
                  l'emploi du verbe "laver" qui est soit transitif (laver qn) ou absolu ("laver" au
                  sens de "se laver les mains" DMF). lavé, puis sont 
                  assisassis au mengier, et sachiez que moult orent et assezLa leçon de V2G atteste la possibilité d'un quasi
                  saut du même au même dans V2 et G, étant donné la proximité entre "assis" et
                  "assez".. Quant il orent dignéSur la forme
                  "digner", cf. analyse linguistique.,
                   dont furent les tables ostees. Myrus se traist pres de Gasus et 
                  li dist : Que vous samble, 
                     sire, de nostre afaire ? Comment avons nous 
                     esploitiébesoignié jusques a ore ?
               Sire, dist Gasus,
                  encore ne voi je point que vous 
                     n’aiés biense bien non besoingnié selonc ce que les paroles sont courues. Mais de tant vous fai je
                  sage que vous avez a faire a i des plus redoutez chevaliers qui 
                     soit ne qui onquespieçaonques issist de Romme.
               Et comment, dist Myrus, a il non ?
               Par foy, 
                     sire, dist Gasus, son non moustre bien que il
                  soit autres que bons, quar il 
                     estest est apelez Malquidant.
               Par Dieu, dist Myrus, je ne me doute de riens que, se je en champ le tieng, mais
                  que autres 

                     de lui ne s’en melle, que 
                     je avec le droit que je 
                     y ai ne doie faire en partie ma volenté de lui.
               Sire, dist Gasus,
                  dites moi vostre non, quar nices 
                     suiai esté que je ne le 
                     vous ai demandé et vous savez bien le mien.
               Sire, dist il, on m’apele Myrus , et fui nez je et mi compaignon en 
                     la terre de
                  Bretaigne.
               
                     Voire ? dist Gasus.En Bretaigne?" dist Gasus. "Sire, voire." Et comment a non vostre compaignon ?
               Sire, dist cil, on m’apele 
                     ElygiousEligiusEligionsIl ne nous paraît pas nécessaire de corriger ce
                     nom propre. La difficulté de son orthographe s'explique facilement : il s'agit
                     de la première occurrence du nom dans le roman..
               Par foy, dist Gasus, merveilles oy que vous en Bretaigne fustes nez ! Et vos 
                     autres compaignons, dont sont il ?
               Sire, nous sommes touz d’un paÿs et d’une
                  terre. Quant il oÿ ce, si leur demanda en
                  quele maniere il estoient venus a Helcanus. Lors li conta Myrus
               l’aventure comment il avoient fait la bataille
                  entre lui et Helcanus, li uns pour le seigneur de Mal Pas et li autres pour la pucelle du Viez
                        Char, et tout aussi comme il est dit
                     devant ou compte, et comment Clyodorus le trouva et lui avec euls, 
                     qui estoient venu pour lui faire aide. Par foy, dist Gasus, ci a gente aventure, et
                  moult faites a prisier, et aussi fait celui a qui vous estes. \pend
            \pstart Ainsi ont parlé ensamble tout le jour de ces choses.
                  Pelyarmenus, qui au Biau Manoir estoit, o lui grant plenté de ses
               barons, appella Malquidant, 
                  sonle chevalier qui Myrus avoit apelé de la
               bataille. Lors li dist : Amis, venez avant et me
                  dites que il vous 
                     est avizsiet de cest chevalier qui ainsi s’est pouroffers
                     devant moi.
               Sire, dist il, sachiez que je sui moult liez de ce
                  que il a empris, quar bien sachiez que moult ert ainz demain heure de 
                     tiercevespres empiriez de ses plaies.
               Ore y parra, dist Pelyarmenuz, que vous ferez. Par Dieu, je vous 
                     aien ai en couvent que, se vous 
                     au desseure de lui pouez venir, je vous acroistrai vostre 
                     terre touz les jourz de x besanzrente de x b. chascun jor.
               Sire, dit il, granz mercis ! Lors ont aucuns
               dit : Sire, ou volez vous que ceste bataille soit
                  ?
               Je veul, dist il, 
                     qu'elleelle eleLa parataxe à l'oeuvre dans V2 n'est pas
                     habituelle. L'autocorrection du copiste sur ce passage va dans ce sens.
                  soit en celle praerie ci dehors. Et
                  mandez a Gasus que il 
                     face 
                           si bien ordener
                     ordenne si bien
                  
                     la chose
                  
                     et si a point
                     aparaut si bien la chose et si ordeneement que je n’en soie repris d’omme qui vaille, quar je sai bien que li messages ne s’en puet 
                     partirtorner sanz domage. Et se ainssi estoit qu’il en venist au desseure, tout cil
                  soient honni qui mon pain 
                     menguentm’envient se il l’en laissent aler a coroies 
                     ointesdesointesLe passage n'est pas simple à comprendre, comme
                     l'attestent la variation entre "m'envient" et "menguent" et la variation entre
                     les deux antonymes "ointes" et "desointes". La variante de G semble la
                     meilleure à en croire la leçon conservée au §149 et la congruence entre un
                     verbe culinaire et la référence au pain, même si le sens culinaire est ici
                     métaphorique. "Manger du pain de qn" signifie "être en bon rapport avec qn"
                     (DMF); formule attestée dans un contexte assez proche du nôtre, où un frère
                     demande aux siens de le venger. L'expression "s'en aler coroies ointes" est
                     elle connue au sens de "s'en aller sain et sauf" (TL II, 883, 26). Dans ce cas,
                     la version de B est un contre-sens, car Pelyarmenus maudit tous ceux qui
                     laisseraient aller le messager du camp adverse sain et sauf.
               ! \pend
            
            \pstart Pas de nouveau § BQuant
               li ancien l’ont 
                  entenduoÿ, si se sont arriere trait et distrent que non 
                  feroientferoit il. Dont monta i chevalier
               et ne fina si vint en 
                  Constentinoblela cité de C.. Cil estoit amis a Gasus. Il vint a lui et li
               dist que Pelyarmenus vouloit que la bataille 
                     si fust en la Prarie des 
                        BiauxAbiax et que il pourveist 
                     telle lieu 
                     quiquilpar quoi il fust convenable, si que il n’en fust repris de nului qui vausist.
                  Nouveau § BGasus, qui
               doutoit ce qui avenir pooit, vint au chevalier et li dist : Sire, je me doi
                  moult en vous fier, et si fais je. Foy que vous devez a Dieu, savés vous nient de
                     mon seigneur, se li messages s’en porroit 
                     alerneent partir sanz domage, se ainssi estoit que 
                     li
                           chevaliers qui doit faire la bataille venist au
                        dessusil venist au deseure de sa besoingne ?
               Foy que je vous doi, dist il, j’ai entendu de mon seigneur, mais que ce soit conseil,
                  que, se li messages avoit victoire, 
                     que il savroit bon gré qui vilonnie li feroit.
               Ha ! dist Gasus,
                  que l’en se puet 
                     en luici petit fier ! Et je vous ai en couvent, dist 
                     ilGasus, que, 
                     s’il en y a nul qui vilonnie li face ne chose
                        descouvenablese il est nus qui y face chose qui ne soit couvenable, que il n’i avra 
                     nul pire anemi de moi, quiconques 
                     ce soitsoit cil qui 
                     lile face.
               Par Dieu, dist cil, et je l’otroi ! Dont s’en
               vint Gasus au lieu et l’a fait aprester moult
               gentement. Dont s’en vint au Biau Manoir et 
                  s’est aparuzse apparus
               
                  adevant
               Pelyarmenus. 
                  Et tantost queQuant il le vit, il est venus a lui et li dist : Sire, 
                     jemoult vous loeroie que vous ceste bataille feissiez faire si souffisaument que
                  vous n’en fussiez repris d’omme qui vausist ne pres ne loing.
               Gasus, dist il, doutez vous dont que ainssi ne vueille je
                  que 
                     il soit fait ?
               Sire, dist il, je non, mais je doute d’aucuns, se il 
                     cuidoient ou veoient que Malcuidant en eust le
                  pieur, qu’il ne se metent avant, par quoi vous en soiez blasmez et 
                     que vous en fussiez trouvez en vostre tort.
               Par Dieu, dist Pelyarmenus, je ne le voudroie mie. Et je vous commant et ai fait 
                     autre foiz que vous si bien soiez pourveuz par quoi nuls tors n’i puisse 
                     estre fais.
               Sire, dist il, volentierSur la chute de la consonne finale, voir analyse linguistique. le
                  ferai. Atant s’est de lui 
                  departizpartiz et 
                  acueillia acueilli chevaliers et serganz tiex que il 
                  cuidoitcuida
               
                  
                  bien que 
                  bienil miex se 
                  peustdeust en euls fier. Dont leur commanda que il fussent a l’endemain tout prest pour
               fournir ceste besoingne dont il les 
                  requeroitavoit requis. Il 
                  respondirentli r. touz que ja de ce ne 
                     se doutast. Il 
                  ens'en vint arriere a son hostel et dist a Myrus ce
               que il 
                     cuidoitcuida que bon fust. \pend
            \pstart Pelyarmenus,
               qui i frere avoit, tel comme devant est dit, 
                  lequel estoit apelez Dyalogus, ciloÿ la parole en tele maniere comme
                  dit estil l'ot parléLa reprise du sujet par "cil" s'explique par l'emploi
                  de nombreuses incidentes. Cf. analyse linguistique que tous ceuz fussent honni qui son pain menjoient s’il en laissoient
                  aler le message a coroies desointes.
               

                  Cil"Cil" désigne ici Dyalogus,
                        comme B le précise dans sa variante. a dit a lui meismes que il fera une partie de la volenté son frere, se
                        il onques puet. Et cil estoitCil avoit non Dyalogus et estoit tiex que par raison, se il estoit traitres et mauvais, il ne fourlignoit
                  mieL'emploi du verbe "fourligner" est tout à fait adapté
                  à la situation: il rappelle la parenté entre Pelyarmenus et Dyalogus, son frère
                  bâtards, tous deux représentants du dévoiement de la lignée du cycle.. Cil
               a fait une cueillete de traitres et de mavais ribausFaire
                  une "cueillette" de personnes, c'est-à-dire une levée d'hommes, est une
                  construction premièrement attestée dans le Roman des Sept
                     Sages et le cycle qui en est issu (Gdf II, 391c). Dans les exemples tirés
                  de cette tradition, on cueille plutôt des gens de bien que des traîtres. Cet
                  emploi est probablement caractéristique de la manière dont le Roman de Pelyarmenus renverse le cycle. et leur dist que, se il vouloient servir Pelyarmenus a gré, que il fussent a l’endemain appareillié pour le chevalier metre a mort qui ainsi s’estoit
                  pouroffers 
                     contre lui et qui la bataille devoit fairepour faire la bataille contre Malcuidant. Cil ont
               respondu que il ne s’en porroit mie 
                     legierement partir sanz perte, ja
                  si bien Gasuz ne 
                     l’ens'en
                  
                     savroitporroit garder. Et celle manie furent bien iic, que gentil 
                  homme que vilain. Ycelui jour passa, et l’endemain vint que Gasus fist aprester Myrus. Et
               sachiez que il fu moult a sa volenté, 
                  et d’armeures et de cheval, et tout aussi furent si compaignon 
                  et 
                        furentil furent tout apresté. Gasus le 
                  fist au mieus quefu si com il le 
                  potpot miex faire. Et 
                  encore avec tout ce fist aprester tout aussi furent cent serganz hardiz et preuz. Et sachiez que la citez fu moult estourmie d’uns
               et d’autres, meismes de bourgois, de dames et de damoiseles qui 
                  vodrentvoudroient veoir le champ de la bataille. Pelyarmenus, de l’autre part, avoit fait 
                  faire
               
                  bongrant hourdeiz et fort, quar il vouloit que il y eust dames et pucelles de la cité
               et de hors meismes, et y fu sa femme a tout grant plenté de compaignie. \pend
            \pstart Gasus
               
                  se muts'en vint de son 
                  hostelchastel, a tout Myrus, 
                  et estoient moult gentementqui moult g. estoit atourné. Il furent grant 
                  plentéplen-de de chevaucheurs qui devoient le 
                  champparc garder. 
                  Lors s’en vindrentS'en vint tout contreval la ville. Qui dont veist 
                  unset uns et autres 
                  desaus fenestres 
                  
                        et des soliers
               
                  pour
               Myrus esgarder, bien peust dire qu’il metoient
               grant paine a lui veoir. Li 
                  unsvins (sic) disoit : C’est li plus biaus 
                     chavaliers du monde ! ; 
                  li secons redisoitl'autre disoit : Bien 
                     resamblesamble chevalier adroit ! ; 
                  li tiersl'autre disoit que moult faisoit 
                     i tel homme a 
                     douterdoutel (sic) ; et li quars 
                  redisoitprioitdisoitDans "redisoit", le préfixe "re-" signifie "pour sa
                  part". que Dieus li donnast l’ouneur de
                  la bataille, quar trop estoit Malquidant
                  felon et outrageus 
                     tenuz de touz. 
                  
                        En tele maniere 
                              comme je vous ai dit

                      avoit 
                        gracegrarce de chascun. Tant ont chevauchié 
                  en tele maniere
               que il sont venu ou champ, et fu 
                  mis
               Myrus en son lieu, mais moult pria avant aus dames
               et aus damoiseles
                  , dont il fu moult esgardez et moult 
                        conjoïscovoitiez des plus souffisans que il priassent 
                  a Dieu que Il li 
                  aidast selonc droiture et donnast victoiredonnast victore s. d.. Sachiez que 
                  dont y ot moult poy de ceuz qui ne 
                  priassentaient prié pour lui. Meismes la femme Pelyarmenus,
               qui Ysidoire avoit non, dit en son cuer : Ja Dieus ne vueille souffrir que tu y soies 
                     ne mors ne vaincus 
                     par homme nul, quar 
                     moult iés biaus 
                     de corsa veoir, et li cuers me dist que moult iés
                     preudoms et de 
                        moult
                     
                        hautbon afaire. \pend
            \pstart A ceste parole vint Pelyarmenus, a tout Malquidant, 
                  et avec lui grant plenté de barons. Et moult se contint Malquidant gentement a l’entrer ou champ, et bien sambloit homme de
               grant hardement et de grant prouesce. Dont chevachaSur la
                  forme "chevacher", cf. analyse linguistique. tout entour et pria
               aus dames et aus 
                  damoiselespucellez que 
                  il priassent 

                        queLa leçon de V2X2 est répétitive par l'emploi
                  redondant du verbe "prier" mais elle est plus correcte: Pelyarmenus demande aux
                  dames et demoiselles de prier pour lui. Dans ce cas, la forme le pronom personnel
                  sujet "il" vaut pour "eles" (Ménard, § 54).  Dieus 
                  lil'enque li donnast l’ouneur. 
                  Li auquant l’ont fait et li autre 
                  ont prié
               
                  a Dieu que Ilque Dieu y
               y meist pays. Dont s’en vint Gasus
               a PeliarmenusIl nous
                  semble que ce segment mérite d'être ajouté à cause du complément qu'on attendrait
                  pour "s'en vint" et de la probabilité d'un saut du même au même sur la finale
                  "-us". et li dist : Sire, je veul que li
                  bans soit 
                     orendroit criez de par vous et de par moi que il n’i ait nul si hardi 
                     
                           en toute l’
                                 assambleassemblee seur la vie perdre qui ci soit en armes s’il n’en est requis de par
                  moi.
               Je l’otroi, dist Pelyarmenus. Dont fu crié 
                  esraument seur la vie 
                  perdrea p. que 
                     il n’i eust nulnul n'i eust qui armez fust se ce n’estoit 
                     lepar le
                     Nous ne corrigeons pas le
                           congié en "par le congié". Dans notre texte "se ce n'estoit" est
                        sémantiquement proche de "sans" et les dictionnaires attestent l'existence
                        d'une construction "sans congé de qn" (DMF).
                  
                  
                     c‹t›[c]ongié du baillif. Quant Dyalogus sot ce, si ne volt mie trespasser le ban, mais
               il ne se moustra 
                  en nule manieremie, ainz 
                  fufurent embuchiés en un lieu 
                  moult privé, la ou Myrus ne 
                  peustse peust, 
                  en nule maniere qui fust, repairier en la cité 
                  de Coustentinoble se par euls non. \pend
         
         
            
                  Comment Myrus se combati
                     encontre Malquidant por Helcanus son seigneur deffendre.
            
               
               Enluminure de 16 UR sur deux colonnes. Mirus se bat contre Malcuidant pour
                  défendre Helcanus.
            
            
            \pstart Pas de nouveau § B
                     Ci endroit dit li comptes
                        que, quant li dui 
                     chevalierchampion
                  
                     Myrus et Malquidant furent ou champ en la maniere comme distSur
                     "dist" pour "dit" voir analyse linguistique. est devant , Myrus s’en vint a Pelyarmenus moult 
                     affaitieementa. sus son cheval et li dist :
               Sire, veez me ci l’escu au col, le glaive el poing,
                  l’elme lacié, armé de toutes armes comme chevalier le puet mieus faire, si comme
                  pour 
                     moustrerdemonstrer la
                     besoingneb. et le mesage
                  
                     de monson
                     de son
                  
                     seigneur. Et s’il estoit nuls qui vousist dire que 
                     bien ne l'eust fait et queje bien ne l'aie fait et le mandement que je vous ai 
                     aportéporté
                  
                     de par lui ne 
                     soitestoit couvenables 
                     et sousfisans et que onques ne vint de traïtour ne d’omme 
                     nul qui ne soit 
                     loiaus et souffisanz, je sui cilz qui le ferai de mon cors contre le sien
                  estable. Dont s’en vint Malquidant a
               lui et li dit : Sire, vez
                     me cimoi
                  
                     tout prest, qui di et ferai estable que celui qui 
                     cesttel mandement vous a envoié a creu conseil de felon 
                     traïtouret de traïteur et que vous ne feistes onques 
                     a nul jour envers lui ne envers les siens 
                     chose nule dont il vous deust 
                           en nule maniere tel mandement envoierpar quoi il vous deust avoir envoié tel mant. Et si di que cilz qui le vuelt faire couvenable est traitres 
                     et mauvais, et l’en rendrai 
                     enhui en cest jour 
                     recreantcorpable
                  
                     devant vous, ou je meismes 
                     demourraiy d. pour tel comme je di. \pend
            \pstart Pas de nouveau § BQuant
               il orent ce dit, si fist on 
                  le bancrier qu’il n’i eust nul si hardi qui n’isist du champ 
                  forshors
               
                  que ceulz qui combatre se devoient. Il firent 
                  moult bien cestui commandement, quar nuls n’i demoura 
                  fors les ii champions 
                        qui combatre se devoientse li dui champion non et iiii autres : 
                  l’unsce fu Pelyarmenus, et le baillif, et 
                  ii hausautre dui grant hommes 
                  qui estoient des plus gentilz seigneurs de toute la cité de
                        Constentinnoble, qui aussi bien estoient de l’une partie comme de l’autre. 
                  Et si sachiez 
                        tout certainnement que, quant la bataille dut estre commenciee, que li dui
                     chevalier furent mis d’une part et Pelyarmenus et le baillif de
                     l’autre part. \pend
            Rubrique: "Ensi comme Mirus fist la bataille en Constantinoble
               contre Malcuidant." + Enluminure + initiale ornée B
            \pstart 
                  DontMont furent 
                  maintenant d’euls quatre mis li doi chevalier en leur 
                  lieulieu .... Dont se sont 
                  moult noblement
                  merveilleusement noblementNous corrigeons la version de V2 car la présence de
                  deux adverbes successifs en "-ment" nous semble signaler une erreur que corrige
                  G. poli en leur armes. Et sachiez que d’eulz ii ne peust on 
                  onquesainç mais
               
                  nul jour de cest monde atendre 
                  nule plus cruel bataille ne 
                  nule plus fiere, 
                  quarcar li contes dit que, quant ce vint aus chevaus poindre 
                  des esperonset, il orent les escus embraciez et les glaives empoingniés. Il 
                  s’entrevindrents'en vindrent tant aigrement li uns 
                  contrevers l’autre que 

                  merveilles fu que on ne sot l’eure. Si 
                  se sontfurent entrencontré si cruelment que li escu fendirent et volerent em pieces, et s’entrencontrerent de si grant vertu que li cheval 
                  s’entreoccistrents'entrecontrerent de si grant vertu que li cheval
                     s'entrocistrent, et euls meismes de corps et de pis s’entrehurterent 
                  si et en tele maniere que les hiaumes leur volerent des chiés et furent 
                  adont telsitel mené que il ne sorent 
                  en nule guise qu’il leur fu avenu ; mais gaires ne furent en tele maniere, et a chascuns 
                  esraument son hyaume 
                  remismis en son chief moult apenseement, et puis mistrent les mains aus espees et 
                  ont maintenant
                  ont les escus embraciez, 
                  puis s’en vindrent li uns vers l’autre en 
                  telleautele maniere comme se il 
                  ne s’entreredoutassent de nule riens 
                        du monderiens ne s'entredoutassent (sic). \pend
            \pstart Pas de nouveau § BLors
               peust on veoir dure escremie et pesme, quar tout aussi feroient li uns seur l’autre
               et de tel force comme se maintenant vousissent morir chascuns par samblant. Nouveau § BMalquidant,
               qui plains estoit de tres grant chevalerie, hasta Myrus a ce qu’il le meist 
                  du tout esraument acomme du t. au souffrir mort. Ce virent li auquant 
                  etqui bien 
                  cuidoientcuidierent
               
                  certainnement qu’il ne deust plus mettre 
                  deffendementdeffense
               
                  ena lui, mais en assez petit d’eure aprés pot on bien veoir la chalenge que il y
               mist, quar, si comme je truis 
                     lisantle sang, Myrus, qui avoit resorti contre
                  Malquidant en tele maniere, li vint seur le
               pis tout a i fais et 
                  le hurta de si grant force que il l’abati tout envers en
                     tele manierel'abati envers de si grant force que on cuida bien que il fust crevez. Myrus,
               qui 
                  de pres le hastason cop sui (sic), li sailli sus la poitrine en tele maniere qu’il n’ot pooir de
               lui desfendre. La li dona il tant de cops que il li 
                  embatiembarra le hyaume en la teste. Et quant il l’ot 
                  en tele manieresi maistroié, si li 
                  deslachadeslaça le hyaume et puis li dist que il se rendist
                  recreant ou il li couperoit le chief. Et cil 
                  li dist que de lui pooit il fere son
                  commandement, car il n’avoit pooir de lui desfendre. Dont veul je, dist Mirus, que tu 
                     te desdies dedesdies ceulz que tu as apelez traitres et mauvés et 
                     de moi aussi, qui ai fait le message.
               Ja par Dieu, dist il, ne m’avendra que pour paour de
                  mort 
                     recevoira recevoir en soie ja desdiz! Dont 
                  n’entendin'atendi
               Myrus a el fors que tant que il en 
                  eustot le chief pris et le 
                  demoustrastmoustramoustrast au pueple. \pend
            \pstart Quant cil 
                  qui el parcquel part (sic) estoient ont ce veu, si sont sailli avant, 
                  et meismesne fust
               Peliarmenus, qui de ceste chose s’est moult 
                  esmerveilliezesbahiz. Gasus et li autre dui sont venuz a Mirrus et li distrent que assez en avoit fait. Lors sont si compaignon venuz a lui et
               l’ont moult conjoï, comme celui qu’il ne 
                  cuidassentcuiderentcuidoient jamais ravoir en vie. Dont commanda Peliarmenus que Malquidans fust
               pris et portez en une abbaÿe 
                  qui pres d’ilec estoitla pres. Quant li bailliz oÿ ce, 
                  siil ne pot aler 
                  a l’encontreencontre, car autrement en eust il fait autre chose. Et dont vint a lui uns messages,
               qui li dist : Sire, sachiez qu’il a grant gent
                  embuschiez au Més Guichart.
               Qui sont il ? dist Gasus.
               Sire, 
                     
                           ce dist il, je croi que ce soit Dyalogus de
                     Romme. Lors fu Myrus remontez sus i cheval, comme celui qui
               n’avoit plaie ne bleceure dont il se dolust. Dont a fait Gasus sa gent metre d’une part et leur dist qu’il se souffrissent tant que il eust parlé a Peliarmenus. Il s’en vint a lui et li dist
               : Sire, on me 
                     donnefet a entendre que Dyalogus a gens cuilliz
                  pour faire vilanie a ces 
                     chevaliersgens de Frise.
               Biau sire, dist il, et qu’en tient il a moi ?
                  Lessiez m’en em pés, bien vous en couviengne ! \pend
            \pstart Quant il entendi 
                  cestesa response, 
                  ilsi s’en vint a sa gent sanz plus dire et leur dist : Biaus seigneurs, ou preudomme ou traïteur, je vous ai touz pris pour
                  droit conduire si avant que chascuns le sache. Il est ainssi que nous sommes ci
                  assemblez pour garder le droit de chascun. Si le faisons comme preudomme, car
                  encore nous porroit il avoir 
                     grantbon mestier. Vez ci ces chevaliers qui sont a Helcanus, qui est filz 
                     Cassidorus, l’empereeur
                        de Costentinnoblel'empereur Cassydorus, qui encore est en vie, 
                     si comme je l’entensselonc ce que j'entent. Et il se sont bien escondit de ce que on leur 
                     a mismetoit sus, si comme de cel champ qui est outrez. Si sont aucun au 
                     
                     i
                  Més Guichart, qui 
                     ensus nostre conduit les 
                     cuidecuidentSur l'effacement possible de la marque -nt dans
                     V2GX2, cf. analyse linguistique. metre a mort. Si vous pri a touz
                  ensemble que vous soiez preudomme en tele maniere que je les puisse mener a
                  ssauveté. Cil ont respondu que de ce ne
                  se doutast il mie et que tout aussi avant com 
                     cil yroient yroient ilil yroit il yroient, que ja n’en seroit espargniez ne bas ne frans. Et dont
               se sont 
                  mis...
               
                  a la voie...au retour moult 
                  ordeneemordeneem.... 
                  Et quant il vindrent 
                        endroit leau
                     Més GuichartGasus el premier chiés. Quant il vindrent endroit le Més, dont sont cil 
                  saillisailli hors les glaives empoigniez et les targes a leur 
                  colzcops
               
                  pendues. Et quant Gasus les a veuz, si leur 
                  a ditdist : Comment ! Biaus seigneurs, voulez vous
                  faire tel outrage en mon conduit ?
               Par Dieu, dist li uns, se vous le traïteur ne nous
                  rendez, mal l’enmenrez sanz nostre volenté ! Onques si tost cil n’ot la
               parole dite que Mirrus hurta le cheval des 
                  esperonsespourons et vint a lui, le glaive 
                  baissiéabaissié, en tele maniere que 
                  onques cilil onques ne se sot si bien garder que parmi le cors ne li ait mis 
                  et ferferle fer et fust. Dont n’i ot plus resne 
                  tenuetenu ne d’une part ne de autre, 
                  ainzsi
               
                  se sont entremeslé contre eulz sisont entremellé tant aigrement que il n’i ot nul 
                  d’eulzentr'euls qui n’eust moult afaireOn peut ici hésiter sur la
                  segmentation d'"afaire".. Qui dont veist Gasus cerchier 
                  les traïteurs l’un aprés 
                        l’autrel'autre chastierles rens et les traïteurs , bien peust dire qu’il ne les amast mie de tout son cuer, car, si comme
                     il apparutli parut, il 
                     enmisten mistLa construction de V2 "il en a mort" ("il en a
                     tué") est un peu raccourcie par rapport à celle des trois autres
                     manuscrits. a mort de 
                     tiexceuls
                le jour qui moult cuidoient estre si ami, mais riens ne 
                  leur valut, car il amoit miex la mort 
                  de ceulzd'euls que la destruction des Bretons. Tout
               aussi fesoient li autre qui le champ orent 
                  
                  i gardé, car il n’en y ot nul qui grant paine ne meist 
                  a metre les traïteursaus traïteurs metre au dessouz. Et sanz faille bien y parut, car eulz meismes se fesoient occirre
               et detrenchier, et trop estoient li felon fort, 
                  car touz jours cuidoient avoir secours et aidecomme cil qui cuidoient avoir secours de Peliarmenus. \pend
            \pstart Pas de nouveau § BMirrus et si compaignon virent que li
               preudomme se fesoient occirre pour 
                  
                  e eulz, si s’esvertuerent chascuns endroit soi. Dont n’en y ot nul qui 
                  leurses
               
                  cops osast atendre. Myrrus vit Dyalogus, qui des siens li fesoit grant damage. Il s’est
               tournez vers lui et fiert 
                  le chevalchevaus des esperons, l’espee ou poing, l’escu embracié. Quant Dyalogus le vit venir, sachiez qu’il ne l’a pas refusé,
               ainçois se sont si grans cops donnez que les escuz depecierent et les hiaumes 
                  froissierentfauserent. Les espees, qui furent dures et pesanz, les ont aussi comme estounez. A cel
               cop vint i
               breton qui Elygius avoit non. Cil prist en tele maniere Dyalogus que, vousist ou non, li couvint les estriers guerpir. La eust
               esté sa besoigne 
                  faite et sa fin venue, mais trop 
                  aigrementtost fu 
                  rescouzsecourus
               
                  des siensde ses gens. \pend
            \pstart Pas de nouveau § B
               Que vous 
                     yroiediroie je 
                     la besoingne atargant?V2 conserve la faute et le copiste de G y
                  réagit, mais "la besoingne atargant" remonte à l'archétype, ce qui justifie notre
                  correction selon B et X2. Tant ferirent 
                  les unsl'un sus 
                  
                        l‹s›[l]es autresl'autre que moult fu li plus fors 
                  afebloiezfoible. Mes en la fin, par la prouesce de Gasus et
               des Bretons, mistrent il ceulz a la voie et
               n’i ot nulz d’eulz qui touz liez ne fust 
                  quiquant il em 

                  peustpot la vie 
                  emporterporter. Gasus et li sien se sont retrait vers la cité et n’ont finé si sont venu au  chastel parmi la ville. La a
               fait Gasus sa gent descendre. Et sachiez que moult
               en y ot de bleciez et de navrez, et meismes des leur en y ot occis vi,
               dont il ot moult grant douleur en la cité, car leur lignage le sot. Dont
               s’en sont bien armé ccc qui la verité sorent de la traïson que Dyalogus avoit 
                  faitefait. Dont sont a cele heure issus aus champs pour ceulz atraper, mais il leur fu
               dit 
                  que il s’estoient destournési se destornerent. Quant il sorent ce, dont orent conseil que il se prendroient au cors
                  Peliarmenus meismes. Il vindrent au
                  Biau Manoir et li firent asavoir que, se
               il ne leur rendoit ceulz qui cele traïson avoient 
                  faitefait
               
                  par quoi leur amis estoient occis, que il se prendroient a 
                  son corslui meismes. \pend
            \pstart Quant Pelyarmenus oÿ ce, sachiez que moult fu esbahiz. Dont s’apensa que
                  
                     il li couvenoit humilier. 
                  Il en vint a eulz etSi leur dist : Biaus seigneurs, gardez vous de
                  faire folie !
               Par mon chief, sire, dist li uns, nous vous faisons
                  bien asavoir que, se vous soustenez les traïteurs qui nos cousins ont occis, nous
                  en 
                     prendronsprendront vengement en tele maniere que ja ne demorrons vostre ami.
               Par Dieu, dist il, et je l’otroi, car par mon los et
                  par ma volenté 
                     n’en ont il fait ce que il en ont faitil n'ont fait ce que fait en estne l'ont il pas faitLe copiste de G opte pour une formule plus courte,
                     plus simple.. Et se il est ainssi que vous puissiez 
                     tenirprendre ne tenir nulz de ceulz qui au fait fussent, je les vous habandonne.
               
                  Bien dites, 
                        distrentfirent il. Atant 
                  sont 
                        repairiéreperié arriere et ont mis ceste chose en respit. Gasus, qui
               moult yriez estoit de ceste chose que on avoit 
                  faitefait aus messages, si leur dist : Biaus seigneurs,
                  or ne vous esmaiez, car sanz moi n’i avrez vous 
                     
                     
                        g
                      garde.
               Sire, distrent il, bien y a paru !
               Dont furent 
                  ses plaies cerchiees dedesarmé et firent leur plaies cerchier a bons 
                  miresmires de la cité, qui moult 
                  mistrenty ont mis grant cure 
                  a lui et a ses compaignons, et distrent qu’il les rendroient touz sainz dedenz viii jours, que
               ja le chevauchier n’en lairoient. Gasus
               
                  en cel jor meismes s’en vint mesmez s'en vint ce jour au Bel Manoir. Et quant Pelyarmenus le vit, si l’apela et 
                  li dist : 
                     AmisAmis, por Dieu, comment avez vous esploitié ?
               Ha ! sire, dist il, pour Dieu, 
                     mauvesementmalement est la chose alee a vostre oés !
               Comment ! 
                     dist il.
                Et ne savez vous 
                     quecomment
                  Dyalogus a fait son pooir de metre a mort ces
                  chevaliers qui 
                     ci sont envoiez ?
               Biau sire, dist il, de chose que Dyalogus
                  
                     ait faitene li autre aient fait, 
                     je n’en tiens de riens a moije neent en trai a moiil n'en tiens (sic) de moi rienL'erreur d'accord de G sur "tiens" provient
                     probablement d'un changement de personne vers un passage de l'expression peu
                     commune "je n'en tiens" ("je n'en possède rien en propre" ou "j'en n'en garde
                     rien de secret"?) à la locution plus courante "tenir qch de qqn" dont le sens
                     est très proche de la construction de B..
               Par foi, dist Gasus, 
                     toutevoiestoute-vois est ce grant honte a vostre oés, car ce ne 
                     cuidentcuiderent il mie.
               Ne m’en chaut, dist il, pour eulz m’en est petit,
                  mais pour ceulz de Costentinnoble qui
                  mort y sont, dont leur ami 
                     m'ontmoult aprouchié, 
                     dont j’ai grantde quoi je suis en doute. Et gardez que vous m’en descorpez ! 
                     EtCar je vous commant 
                     que
                     que-que (sic), se vous en pouez nul prendre, que vous leur delivrez.
               Par Dieu, dist il, bien dites ! Et je le ferai
                  volentiers. 
                     EtMais que voulez vous que je face de ces messages ? Car il sont moult navré et
                  blecié. Si n’en faites chose dont vous 
                     soiezn'en soiez blasmez, 
                     jecar je le vous lo.
               Gasus, dist
                     Peliarmenus, vous savez bien quel
                  mandement cilz m’a fait. Et sachiez que je moult folz seroie se je hors de ma
                  force 
                     aloiela aloie pour moi metre en abandon de ma vie perdre. Sachiez que, pour
                  chose que je encore 
                     enne sache, je ne lairai l’empire devant que plus fort m’en getera.
               Dont n'i
                     aa el, dist il, 
                     autre chosefors que vous ceste chose 
                     remandezmandez arriere par vostre lectre, sanz 
                     pluspis faire leur 
                     que fet en avezque l'en fait leur aque//avez.
               Je le veul, dist Peliarmenus. Atant se parti Gasus de
               lui et revint 
                  arriere en Costentinnoble aus messages et
               leur conta ce qu’il avoit trouvé em Peliarmenus. Par foy, dist Mirrus, poi nous poons loer d’aucuns de cest païs, 
                     caret moult de tiex en y a. Dont ne se pot tenir 
                  queMyrus que il n’ait 
                  conteeconté l’aventure 
                  de Ascanus comment ele
                     leur estoit avenuequi leur estoit avenue d'Ascanus. \pend
            \pstart Pas de nouveau § BQuant
                  Gasus l’entendi, si dist : Ha ! du traïteur, pourquoi le m’avez vous celé si longuement
                  ?"Ah! ce qui concerne ce traîrte, pourquoi le
                  l'avez-vous caché si longtemps?"
               
                     Par foi, 
                           ce dist Mirrus, porce quepour ce, dist Mirus, que nous ne savons envers qui nous faisons 
                     ne
                  
                     
                     i mal ne bien, car 
                     nous doutonsdoutions qu’i ne fust si 
                     vostrenostre ami qu’il ne vous en anuiast.
               Or sachiez, dist il, que il est si mes amis que de
                  chose dont je n’i aie a meller 
                     ilil n'est tex que il ne s’en ose entremetre.
               Par Dieu, 
                     ce dist Mirrus, de ce sui je moult liez, 
                     et bien vous sera guerredonné. Et sachiez qu’il nous fera encore a souffrir, se il en a le pooir.
               Je vous ferai avoir bon conduit, dist Gasus, que ja tel ne sera 
                     queque il mal vous face 
                     ne nene face faire.
               Sire, 
                     dientdistrent il, granz merciz ! \pend
            \pstart Ainssi avint que li més furent avec Gasus tant que il furent bien gueri. Et quant ce vint au congié prendre, Gasus leur fist si
               toutes leur besoignes que bien cuidierent 
                  avoiravoir savoir sauf conduit par la 
                  lectreterreLa leçon de BX2 correspond plus exactement au
                  contexte de droite qui évoque l'existence d'un message envoyé à Ascanus.
                  Cependant, l'expression de V2G "avoir sauf conduit de la terre" reste plausible.
                  Elle désigne que le chemin et libre et assure une voie saine et sauve., car
                  Peliarmenus avoit envoié 
                  et mandé a Ascanus qu’il n’eussent garde de lui.
               Mais Dyalogus, qui n’estoit mie recreanz de son
               malice, s’estoit embuschiez en la Forest de
                  Vulgue a tout c traïteurs, qui touz avoient esté en l’autre
               aguet en tel maniere comme vous avez oÿ. Gasus, qui 
                  les messages avoit asseurez et qui de ce garde ne se
                     donnoitmie ne s'en doutoit, a les messages asseurez, si leur dist que hardiement s’en pooieent 
                  repairierrepairierier, car il n’avoient garde de nului. Dont 
                  se sont 
                        asseuréapresté et ont pris congié 
                  de laa euls
                  d'elsOn prend plus couramment congé de quelqu'un de d'un
                  lieu. On note toutefois une attestation de "prendre congé de qq. part", à la fin
                  du XVe siècle (DMF). Le caractère inhabituel de la formule explique peut-être
                  l'omission de G. et se sont mis en leur chemin, si n’ont finé par leur
               jornees 
                  tant que il 
                        se sont approchié de la
                     forest, qui moult grant estoit, si comme devant est
                        dit. 
                        Par aventurepor toutes aventuresqu'ilz sont venus en la Forêt de Vulgue il furent de leur armes 
                  garnibien garnis, 
                  si sont entré en la
                        forest
                     
                         par ou il estoient venuque il erent venu. Dont 
                  n’ontn'orent mie 
                  
                  g granment chevauchié 
                  
                  i quant il ont encontré i escuier qui leur dist : Ha ! seigneur, tout estes mort se vous plus avant alez !
               Comment ! dist Mirrus. Amis, et que as tu veu ?
               Par foi, 
                     ce dist cil, ci devant sont embuschié l armeures de fer qui
                  n’atendent se vous non. Quant Mirrus oÿ
               ce, si fu touz esbahiz et dist : Amis, porrions nous
                  par ailleurs 
                     alerpasser ?
               Par Dieu, dist cil, oïl. Je vous menroie bien par
                  ailleurs, mais que ma poine y fust sauve.
               Par Dieu, 
                     ce dist Mirrus, amis, bien te sera
                  guerredonné ! Dont s’est cil mis au chemin au travers 
                  de la forest et 
                  leuril dist que il le 
                     suiventsuivissent, et il si firent. Et tant 
                  chevauchierent quechevauchie-queont chevauchié queD'après le passage à la ligne après "chevauchie", il
                  y a fort à penser que la bonne leçon sur laquelle V2 commet une erreur était
                  "chevauchierent". C'est pourquoi nous adoptons la leçon de G. il 
                  furent moult esloigniémoult sont loing issi de leur chemin. Dont les a cil menez en i parfont val ou Dyalogus estoit a tout ceulz dont je
                  vous ai devant dit. Quant il y furent, si sont sailli et devant et derriere
               et ont les messages assailliz. \pend
            \pstart Pas de nouveau § BQuant
                  Mirrus et si compaignon ont ce veu, si 
                  ont reclaméreclaiment Dieu, car il virent bien que cil les avoit traÿs et que leur vie estoit alee.
               Adont se sont touz mis ensemble et cil les ont 
                  assaillisassailliez
               
                  devant et derrierede toutes pars, mais bien sachiez 
                  que il ne se 
                  lessoientlaissierent approchier, ainz se sont tant deffendu que il leur ont occis x des
               leur avant que nulz 
                  des lord’eulz leur y fust bleciez 
                  ne navrez. Et quant Dyalogus vit qu’il se
               desfendoient si aigrement, si a ses traïteurs escriez et dist : Seigneurs, vous lairez vous en tel maniere occirre a ces ribaus
                  ? Dont se sont esvertué et ont feru seur eulz en tele maniere que seur
               chascun en 
                  feroientferirentferoit bien x, et qui n’i pooit avenir, il y ruoit. \pend
            \pstart Pas de nouveau § BEn
               tele maniere feroient cil seur les Bretons.
               Et quant Mirrus vit que morir le couvenoit, si
               s’est esvertuez et feri cheval des esperons. Il estoit richement montez au miex de
               touz. Il se feri parmi eulz et 
                  commence a cerchier les rensferi a destre et a senestre, en tele maniere que 
                  il n’i ot nul si hardimal fust de celui qui voie ne li feist, et 
                  en tele guisetout aussien autele guise ont fait si compaignon. Mirrus vit Dyalogus, qui i des siens avoit mis 
                  ajus du cheval a mort. Dont vint 
                  avers lui en tele 
                  maniereravine qu’il l’a 
                  si feru de l’espee qu’il l’abati jus du destrier, vousist ou non. Et bien
               cuidierent li sien 
                  que il fustque il fust que il fust (sic) mors, 
                  
                        maism-mais (sic) non estoitet sachiez de voir que si eust il esté se il eust esté
                     preudons. Mais il 
                  estoit tel atorné que ilnon, ainçois fu tel mené que la destre espaule 
                  avoitot toute rompue au cheoir que il fist. Dont l’ont li sien trait 
                  horsd'une part hors du chapleis. Que vous diroie je ?
               
                  Il ne porent au lonc durer, c’est a ssavoir li messageNe porent au lonc durer li messageLez messages ne porent au lonc durer, quar 
                  ilsi furent 
                  
                        si afebloié des cops qu’il donnoient et de ceulz que il 
                        avoient receuzrecevoient que recroire les couvinttuit recreuz. 
                  Li felon traïteurCil, qui nule pitié n’avoient d’eulz, 
                  les ont tiex menez que 
                  des x ont les viiide x lez viii en ont occis, et 
                  lescil
               ii qui demorerent furent 
                  Mirrus et Elygius. Et quant il virent que leur
               compaignons 
                  estoientfurent mort, si leur ont les dos livrez, mes il ne fuioient mie, que 
                  au retourner que il faisoient aucune foiz contre 
                        ceulz qui les chasçoient les chastierentau retour ne les chastiassent, 
                  sisi bien que 
                  iltuit ont lessiee leur chace. Et quant il 
                  ont ce veuvirent ce, si se sont mis en la grant
                  forest espesse, comme cil qui n’avoient cure d’eulz moustrer, car il ne
               savoient qui amis leur estoit ne qui anemis. Dont chevauchierent moult longuement et
               ne sorent quel part il furent, 
                  etIl fu tart et
               
                  il n’orent en tout le jour 
                  nebeu ne mengié. 
                  EtCar de leur vie estoit il petit, car si blecié 
                  estoientse sentoient et si navré que
                  de nule riensd'autre chose  ne leur souvenoit. Ha ! Myrrus, dist Elygius, je me muir ! Dont s’est pasmez sus son cheval quant
                  Mirrus
               
                  
                  i l’a secoru. Et le tint entre ses braz tant qu’il fu revenu 
                  de pasmoison. \pend
            \pstart 
                  Moult leur avintDont leur vint bien, car illeuc avoit une 
                  gentemoult gente fontaine dessouz i arbre. Il sont descendu de leur chevaus, 
                  
                  i
               
                  carqui moult estoient blecié et navré. Il 
                  leur ontl'ont les frains 
                  ostez
                  ostez es chevaux
               
                  et les ont
               
                  lessié
               
                  aleraleur, et 
                  il se sont prissont alés
               
                  a paistreau paistre de l'erbepaistre. Li dui compaignon sont venu a la fontaine et ont leur hyaumes ostez, qui
               n’estoient mie entiers, 
                  puis ont de leur cotes a armer"cotes a armer" : "leurs cottes faites pour s'armer". leur plaies
                     estoupees, dont moult de sanc estoit issu. Que vous 
                     conteroieceleroiediroie je d’eulz ?
               
                  Tel estoient mené que trop grant pitié estoit d’eulz veoir. Dont commencierent
               leur compaignons a regreter et eulz meismes a dementer 
                  et grans souspirs et granz lamentacions a faire. Ainssi comme il estoient en tel
               point, 
                  esteses vous que i
               joenes hom en habit de relegion
               
                  a tout
               i baril en sa main est ve
                  n‹u›[n]uz a la fontaine. 
                  QuantLors quant il vit Mirrus
               
                  et son compaignonson compaignon, si ot aussi comme paour et 
                  s’est traitse trait
               i
               
                  poi arrierearriere. Quant Mirrus
               
                  le vitl'a veu, si s’est levez a moult grant poine et 
                  est venuzvint a lui. Et quant cil le vit venir, si s’est a la voie mis. Et Mirrus commença a 
                  crierhuchier aprés lui qu’il n’eust doute. Et
               quant plus li disoit, et mains
                  s’asseuroits'asseuroit de lui cil. Et dont se 
                  mistmist grant aleure
               a la voie vers i
               hermitage qui pres d’ileuc estoit; et
               quant il y vint, si est enz entrez et a son huis fermé devant Mirrus. Et quant Mirrus vit ce, si
               ne voult lessier qu’il ne soit par la venuz. \pend
            \pstart Dont commença a huchier et dist : Frere, pour Dieu, lessiez moi enz entrer, car grant mestier ai
                  d’aide ! Quant cil 
                  entendi cel'entendi, 
                  sidont en vint en une chambre ou il avoit i
               viel homme
               
                  d’aagede tres grant aage, et estoit si viex 
                  qu’il ne veoit gouteque la veue il avoit perdue
               
                  et 
                        ooitdort ou poi ou nient. Dont cria li jouvenciaus : Sire, pour Dieu,
                  ne sai quiex gens ai trouvez a la fontaine. Et sont touz couvers de fer, et je
                  m’en sui ci 
                     afouifouiLa leçon de G apparaît comme une banalisation du
                     verbe "afuir" dont les attestations sont pourtant nombreuses dans les
                     dictionnaires., et li uns si m’a suivi 
                     jusques 
                           ciicy et veult ceenz entrer.
               Amis, dist il, laisse le enz entrer et le fai parler
                  a moi. Cilz ne voult mie trespasser son commandement, ainz vint a l’uis
               tout 
                  tremblanttramblant de paor et l’a ouvert et dist : Sire, bien soiez vous
                  venuz !
               Amis, dist il, Diex vous beneyee ! N’aiez ja paour
                  de moi.
               Sire, dist il, si ai eu. Mais venez ça parler a mon mestre. Dont enmena Mirrus en une chambre ou li
                     preudonsvieux hons se gisoit. Quant il le vit, si 
                  s’assist d’encostes'est assis dejouste li et l’a salué, mais ce ne fu pas si haut que cilz l’ait entendu. Quant li jouvenciaus li dist Sire, parlez 
                     hauthaut, car il n'ot mie bien, adont parla si 
                  
                  l
               
                  haut que bien l’entendi. Et quant li preudoms
               l’oÿ, si 
                  demandali demanda
               qui il estoit et dont il venoit et comment il li
                  estoit avenu. Et cil li conta tout en la
                     maniere que avenu li estoit.  \pend
            \pstart Quant li preudom l’
                  aot entendu, si 
                  s’estest tournez vers lui, ce que il n’avoit 
                  pieçamoult grant piece fait se on ne li avoit aidié. Dont dist : Sire chevaliers, 
                     sachiez que moult aipar Dieu, bien ai mainte foi oÿ parler de celui a qui vous dites que vous estes. Et sachiez que sa mere demoura ceenz avec moi pres de
                     v anz et demi. Quant Mirrus
               oÿ ce, si ot 
                  grantsi grant merveille si que il 
                  oubliaentroublia toute 
                  sala douleur que il sentoit. Dont dist : Ha !
                  sire, pour Dieu, souffrez vous que je voise querre mon compaignon qui moult est au dessouz de sa force ? Car 
                     je ai grant 
                     paourpaor ai de lui se je tant demouroie que 
                     vous m’eussiez dit aucunes nouvelesje seusse de vous aucunes nouveles et aucunes choses que je vous veul demander.
               Alez, dist il, car vous mais hui ne demain ne 
                     departirezpartirez vous de ci, se vous 
                     m’enme creez.
               Sire, dist il, grans merciz ! Dont 
                  est venuss'en vint
               Mirus a son
                  compaignon et le trouva a 
                  grantmoult grant meschief et li dist : Amis, nous yrons ci
                  pres 
                     
                     a a i
                  hermitage. Et la a i
                  saint homme qui nous aidera et conseillera mais
                     hui. Quant il oÿ ce, si fu moult liez et s’est 
                  levezdreciez a moult grant paine, et Mirrus a pris 
                  lesleur chevaus, et s’en 
                  sontsont tot en tele maniere alez a pié, 
                  tant que il furent a l’ermitagedesi a l'ermitagea l'ermitage. \pend
            \pstart Quant il furent la venu, li jones 
                     homhermites a 
                  presentement pris 
                  leurles les chevaus et les a mis en une açainte et leur donna orge, 
                  dont il 
                        mengierent moultont mengié volentiers. 
                  DontLorsAprés est 
                  repairiezretornez
               
                  arriere a eulz et leur dist : Biaus seigneurs, 
                     vous aiderai jeaiderai vous plus de riens ?
               Amis, distrent il, bien avrions 
                     
                     i mestier d’aide, mais mauvaisement en estes aaisiez. Dont ne sorent
               que faire de eulz desarmer pour leur plaies, et d’autre part il n’avoient nules robes
               dont il se peussent vestir, et tant qu’il 
                  ses'en sont moult complaint. Et quant li jouvenciaus oÿ ce, si 
                  le distl'a dit au viel homme. Et dont 
                  apela illes apela
               
                  Mirrus
               
                  et li dist et 
                  ilMirus
               
                  vinty est venus
               
                  a lui et dist : Sire, que vous plaist ? Je sai,
                  dist il, que vous estes 
                     trop fort navré. 
                     DesvestezDesgarnissiez vous de vos armes et vous vestez de tiex 
                     vesteuresdrasarmes
                        vesteures que je vous porrai donner dusques a tant que je vous ensaignerai qui de vos
                  plaies vous porra 
                     gueriralegier.
               Sire, dist il, 
                     gransmoult granz mercis ! Dont a commencié a desarmer Eligius. Et li
                  jouvenciaus li a aporté une cote chaude, et i gris mantel li a
               fait 
                  afublervestir. Et quant il 
                  se vit ainssi appareilliezle fu, dont ne se pot tenir que il 
                  ne sourrisistn'ait souzrizLa variante de B, qui ne correspond pas à la
                  concordance habituelle du texte, s'explique peut-être par la rareté de la forme
                  "sourrisist" dont aucun dictionnaire ne rend compte., et tout aussi fist
                  Mirrus. Aprés 
                  s’est faitsi s'estse fist
               
                  desarmer et se vestivestir tout en 
                  autretelautele maniere comme 
                  son compaignon
                     avoitcil estoit fait. Quant il furent ainssi atourné, sachiez que
               
                  moult bien semblerent hermite. Dont s’en vindrent devant le preudomme et li prierent pour Dieu que il les conseillast de
               leur plaies, se il savoit. Par Dieu, dist il, 
                     bienoïl, bien vous en est 
                     avenucheu, car en nul mois de l’an on ne porroit recouvrer de 
                     l’erbel'erbe fors en cestui dont je vous garirai 
                     fors en cestui 
                           porpristous si com de plaie d'armeure trenchant ! Dont apela le jone
                  hermite et li dist : 
                     Amis, alez en cele court, la derriere, et prenez 
                     enulaemyla (sic)L'erreur commune à V2GX2 s'explique certainement
                     par le caractère spécifique et savant du nom botanique "enula" du latin "inula"
                     "aunée" (DEAFplus). Le terme est en effet plus attesté en ancien et moyen
                     français sous sa forme héréditaire "eaune" (FEW IV, 784b-785a).  et
                  autres herbes que vous y trouverez delez, 
                     quilesqueles sont trop bonnes. 
                     Aprés siEt les me lavez moult bien et les triblez entre vos mains, si 
                     espandez le jus susleur espandez en leur plaies partout la ou il les vous mousterront. Cilz 
                  l’aa fait tout en tele maniere comme 
                  li preudom li
                     enseignail l'a dit. Et quant cil sentirent la medecine, si leur commença 
                  plus etde plus en plus la douleur 
                  de leur plaies
               
                  a assouagierassouagier. Dont loerent Dieu de ceste chose. Lors leur demanda li hermites
               comment 
                     il leur estoitil se sentoient, et il distrent : Sire, il nous
                  semble que nous soions em paradis !
               Or vous souffrez, dist il, car dedenz iii
                  jours 
                     vous n’en lairez jane vous couvendra la por ce laissier a chevauchier. Mais 
                     orce me dites 
                        puis quant vous 
                        ne mengastes ?
               Par foi, sire, 
                     distrent il, nous ne beumes hui ne ne menjames.
               Si fis, dist Elygius. Je bui 
                     orainsorehui de cele fontaine en cele forest. Dont leur a fait aporter li hermites blanc pain d’orge et de segle et yaue
               clere, ytele viande comme il menjoitLe pain d'orge et de
                  seigle, de mauvaise qualité, constituent la nourriture des ermites et des moines.
                  Voir à ce sujet un exemplum 568 du Ci nous dit. Le terme
                  "viande" désigne ici plus largement leur nourriture, très frugale., et dist
               : Biaus seigneurs, mengiez de 
                     cel pain tel comme nous l’avonstel pain com nous avons, 
                     car sachiez, se 
                           je miex eusse, plus volentiers le vous donnasse.
               Ha ! sire, distrent il grans mercis ! Dont
               ont mengié et beu 
                  apar talent. Et quant il 
                  ont cel'ont fait, si ont mis le preudomme a raison 
                  et li ont demandéUne correction n'est peut-être pas nécessaire ici.
                  Une construction elliptique avec "que" figure par exemple dans le Miracle de saint Panthaleon (DMF).
               comment ce avoit esté que la mere Helcanus avoit
                  demouré avec lui tant comme il disoit. Dont leur a compté 
                     en quele manieretout ainsi comment
                  ele estoit la venue et comment ele avoit
                  demoré en l’ermitage, et tout en
                  autele maniere comme li traïteur du paÿs l’avoient traÿe et comment ele
                  fu arriere en 
                     lasa seignorie. Tout leur 
                  cc‹...›[o]mpta, 
                  queque il de riens n’i failli, 
                     en tele maniere comme devant est dit. Et quant il ont ce oÿ, sachiez que moult 
                  s’en merveillierents'esmerveillierent, car onques mais n’en avoient oÿ parler. En tele maniere demorerent la nuit o
                  l’ermite, qui les aaisa de ce que il pot. \pend
            \pstart Quant ce vint a l’endemain et il fu grant heure,
               dont firent leur plaies remuer, et leur sembla qu’il guerissoient 
                  tout. Dont dist li uns a l’autre : Comment porrons
                  nous partir de cest paÿs ? Nous savons bien que, se nous sommes 
                     aperceuzperceu, que nous sommes 
                     
                     t mort. Dont souvint a Mirrus de la pucele qui si grant bien leur avoit fait et 
                  puisque ele leur pria que il par 
                  eulzli
               
                  repairassentretornassent, mais puis penserent a ce qu’il avoient son
                  frere mort, si ne sorent que faire : ou d’envoier arriere pour secours a
                  Gasus, ou a la
                  pucele pour savoir qu’ele voudroit dire de ce qu’il 
                  devoientdurent par 
                  la retornerli revenir. Par foi, ce dist Mirrus, nous nous en conseillerons a nostre oste.
               
                  Et il si firentDont s'en vindrent au preudomme et 
                  lileur distrent ceste chose. Quant il
               l’entendi, si 
                  leur dist : Je vous dirai que vous ferez. Bien a passé xxx anz que 
                     le pere a la pucelle et la mereli peres et la mere a la pucelele pere a la pucele et a la mere
                  
                     m'ontmoult
                  
                     soustenuausi comme soustenu en cestui paÿs. Et sachiez qu’ele est 
                     de sid'ausi bonne vie et si sage que ele ne feroit chose 
                     ou il n’eustsanz bonne raison. Je li 
                     
                     i manderai par mon
                     desciple que vous 
                     estesci estes en tel point ci et que ele mete conseil en vous par quoi vous
                  soiez arriere en Frise sanz le seu 
                     de celui et de ceulz qui vous ont mis en tel point, et je sai bien que ele en fera son
                  pooir. Et puisque ele 
                     s’en meslerale fera, dont sai je bien que la besoigne 
                     seraen sera bien faite.
               Sire, 
                     dientdistrent il, pour Dieu, pensez 
                     enty ! \pend
            \pstart Dont a maintenant li
                  preudons apelé son
                  desciple et li dist qu’il alast a 
                     GonfortConfort et parlast a la pucele, qui avoit non Melode, et li deist de par lui que ii
                  
                     chevalierssi grant ami li estoient venu aussi comme 
                     d’aventurepar grant aventure, et chevalier estoient, 
                     liquel estoient a si grant meschief que plus ne pooient
                        estre, si 
                     qu’ele meist conseil qu’il peussent sauvement aler en Frise, dont il estoient. Ha ! sire, pour Dieu, dist Mirrus, merci. Je
                  vous ai dit que i
                  sien frere fu a la bataille occis de l’agait
                  qui nous fu fait au Vil Pas. Ja si tost
                  n’orra parler de Frise
                  
                     qu’ele savra bienque maintenant savra que nous soumes cil qui son frere
                  
                     avonsont occis.
               Ne vous chaut ! dist li preudom. S’il est ainssi qu’ele ne vous veulle aidier, ele ne vous
                  grevera mie, car encore vaudroit ce pis se elle savoit que vous fussiez autres. Et
                  puis quant vous venriez et ele vous recongneust, si en fust esbahie. Et puis se
                  ele vous vouloit mal, si 
                     en seriez plus entrepris.
               Sire, distrent cil, bien dites. Dont fu cil
               bien escolez et se mist a la voie et ne fina si vint a 
                  GonfortConfort, car il n’i ot pas plus de v lieues 
                  petites. \pend
            \pstart Pas de nouveau § BQuant
               il fu la venus, 
                  si fu bien congneus de chascun, comme cil qui 
                  souventchascun jour y venoit. Dont fist il tant que il parla a la
                  pucele et la salua de par son maistre.
                  Bien ait il ! dist ele. Et comment le fait
                     Ydoinies, mes bons amis ?
               Par foi, 
                     damoisele,damoisele, dist il,
                  
                     il est sisi est mais viex que de riens ne se puet mes aidier. Si m’envoie 
                     aci a vous pour une besoigne dont il a en vous moult grant fiance, car
                     ii
                  
                     de ses bons amissi grant amide ses granz amis le vindrent ceste semaine veoir 
                     adont a moult grant doutance 
                     et tout ausi 
                           sont il en grant douteont il d’aucuns maufeteurs dont il sont aguetiez, si n’osent mie apertement aler
                  que il ne soient cougneus.
               Et dont sont il ? dist la damoisele.
               Il sont, dist il, de Frise.
               Et quel gent sont 
                     ilce?, dist ele. 
                     Damoisele, 
                           dist il, il sont chevalier. Et quans sont il ? dist
               ele. Damoisele, 
                     il sont iideus, 
                  dist il. \pend
            \pstart Quant la 
                     damoiselepucele oÿ celui 
                  en tele manieresi faitement parler, si pensa aus messages 
                  qui par lui 

                        vindrentvind-drent (sic), qui bien estoient x
                     
                        ou plusqui la estoient, mais plus de x estoient. Si fu aussi comme toute seure que 
                  ce n’estoient nulz de ceulznul de ceuls n'estoient. Et nonpourq
                  uan‹u›[uan]t les atendoit ele de jour en jour et les fesoit espier par 
                  les cheminsle chemin ou il devoient passer, que pour ce ne lessassent mie a venir 
                  par ileuc pour cause de son
                        frere se il avoit esté occispar li que ses freres avoit esté mors en l’aguet, car il l’avoient fet sus leur droit. 
                     Amis, dist la pucele
                        
                           a celi,Si dist la pucele a celui dites 
                     a mon bon ami Ydoine qu’il m’envoit ce
                  qu’il voudra, et je ferai 
                     pour eulzd'euls bonnement leur requeste. Dont dist cilz : Dont les atendez a 
                     demaindemain au soir que il mouveront de nous bien tart et aiez le portier tel 
                     atourné que il puissent ceenz entrer sanz 
                     destourbierdebat, et nulz fors que vous ne soit sages d’eulz.
               Bien en ferai la besoingne, dist ele. Dont li
               a 
                  chargiéchargiés
               ii barilz de vin et pain blanc de fourment et iiii pastez de
               chapons et autres viandes assez et 
                  leur mandali dist que, se il avoient mestier de chose que ele
                  eust tant qu’il fussent 
                     en cest paÿschiez lui, qu’il 
                     renvoiassenten venist a li. Damoisele, dist cil,
                  volentiers. Adont 
                  se mists'est cis (sic) mis au retour et ne fina si vint a son
                  hermitage. Il fu tart 
                  ainçois que il y venistquant il y fu venuz. Et quant li dui compaignon 
                  le virent revenirl'ont veu, si en furent moult lié et moult 
                  joiantjoyeux. Nouveau § B

                  Il liLors enquistrent comment il avoit
                  besoignié, et cil leur a compté tout en
                  autel maniere comme la damoisele leur avoit 
                     mandémandé par lui. Dont furent 
                  joiantjoyeux et si en ont Dieu loé. Ceste 
                  
                  i nouvele ont 
                  conteeconté au viel hermite, qui moult 
                  en fu 
                        joianzjoyeus et moult en fist grant joie. Dont 
                  s’apresterentse sont apresté de souper, car 
                  il en avoient 
                  grantbien mestier. Mais onques li 
                  dui hermiteviex preudons ne li jones ne voudrent boivre ne mengier de viande qu’il eust aportee, car 
                  onquesil onques en nul temps ne menjoient que une foiz le jour, et ce estoit pain et yaue 
                  tant seulement
                  seulement. \pend
            \pstart Pas de nouveau § BQuant
               li dui compaignon se furent bien aaisié, 
                  ilsi alerent dormir. Et quant ce vint au matin, il se leverent et firent prendre
               garde a leur plaies, qui moult bien guerissoient, ce leur sembloit. Et furent toute 
                  jour ileucle jour jusques a heure de vespres, que il ont escouté moult grant 
                  fraintefrienteLes deux synonymes dérivent de deux étymons
                  différents (frangere FEW III, 753a et fremitus FEW III, 774a). Sur un commentaire du FEW sur fremitus, qui
                  explique un rapide changement de genre en afr > fremita et le lien possible vers
                  crainte/fremour, voir FEW III, 774a. de chevaus. Adont ont eu en doute
               qu’il ne fussent espié, si n’ont eu plus de secours 
                  qu’ilfors que il sont venuz 
                  devantseoir devant
               Ydonie.
                  Lors ne demoura gueres que li
                  hermitages fu touz pourpris de maufeteurs qui par le bois avoient quis et
               queroient les ii compaignons qui eschapez leur estoient. Dyalogus, aussi comme 
                  par affitaffit, quant il vit l’ermitage, si
               dist que il se vouloit confesser a l’ermite et mist pié a terre et est
               venuz en l’ermitage et trouva le jone hermite seant enmi la
               maison, 
                  sidont li demanda : Quel gent a il ceenz ?
               Sire, dist cil, nous y sommes.La réponse de l'ermite ne semble pas tout à fait convenir à la
                  question posée. Peut-être est-ce une façon de traduire la terreur du jeune
                  homme.
               Et ou en a plus ? 
                     dist il.
               Sire, dist cil qui ot paour, en cele chambre est mon mestre et ii autres
                  preudommes. Dont vint Dyalogus en la
               chambre et trouva 
                  le viel hermite et les
                        ii chevaliers en 
                        l’abitl'abit dez autres hermitezles ii preudommes qui ainsi estoient en abit, 
                     comme devant est dit. Dont cuida qu’il fussent
                  hermite et les salua et dist : Liquiex est
                  li maistres 
                     de touzde vous ?
               
                  Et quant Mirrus l’a
                     veuMyrus, quant il le vit, si l’a cougneu 
                  apar ses armes. Dont 
                  neil ne li voult respondre de courrouz, ainz li fist signe que 
                  ce estoit cilz qui estoit el litcil du lit estoit li mestresce estoit cil du lit. Dont sont andui 
                  issuissi de la 
                  chambrechamble (sic), et Dyalogus s’est assis delez lui et li
               dist : Sire, 
                     tornez vous!dormez vous? Dont ne li a mie 
                  cil respondu, qu’il ne l’
                  aavoit pas oÿ. A l’autre foiz l’a si haut escrié que cil qui dehors l’atendoient s’en
               sont gabé. \pend
            \pstart 
               Pas de nouveau § BQuant Ydoine oÿ Dyalogus, si
               li 
                  a demandédemanda
               qui il estoit. Je sui, dist il, i chevalier malfaiteur. Et dont
               li a commencié a compter moult de douleurs que il
                  avoit 
                     faitesfait et de teles qui onques de lui n’avoient esté assouvies, mais ce
               disoit il pour lui tempter, meesmement 
                  commentli commença il a conter comment il avoit fait les messages traïr et occirre. Et quant 
                     YdoniesYdoinePour les variantes orthographiques de
                  ce nom nous renvoyons à l'index. l’ot escouté, si li dist : Amis, 
                     en estn'est ce malfait?
               Par foi, dist Dyalogus, je sai bien que ce n’est mie si bien fait 
                     comme une fine merveille, mes ainssi me plaist. Dont li dist Ydonies : Dont n’en estes vous mie
                  repentanz ?
               Oïl, dist il, en tele maniere que je voudroie 
                     
                           jala tenir les autres ii qui hier nous eschaperent en tele que je peusse tenir les autres ii qui hier nous
                        eschaperent par maniere que jamais jour de ma vie ne deusse bien faire ! \pend
            \pstart Quant Ydonie l’a
               entendu, si a fait le signe de la vraie crois seur lui et 
                  li dist :Va de ci, 
                     anemianenusavenu! Je te conjur de par celui 
                     quequi j’ai servi grant temps que tu n'aies

                     nul pooir de ceenz plus arrester, se ainssi n’est que 
                     tu aies volentévolenté més de toi amender. Dont sembla a Dyalogus que x
               
                  deables le 
                  pristrent par les costez etpreissent li un par les costez, li autre par les espaules et l’ont porté 
                  hors de leenz par tel vertu que il cuida 
                  bien estre mors. Et quant il vint hors, si l’ont 
                  mis jus en tel manieresi mis jus qu’il n’ot pooir de lui aidier. Et quant il vit ce, si sot 
                  maintenant que ce estoit aussi comme venjance de Dieu, et pour ce ne se repenti il mie,
               ainz jura que, s’il eust illecques feu, 
                     queil 
                     arsistardroit
                  l’ermitage et ceulz qui
                  dedenz 
                     estoientqui laiens estoient. Dont dist li uns : 
                     CeHa! sire, ce ne ferez vous mie : on ne doit mie faire touz les maus
                     que on porroitLa tournure proverbiale ne se
                     retrouve pas dans les recueils mais établit une logique propre au roman, qui ne
                     cesse de réfléchir au mal, qu'incarne notamment Pelyarmenus., 
                     car sachiez 
                     que il sont bonne gent.
               Pource, 
                     distce dist
                  li desloiaus, que je sai que il sont bonne
                  gent le voudroie je faire ! Atant ont remonté Dyalogus et 
                  se sont d’ilec partiz par tele aventures'en sont ainsi alés
               
                  
                           comque vous avez oÿ. \pend
            \pstart 
               Quant li dui compaignon ont ce 
                  veuoÿ, si 
                  ont ditdistrent que Diex les avoit visitez par biau miracle et 
                  si distrent qu’il ne se devoient mie desesperer pour 
                  chosenule chose qui avenue leur fust quant 
                  Diexainsi les avoit delivrez de ceulz qui pour occirre les queroient et si 
                  lesles
               
                  avoient veuzveoient a leur iex. Dont sont venu a Ydonie et li
               ont dit : Sire, 
                     ceulz de ci s’en vontcil qui deci en vont, ce sont celz qui nous ont mis en tel point comme nous vous avons
                  compté.
               Mi bel enfant, 
                     dist il
                     dist-dist (sic) il, j'aSur "j'a" pour "j'ai" (BGX2), voir l'analyse
                     linguistique. ce feu par ce deable qui de ci s’en va, car il venoit a
                  moi a confession et si n’estoit de riens repentanz. Et Dieu, qui maint bien m’a
                  fait, m’en a delivré. Dont parlerent de
                  mout de choses et meesmement a la requeste de l’ermite
               
                  se sont li dui compaignonles a an ii
                  Nous écrivons habituellement "andeu" en un mot, sauf
                     ici où cela est impossible.
                confessé moult curieusement, et les a 
                  li sains hom
               
                  endoctrinezdoctrinez et mis en bonne voie de bien faire. Et quant il orent ce fait, si fu nuit, et
               se sont apresté d’aler 
                  ent pource que il 
                  savoientsorent
               
                  bien que la pucele les atendoit. 
                  IlLa nuit se sont garni de leur armes et ont pris congié 
                  a l’ermiteau preudomme moult 
                  doucementbonement et l’ont moult me
                  r‹c›[r]cié de ce que fait leur avoit. Atant sont monté 
                  seur leuresseur les
               chevaus, et li joenes
                  hermites les a mis a la voie, qui ne leur 
                  pooitpot faillir jusques a 
                     GonfortConfort. Quant il furent la venu, si fu moult grant 
                  partiepiece de la nuit alee. Il 
                  sont venuvindrent a la porte et ont huchié que on les 
                  lessast enz entrerouvrist. Li portiers, qui sages en
               estoit, 
                  au plus tost que il pot leur 
                        a ouvertea ouvert la porteleur ouvri et leur dist : Biaus seigneurs, 
                     vous soiez les bienvenuzbien soiez vous venus !
               Amis, distrent il, Diex te beneye ! Dont les
               a cilz fait descendre et leur dist que il
                  l’atendissent illeuc tant qu’il eust mis leur chevaus 
                     en l’establea estable, 
                  et il si firent. Dont n’est mie cilz trop demorez que 
                  il revint 
                  a eulz et leur dist : Biaus seigneurs, venez aprés
                  moi.
               
                  Dont les a cilEt il lez a
               
                  menezmenez amontamenés en une moult riche chambre, et de cele en une autre, et 
                  dont
               
                  ont trouvétroverent
               la pucele, qui les atendoit, o lui une 
                  damoiselepucelle. Et quant ele a veu les chevaliers, si 
                  est venuevint contre eulz et les a moult 
                  tres belbienbel saluez. Et ceulz li ont 
                  leur salus rendusson salu renduson salu rendus, et puis li dist Mirrus : 
                     Damoisele,Damoisele, pour Dieu, comment vous est ?
               Sire, dist ele, ainssi comme 
                     a Dieu plest. Et qui estes vous ? 
                     Me connoissiez vous de noient ?
               
                     Par foi, dist il, si fais, maisHa! damoiselle il me semble que vous m’avez 
                     ja mescongneu, si n’est mie grant merveille.
               Sire, dist ele, se je vous veoie desarmé, espoir que 
                     je vous congnoistroie miex. Dont 
                  
                  i leur a dit qu’i se desarmassent, car il
                  estoient hebergié. Voire, damoisele, se
                  vous nous 
                     voulezvoliez asseurer.
               
                  Et quant ele l’aMaintenant que ele l'ot entendu, si l’a congneu a la parole et li dist : Ha ! Mirrus, estes vous ce
                  ?
               Damoisele, dist il,
                  voirement sui je ce.
               
                     Et ou sont vostre compaignon?
               Par foi, damoisele, dist il, 
                     la veritévoirement
                  
                     vous enle vous dirai assez a temps, mais que, pour Dieu, que nous 
                     soionssions (sic) asseuré de vous.
               Certes, sire, dist elle, 
                     soiez certainssachiez que je ne vous veul mie traïr. Ne fust ore que 
                     poursi por
                  le preudomme
                  
                     quinon qui ci vous a envoiez et m’eussiez mon
                     frere occis a tort, si n’avriez vous garde de moi 
                     pour l’amour de lui, 
                     siainz vous 
                     metraimetroiemetrai je a sauveté et 
                     nient ne fustd'autre part ne fust riens de lui, si vous ai je fait atendre a touz les trespas de vostre chemin et
                  vous fesoie asavoir que pour mon frere ne
                  lessissiez mie que vous par moi ne 
                     revenissiezvenissiez, car j’avoie 
                     grantmoult grant desir de parler a vous.
               Damoisele, 
                     distrentdient il, bien va la besoigne selonc toutes aventures. Dont se sont fait
               desarmer a ii damoiseles, 
                  quique il trouverent leur haubers rous et despanez et eulz meismes 
                  trop durement navrezmoult navrez et distrentmoult navrez. 
                     Certes, dist la
                        damoisele, bienHa! biax seigneurs,
                  
                     avezvous avez eu mauvés encontre puis que vous de ci partistes !

                     Dites moi comment ce a esté.
               Ce vous dirons nous bien, dist Myrrus. Quant il furent desarmez, si leur 
                  furent aportees robes que il vestirentaporterent robes a leur mesure. Sire, 
                     ce dist la pucele, je croi que vous ne menjastes 
                     huianuit. Avant que je plus vous enquiere voil je que vous soupez. Dont a
               fait metre i petit banc 
                  devant
               i lit et 
                  la al'a fait estendre une nape et les a fait souper tant que il furent 
                  bien servi a leur volenté et de bonnes viandes. Quant il orent soupé,
               si leur a 
                  enquistout enquis
               la pucele
               comment il avoient esploitié depuis que il se 
                     partirentfurent parti de leens. Par foi, damoisele, dist
                     Mirrus, volentiers le vous dirai. \pend
            \pstart Dont ne volt lessier 
                  que, tout en cele maniere comme avenu 
                        lileur estoit depuis que 
                        d’ilec s’estoit departizdesi la ou il estoitdesi la, que tout ne 
                        lileur ait compté de chief en chiefqu'il ne li ait tout compté tout en autele maniere comme il li
                     estoit avenu. Quant la pucele l’entendi, si a geté
               moult granz souspirs et loa moult Nostre Seigneur de touz ses consentemenz et dist :
                  Biaus sire Diex, voirement consentez vous les
                  malices ! Mais au derrenier paiez vous a chascun sa desserte. Lors dist :
                  Biaus seigneurs, 
                     loeztoutes voies loez Dieu de ce que Il vous a 
                     jusques ci amenez a sauveté, car bien sachiez que, se vous fussiez alez vostre
                  chemin, 
                     que vous 
                     fussiezeussiez esté destruit. Et 
                     sachiezbien sai que il ne sera 
                     jamaismais heure 
                     deci adevant
                  i mois que li chemin ne soient aguetié, soit de Dyalogus
                  
                     ou soit de Ascanus. Mais il vous pueent
                  bien aguetier pour noient, car vous de ceenz n’istrez 
                     devantdesi adont que vous 
                     soiezserez
                  
                     bien asseurez d’eulz, se engins et sens ne me faut.
               Damoisele, distrent il, grans mercis, car poi 
                     vaudroitvaloit la force de ii chevaliers contre xx ou
                  xxx. 
                     EtCar sachiez de voir que, se nous de nos plaies estions gueriz, ja pour
                     v chevaliers 
                     oune pour vi
                  
                     nenous ne lairions nostre chemin a aler ne ne sejornerions en cest paÿs, si vendrions
                  en Frise.
               Par Dieu, 
                     ce dist la pucele,
                      de ce vous veul je bien croire. Mais vous soufferrez tant que vous serez 
                     touz
                  gueriz, et je vous aiderai de ce 
                     quedont vous avrez mestier, et ceenz n’avez vous garde 
                     se bonne non. En tele maniere demorerent 
                  li dui compaignon avec la pucelle. Si me
                  veul ore de eulz taire, 
                     car bien y savrai repairier quant il en sera 
                           tempspoins et lieus, et veul veniret repairerai a a i de leur
                     garçons qui eschapa des traïteurs quant li viii furent occis 
                     de leur compaignie sanz leur garçons, et cilz s’en vint par fiere aventure par divers paÿs en Frise. \pend
         
         
            
                  Comment le
                        garçon eschapa des traïteurs et s’en ala en Frise a Japhus, 
                        le filz au roy de
                                 Frise
                  Ensi comme li mes conta a Helcanus que Myrus et ses compaignons
                     furent occis el message.
            
               
               Enluminure de 11 UR sur une colonne. L’enfant s’échappe des traitres pour
                  rejoindre Japhus en Frise
            
            \pstart Ci endroit dit li comptes
               que, quant li garçons
               
                  fuse fu eschapez des traïteurs, il a tant alé que d’une part que d’autre que il 
                  est venusvint en Frise. Adont a enquis ou il 
                  trouveroitporroit trouver
               Japhus, le filz au roy de Frise. Il li fu 
                  enseigniédit en i chastel, et cilz
                  ne fina tant que il y est venus. Et la trouva touz yceulz que il demandoit. Et quant li 
                  auquantaucun l’ont veu, si li ont enquis nouveles. Il dist 
                  que il n’en savoit nulles qui bonnes fussent. Dont s’en vint a Helcanus et s’est lessiez cheoir a ses piez et dist :
                  Ha ! sire, tout sont mort et occis vostre bon ami
                  qui el message alerent ! Quant Helcanus
               l’entendi, dont n’eust 
                  i mot dit pour tout le monde, et 
                  n’enne failli gueres qu’il ne tresala. Mais cil, qui ot cuer suer Il manque le premier adjectif du doublet, ce qui peut
                  s'expliquer par une haplographie que nous corrigeons. et couvenable, a geté
                  i moult grant souspir et dist au garçon : Lieve sus
                  et ne di mot deci adont que je 
                     t’ente semondrai. Lors a apelé le duc de
                     Lembourt et son cousin, le filz au roy de
                     Frise. Meismes 
                  CliodorusElyodorus ne volt il mie oublier. Et 
                  dont sont entré en une chambre et sont assis li uns delez l’autre. Et dont fu li garçons apelez, et li distrent que
                  il leur contast comment il leur estoit
                  avenu. Dont dist cilz : Biaus seigneurs,
                  que voulez vous que je vous die ?
               Amis, dist Helcanus, conte nous comment il est avenu de tes mestres.
               Dont leur a cil conté comment Ascanus les avoit fait gaitier en la Forest de Vulgue et comment il 
                     ens'en estoient partiz par bataille, aprés comment il estoient venu au chastel de
                     
                        GonfortConfort, ou la pucele leur avoit
                  fait tele feste. Aprés leur compta comment Gasus avoit 
                     esté en Costentinnoble conseillieren Costentinnoble conseillié les messages et comment Mirrus prist
                  bataille 
                     contrea
                  Malquidant et 
                     l’occistcomment il l'ocist et comment il fu guetiez de Dyalogus et 
                     comment
                  Gasus l’en delivra, et puis aprés comment il 
                     se partirent de Costentinnoble et
                  s’en vindrent 
                     endesi a la Forest de Vulgue, et la
                  furent 
                     
                     i
                  
                     morttraÿ et mort et occis de celui 
                        AscanusEscanus, 
                     qui a l’aler les avoit 
                     faitfais gaitier, si comme il cuidoit. \pend
            \pstart Quant li baron 
                  ontorent celui 
                  entenduoÿ, si ont 
                  faitentr'euls fait moult grant duel et moult piteus. Qui dont oïst Helcanus
               
                  comment ilqui les regretoit, il ne fust nus qui grant pitié n’en deust avoir. Meismes
                  Mirrus regretoit il tant piteusement que ce
               estoit merveilles. Et quant il orent cessé 
                  leur duel, si 
                  demanderentdemanda au 
                  
                  i
               garçon se cil estoient tant de gent qui les siens avoient assailliz que par
                  nule aventure peust estre eschapez 
                     nulzde nul d’eulz. Et cil dist que 
                     touzil furent sorpris, et devant et desriere, et que bien 
                     furenten y ot plus de
                  c seur eulz x. Et quant il oÿrent ce, si sorent bien
               que il s’estoient mis a desfense et 
                  que avant se feissent occirre qu’il se fussent rendu. Par Dieu, dist 
                     Helcanuschascun, ce est tout voir. \pend
            \pstart Pas de nouveau § B
                     SireOre, dist li dus de Lemborc, bien est drois que on ait conseil de ceste
                  chose. Dont parla 
                  ClyodorusElyodorus : Merveilles ai, dist il, que nous 
                     n’avonsn'avons eu encore nul de ceulz qui 
                     alerentfurent envoié aus princes. Il ne puet estre, dist il, que nous n’en 
                     aionsoions nouveles temprement.
               Par foi, dist Helcanus, 
                     mar en quiermais n'en puis oïr nouveles 
                     puisquequant j’ai perduz mes bons amis !
               Sire, dist li
                     dus, souffrez vous : espoir que vous bonnes nouveles en orrez encore dont
                  vous touz liez serez. Mes 
                     orce me dites : qui est cil 
                     GasusGasus est (sic) qui si bien se prouva envers 
                     vosnosvous messages ?
                Par 
                     Dieufoi, 
                     ce dist Cliodorus, c’est i des 
                     plusL'absence du superlatif dans B n'est pas
                     commune. souffisanz chevaliers du monde. Et vous di que il est preudom de son cors et est souverain baillif de toute la terre de Costentinnoble, et me dist i jour qui passez
                  est que sus s'ame il n’estoit mie 
                        pour el elou paÿs 
                        fors pourcefors que qu’il ne vouloit mie que autres y fust qui le paÿs destruisist.
                  Et bien 
                     me distm'a dit que, de quele heure qu’il 
                        puisse savoirpuist
                     
                        qu’il y ait nul homicide ne que l’en veulle faire nul
                           tort aus hoirsque nul y ait des hoirs de Galilee, que ja puis confort ne
                     aide 
                        n’en avradoie avoir de lui Peliarmenus
                  .
               Par Dieu, dist Helcanus, ce veul je bien croire ! Et bien 
                     m’a moustrémoustre semblant d’amour, a ce que 
                     je en ai oÿ, etcil a dit qu’il a fait de mes messages. Et sanz faille je ai moult souvent mon seigneur de pere oÿ loer de lui.
               
                  Dont ont entr’eulz 
                        moult de paroles dites et moult de conseulz eusdit moult de paroles et eu moult de conseuls. Au chief 
                        dedu tout, Helcanus distEt puis dist Helcanus a Japhus son 
                  neveucousin : Quele aide avrai je de vous ?
               Cousin, dist il, toute l’aide 
                     que je vous porrai faire, et de moi et des miens, je la
                        vous ferai volentierset le confort de moi et des miens que nous porrons faire, cousin,
                        sachiés que nous le vous ferons, et vous parlerez a mon seigneur de
                     pere et vous 
                     humelierezhumiliez contre lui et li dites que il ne vous
                     doit faillir au besoing et que il mete conseil en ceste chose, et je sai
                  vraiement que il le fera. Et se ainssi estoit que il ne le feist, je et mes freres
                  en ferions nostre pooir.
               Biaus cousins, 
                     ce dist Helcanus, grans mercis
               ! \pend
            \pstart Dont orent conseil que il yroient au roy
               
                  etet que parleroient a lui. Il s’apresterent 
                  l’endemaina l'endemain et 
                  n’ont finé si vindrent la ou li roys estoit. 
                  Et quant il furent la venu, si li ont acointiéEt li ont compté
               ceste chose, comment Pelyarmenus
               
                  avoitot esploitié déSur la chute de la consonne finale de
                  "des", voir l'analyse linguistique. messages et 
                  commentainsi comme il cuidoit qu’il les eust 
                     faisfait occirre. Quant il oÿ ce, sachiez que moult 
                  enil en fu yriez, et dist : Biaus niez, sachiez que 
                     trop suimais sui trop viex por aler si loing. Mais vez ci Japhus mon filz, que je pri de tot mon cuer que il face tant de
                  ceste chose que il n’en soit repris de 
                     nulzvous, car de ma personne n’en puis je 
                     autre choseel faire. Quant Japhus oÿ ce, si dist
               : Et je el ne demant. Dont fist signe a
                  Helcanus qu’il en merciast son pere, et il si fist de ce que il pot. Adont se 
                  mistrenttrestrent d’une part et orent conseil que il
                  envoieroient par tout le roiaume la ou il porroient gent avoir, et il si
               firent, et que le tresor le roi estoit habandonnez a touz ceulz qui avoir en 
                  vouloientvodroient. \pend
            \pstart Pas de nouveau § BDont furent
               moult esmeuz par toute la terre et en mistrent en poi de tens a leur volenté 
                  pourde mouvoir du jour a l’endemain 
                  xxxvibien xxxvi mille. Quant li dus de Lembourc vit ce, si en fu 
                  moult
               
                  joiansjoyeux et dist que de la seue part en avroit 
                     bien
                  xm, que chevaliers que sergens, dont li moins 
                     vaillanzpuissanz
                  
                     cuideroitcuidoit
                  
                     bien valoir i conte endroit 
                     deque de sa
               personne.
                  Biau sire, dist Helcanus, grans mercis.
               Voire, dist il, et si vous en ai bien autant
                  porchacié qui vendront 
                     
                     q de Galylee, de quele heure qu’il 
                     en soient semons.
               Par mon chief, dist Helcanus, encore nous vendront 
                     gensgentg-genz (sic) d’ailleurs, car j’envoierai a mon
                     oncle le roy d’Arragon et li ferai savoir comment il
                  m’est.
               Sire, dist li
                     dus, vous ferez bien. Mais qui y porrons nous envoier ?
               Par foy, dist il, je ne sai.
               Sire, dist Clyodorus, je yrai el message, se il vous plest.
               Par Dieu, dist il, amis, et je l’otroi. Dont
               li fu li messages enchargiez, et 
                  il s’est aprestez, o lui i escuier et i garçon sanz plus,
               mais sanz armes 
                  n’ine volt il mie aler, 
                  caret
               Helcanus li conseilla. \pend
            \pstart En tele maniere se mist Clyodorus en son chemin, et li autre demorerent, qui de jour en jour
               atendoient response des princes de Costentinnoble. Ne demoura que, viii jours aprés, 
                  quequant
               iii messages vindrent, qui Grece
               avoient cerchiee et aportoient la response de
                  
                  c chascun. Mais tout en tele maniere comme li uns mandoit ne 
                  mandoitfaisoit mie li autres. 
                  QuantCar li uns disoit : 
                     Quantque li emperieres Ca
                        s‹r›[s]sidorus vendra en sa terre, si comme il devra, nous ferons de
                  lui comme de 
                     nostreno seigneur ; li secons 
                  mandoitremandoit que, quant Helcanus, li ainsnez filz 
                     Cassidorusde l'empereur Cassidorus, vendroit en la terre quant il le voudroit faire, il en feroient tant que ja
                     
                     b
                  

                     blasme n'en feroientblasmerblasmez d’omme sage a son 
                     esgartesgart n'en seroit ; li tiers mandoit que 
                        HelcanusHelsanus (sic)
                  
                     venisten traist hardiement en la terre 
                     par lequelauquel chief que il voudroit, 
                     et il yroit a l’encontre de 
                     lui a toutLa variante de B renverse la perpective.
                  vi mille hommes, qui touz li seroient contraires de jusquesLa structure "de jusques" est répétitive. Peut-être
                     provient-elle de "dusques". a la mort. De ceulz y ot le plus qui
               ce li ont mandé, et si en y ot d’autres qui le contraire 
                  remanderentreremanderent (sic), mais il ne voudrent mie que touz le seussent que hardiement venist
               contre Pelyarmenus
               
                  amis a bataille, et il li 
                  aideroientaideroient a destruire, de ce fust 
                  il touz fiz. \pend
            \pstart Pas de nouveau § BQuant
                  Helcanus oÿ ce, si dist que il n’i avoit el que cil qui 
                     aidiereudier (sic) li voudroient 
                     s’appareillassents'apareillenssent (sic)"il dit qu'il fallait seulement que ceux
                  qui voudraient l'aider se préparassent"., car 
                  plus ne vouloientil plus ne voloit atendre de aler sus Peliarmenus, et
               premierement sus Ascanus, car la estoit li
               passages. Dont se mist li dus au chemin et dist qu’il apresteroit sa gent si hastivement 
                     que bien cuideroit tostcom il bien cuidoit entrer en la terre, mais 
                     queque il trop ne se hastassent. Sire, dist
                     Japhus, alez et si hastez vostre besoingne,
                  car de ma partie 
                     n’atendraiL'erreur de G ne s'explique nullement ni par un
                     saut, un passage à la ligne ou changement de colonne de V2 je 
                     ne vous ne autrevous ne autrui, car bien ai gent pour entrer en la terre. Dont se mist li dus au chemin, et Japhus et sa gent 
                  se mistrentmis ensemble. Et bien en y ot dedenz xv jours xl mille. Ce
               fu iiii mille plus que il 
                  n’enne cuidoit avoir du mandement premier, mais aussi comme tout le paÿs le suivoit
               pource qu’il avoit le non d’estre larges. Il 
                  ens'en vindrent a la mer et 
                  trouverentfurent les nés
               prestes.
               Dont entrerent enz et orent 
                  bon vent a leur volenté, et n’ont finé si sont 

                  arrivé 
                        et venu a terrebien et belvenus a terre seche. \pend
            \pstart Dont sont 
                  les nouveles 
                  couruescouruees (sic) parmi 
                     Alemaingnel'Alemaigne et par les terres que Frison estoient hors issu pour aler en Grece. 
                  Et lesCesLes nouveles 
                  vindrent 
                        lajusques laalerent desi la et estoient 
                  aussi comme toutja aussi com pourveu, comme cil qui de jour en jour n’atendoient autre chose. Ascanus, qui bien savoit que il aroit 
                  a faire aule premier assaut, a fait sa gent entrer en ses chastiaus et 
                  a bien 
                        etet/et (sic) richement garni sa cité de Nisebien garnir. Sa cité de Nise avoit il fait moult richement
                     aprester, comme cele qui si fort estoit, 
                     comme ilque bien sera dit ci aprés.
               Frison, qui ne targoient se poi non, 
                  n’ont finéont tant alé et chevauchié par leur jornees 
                  tant qu’il sont entré en Ytalie Majour – ce
               estoit la terre ou Ascanus demouroit. Lors sont
                  Frison enz entré et ont couru par le paÿs et fait moult grans domages a ceulz
               de la terre. Dont envaïrent villes et chastiaus 
                  et n’en trouvoient nulz qui longuement se peust tenir contre eulz, si que nul
               n’en 
                  pristrentprenoient a force 
                  qu’il ne meissent a l’espee ceulz qui dedenz estoientque touz ne meissent a l'espee. Que vous 
                     feroie je mention de touz les assausdiroie je dez grans assauz que il firent en la terre ? Sachiez que trop aroie a faire, car avant que
                  je 
                     mem'en taise, cuit je 
                     bienmoult anuier avant que j’aie raconté les grans assaus 
                     et les grans paines que Helcanus et li sien
               souffrirent, avant que vii anz fussent passé que il furent en Grece, avant que 
                  il eussent gueres de repos ne de sejourpou eussent de repos. \pend
            \pstart Helcanus et son cousin
               
                  ne finerentn'ont finé, 
                  ainz vindrent la terre destrui
                        
                        ssant etsi vindrent par la terre destruisant le paÿs et touz ceulz qui contre 
                  eulzlui estoient, 
                  
                        etil oïrent parlerIl oïrent nouveles que Ascanus avoit o lui sa baronnie en la cité de Nise. Dont ne voudrent lessier que il ne se meissent cele
               part. Il ont tant alé qu’il ont la cité aprouchiee.
                  Ascanus, qui bien 
                  savoitsot leur 
                  venuvenueLa première attestation de "venu" comme mot masculin
                  sans "-e" final est plutôt tardive (1440) d'après le DMF. Cependant, la chute de
                  la voyelle finale "-e" se retrouve dans d'autres passages comme le §151, si bien
                  qu'il nous paraît préférable de maintenir la leçon de V2, avec une hésitation sur
                  le genre du nom "venu"., pensa qu’il ne
                  se lairoit mie assegier sanz 
                     tropcopLa leçon de BX2 paraît supérieure à celle de V2G
                     dans la mesure où elle reprend une formule "sanz cop ferir" bien connue encore
                     aujourd'hui. Cependant, étant donné que l'emploi absolu du verbe "ferir"
                     "donner des coups" est bien attesté (DMF), la leçon de V2G nous semble
                     sensée. ferir. Il ot sa gent aprestee, et furent bien xxxm, car Pelyarmenus li avoit envoié grant gent. 
                  MeismesMes
               Dyalogus
               
                  y estoit, 
                  avec lui ses traïteurs, si comme Pelyarmenus
               
                  les avoit banisavoit fait banir de sa terre par la volenté Gasus, si comme devant est dist. 
                  LorsDont sont cil de la cité
               
                  issuissi hors et se mistrent contre eulz. Li dui 
                  frerefrere Japhus, qui toute jor avoient chevauchié par le paÿs, 
                  si comme dessus est dit, dont li ainsnez avoit 
                   nona nom
               
                  
                  i
               Heleus et li autres Nazareus, distrent qu’il vouloient
                  estre les premiers a la cité asseoir. Dont leur fu otroié, et 
                  avoientavoit chascun xm homes en 
                  leurson conduit. Japhus en avoit xvm et Helcanus en 
                  avoit bien xxmravoit vm qui 
                  assezadés li estoient 
                  presmoult pres. Dont chevaucha premiers Nazareus et 
                  vitvirent sa gent que cil de la cité estoient issu 
                  dehorshors. 
                  Lors fistIl l'ont fait assavoir a leur gent qu’il aroient estour. Dont 
                  chevauchieretchevauchierentSur l'effacement de la marque du pluriel, trait
                  linguistique attesté dans l'Est, voir l'analyse linguistique XXX. serré et
               tant que 
                  entreen la gent Nazareus avoit i
               chevalier qui moult estoit prisiez d’armes. Cil
               s’apensa que il ferroit avant. 
                  Et cilz chevaliers estoit apelezCil ot non
               Melior. Si feri a l’aprochier cheval des esperons
               tant aigrement que trop fu bel a veoir. Et quant li autre l’ont veu, Dyalogus, qui el premier chief estoit, 
                  ne se fist pas prier, ainz a fait le 
                  chevalsien cheval sentir les esperons et vint contre Melior si
               adroit qu’il s’entrencontrerent de si 
                  granttres grant force que il ont mis les escuz en pieces et les 
                  lancesglaives en asteles et les hiaumes hors des chiez, mes 
                  onquesil onques pis ne se firent, ainz passerent chascuns 
                  
                  o outre, joinz en 
                  leurses armes. A ce cop sont feruz ensemble. 
                  LaSi y ot i estour tres dur. 
                     NazareusNazarus, qui moult estoit fort et plains de grant chevalerie, s’est
               feruz en 
                  l’estourl'estour moult aigrement, et aussi firent cilz qui avec lui estoient. Et Dyalogus et li sien si aidierent moult esforciement, meismes uns
               princes d’Ytalie, qui Argus
               
                  avoitot non, qui la premiere bataille avoit, si esforcieement si aida 
                  luiilli et li sien 
                  quequi moult donnerent a souffrir a Nazareuz et
               aus siens. Que vous diroie je ? Tant ont li i et
               li autre chaplé que moult sont las et travaillié. \pend
            \pstart Adont s’est Heleus
               feru en la bataille, 
                  luiilli et les siens. Et d’autre part y est venuz Carus. Cil estoit freres Ascanus ;
               cil avoit o lui v
               
                  chevaucheureschevaucheurs. Quant cil se furent 
                  misferu el tas, 
                  dont ysachiez que moult peust on veoir 
                  desde mors 
                  a grant foison en la plaine. Melior, qui les rens aloit 
                  assez cerchant, n’estoit pas oiseus, ainçois, si comme 
                     li comptes 
                           racontele racontedit li contes, encontrerent li uns l’autre 
                  entre lui etde lui et de
               Dyalogus. A celui estour eust esté pais 
                  de Dyalogus le traitre
                     se n’eussent estédu traïteur, ne fussentli sien, qui le secorurent. 
                  MoultAinsi se maintenoient l’une partie et l’autre 
                  bien, et tant que 
                  ce vint a la parfin 
                  que la gent Ascanus furent au dessouz, 
                  si comme 
                           l’enle pot bien veoir. \pend
            \pstart Pas de nouveau § BQuant li
               auquant virent ce, si se sont mis avant. Meismes Helcanus
               
                  et liil et li (sic) sien ne voudrent plus souffrir que il ne soient desrengié et feru en la
               bataille. Qui dont veist Helcanus sa terre
               chalengier, bien peust dire : Cilz n’a 
                     curesoing que nul tort l’en 
                     li face !, car il aloit par la bataille ferant a destre et a senestre.
                  Que mal de celui qui voie ne li feist !
               
                  Et d’autre part aloitDe l'autre part se desrenga
               Ascanus. Et cilz n’estoit pas hom comme autres,
               car si 
                  esragiementenragiement se penoit d’eulz confondre que bien cuidoit que nulz ne se 
                  peust prendredeust tenir a lui. 
                  Dont feroit d’une part et d'autreet feroit en tel maniere que moult estoient si cop redouté. Il avint par aventure que 
                  lui et Helcanus
                     
                        s'entrecontrerents’entrenconrent (sic)il avinrent l'un contre l'autre Helcanus et celui Ascanus. 
                  
                     
                        Adont ne s’entrecongnurentet ne se cognurent mie, comme cil qui n’estoient mie bon ami ensemble, ainz avoient moult
                     grant desir de faire d’armes, et le jour en firent tant que bien fait a
                     recorder.Dont ne s'entreconnurent mie comme ceuls qui n'estoient pas garni de
                     leur connoissances, ainz se doutoient que il ne fussent conneus par quoi l'en
                     meist plus grant force a euls occire ou prendre et il voloient faire d'armes,
                     et il le jour en firent tant que bien fist a recorder, porce que il avoient veu
                     comment il uns faisoit damage a l'autre Il s’entrevindrent de si grant force 
                  et de 
                        si granttel aïr, les espees es mains et les escuz mis avant, et 
                  leurles chevaus 
                  estoient fors et 
                        roidesrades (sic)roides et courans, et eulz meismes de 
                  grant forcetres grant ire esmeu. Dont ont ferus si adroit sus leur armes que 
                  moult s’en tint a chargié tous li plus preus"le plus vaillant se sentait tout à fait
                        attaqué".
                     . Mais
                        Ascanus fu si durement ferus d’un cop
                     merveilleus que Helcanus li douna en tel
                     maniere et de si grant force que il ne pot le cop endurer, ainz trebuscha a
                     terrebien le pot l'en veoir entour euls. Et cil qui miex fu ferus et de
                     braz plus couvenable, ce fu Ascanus, qui le cop ne pot endurer, ainz couvint , lui et le 
                  chevalcheval cheoir tout en i mont.
                  LorsLa s’arresta Helcanus seur lui 
                  et le cuidaqui le cuida du tout metre a mort, mais tant y ot des siens que bien y parut, car, se li comptes ne ment, la eust Helcanus esté mort ou pris se il ne s’i 
                  fustfu si bien maintenu, et meesmement li sien, qui partout le suioient, si comme
               faon leur mere. Que vous diroie je 
                     lonc 
                           prologueprolongueD'après les dictionnaires, l'expression habituelle
                     est plutôt "Que vous feroie long prologue?" avec une colocation de "prologue"
                     avec "faire" plutôt que dire (GD X, 430a; TL VII, 1965, 38; DMF). L'association
                     avec "dire" se retrouve néanmoins hors de questions rhétoriques stéréotypées.
                     La leçon de B se rattache au même mot "prologue" dont elle est une variante (TL
                     VII, 1965, 48). La leçon de G permet elle de revenir à une construction mieux
                     connue et plus simple. ? Tant 
                  y
               feri Helcanus et 
                  lui etL'omission de "lui et" dans BX2 change le référent de
                  "li sien". On y lit B un accord de proximité par lequel "Helcanus" et "li sien"
                  sont sujet de "feri". Dans la version de V2G, la référence à "li sien" est
                  peut-être différente: elle renvoie aux chevaliers d'Ascanus ; "li sien" est
                  complément d'objet. Le segment "et lui et li sien" peut aussi former l'autre sujet
                  de "ferir", quoique la leçon soit alors répétitive. li sien que par droite
               force Ytalien se sont retrait vers la cité avant que 
                  mestierbesoing fust que Japhus
               
                  et li sien s’en fussent meslés'en fust ja meslez ne li sien. Dont virent il que noient 
                  porroientne porroientpourroient il faire 
                  fors que de eulz faire occirre. 
                  LorsSi sont entré 
                  en la villeenz entré ceulz qui 
                  entrer y 
                        porentpouoientfaire le porrent. Et sachiez que moult en y ot de mors et de pris 
                  et moult plus 
                        perilleuscrueus chaple 
                        y ot que on n’avoit pieça veuque l'en eust veu pieça. Dont pristrent terre et d’une part et d’autre et ne demoura gueres que tot
               furent devant la cité. Et ont tendu 
                  tentes et paveillonspaveillons et trés partout la ou il porent et furent assez tost logiez. En tele maniere fu la cité assise de ceulz de Frise. Il fu tart, car la bataille avoit grant piece duré. Li pluseur
               qui navré se sentirent se firent desarmer et ont fait prendre garde a leur plaies. Il
               orent bons mires et furent bien gardé. Cil qui du souper 
                  s’entremistrentse durent entremetre le firent. Et quant il fu temps de souper, 
                  si s’assistrent et mengierent cil qui mestier en avoient,
                     puissi l'ont moult tost fait comme cil qui el ne queroient, ainz fu commandé 
                  le gaitl'agait a xm
               Frisons, et en fu souverains i
               duc de 
                        Landela terre, qui moult estoit bons chevaliers. La nuit passa et
               l’endemain vint. 
                  Dontquesi sont li auquant alé veoir la cité. Dont l’ont 
                  veuetrouvee si fort que bien leur 

                  semblesambla
               
                  pourque, pour assaut que l’en y feist 
                  que jamais ne deust estre prise. Dont 
                  dist Japhus a Helcanusfu ce dit a Japhus et a Helcanus : 
                  Or ne vous esmaiezL'autocorrection de V2 indique peut-être une hésitation sur la
                           diphtongue "ai" qui peut ailleurs être rendue par la graphie "a", cf.
                           analyse linguistique., car encore n’i avrons nous em piece esté
                           ix anz ne x. Et sachiez certainement que je ne m’en
                        quier jamais partir si avrai 
                           
                           i
                        mis Ascanus au dessouz ou il
                        moi.
                  qui distrent : Encor n'i avons nous
                        pas esté ix ans ne x. Par Dieu, dist Japhus, je ne m'en cuit partir, si aurai
                        Helcanus mis a merci ou il moiEt quant si homme l’ont oÿ, si 
                  onts'est chascuns pensé d’enforcier sa
               loge.
                  Si me veul 
                     ore
                  i poi d’eulz taire et veul 
                     repairiervenir a Myrrus, qui de tout ce ne savoit encore
                  riens. \pend
         
         
            
                  Comment Mirrus
                     
                        est demourédemoura avec la pucele et y demoura par
                        ii mois 
                        entre lui et son 
                           compaignoncompaignon, EligiusEnsi comme Myrus et son compagnon estoient en Gonfort avuec la
                     pucele.
            
               
               Enluminure de 12 UR sur une colonne. Mirus reste deux mois avec la Mélode,
                  entouré d’un jeune homme et d’une jeune femme sous un pavillon devant le
                  château
            
         
      
   

\end{pages}
\end{document}
        